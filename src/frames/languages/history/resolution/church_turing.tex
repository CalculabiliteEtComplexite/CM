% SPDX-License-Identifier: CC-BY-SA-4.0
% Author: Matthieu Perrin
% Part: <Nom de la partie>
% Section: <Nom de la section>
% Sub-section: <Nom de la sous-section>  % (facultatif, laisser vide si non utilisé)
% Frame: <Titre de la slide>

\begingroup

\begin{frame}{Qu'est-ce qu'une procédure mécanique ?}
 
  \onBlock[right=.7\textwidth, top=-2mm]{Développement du $\lambda$-calcul (années 1930)}{
    \begin{itemize}
    \item Tout est fonction
    \item Le calcul est une réécriture symbolique
    \item Formalise une notion abstraite de calcul effectif
    \item Modèle de la \structure{programmation fonctionnelle}
    \end{itemize}
  }
 
  \onBlock[left=.7\textwidth, bottom=10mm]{La machine de Turing (1936)}{
    \begin{itemize}
    \item Modèle mécanique et opérationnel
    \item Formalisme proche des automates
    \item Définit une notion de fonction calculable
    \item Modèle de la \structure{programmation impérative}
    \end{itemize}
  }

  \on[bottom=6mm]{\alert{Les deux modèles définissent les mêmes fonctions}}

  
  \onImage[x=-.35\textwidth,top]{%
    height=3cm,
    title={Alonzo Church},
    licenselogo={$\copyright$},
    license*={$\copyright$ - Princeton University Library (voir \href{https://en.wikipedia.org/wiki/File:Alonzo_Church.jpg}{Wikimedia}). Utilisation non commerciale à des fins pédagogiques (fair use)},
    img={Church.jpg}
  }
 
  \onImage[x=.35\textwidth,bottom=10mm]{%
    height=3cm,
    title={Alan Turing},
    licenselogo={\ccPublicDomain},
    license*={Domaine public UK (1951, \href{https://commons.wikimedia.org/wiki/File:Alan_Turing_(1951).jpg}{Wikimedia})},
    img={Turing.jpg}
  }
  
\end{frame}

\endgroup
