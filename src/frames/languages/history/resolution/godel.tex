% SPDX-License-Identifier: CC-BY-SA-4.0
% Author: Matthieu Perrin
% Part: <Nom de la partie>
% Section: <Nom de la section>
% Sub-section: <Nom de la sous-section>  % (facultatif, laisser vide si non utilisé)
% Frame: <Titre de la slide>

\begingroup

\begin{frame}{Vers une axiomatisation nécessairement incomplète}
 
  \onBlock[right=.75\textwidth, top=-2mm]{Théorème d'incomplétude de Gödel (1931)}{
    Il n'existe pas de système de preuve :
    \begin{itemize}
    \item formel (\textsc{Vérification des preuves} est décidable)
    \item suffisamment expressif pour l'arithmétique entière
    \item à la fois complet et cohérent
    \end{itemize}
  }
 
  \onBlock<2->[left=.71\textwidth, bottom=-3mm]{Vers une théorie des ensembles cohérente}{
    \begin{itemize}
    \item Schéma de compréhension restreint
      $\structure{\{x \alert{\in E} \mid P(x) \}}$
    \item La base de l'axiomatisation moderne (ZFC)
      \begin{itemize}
      \item Complétée par Fraenkel, Skolem, Von Neumann...
      \end{itemize}
    \item Cohérence toujours indémontrée
    \end{itemize}
    \begin{minipage}{.9\textwidth}
      \myquote{David Hilbert}{Personne ne doit pouvoir nous chasser du paradis que Cantor nous a créé}
    \end{minipage}
  }
  
  \onImage[x=-.4\textwidth,top]{%
    height=25mm,
    title={Kurt Gödel},
    licenselogo={\ccPublicDomain},
    license*={Domaine public (Vienne, 1925. \href{https://commons.wikimedia.org/wiki/File:Young_Kurt_G\%C3\%B6del_as_a_student_in_1925.jpg}{Wikimedia})},
    img={Godel.jpg}
  }
 
  \onImage<2->[x=.4\textwidth,bottom=5mm]{%
    height=30mm,
    title={Ernst Zermelo},
    licenselogo={\ccPublicDomain},
    license*={Domaine public (Italie, vers 1900. \href{https://commons.wikimedia.org/wiki/File:Ernst_Zermelo_1900s.jpg}{Wikimedia})},
    img={Zermelo.jpg}
  }
  
\end{frame}




%\begin{frame}{Thèse de Church -- Turing}
% 
%  \onBlock[left=.75\textwidth, y=22mm]{Thèse de Church -- Turing (1936-1937)}{
%    \centering
%    \structure{La définition des \og fonctions calculables \fg par des \\
%      \alert{Machines de Turing déterministes}  \\
%      caractérise la notion intuitive de \og procédure effective \fg.}
%  }
%  
%  \onBlock[y=-15mm]{Arguments en faveur de la thèse}{
%    \begin{itemize}
%    \item Équivalence entre formalismes
%      \begin{itemize}
%      \item Machines de Turing, $\lambda$-calcul, langages de programmation...
%      \end{itemize}
%    \item On ne connaît pas de machine plus puissante 
%      \begin{itemize}
%      \item Les formalismes plus expressifs necessitent des \og oracles \fg
%        \begin{itemize}
%        \item Par exemple, on peut \emph{définir} des objets mathématiques non-calculables
%        \end{itemize}
%      \item Les limites au formalisme sont internes
%        \begin{itemize}
%        \item L'indécidabilité du \structure{problème de l'arrêt} vient d'un paradoxe
%        \end{itemize}
%      \end{itemize}
%    \item Possibilité de simuler l'univers...
%      \begin{itemize}
%      \item ... donc toute machine qui peut y être effectivement construite 
%      \end{itemize}
%    \end{itemize}
%  }
% 
%  \onImage[x=42mm,y=15mm]{%
%    width=2.3cm,
%    title={Alonzo Church},
%    license={$\copyright$ - Princeton University Library (voir \href{https://en.wikipedia.org/wiki/File:Alonzo_Church.jpg}{Wikimedia}). Utilisation non commerciale à des fins pédagogiques (fair use)},
%    img={Church.jpg}
%  }
% 
% 
%%  Constructivisme mathématique
%%  
%%  L.E.J. Brouwer, Intuitionism and Formalism (1908) :
%%  
%%  « A mathematical assertion is true only if we have a construction for it. »
%  
%\end{frame}



\endgroup
