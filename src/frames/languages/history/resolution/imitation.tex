% SPDX-License-Identifier: CC-BY-SA-4.0
% Author: Matthieu Perrin
% Part: <Nom de la partie>
% Section: <Nom de la section>
% Sub-section: <Nom de la sous-section>  % (facultatif, laisser vide si non utilisé)
% Frame: <Titre de la slide>

\begingroup

\begin{frame}{Le jeu de l'imitation}

  \on[y=-5mm]{
    \begin{tikzpicture}
      \node [faded background picture=Village,    text width=\paperwidth/3] (A) at (-\paperwidth/3,0) {};
      \node [faded background picture=Salon, text width=\paperwidth/3]      (B) at ( 0            ,0) {};
      \node [faded background picture=Machineroom, text width=\paperwidth/3] (C) at ( \paperwidth/3,0) {};

      \node[anchor=south, outer sep=0pt, inner sep=0pt] at (A.south) {\includegraphics[height=22mm]{Bob}};
      \node[anchor=south, outer sep=0pt, inner sep=0pt] at (B.south) {\includegraphics[height=22mm]{Carole}};
      \node[anchor=south, outer sep=0pt, inner sep=0pt] at (C.south) {\includegraphics[height=22mm]{Robot}};
    \end{tikzpicture}
  }

  \on<2>[x=-\paperwidth/3, y=-15mm]{
    \chatBubble[color=example]{Thé, mails, métro}
  }
  \on<3>[x=-\paperwidth/3, y=-15mm]{
    \chatBubble[color=example]{Raté !}
  }

  \on<1>[x=0, y=-15mm]{
    \chatBubble[color=alert]{Décris ta matinée en trois mots}
  }
  \on<3>[x=0, y=-15mm]{
    \chatBubble[color=alert]{Humain ?}
  }

  \on<2>[x=\paperwidth/3, y=-15mm]{
    \chatBubble[color=structure]{Café, mails, métro}
  }
  \on<3>[x=\paperwidth/3, y=-15mm]{
    \chatBubble[color=structure]{Raté !}
  }

  \on[x=-\paperwidth/3, y=15mm]{
    \begin{chat}[color=example]{Carole}
      \chatRecv[color=alert]{Décris ta matinée en trois mots}
      \only<2->{\chatSend{Thé, mails, métro}}
    \end{chat}
  }

  \on[x=0, y=15mm]{
    \begin{chat}[color=alert]{Mystère}
      \chatSend{Décris ta matinée en trois mots}
      \only<2->{\chatRecv[color=structure]{Café, mails, métro}}
    \end{chat}
  }

  \on[x=\paperwidth/3, y=15mm]{
    \begin{chat}[color=structure]{Carole}
      \chatRecv[color=alert]{Décris ta matinée en trois mots}
      \only<2->{\chatSend{Café, mails, métro}}
    \end{chat}
  }

\end{frame}

\endgroup
