% SPDX-License-Identifier: CC-BY-SA-4.0
% Author: Matthieu Perrin
% Part: <Nom de la partie>
% Section: <Nom de la section>
% Sub-section: <Nom de la sous-section>  % (facultatif, laisser vide si non utilisé)
% Frame: <Titre de la slide>

\begingroup

\begin{frame}{Une machine peut-elle penser ?}
  
  \begin{block}{L'intelligence humaine comme fonction entre des langages}
    Dans le test de Turing, l'homme est vu comme une \structure{relation} : 
    \begin{itemize}
    \item on lui donne un texte en entrée
    \item il répond un texte en sortie
    \end{itemize}
    La \structure{manière} dont la réponse est produite est hors de portée de l'observateur :
  \myquote{Alan Turing}{La seule manière d’être certain qu'une machine pense serait d’être cette machine et de se sentir penser.}
  \end{block}

  \vspace{-5mm}
  \begin{block}{Quelle est la puissance de calcul de l'intelligence humaine ?}
    \begin{itemize}
    \item si l'humain est observé par ses seules productions symboliques,
    \item en quoi serait-il calculatoirement plus puissant qu'une machine ?
    \end{itemize}
  \myquote{Alan Turing}{On peut espérer que les machines finiront par rivaliser avec l'homme dans tous les domaines purement intellectuels.}
  \end{block}

\end{frame}

\endgroup
