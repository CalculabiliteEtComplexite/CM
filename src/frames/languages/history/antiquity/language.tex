% SPDX-License-Identifier: CC-BY-SA-4.0
% Author: Matthieu Perrin
% Part: 
% Section: 
% Sub-section: 
% Frame: 

\begingroup

\begin{frame}{Au commencement était le Verbe\footnote{\structure{Évangile selon Saint-Jean} : Le mot \structure{``Verbe''} est ici la traduction de \structure{``Logos''}.}}

  \onBlock[top=-3mm]{Penser, c'est parler}{}

  \onBlock[y=-3mm]{Finitude des descriptions, infinité des objets}{
    \vspace{-2mm}
    \begin{itemize}
    \item Un langage est un ensemble de mots sur un alphabet
    \item Les mots peuvent avoir une \structure{structure interne} et représenter des données
    \item Un langage définit un prédicat sur des données à \alert{représentation finie}
    \end{itemize}
  }

  \on[width=.55\textwidth, x=-32mm, top=6mm]{
    \myquote{Ernst Cassirer}{
      Le langage n’est pas un simple moyen de communication, mais un médium universel de la pensée.
    }
  }

  \on[width=.45\textwidth, x=26mm, top=6mm]{
    \myquote{Lev Vygotsky}{
      La pensée ne s’exprime pas simplement par des mots : elle prend existence à travers eux.
    }
  }
    
  \on[width=.45\textwidth, x=-37mm, bottom=2mm]{
    \myquote{Stephen Kleene}{
      Les moyens finis constituent le seul accès possible aux processus infinis.
    }
  }
  
  \on[width=.55\textwidth, x=22mm, bottom=-1mm]{
    \myquote{Noam Chomsky}{
      Le langage humain repose sur un ensemble fini d’éléments qui peuvent être combinés d’un nombre infini de manières.
    }
  }

\end{frame}

\endgroup
\endinput

