% SPDX-License-Identifier: CC-BY-SA-4.0
% Author: Matthieu Perrin
% Part: 
% Section: 
% Sub-section: 
% Frame: 

\begingroup

\begin{frame}{Mathématiques de l'âge du bronze}
 
  \onBlock[left=.7\textwidth, top=-2mm]{Mathématiques mésopotamiennes}{
    Approche procédurale de calculs avancés
    \begin{itemize}
    \item Multiplication, inverse, racine...
    \item Triplets pythagoriciens
    \item Numération en base 60
    \end{itemize}
  }
 
  \onBlock<2->[right=.6\textwidth, bottom=6mm]{Géométrie égyptienne}{
    Procédures systématiques de construction
    \begin{itemize}
    \item Problèmes de géométrie pratique résolus
    \item Calcul d’aires et de volumes
    \item Numération en base 10
    \item Procédures pour approcher l'aire du cercle par celui d'un carré
    \end{itemize}
  }
 
  \onImage[x=.3\textwidth,top]{%
    height=25mm,
    title={Tablette Plimpton 322},
    licenselogo={\ccPublicDomain},
    license*={Domaine public (Irak, vers -1700 : Photo : \href{https://www.sciencesetavenir.fr/archeo-paleo/archeologie/la-trigonometrie-version-babylonienne_116888}{Andrew Kelly})},
    img={plimpton.jpg}
  }
 
  \onImage<2->[x=-.3\textwidth,bottom=5mm]{%
    height=3.5cm,
    title={Papyrus Rhind},
    licenselogo={\ccPublicDomain},
    license*={Domaine public (Égypte, vers -1650. Photo : \href{https://old.maa.org/press/periodicals/convergence/mathematical-treasure-the-rhind-and-moscow-mathematical-papyri}{Frank J. Swetz})},
    img={rhind.png}
  }
  
\end{frame}

\endgroup
\endinput

