% SPDX-License-Identifier: CC-BY-SA-4.0
% Author: Matthieu Perrin
% Part: 
% Section: 
% Sub-section: 
% Frame: 

\begingroup

\begin{frame}{Des fondements qui se fissurent}

  \onBlock[left=.6\textwidth, top=-3mm]{Coordonnées cartésiennes}{
    Approche algébrique de la géométrie
    \begin{itemize}
    \item Rupture avec la géométrie constructive
    \item Les figures ne sont plus le centre
      \begin{itemize}
      \item Nombres, tuples, ensembles
      \end{itemize}
    \item Pas d'axiomatisation formelle
    \end{itemize}
  }

  \onBlock<2->[right=.65\textwidth, bottom=5mm]{Limites de l'algèbre géométrique}{
    Construction à la règle et au compas : 
    \begin{itemize}
    \item Caractérisation des nombres \structure{constructibles}
      \begin{itemize}
      \item Théorème de Wantzel (1837)
      \end{itemize}
    \item Le nombre $\pi$ est \structure{transcendant}
      \begin{itemize}
      \item Théorème de Lindemann–Weierstrass (1882)
      \end{itemize}
    \item Donc la quadrature du cercle est impossible
    \end{itemize}
  }

  \onImage[x=.33\textwidth,top]{%
    height=2.8cm,
    title={René Descartes},
    licenselogo={\ccPublicDomain},
    license*={Domaine public (Frans Hals. \emph{Portret van René Descartes}. 1649. \href{https://commons.wikimedia.org/wiki/File:Frans_Hals_-_Portret_van_Ren\%C3\%A9_Descartes.jpg}{Wikimedia})},
    img={Descartes.jpg}
  }

  \onImage<2->[x=-.33\textwidth,bottom=5mm]{%
    height=2.8cm,
    title={Ferdinand von Lindemann},
    licenselogo={\ccPublicDomain},
    license*={Domaine public (1852-1939. \href{https://commons.wikimedia.org/wiki/File:Carl_Louis_Ferdinand_von_Lindemann.jpg}{Wikimedia})},
    img={Lindemann.jpg}
  }
  
\end{frame}

\endgroup
