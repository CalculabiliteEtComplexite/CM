% SPDX-License-Identifier: CC-BY-SA-4.0
% Author: Matthieu Perrin
% Part: <Nom de la partie>
% Section: <Nom de la section>
% Sub-section: <Nom de la sous-section>  % (facultatif, laisser vide si non utilisé)
% Frame: <Titre de la slide>

\begingroup

\newcommand\PROBLEM{\textsc{problem}}

\begin{frame}{Expressivité des langages de description}

  On veut un langage $\langle \mathcal{L}, \llbracket \cdot \rrbracket \rangle$ où tout problème de décision est représentable :

  \vspace{-2mm}
  $$\alert{\forall \PROBLEM \in \textsc{lang},~ \exists u \in \mathcal{L},~ \PROBLEM = \llbracket u \rrbracket}$$
  
  \begin{block}{Théorème -- Corollaire du théorème de Cantor}
    Il n'existe pas de langage de description $\langle \mathcal{L}, \llbracket \cdot \rrbracket \rangle$
    où $\llbracket \cdot \rrbracket$ est surjective.
  \end{block}

  \vspace{-1mm}
  \begin{alertblock}{Démonstration}
    \begin{itemize}
    \item Supposons (par contradiction) qu'il existe un tel $\langle \mathcal{L}, \llbracket \cdot \rrbracket \rangle$.
    \item Posons $E = \{ x \in \mathcal{L} \mid x \notin \llbracket x \rrbracket\}$.
    \item Par surjectivité, il existe $y \in \mathcal{L}$ tel que $E = \llbracket y \rrbracket$.
    \item A-t-on $y \in \llbracket y \rrbracket$ ?
    \end{itemize}
  \end{alertblock}

  \vspace{-1mm}
  \begin{block}{Remarques}
    C'est encore le paradoxe du barbier, avec :
    \begin{itemize}
    \item La relation $x \bowtie y \eqdef x \in \llbracket y \rrbracket$ joue le rôle de ``est rasé par''.
    \item $y$ joue le rôle du barbier.
    \end{itemize}
    $\mathcal{L}$ est \structure{dénombrable} et $\textsc{lang}$ est \structure{indénombrable}.
  \end{block}
  
\end{frame}

\endgroup
\endinput
