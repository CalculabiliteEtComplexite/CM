% SPDX-License-Identifier: CC-BY-SA-4.0
% Author: Matthieu Perrin
% Part: <Nom de la partie>
% Section: <Nom de la section>
% Sub-section: <Nom de la sous-section>  % (facultatif, laisser vide si non utilisé)
% Frame: <Titre de la slide>

\begingroup

\begin{frame}{Reformulation du programme de Hilbert}

  \begin{block}{Le programme de Hilbert}
    Trouver un système de preuve $\langle \Phi, \Pi, \valid \rangle$ :
    \begin{itemize}
    \item cohérent et complet ; 
    \item suffisamment expressif pour formaliser les mathématiques ;
    \item tel que ces problèmes soient \alert{décidables par une procédure mécanique} :
    \end{itemize}

    \hspace\fill
    \begin{minipage}[t]{.45\textwidth}
      \begin{problembox}
        \textsc{Vérification des preuves}\\
        \structure{Instance :}
        \begin{itemize}
        \item Une preuve $\pi \in \Pi$ 
        \item Une proposition $\varphi \in \Phi$ 
        \end{itemize}
        \structure{Question :}
        \begin{itemize}
        \item Est-ce que $\pi \valid \varphi$ ?
        \end{itemize}
      \end{problembox}
    \end{minipage}
    \hspace\fill
    \begin{minipage}[t]{.5\textwidth}
      \begin{problembox}
        \textsc{Entscheidungsproblem}\\
        \structure{Instance :}
        \begin{itemize}
        \item Une proposition $\varphi \in \Phi$ 
        \end{itemize}
        \structure{Question :}
        \begin{itemize}
        \item Est-ce que $\exists \pi \in \Pi,   \pi \valid \varphi$ ?
        \end{itemize}
      \end{problembox}
    \end{minipage}
    \hspace\fill
  \end{block}
  
  \begin{center}
    \alert{Que signifie ``décidable par une procédure mécanique'' ?} 
  \end{center}

\end{frame}

\endgroup
\endinput
