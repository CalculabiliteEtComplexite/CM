% SPDX-License-Identifier: CC-BY-SA-4.0
% Author: Matthieu Perrin
% Part: <Nom de la partie>
% Section: <Nom de la section>
% Sub-section: <Nom de la sous-section>  % (facultatif, laisser vide si non utilisé)
% Frame: <Titre de la slide>

\begingroup

\newcommand\PROBLEM{\textsc{problem}}

\begin{frame}{Problème de décision}

  \begin{block}{Définition -- Problème de décision formel}
    Un \structure{problème de décision formel} est un langage \alert{$\PROBLEM{} \in \textsc{lang}$}. 
  \end{block}

  \begin{block}{Définition -- Problème de décision (à domaine) contraint}
    Un \structure{problème de décision contraint} est un couple \alert{$\PROBLEM{} = \langle I, P \rangle$}, où :
    \begin{description}
    \item[$I \in \textsc{lang}$] est le langage des \structure{instances} de \PROBLEM{}
    \item[$P \subseteq I$] est le langage des \structure{instances positives} de \PROBLEM{}
    \end{description}
  \end{block}
  
  \begin{block}{Notations}
    \begin{tikzpicture}[2Darray, x=25mm, y=6mm]
      \arrayColumn[header=\structure{Notation}]{
        \arrayLine{\structure{$\Sigma(\PROBLEM)$}}
        \arrayLine{\structure{$\mathcal{I}(\PROBLEM)$}}
        \arrayLine{\structure{$\textsc{pos}(\PROBLEM)$}}
        \arrayLine{\structure{$\textsc{neg}(\PROBLEM)$}}
      }
      \arrayColumn[header=\structure{$L \in \textsc{lang}$}]{
        \arrayLine{$\Sigma(L)$}
        \arrayLine{$\Sigma(L)^\star$}
        \arrayLine{$L$}
        \arrayLine{$\Sigma(L)^\star \setminus L$}
      }
      \arrayColumn[header=\structure{$\langle I, P\rangle \in \textsc{lang}^2$}]{
        \arrayLine{$\Sigma(I)$}
        \arrayLine{$I$}
        \arrayLine{$P$}
        \arrayLine{$I \setminus P$}
      }
      \arrayColumn[width=30mm, header=\structure{Notion}]{
        \arrayLine{Alphabet}
        \arrayLine{Instances}
        \arrayLine{Instances positives}
        \arrayLine{Instances négatives}
      }
    \end{tikzpicture}
  \end{block}

\end{frame}

\endgroup
\endinput
