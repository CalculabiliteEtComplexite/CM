% SPDX-License-Identifier: CC-BY-SA-4.0
% Author: Matthieu Perrin
% Part: <Nom de la partie>
% Section: <Nom de la section>
% Sub-section: <Nom de la sous-section>  % (facultatif, laisser vide si non utilisé)
% Frame: <Titre de la slide>

\begingroup


\begin{frame}{Conséquence}

  \begin{block}{Tout langage de formalisme a des limites}
  Pour tout langage de formalisme $\langle \mathcal{L}, \llbracket \cdot \rrbracket \rangle$, il existe des problèmes de décision qui ne sont représentés par aucun mot de $\mathcal{L}$.
  \end{block}
  
  \begin{exampleblock}{Exemples}
    \begin{itemize}
    \item Si $L$ est le langage des \structure{expressions rationnelles} (flex)
      \begin{itemize}
      \item Il existe des langages qui ne sont \alert{pas rationnels}
      \end{itemize}
    \item Si $L$ est le langage des \structure{grammaires algébriques} (bison)
      \begin{itemize}
      \item Il existe des langages qui ne sont \alert{pas algébriques}
      \end{itemize}
    \item Si $L$ est la théorie de la \structure{géométrie euclidienne}
      \begin{itemize}
      \item Il existe des nombres qui ne sont \alert{pas constructibles}
      \end{itemize}
    \item Si $L$ est le langage des \structure{mathématiques}
      \begin{itemize}
      \item Il existe des problèmes qui ne sont \alert{pas définissables}
      \end{itemize}
    \item Si $L$ est un \structure{langage de programmation}
      \begin{itemize}
      \item Il existe des problèmes qui ne sont \alert{pas décidables} dans ce langage
      \end{itemize}
    \end{itemize}
  \end{exampleblock}

  \begin{center}
    \alert{Quels problèmes ne sont pas décidables par des algorithmes ?}
  \end{center}

  
\end{frame}

\endgroup
\endinput
