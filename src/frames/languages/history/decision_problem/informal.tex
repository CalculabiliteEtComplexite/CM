% SPDX-License-Identifier: CC-BY-SA-4.0
% Author: Matthieu Perrin
% Part: <Nom de la partie>
% Section: <Nom de la section>
% Sub-section: <Nom de la sous-section>  % (facultatif, laisser vide si non utilisé)
% Frame: <Titre de la slide>

\begingroup

\newcommand\PROBLEM{\textsc{problem}}

\begin{frame}{Description informelle d'un problème de décision}

  \on[top, text]{
    Un problème de décision \PROBLEM{} est décrit par :
    \begin{description}
    \item[Instance :] une entrée possible de \PROBLEM
    \item[Question :] une question fermée (oui/non) sur les instances
      \begin{itemize}
      \item Une instance est \alert{positive} si la réponse est ``\structure{oui}''
      \item Une instance est \alert{négative} si la réponse est ``\structure{non}''
      \end{itemize}
    \end{description}
  }

  \onExampleBlock[left=.5\textwidth, y=10mm, anchor=north]{Exemple : langage}{
    Soit $L = a^+ b^+$
    
    \Probleme{\example{Decision$_L$}}{
      Un mot $u \in \{a, b\}^\star$.
    }{
      Est-ce que $u \in L$ ?
    }
    
    \begin{itemize}
    \item Instances positives : \example{$aab, abb$}
    \item Instances négatives : \example{$bab, a$}
    \end{itemize}
  }

  \onExampleBlock[right=.5\textwidth, y=10mm, anchor=north]{Exemple : test de primalité}{
    ~
    
    \Probleme{\example{Primalité}}{
      Un entier $n \in \mathbb{N}$.
    }{
      $n$ est-il premier ?
    }
    
    \begin{itemize}
    \item Instances positives : \example{$2, 7, 13$}
    \item Instances négatives : \example{$1, 9, 14$}
    \end{itemize}
  }

\end{frame}

\endgroup
\endinput
