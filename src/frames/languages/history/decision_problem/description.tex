% SPDX-License-Identifier: CC-BY-SA-4.0
% Author: Matthieu Perrin
% Part: <Nom de la partie>
% Section: <Nom de la section>
% Sub-section: <Nom de la sous-section>  % (facultatif, laisser vide si non utilisé)
% Frame: <Titre de la slide>

\begingroup

\begin{frame}{Comment décrire les problèmes de décision ?}

  \begin{alertblock}{Remarque}
    \begin{itemize}
    \item Un problème doit être décrit dans un formalisme centré sur un langage
    \end{itemize}
  \end{alertblock}
  
  \begin{exampleblock}{Exemples : langage composé des seuls mots $aa$ et $ab$}
    \begin{description}[Grammaires algébriques :]
    \item[Mathématiques :]
      $\{aa, ab\}$
    \item[Expressions rationnelles :]
      $a (a\mid b)$
    \item[Grammaires algébriques :]
      $\{S \rightarrow aa \mid ab\}$
    \item[Langage C++ :]
      \lstinline0bool decide(string s) \{return s=="aa" || s=="ab";\}0
    \end{description}
  \end{exampleblock}

  \begin{block}{Définition -- Langage de descriptions}
    Un \structure{langage de descriptions} $\langle \mathcal{L}, \llbracket \cdot \rrbracket \rangle$ est un langage muni d'une sémantique :
    \begin{description}[$\llbracket \cdot \rrbracket : \mathcal{L} \rightarrow \textsc{lang}$]
    \item[$\mathcal{L} \in \textsc{lang}$] est la \alert{syntaxe} du langage de descriptions
    \item[$\llbracket \cdot \rrbracket : \mathcal{L} \rightarrow \textsc{lang}$] est la \alert{sémantique} du langage de descriptions
    \end{description}
    \begin{itemize}
    \item Un mot \structure{$u \in \mathcal{L}$} est la représentation du problème formel \structure{$\llbracket u \rrbracket \in \textsc{lang}$}.
    \end{itemize}
  \end{block}
  
\end{frame}

\endgroup
\endinput
