% SPDX-License-Identifier: CC-BY-SA-4.0
% Author: Matthieu Perrin
% Part: 
% Section: 
% Sub-section: 
% Frame: 

\begingroup

\begin{frame}{La crise des fondements}

  \on[top]{
    Sur quoi repose la légitimité des objets mathématiques ?
  }

  \onBlock[bottom=-3mm]{Une crise existentielle : réalistes\footnote{\structure{Réalisme :} Le monde a une existence \emph{ontologique}, c'est-à-dire indépendante de notre représentation} contre idéalistes\footnote{\structure{Idéalisme :} La réalité n'existe qu'à travers des représentations de l'esprit.}}{

    Des positions radicalement opposées se dégagent :

    \begin{description}
    \item[Platonicisme :] existence ontologique des objets mathématiques.
    \end{description}
    \myquote{Charles Hermite}{
      Il existe (...) un monde tout entier, qui est la totalité des vérités mathématiques, (...) comme il existe un monde de réalités physiques.
    }
    
    \vspace{-7mm}
    \begin{description}
    \item[Finitisme :] seuls les objets finis ont une existence légitime.
    \end{description}
    \myquote{Leopold Kronecker}{
      Dieu a créé les entiers, tout le reste est \oe uvre humaine.
    }

    \vspace{-7mm}
    \begin{description}
    \item[Formalisme :] les objets existent dans un cadre formel, s'il est cohérent.
    \end{description}
    \myquote{David Hilbert}{
      Les mathématiques sont un jeu joué selon certaines règles simples avec des symboles dénués de signification.
    }
  }

\end{frame}

\endgroup
\endinput
