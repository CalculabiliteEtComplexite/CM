% SPDX-License-Identifier: CC-BY-SA-4.0
% Author: Matthieu Perrin
% Part: 
% Section: 
% Sub-section: 
% Frame: 

\begingroup

\begin{frame}{Le paradoxe du barbier}

  \on[y=-5mm]{
    \begin{tikzpicture}
      \node [faded background picture=Village,    text width=\paperwidth/2] (A) at (-\paperwidth/4,0) {};
      \node [faded background picture=Barbershop, text width=\paperwidth/2] (B) at ( \paperwidth/4,0) {};

      \node[anchor=south, outer sep=0pt, inner sep=0pt] at (A.south) {\includegraphics[height=30mm]{shaving}};
      \node[anchor=south, outer sep=0pt, inner sep=0pt] at (B.south) {\includegraphics[height=30mm]{barber}};
    \end{tikzpicture}
  }

  \on[x=-.25\paperwidth, y=-5mm]{
    \chatBubble[color=example]{Je me rase moi-même}
  }

  \on[x=.25\paperwidth, y=-5mm]{
    \chatBubble[color=alert]{Je rase ceux qui ne se rasent pas}
  }

  \on[y=22mm]{
    \begin{tikzpicture}
      \node[
        every chat bubble,
        draw=structure,
        top color=structure!10,
        bottom color=structure!30,
      ]{
        \begin{minipage}{.95\textwidth}
        Dans un petit village suisse, vivait un barbier.

        Tous les villageois étaient bien rasés.
        \vspace{-2mm}
        \uncover<2->{%
          $$\forall x,~ x\ \structure{\text{est~rasé~par}}\  \mathit{le~barbier} \Leftrightarrow \lnot (x\ \structure{\text{est~rasé~par}}\ x)$$%
        }%
        \vspace{-6mm}%
        \uncover<3->{%
          \begin{center}%
            \emph{\Large\structure{Qui rase le barbier ?}}%
          \end{center}%
        }%
        \vspace{2mm}
        \end{minipage}%
      };
    \end{tikzpicture}
  }
  
\end{frame}

\endgroup
