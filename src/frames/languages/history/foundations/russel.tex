% SPDX-License-Identifier: CC-BY-SA-4.0
% Author: Matthieu Perrin
% Part: 
% Section: 
% Sub-section: 
% Frame: 

\begingroup

\begin{frame}{Le paradoxe de Russell}

  \onBlock[top=-2mm]{Conséquence : un tel barbier n'existe pas}{
    On généralise à n'importe quelle \structure{relation homogène} $\bowtie$ :
    $$\structure{\nexists y, \forall x, x \bowtie y \Leftrightarrow \lnot (x \bowtie x)}$$
    
    Dans le paradoxe du barbier, on a :
    \begin{itemize}
    \item $y$ est le barbier
    \item $\bowtie$ est la relation ``est rasé par''
    \end{itemize}
  }  

  \onAlertBlock<2->[bottom=3mm]{L'argument de Russell}{
    \begin{itemize}
    \item Application à l'appartenance ensembliste $\in$ :
      \begin{itemize}
      \item $\nexists y = \{ x \mid x \notin x \}, \quad \forall x, x \in y \Leftrightarrow x \notin x$
      \end{itemize}
    \item Définition des ensembles par compréhension :
      \begin{itemize}
      \item $\exists y = \{ x \mid x \notin x \}, \quad \forall x, x \in y \Leftrightarrow x \notin x$
      \end{itemize}
    \end{itemize}
    \begin{center}
      \alert{La théorie naïve des ensembles est \emph{incohérente}}
    \end{center}
  }  

  \onImage<2->[x=.35\textwidth, bottom=10mm]{%
    height=4cm,
    title={Bertrand Russell},
    licenselogo={\ccPublicDomain},
    license={Domaine public (Bassano Ltd, 1936. \href{https://commons.wikimedia.org/wiki/File:Bertrand_Russell_photo.jpg?uselang=fr}{Wikimedia})},
    img={Russell.jpg}
  }
  
\end{frame}

\endgroup
\endinput
