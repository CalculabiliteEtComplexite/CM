% SPDX-License-Identifier: CC-BY-SA-4.0
% Author: Matthieu Perrin
% Part: 
% Section: 
% Sub-section: 
% Frame: 

\begingroup

\begin{frame}{Cohérence et complétude d'un système de preuves}

  \begin{block}{Système de preuves}
    Un système de preuves est un triplet \structure{$S = \langle \Phi, \Pi, \valid \rangle$}, constitué de :
    \begin{description}[Complétude :]
    \item[$\Phi\in\textsc{lang}$ :] un langage de \structure{propositions} \hspace{\fill} \example{exemple : formules logiques}
      \begin{itemize}
      \item On suppose que $\Phi$ possède un opérateur de négation $\lnot$
      \end{itemize}
      \item[$\Pi\in\textsc{lang}$ :] un langage de \structure{preuves} \hspace{\fill} \example{exemple : arbres d'inférence}
      \item[$\valid \subseteq \Pi \times \Phi$ :] une relation entre les preuves et les propositions
      \begin{itemize}
      \item ``\structure{$\pi \valid \varphi$}'' signifie ``$\pi$ est une preuve valide de $\varphi$''
      \end{itemize}
    \end{description}
    Une proposition $\varphi \in \Phi$ est dite \structure{démontrable dans $S$} si \alert{$\exists \pi\in \Pi, \pi\valid\varphi$}.
  \end{block}

  \pause
  \begin{block}{Deux propriétés fondamentales}
    Un système de preuves $S=\langle \Phi,\Pi,\valid\rangle$ est dit :
    \begin{description}[Complétude :]
    \item[Complet] si pour toute proposition $\varphi \in \Phi$, \structure{$\varphi$ ou $\lnot \varphi$ est démontrable}
      \begin{itemize}
      \item sinon, \alert{certaines propositions restent indécidables}
      \end{itemize}
    \item[Cohérent] si pour toute proposition $\varphi \in \Phi$, \structure{$\varphi$ ou $\lnot \varphi$ est indémontrable}
      \begin{itemize}
      \item sinon, \alert{toute proposition est démontrable} : $(\varphi\land \lnot \varphi) \Rightarrow \psi$
      \item donc le système logique devient inutile
      \end{itemize}
    \end{description}
  \end{block}

\end{frame}

\endgroup
