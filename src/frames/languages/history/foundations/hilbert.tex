% SPDX-License-Identifier: CC-BY-SA-4.0
% Author: Matthieu Perrin
% Part: <Nom de la partie>
% Section: <Nom de la section>
% Sub-section: <Nom de la sous-section>  % (facultatif, laisser vide si non utilisé)
% Frame: <Titre de la slide>

\begingroup

\begin{frame}{Assurer les fondements des mathématiques}
  
  \onBlock[left=.7\textwidth, top=9mm]{Le programme de Hilbert (d'après Frege)}{
    \begin{itemize}
    \item Formaliser rigoureusement les mathématiques
    \item Garantir la \structure{cohérence} des théories formelles
    \item Idéalement, assurer leur \structure{complétude}
    \item Rendre les preuves \structure{mécaniquement vérifiables}
    \end{itemize}
  }

  \on[left=.6\textwidth, top=-1mm]{
    \myquote{David Hilbert\footnote{En réponse à Emil du Bois-Reymond : ``Nous ne savons pas, nous ne saurons pas''}}{
      Nous devons savoir, nous saurons.
    }
  }
  
  \onAlertBlock[bottom=7mm]{Der Entscheidungsproblem (Le Problème de la Décision)}{
    Existe-t-il une procédure mécanique permettant de déterminer, pour toute formule logique donnée, si celle-ci est logiquement valide ?
    \begin{itemize}
    \item Qu'est-ce qu'un \og \alert{problème de décision} \fg, en général ?
    \item Qu'est-ce qu'une \og \alert{procédure mécanique} \fg ?
    \end{itemize}
  }

  \onImage[x=40mm,top]{%
    width=2.5cm,
    title={David Hilbert},
    licenselogo={\ccPublicDomain},
    license={Domaine public (Göttingen, 1912, \href{https://commons.wikimedia.org/wiki/File:Hilbert.jpg}{Wikimedia})},
    img={Hilbert.jpg}
  }
  
\end{frame}

\endgroup
