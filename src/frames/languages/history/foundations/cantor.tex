% SPDX-License-Identifier: CC-BY-SA-4.0
% Author: Matthieu Perrin
% Part: 
% Section: 
% Sub-section: 
% Frame: 

\begingroup

\begin{frame}{Vers une théorie ensembliste de l'arithmétique}

  \onBlock[top=-4mm, left=.62\textwidth]{Théorie (naïve) des ensembles (Frege)}{      
    \begin{itemize}
    \item Tout objet mathématique est un ensemble 
    \item Ensemble = prédicat logique
      \begin{itemize}
      \item Par exemple, $\{1, 2\} = \{x \mid \alert{x=1 \lor x=2}\}$
      \end{itemize}
    \end{itemize}
    \myquote{Gottlob Frege}{Les lois de l'arithmétique doivent être déduites des lois de la logique}
  }

  \onBlock<2->[bottom=6mm, right=.7\textwidth]{L'infini comme objet mathématique (Cantor)}{
    \begin{itemize}
    \item Étude systématique des ensembles infinis
    \item Théorème de Cantor :
      \begin{itemize}
      \item $\forall E, | \mathscr{P}(E) | > | E |$
      \item En particulier, $| \mathbb{R} | > | \mathbb{N} |$
      \item Argument de diagonalisation
      \end{itemize}
    \item Infini non unique : \structure{hiérarchie des infinis}
    \end{itemize}
  }
  
  \onImage[x=.37\textwidth,top]{%
    height=28mm,
    title={Gottlob Frege},
    licenselogo={\ccPublicDomain},
    license*={Domaine public (vers 1879. \href{https://commons.wikimedia.org/wiki/File:Young_frege.jpg}{Wikimedia})},
    img={Frege.jpg}
  }

  \onImage<2->[x=-.37\textwidth,bottom=7mm]{%
    height=28mm,
    title={Georg Cantor},
    licenselogo={\ccPublicDomain},
    license*={Domaine public (vers 1900. \href{https://commons.wikimedia.org/wiki/File:Georg_Cantor2.jpg}{Wikimedia})},
    img={Cantor.jpg}
  }
  
\end{frame}

\endgroup
