% SPDX-License-Identifier: CC-BY-SA-4.0
% Author: Matthieu Perrin
% Part: <Nom de la partie>
% Section: <Nom de la section>
% Sub-section: <Nom de la sous-section>  % (facultatif, laisser vide si non utilisé)
% Frame: <Titre de la slide>

\begingroup

\begin{frame}{Généralités}

  \begin{block}{Prérequis}
    \begin{description}[Mathématiques : ]
    \item[Mathématiques : ] logique, théorie des ensembles
    \item[Informatique : ] algorithmique, théorie des langages formels
    \end{description}
  \end{block}

  \begin{block}{Objectifs du cours}
    \begin{itemize}
    \item Comprendre les \structure{fondements de la théorie de la calculabilité}
    \item Modéliser des algorithmes simples à l’aide de \structure{machines de Turing}
    \item Comprendre et manipuler les classes de complexité \structure{P} et \structure{NP}
    \item Appliquer des \structure{techniques de réduction} de calculabilité et de complexité
    \end{itemize}
  \end{block}

  \begin{block}{Bibliographie}
    \begin{itemize}
    \item P. Wolper. \structure{\emph{Introduction à la calculabilité}} -- 3e Ed., Dunod 2006
    \item O. Carton. \structure{\emph{Langages Formels -- Calculabilité et complexité}}, Vuibert 2008
    \end{itemize}
  \end{block}
  
\end{frame}

\endgroup
