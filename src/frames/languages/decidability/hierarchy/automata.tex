% SPDX-License-Identifier: CC-BY-SA-4.0
% Author: Matthieu Perrin
% Part: <Nom de la partie>
% Section: <Nom de la section>
% Sub-section: <Nom de la sous-section>  % (facultatif, laisser vide si non utilisé)
% Frame: <Titre de la slide>

\begingroup

\begin{frame}{Hiérarchie de Chomsky et automates}

  \vspace{-1mm}
  \centering
  \begin{tikzpicture}[2Darray, x=23mm, y=15mm]
    \arrayColumn[width=8mm, header=Type]{
      \arrayLine{0}
      \arrayLine{1}
      \arrayLine{2}
      \arrayLine{3}
    }
    
    \arrayColumn[header=Grammaire]{
      \arrayLine{Non-restreinte\\ $\alpha A \beta \rightarrow \gamma$}
      \arrayLine{Contextuelle\\   $\alpha A \beta \rightarrow \alpha \gamma \beta$}
      \arrayLine{Algébrique\\     $A \rightarrow \beta$}
      \arrayLine{Rationnelle\\    $A \rightarrow a B \mid b \mid \varepsilon$}
    }
    
    \arrayColumn[header=Langage]{
      \arrayLine{Récursivement\\ énumérable}
      \arrayLine{Contextuel}
      \arrayLine{Algébrique}
      \arrayLine{Rationnel}
    }
    
    \arrayColumn[width=25mm, header=Automate]{
      \arrayLine{Machine\\de Turing}
      \arrayLine{Automate linéairement borné}
      \arrayLine{Automate à pile\\non déterministe}
      \arrayLine{Automate fini}
    }
    
    \arrayColumn[width=30mm, header=Déterminisable]{
      \arrayLine{Oui. Complexité ? \\ \footnotesize $\textsc{p} \stackrel{?}{=} \textsc{np}$}
      \arrayLine{Inconnu\\ \scriptsize $\textsc{nspace}(n) \stackrel{?}{=} \textsc{dspace}(n)$}
      \arrayLine{Impossible}
      \arrayLine{Oui. Exponentiel\\ en nombre d'états}
    }
  \end{tikzpicture}
  
\end{frame}

\endgroup
