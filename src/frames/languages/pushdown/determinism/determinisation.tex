% SPDX-License-Identifier: CC-BY-SA-4.0
% Author: Matthieu Perrin
% Part: 
% Section: 
% Sub-section: 
% Frame: 

\begingroup

\begin{frame}{Déterminisation d'automate à pile}

  \begin{block}{Définition -- Langage algébrique déterministe}
    \begin{itemize}
    \item Un langage algébrique $L$ est dit \structure{déterministe} s'il est \alert{reconnu par un automate à pile déterministe}.\\
    \item La classe des langages algébriques déterministes est notée \alert{$\textsc{det}$}.
    \end{itemize}
  \end{block}
  
  \begin{block}{Théorème}
    
    \vspace{-4mm}
    $$\alert{\textsc{det} \subsetneq \textsc{alg}}$$

    \vspace{-1mm}
    \structure{Démonstration :}
    \begin{enumerate}
    \item Un automate à pile déterministe est un automate à pile : $\alert{\textsc{det} \subseteq \textsc{alg}}$.
    \item $\textsc{det}$ est stable par complémentation
      \begin{itemize}
      \item On peut le compléter, puis inverser les états finaux et non-finaux
      \end{itemize}
    \item $\textsc{alg}$ n'est pas stable par complémentation
      \begin{itemize}
      \item Sinon, $\textsc{alg}$ serait stable par intersection, car \structure{$L_1 \cap L_2 = \overline{\overline{L_1} \cup \overline{L_2}}$}
      \item Or $\textsc{alg}$ n'est pas stable par intersection, car $$\structure{\{a^n b^n c^n \mid n\in \mathbb{N}\} = \{a^n b^m c^n \mid m, n\in \mathbb{N}\} \cap \{a^n b^n c^m \mid m, n\in \mathbb{N}\}}$$
      \end{itemize}
    \item Donc $\alert{\textsc{det} \neq \textsc{alg}}$.
    \end{enumerate}
  \end{block}

\end{frame}

\endgroup
