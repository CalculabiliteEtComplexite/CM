% SPDX-License-Identifier: CC-BY-SA-4.0
% Author: Matthieu Perrin
% Part: 
% Section: 
% Sub-section: 
% Frame: 

\begingroup

\begin{frame}{Automates déterministes et complets}

  Soit $A = \langle \Sigma, Q, q_0, F, \rightarrow \rangle$ un AFN.
  
  \begin{block}{Définition -- Automate fini déterministe (AFD)}
    On dit que $A$ est \structure{déterministe} si toutes les conditions sont vérifiées
    \begin{description}[relation fonctionnelle :]
    \item[$\varepsilon$-liberté :] $A$ ne possède pas d'$\varepsilon$-transition

      \vspace{-2mm}
      $$\alert{\rightarrow\, \subseteq Q \times \Sigma \times Q}$$

    \item[relation fonctionnelle :] pour chaque état $q$ et chaque symbole $a$, il existe \alert{au plus} une transition sortant de $q$ étiquetée $a$

      \vspace{-2mm}
      $$\alert{\forall q, q_1, q_2\in Q,  \forall a\in \Sigma, \left(\structure{q\xrightarrow{a} q_1} \land \structure{q\xrightarrow{a} q_2}\right) \structure{\Rightarrow q_1 = q_2}}$$
    \end{description}
  \end{block}

  \vspace{-2mm}
  \begin{block}{Définition -- Automate fini complet}
    On dit que $A$ est \structure{complet} si la condition suivante est vérifiée
    \begin{description}[relation fonctionnelle :]
    \item[relation totale :] pour chaque état $q$ et chaque symbole $a$, il existe \alert{au moins} une transition sortant de $q$ étiquetée $a$

      \vspace{-2mm}
      $$\alert{\forall q\in Q,  \forall a\in \Sigma, \structure{\exists q' \in Q,~q \xrightarrow{a} q'}}$$
    \end{description}
  \end{block}
  
\end{frame}

\endgroup
\endinput
