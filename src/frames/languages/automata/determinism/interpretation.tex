% SPDX-License-Identifier: CC-BY-SA-4.0
% Author: Matthieu Perrin
% Part: 
% Section: 
% Sub-section: 
% Frame: 

\begingroup

\begin{frame}{Interprétation du non-déterminisme}

  \begin{block}{Plusieurs formes de non-déterminisme}
    \begin{description}
    \item[Aléatoire :] assigner \structure{une probabilité} à chaque transition possible
      \begin{itemize}
      \item Arrivée dans un état final \structure{avec grande probabilité}
      \end{itemize}
    \item[Adversaire :] la transition à emprunter est décidée par \structure{l'environnement}
      \begin{itemize}
      \item Arrivée dans un état final \structure{quel que soit le chemin}
      \end{itemize}
    \item[\alert{Ubiquitaire :}] la transition à emprunter est décidée par \alert{un oracle}
      \begin{itemize}
      \item Arrivée dans un état final \alert{pour au moins un chemin}
      \end{itemize}
    \end{description}
  \end{block}
  
  \begin{block}{Le non-déterminisme comme de l'ubiquité}
    \begin{itemize}
    \item L'automate se trouve dans un sous-ensemble des états
    \item Le mot est reconnu si l'un des états du sous-ensemble est final
    \item Ces sous-ensembles forment un nouvel automate, qui est déterministe
    \end{itemize}
  \end{block}

\end{frame}

\endgroup
\endinput
