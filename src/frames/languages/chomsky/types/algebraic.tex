% SPDX-License-Identifier: CC-BY-SA-4.0
% Author: Matthieu Perrin
% Part: 
% Section: 
% Sub-section: 
% Frame: 

\begingroup

\begin{frame}{Grammaires et langages algébriques}

  \begin{block}{Définition -- Grammaire algébrique}
    Une grammaire $G = \langle \Sigma, \Gamma, S, \rightarrow \rangle$ est dite \structure{algébrique} si les membres gauches de toutes ses règles sont un non-terminal :

    $$\forall \langle \alpha, \beta\rangle \in \rightarrow, \quad \alert{\alpha \in \Gamma}$$
  \end{block}

  \vspace{-2mm}
  \begin{block}{Définition -- Langage algébrique}
    Soient $\Sigma$ un alphabet et $L$ un langage sur $\Sigma$.
    \begin{itemize}
    \item $L$ est dit \structure{algébrique} s'il est engendré par une grammaire algébrique
    \item L'\structure{ensemble des langages algébrique} sur $\Sigma$ est noté \alert{$\textsc{alg}_\Sigma$}
    \item La \structure{classe des langages algébrique} est noté \alert{$\textsc{alg}$} (indépendant de $\Sigma$)
    \end{itemize}
  \end{block}

  \begin{exampleblock}{Exemple -- Langage $\{a^n b^n \mid n\in \mathbb{N}\}$}
    \begin{itemize}
    \item $G = \langle \{a, b\}, \{S\}, S, \left\{S \rightarrow a S b \mid \varepsilon \right\} \rangle $
    \item Génération de $aabb$ : $ S \vdash aSb \vdash aaSbb \vdash aabb $
    \end{itemize}
  \end{exampleblock}
  
\end{frame}

\endgroup
\endinput
