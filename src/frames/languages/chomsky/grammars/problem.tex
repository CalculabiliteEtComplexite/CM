% SPDX-License-Identifier: CC-BY-SA-4.0
% Author: Matthieu Perrin
% Part: 
% Section: 
% Sub-section: 
% Frame: 

\begingroup

\begin{frame}{Comment décrire un problème de décision ?}

  \onAlertBlock[top=-5mm]{Problème -- Définition d'un langage}{
    On cherche un langage de description \alert{$\langle\mathcal{L}, \llbracket \cdot \rrbracket \rangle$}
    le plus expressif possible pour décrire des langages sur un alphabet $\Sigma$
  }

  \onBlock[y=12mm]{Approche générative}{
    \vspace{-2mm}
    \begin{itemize}
    \item Donner une \structure{description syntaxique} de la forme des mots de $L$
    \item Une \structure{grammaire} est un ensemble de règles qui \structure{génèrent} les mots de $L$
    \end{itemize}
  }
  
  \onBlock[y=-7mm]{Approche algorithmique}{
    \vspace{-2mm}
    \begin{itemize}
    \item Décrire une \structure{procédure de décision} pour l'appartenance à $L$
    \item Un \structure{automate} spécifie un calcul effectif sur les mots
    \end{itemize}
  }

  \on[bottom, width=.53\textwidth, x=-26mm]{
    \Probleme{\structure{Decision$_L$} \normalfont{(pour $L \in \mathcal{L}$)}}{
      \vspace{2.3mm}Un mot \structure{$u \in \Sigma^\star$}
    }{
      \vspace{2mm}Est-ce que \structure{$u \in \llbracket L \rrbracket$} ?
    }
  }
  \on[bottom, width=.55\textwidth, x=27mm]{
    \Probleme{\structure{Universal$_{\langle\mathcal{L}, \llbracket \cdot \rrbracket \rangle}$}}{
      \begin{itemize}
      \item Un langage \structure{$L \in \mathcal{L}$}
      \item Un mot \structure{$u \in \Sigma^\star$}
      \end{itemize}
    }{
      Est-ce que \structure{$u \in \llbracket L \rrbracket$} ?
    }
  }

\end{frame}

\endgroup
\endinput
