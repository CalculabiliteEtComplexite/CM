% SPDX-License-Identifier: CC-BY-SA-4.0
% Author: Matthieu Perrin
% Part: 
% Section: 
% Sub-section: 
% Frame: 

\begingroup

\begin{frame}{Compresser les rubans}

  \vspace{-3mm}
  \begin{block}{Théorème -- Puissance d'une machine à $k$ rubans}
    Tout langage reconnu par une machine à $k$ rubans est reconnaissable.
  \end{block}
  \pause
  \begin{block}{Démonstration}
    Soit $M = \langle \Sigma, \Gamma, \langle\blank_1, ..., \blank_k\rangle, Q, q_0, F, \rightarrow \rangle$ une machine à $k$ rubans.

    On construit $M' = \langle \Sigma, \Gamma', \blank', Q', q_0, F, \rightarrow' \rangle$ telle que $\mathcal{L}(M') = \mathcal{L}(M)$ :

    \begin{itemize}
    \item \structure{Encodage des rubans}
    \begin{itemize}
    \item $\Gamma' = \Gamma^k \cup \{\text{\faMapMarker}\}$, $\blank' = \langle\blank_1, ..., \blank_k\rangle$
    \item \faMapMarker{} est à gauche de la position des têtes de lecture
    \item Les autres cases contiennent un symbole de chaque ruban
    \end{itemize}
    \item \example{Exemple avec $\Gamma = \{a, b, \blank\}$ et $k = 2$ :}
    \end{itemize}

  \scalebox{.7}{\begin{tikzpicture}
    \draw[example] (0,.25) node[left]{$1 :$};
    \draw[fill=example!10] ( 0.0,0) -- +(.5,0) -- +(.5,.5) -- +(0,.5) +(.25,.25) node{$\ldots$};
    \draw[fill=example!10] ( 0.5,0) rectangle +(.5,.5) +(.25,.25) node{$\blank$};
    \draw[fill=example!10] ( 1.0,0) rectangle +(.5,.5) +(.25,.25) node{$\blank$}; 
    \draw[fill=example!10] ( 1.5,0) rectangle +(.5,.5) +(.25,.25) node{$a$}; 
    \draw[fill=example!10] ( 2.0,0) rectangle +(.5,.5) +(.25,.25) node{$b$}; 
    \draw[fill=example!10] ( 2.5,0) rectangle +(.5,.5) +(.25,.25) node{$b$}; 
    \draw[fill=example!10] ( 3.0,0) rectangle +(.5,.5) +(.25,.25) node{$a$};
    \draw[fill=example!10] ( 3.5,0) rectangle +(.5,.5) +(.25,.25) node{$b$};
    \draw[fill=example!10] ( 4.0,0) rectangle +(.5,.5) +(.25,.25) node{$a$};
    \draw[fill=example!10] ( 4.5,0) rectangle +(.5,.5) +(.25,.25) node{$b$};
    \draw[fill=example!10] ( 5.0,0) rectangle +(.5,.5) +(.25,.25) node{$\blank$};
    \draw[fill=example!10] ( 5.5,0) rectangle +(.5,.5) +(.25,.25) node{$\blank$};
    \draw[fill=example!10] ( 6.0,0) +(.5,0) -- +(0,0) -- +(0,.5) -- +(.5,.5) +(.25,.25) node{$\ldots$};
    \fill[example] (3.5,0) +(.1,-.3) -- +(.25,0) -- +(.4,-.3);
  \end{tikzpicture}}
  \scalebox{.7}{\begin{tikzpicture}
    \draw[example] (0,.25) node[left]{$2 :$};
    \draw[fill=example!10] ( 0.0,0) -- +(.5,0) -- +(.5,.5) -- +(0,.5) +(.25,.25) node{$\ldots$};
    \draw[fill=example!10] ( 0.5,0) rectangle +(.5,.5) +(.25,.25) node{$\blank$};
    \draw[fill=example!10] ( 1.0,0) rectangle +(.5,.5) +(.25,.25) node{$\blank$}; 
    \draw[fill=example!10] ( 1.5,0) rectangle +(.5,.5) +(.25,.25) node{$b$}; 
    \draw[fill=example!10] ( 2.0,0) rectangle +(.5,.5) +(.25,.25) node{$a$}; 
    \draw[fill=example!10] ( 2.5,0) rectangle +(.5,.5) +(.25,.25) node{$a$}; 
    \draw[fill=example!10] ( 3.0,0) rectangle +(.5,.5) +(.25,.25) node{$a$};
    \draw[fill=example!10] ( 3.5,0) rectangle +(.5,.5) +(.25,.25) node{$a$};
    \draw[fill=example!10] ( 4.0,0) rectangle +(.5,.5) +(.25,.25) node{$b$};
    \draw[fill=example!10] ( 4.5,0) rectangle +(.5,.5) +(.25,.25) node{$b$};
    \draw[fill=example!10] ( 5.0,0) rectangle +(.5,.5) +(.25,.25) node{$\blank$};
    \draw[fill=example!10] ( 5.5,0) rectangle +(.5,.5) +(.25,.25) node{$\blank$};
    \draw[fill=example!10] ( 6.0,0) +(.5,0) -- +(0,0) -- +(0,.5) -- +(.5,.5) +(.25,.25) node{$\ldots$};
    \fill[example] (3.0,0) +(.1,-.3) -- +(.25,0) -- +(.4,-.3);
  \end{tikzpicture}}

  
  \scalebox{.8}{\begin{tikzpicture}
    \draw[fill=structure!10] ( 0,0) -- +(1,0) -- +(1,1) -- +(0,1) +(.5,.5) node{$\ldots$};
    \draw[fill=structure!10] ( 1,0) rectangle +(1,1) +(.5,.7) node{$\blank$} +(.5,.3) node{$\blank$};
    \draw[fill=structure!10] ( 2,0) rectangle +(1,1) +(.5,.7) node{$a$} +(.5,.3) node{$\blank$};
    \draw[fill=structure!10] ( 3,0) rectangle +(1,1) +(.5,.7) node{$b$} +(.5,.3) node{$b$};
    \draw[fill=structure!10] ( 4,0) rectangle +(1,1) +(.5,.7) node{$b$} +(.5,.3) node{$a$};
    \draw[fill=structure!10] ( 5,0) rectangle +(1,1) +(.5,.7) node{$a$} +(.5,.3) node{$a$};
    \draw[fill=structure!10] ( 6,0) rectangle +(1,1) +(.5,.5) node{\faMapMarker};
    \draw[fill=structure!10] ( 7,0) rectangle +(1,1) +(.5,.7) node{$b$} +(.5,.3) node{$a$};
    \draw[fill=structure!10] ( 8,0) rectangle +(1,1) +(.5,.7) node{$a$} +(.5,.3) node{$a$};
    \draw[fill=structure!10] ( 9,0) rectangle +(1,1) +(.5,.7) node{$b$} +(.5,.3) node{$b$};
    \draw[fill=structure!10] (10,0) rectangle +(1,1) +(.5,.7) node{$\blank$} +(.5,.3) node{$b$};
    \draw[fill=structure!10] (11,0) rectangle +(1,1) +(.5,.7) node{$\blank$} +(.5,.3) node{$\blank$};
    \draw[fill=structure!10] (12,0) +(1,0) -- +(0,0) -- +(0,1) -- +(1,1) +(.5,.5) node{$\ldots$};
    
    \fill [structure] (7,0) +(.25,-.5) -- +(.5,0) -- +(.75,-.5);
  \end{tikzpicture}}
  \end{block}

  \vspace{-5mm}
  \begin{itemize}
  \item \structure{Simulation des transitions}
    \begin{itemize}
    \item Problème : il faut garder les têtes de lecture au même endroit
    \item Au lieu de déplacer le tête de lecture, on déplace le ruban à l'envers
    \end{itemize}
  \end{itemize}
\end{frame}

\endgroup
