% SPDX-License-Identifier: CC-BY-SA-4.0
% Author: Matthieu Perrin
% Part: <Nom de la partie>
% Section: <Nom de la section>
% Sub-section: <Nom de la sous-section>  % (facultatif, laisser vide si non utilisé)
% Frame: <Titre de la slide>

\begingroup

\begin{frame}{De semi-décidable à décidable}
  \begin{block}{Lemme}
  \vspace{-1mm}  
    Soit $L$ un langage tel que $L$ et $\overline{L}$ sont semi-décidables.
    Alors $L$ est décidable.
  \end{block}

  \vspace{-1mm}  
  \begin{block}{Démonstration}
  \vspace{-1mm}  
    \begin{itemize}
    \item Il existe $\structure{M_1 = \langle \Sigma, \Gamma_1, \blank_1, Q_1, q_{1}^0, F_1, \rightarrow_1 \rangle}$ une MTD qui semi-décide $L$.
    \item Il existe $\example{M_2 = \langle \Sigma, \Gamma_2, \blank_2, Q_2, q_{2}^0, F_2, \rightarrow_2 \rangle}$ une MTD qui semi-décide $\overline{L}$.
    \item Alors $L$ est décidé par la machine de Turing déterministe à 2 rubans qui
      \begin{enumerate}
      \item recopie l'entrée sur les deux rubans;
      \item exécute $\alert{M_{1\times 2} = \langle \Sigma, \Gamma_{1\times 2}, \blank_{1\times 2}, Q_{1\times 2}, q_{1\times 2}^0, F_{1\times 2}, \rightarrow_{1\times 2} \rangle}$, avec :
      \end{enumerate}
    \end{itemize}

    \vspace{-4mm}
    $$\begin{array}{rcl}
      \alert{Q_{1\times 2}} &=& \structure{Q_1} \times \example{Q_2}  \hspace{7mm}
      \alert{F_{1\times 2}} \hspace{3mm}=\hspace{3mm} \left( \structure{F_1} \times \example{Q_2} \right) \cup \left( \structure{Q_1} \times \example{(Q_2 \setminus F_2)}\right)\vspace{2mm}\\
      \alert{q_{1\times 2}^0} &=& \langle \structure{q_1^0}, \example{q_2^0}\rangle\hspace{7.5mm}
      \alert{\Gamma_{1\times 2}} \hspace{3mm}=\hspace{3mm} \structure{\Gamma_1} \cup \example{\Gamma_2} \hspace{7mm}
      \alert{\blank_{1\times 2}} \hspace{3mm}=\hspace{3mm} \langle \structure{\blank_1}, \example{\blank_2}\rangle \vspace{2mm}\\
      \alert{\rightarrow_{1\times 2}} &=& \left\{  \langle \structure{q_1}, \example{q_2} \rangle \xrightarrow{\small \begin{array}{l} \structure{1 : a_1/b_1,d_1} \\ \example{2 : a_2/b_2,d_2} \end{array}} \langle \structure{q'_1}, \example{q'_2} \rangle
      \middle|  \begin{array}{rl} &\structure{q_1 \xrightarrow{a_1/b_1,d_1} q'_1 \in \rightarrow_1} \\ \land& \example{q_2 \xrightarrow{a_2/b_2,d_2} q'_2 \in \rightarrow_2} \end{array} \right\}\vspace{1mm}\\
    \end{array}$$
  \end{block}

\end{frame}

\endgroup
