% SPDX-License-Identifier: CC-BY-SA-4.0
% Author: Matthieu Perrin
% Part: <Nom de la partie>
% Section: <Nom de la section>
% Sub-section: <Nom de la sous-section>  % (facultatif, laisser vide si non utilisé)
% Frame: <Titre de la slide>

\begingroup

\begin{frame}{Caractérisation des langages décidables}

  \onBlock[top=-3mm]{Théorème --- Clôture par complémentaire}{
    Soit $L$ un langage et $\overline{L}$ son complémentaire.\\
    Les trois conditions suivantes sont équivalentes :
    \begin{enumerate}
    \item $L$ est décidable
    \item $\overline{L}$ est décidable
    \item $L$ et $\overline{L}$ sont tous les deux semi-décidables
    \end{enumerate}
  }
  
  \onBlock<2->[anchor=north, y=2mm]{Démonstration}{
    \begin{description}[$1 \Rightarrow 2$ :]
    \item<2->[$1 \Rightarrow 2$ :]
      Si $M = \langle \Sigma, \Gamma, \blank, Q, q_0, \alert{F}, \rightarrow \rangle$ décide $L$,\\
      alors $\overline{M} = \langle \Sigma, \Gamma, \blank, Q, q_0, \alert{Q \setminus F}, \rightarrow \rangle$ décide $\overline{L}$.\\
    \item<3->[$2 \Rightarrow 3$ :] Décidable $\Rightarrow$ semi-décidable par définition.
    \item<4->[$3 \Rightarrow 1$ :] On exécute un \structure{produit} des deux machines.\\
      L'une finit par s'arrêter en donnant la réponse.
    \end{description}
    \uncover<4->{\footnotesize
      $$
      \alert{\left\langle \structure{q_1}, \example{q_2} \right\rangle
        \xrightarrow{\smGroup{\smTMtrans[\structure{\textsc{l}}]{\structure{a_1}}{\structure{b_1}}{\structure{d_1}}
            \smTMtrans[\example{\overline{\textsc{l}}}]{\example{a_2}}{\example{b_2}}{\example{d_2}}}}_{1\times 2}
        \left\langle \structure{q'_1}, \example{q'_2} \right\rangle}
      \quad\Leftrightarrow\quad
      \structure{q_1 \xrightarrow{a_1/b_1,d_1}_1 q'_1}
      \quad\land\quad
      \example{q_2 \xrightarrow{a_2/b_2,d_2}_2 q'_2}
      $$
    }
  }

  \on<4->[top, x=40mm, y=15mm]{
    \begin{tikzpicture}[turingMachine, x=5mm, y=5mm]\scriptsize
      \begin{scope}[draw=structure, text=structure]
        \state[structure on=<-5>, initial] (0) at (0,5) {$0$}; 
        \state[structure ob=<6>,accepting] (1) at (3,5) {$1$}; 
        \path     (0) edge[loop above]   node         {\smAlign{\smTMtransR[\textsc{l}]{b}{\blank}\smTMtransL[\textsc{l}]{\blank}{\blank}}}  (0);
        \path     (0) edge               node         {\smTMtransL[\textsc{l}]{a}{a}}      (1);
      \end{scope}
      
      \begin{scope}[draw=example, text=example]
        \state[example, initial right] (a) at (5,3) {$a$}; 
        \state[accepting]              (b) at (5,0) {$b$}; 
        \path                  (a) edge[loop above]   node         {\smAlign{\smTMtransR[\overline{\textsc{l}}]{b}{b}\smTMtransS[\overline{\textsc{l}}]{a}{a}}}           (a);
        \path                  (a) edge               node         {\smTMtransL[\overline{\textsc{l}}]{\blank}{\blank}}      (b);
      \end{scope}
      
      \begin{scope}[draw=alert, text=alert]
        \state[alert on=<-5>,  initial] (a0) at (0,3) {$a0$}; 
        \state[                       ] (b0) at (0,0) {$b0$}; 
        \state[alert ob=<6>, accepting] (a1) at (3,3) {$a1$}; 
        \state[                       ] (b1) at (3,0) {$b1$}; 
        \path (a0) edge[loop above] (a0);
        \path (a0) edge             (b0);
        \path (a0) edge             (a1);
      \end{scope}
    \end{tikzpicture}
  }

  \on<4->[y=30mm, x=40mm]{
    \begin{tikzpicture}[tape, draw=structure, text=structure, x=5mm, y=5mm]
      \cell{\only<4>{b}} \smhead<4,6> 
      \cell{a} \smheadb<5> 
      \cell{b} 
    \end{tikzpicture}
  }

  \on<4->[y=20mm, x=40mm]{
    \begin{tikzpicture}[tape, draw=example, text=example, x=5mm, y=5mm]
      \cell{b} \smhead<4>
      \cell{a} \smheadb<5->
      \cell{b}
    \end{tikzpicture}
  }

\end{frame}

\endgroup
