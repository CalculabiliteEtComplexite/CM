% SPDX-License-Identifier: CC-BY-SA-4.0
% Author: Matthieu Perrin
% Part: 
% Section: 
% Sub-section: 
% Frame: 

\begingroup

\begin{frame}{Un automate à ruban}

  \on[text,top]{
    \begin{itemize}
    \item Une \structure{machine de Turing} est un \alert{automate} à \alert{ruban}.
      \begin{itemize}
      \item en : ``\structure{\textit{Infinite tape}}'' = fr : \og \structure{\textit{ruban infini}} \fg
      \item en : ``\structure{\textit{Magnetic tape}}'' = fr : \og \structure{\textit{bande magnétique}} \fg
      \end{itemize}
    \end{itemize}
  }

  \onImage[y=4mm]{%
    height=2.2cm,
    title={Cassette VHS},
    license={{}\ccbysa{} \href{https://creativecommons.org/licenses/by-sa/4.0/}{CC BY-SA 4.0} -- Toby Hudson 2012 -- \href{https://commons.wikimedia.org/wiki/File:VHS_cassette_tape_12.JPG}{Wikimedia}},
    img={vhs.jpg}
  }
  
  \on[text,bottom=3mm]{
    \begin{itemize}
    \item Exécution du \structure{contrôleur} :
      \begin{itemize}
      \item Un \structure{automate fini} décide des opérations sur la mémoire
      \item L'exécution dure aussi longtemps qu'il y a des actions possibles
      \end{itemize}
    \item Opérations sur le \structure{ruban} :
      \begin{itemize}
      \item \alert{Lire/écrire} à la position de la tête de lecture/écriture
      \item \alert{Déplacer} la tête de lecture/écriture vers la gauche ou vers la droite
      \item Pour simplifier : une lecture, une écriture, et un déplacement par étape
      \end{itemize}
    \end{itemize}
  }
  
\end{frame}

\endgroup
