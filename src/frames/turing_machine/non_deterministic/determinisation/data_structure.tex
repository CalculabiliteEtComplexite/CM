% SPDX-License-Identifier: CC-BY-SA-4.0
% Author: Matthieu Perrin
% Part: <Nom de la partie>
% Section: <Nom de la section>
% Sub-section: <Nom de la sous-section>  % (facultatif, laisser vide si non utilisé)
% Frame: <Titre de la slide>

\begingroup

\SetKwFunction{Deterministe}{semi\_decide}
\SetKwFunction{Enfile}{enfile}
\SetKwFunction{Defile}{defile}

\begin{frame}{Structures de données}

  \onBlock[top]{$M_D$ utilise deux rubans $c$ et $f$}{
    \begin{itemize}
    \item $c$ : configuration $\langle \langle G, D \rangle, q \rangle$ de $M$, représentée par la chaîne $GqD$
      \begin{itemize}
      \item Contient le mot $u$ initialement
      \end{itemize}
    \item $f$ : file de configurations, précédées du symbole $\mid$
      \begin{itemize}
      \item \Defile à gauche, \Enfile à droite 
      \item Utilisation d'un symbole blanc différent $\sharp$
      \end{itemize}
    \end{itemize}
  }

  \obBlock<1>[anchor=north,y=5mm]{Pré- et post- conditions}{
    Entre chaque sous-machine, les invariants suivants sont vérifiés : 
    \begin{itemize}
    \item La tête de lecture de $c$ centrée sur l'état : $\alert{\mathit{lire}(c) \in Q}$
    \item La tête de lecture de $f$ est située à droite du ruban : 
      $\alert{f = \langle G_f , \varepsilon \rangle}$
    \end{itemize}
  }

  \obBlock<2>[anchor=north,y=5mm]{Initialisation}{
    \begin{itemize}
    \item L'entrée $u$ de $M$ est située sur le ruban $c$
    \item On commence par écrire $q_0$ à gauche de l'entrée, pour encoder $\langle \langle \varepsilon, u \rangle, q_0 \rangle$
    \end{itemize}
  }
 
  \on<1>[x=-25mm, y=-25mm]{
    \begin{tikzpicture}[tape, x=5mm, y=5mm]
      \node{$c$ :};
      \cell{$\blank$}
      \cell{$g_1$}
      \cell{$g_2$}
      \cell{$q$}   \smhead
      \cell{$d_1$}
      \cell{$d_2$}
      \cell{$d_3$}
      \cell{$\blank$}
    \end{tikzpicture}
  }
 
  \on<1>[x=-25mm, y=-35mm]{
    \begin{tikzpicture}[tape, x=5mm, y=5mm]
      \node{$f$ :}; 
      \cell{$\sharp$}
      \cell{$|$}
      \cell{$c_1^1$}
      \cell{$c_1^2$}
      \cell{$|$}
      \cell{$c_2^1$}
      \cell{$c_2^2$}
      \cell{$\sharp$}  \smhead
    \end{tikzpicture}
  }
 
 
  \on<2->[x=-25mm, y=-25mm]{
    \begin{tikzpicture}[tape, x=5mm, y=5mm]
      \draw node{$f$ :}; 
      \cell{$\sharp$}
      \cell{$\sharp$}
      \cell{$\sharp$}
      \cell{$\sharp$}   \smhead
      \cell{$\sharp$}
      \cell{$\sharp$}
      \cell{$\sharp$}
      \cell{$\sharp$}
    \end{tikzpicture}
  }
      
  \on<2->[x=-25mm, y=-35mm]{
    \begin{tikzpicture}[tape, x=5mm, y=5mm]
      \draw (0.5,0.25) node{$c$ :};
      \cell{$\blank$}
      \cell{\alt<-3>{$\blank$}{$q_0$}} \smhead<3->
      \cell{$u_1$}                 \smhead<2>
      \cell{$u_2$}
      \cell{$u_3$}
      \cell{$u_4$}
      \cell{$u_5$}
      \cell{$\blank$}
    \end{tikzpicture}
  }
 
  \on<2->[x=25mm, y=-30mm]{
    \begin{tikzpicture}[turingMachine]
      \state[example, alert ob=<2>, initial]  (0) at (0,0) {0}; 
      \state[example, alert ob=<3>         ]  (1) at (1,0) {1}; 
      \state[example, alert ob=<4>         ]  (2) at (2,0) {2}; 
      
      \path (0) edge node[above] {\smTMtransL[c]{x}{x}} node[below] {$\forall x \in \Sigma$} (1);
      \path (1) edge node        {\smTMtransR[c]{\blank}{q_0}}                               (2);
 
      \begin{scope}[background]
        \state[rectangle, draw=example, fill=example!10, fit=(0.center)(2.center), minimum height=20mm, inner sep=0] (box) {};
        \node[example, anchor=north west] at (box.north west) {init};
      \end{scope}
    \end{tikzpicture}
  }

\end{frame}

\endgroup
