% SPDX-License-Identifier: CC-BY-SA-4.0
% Author: Matthieu Perrin
% Part: <Nom de la partie>
% Section: <Nom de la section>
% Sub-section: <Nom de la sous-section>  % (facultatif, laisser vide si non utilisé)
% Frame: <Titre de la slide>

\begingroup

\begin{frame}{Équivalence entre formalismes}

  \on[text,top]{
    \begin{itemize}
    \item Les cinq classes suivantes sont égales
    \item On nomme la classe commune \alert{\textsc{re}} (pour \structure{récursivement énumérable})
    \end{itemize}
  }
  
  \on[y=-10mm]{
    \begin{tikzpicture}[x=38mm, y=33mm]\small
      \tikzset{box/.style={structure, smBox, text width=27mm, minimum height=12mm},}
      \node[box] (eng) at (1,1) {Langage\\\alert{engendré} par une\\ grammaire\\ non-restreinte};
      \node[box] (rec) at (2,1) {Langage\\\alert{reconnu} par une \\ machine de Turing \\non déterministe};
      \node[box] (gen) at (2,0) {Langage\\\alert{généré} par une \\ machine de Turing \\non déterministe};
      \node[box] (sem) at (3,1) {Langage\\\alert{semi-décidé} par une \\ machine de Turing \\déterministe};
      \node[box] (enu) at (3,0) {Langage\\\alert{énuméré} par une \\ machine de Turing \\déterministe};

      \footnotesize
      \path[-latex, densely dotted] (rec) edge            node[auto]               {énumération de $\Sigma^\star$}  (gen);
      \path[-latex, densely dotted] (gen) edge            node[auto, sloped]       {encodage}                      (eng);
      \path[-latex, densely dotted] (eng) edge[bend left] node[above]              {force brute}                   (rec);
      \path[-latex, densely dotted] (rec) edge[bend left] node[above]              {déterminisation}               (sem);
      \path[-latex, structure]      (sem) edge[bend left] node[below]              {inclusion}                     (rec);
      \path[-latex, densely dotted] (gen) edge[bend left] node[above]              {déterminisation}               (enu);
      \path[-latex, structure]      (enu) edge[bend left] node[below]              {inclusion}                     (gen);
    \end{tikzpicture}
  }
\end{frame}

\endgroup
