% SPDX-License-Identifier: CC-BY-SA-4.0
% Author: Matthieu Perrin
% Part: <Nom de la partie>
% Section: <Nom de la section>
% Sub-section: <Nom de la sous-section>  % (facultatif, laisser vide si non utilisé)
% Frame: <Titre de la slide>

\begingroup


\begin{frame}{Classes de décidabilité}
  Soit $\Sigma$ un alphabet.
  \begin{block}{Expressivité des machines de Turing}
    Les MTD permettent d'exprimer deux notions sur les langages. On note :
    \begin{description}
    \item[$\textsc{r}_\Sigma$] pour \structure{récursif} :\\
      $\textsc{r}_\Sigma$ est l'ensemble des langages sur $\Sigma$
      \begin{itemize}
      \item \alert{décidables}
      \end{itemize}
    \item[$\textsc{re}_\Sigma$] pour \structure{récursivement énumérable} :\\
      $\textsc{re}_\Sigma$ est l'ensemble des langages sur $\Sigma$ 
      \begin{itemize}
      \item \alert{semi-décidables},  
      \item \alert{reconnaissables},
      \item \alert{récursivement énumérables},
      \item \alert{de type $0$}...
      \end{itemize}
    \end{description}
  \end{block}
  \begin{block}{Remarque}
    On parle des \structure{classes} $\textsc{r}$ et $\textsc{re}$, indépendantes de $\Sigma$.
  \end{block}
\end{frame}

\endgroup
