% SPDX-License-Identifier: CC-BY-SA-4.0
% Author: Matthieu Perrin
% Part: <Nom de la partie>
% Section: <Nom de la section>
% Sub-section: <Nom de la sous-section>  % (facultatif, laisser vide si non utilisé)
% Frame: <Titre de la slide>

\begingroup

\begin{frame}{Un grand problème ouvert : $P \stackrel{?}{=} NP$}

  \on[width=.7\textwidth, x=9mm, y=25mm]{
    \myquote{William Gasarch}{
      La plupart des théoriciens pensent que $\textsc{p} \neq \textsc{np}$.
    }
  }

  \on[width=.7\textwidth, x=-21mm, y=7mm]{
    \myquote{Stephen Cook}{
      Une réponse positive aurait des conséquences pratiques et philosophiques profondes.
    }
  }

  \on[width=\textwidth, x=-5mm, y=-12mm]{
    \myquote{Scott Aaronson}{
      Si $\textsc{p} = \textsc{np}$, trouver des solutions serait aussi facile que les
      vérifier ; la plupart des systèmes cryptographiques s'effondreraient.
    }
  }

    \onAlertBlock[bottom]{L'un des sept \og Millennium Prize Problems \fg}{
    \begin{itemize}
    \item L'équivalent des problèmes de Hilbert pour le XXI$^{\text{e}}$ siècle
    \item Le \structure{Clay Mathematics Institute} offre \alert{1\,000\,000\$} à quiconque y répond
    \end{itemize}
  }

\end{frame}

\endgroup
