% SPDX-License-Identifier: CC-BY-SA-4.0
% Author: Matthieu Perrin
% Part: <Nom de la partie>
% Section: <Nom de la section>
% Sub-section: <Nom de la sous-section>  % (facultatif, laisser vide si non utilisé)
% Frame: <Titre de la slide>

\begingroup

\begin{frame}{Complexité non-déterministe d'un problème}

  \begin{block}{Classes de complexité \alert{non-déterministe} d'un problème}
    Soit $f$ une suite de $\mathbb{N}$ dans $\mathbb{R}^+$. 
    On définit les \structure{classes de complexité} : 
    \begin{description}[$\textsc{co-nspace}(f)$ :]
    \item[$\textsc{\alert{n}time}(f)$ :] langages reconnaissables en temps $\mathcal{O}(f)$ \structure{par une MT\alert{N}D} :
      
      \vspace{-2mm}
      $$\structure{\textsc{\alert{n}time}(f) \eqdef \left\{L \in \textsc{lang} \mid \exists M, L = \mathcal{L}(M) \land \alert{N}T_M \in \mathcal{O}(f) \right\}}$$
    \item[$\textsc{\alert{n}space}(f)$ :] langages reconnaissables en espace $\mathcal{O}(f)$ \structure{par une MT\alert{N}D} :

      \vspace{-2mm}
      $$\structure{\textsc{\alert{n}space}(f) \eqdef \left\{L \in \textsc{lang} \mid \exists M, L = \mathcal{L}(M) \land \alert{N}S_M \in \mathcal{O}(f) \right\}}$$
    \end{description}

    On s'intéresse aussi à la complexité de reconnaissance du complémentaire :
    \begin{description}[$\textsc{co-nspace}(f)$ :]
    \item[$\textsc{\alert{co}-\alert{n}time}(f)$ :] langages dont le complémentaire est dans $\textsc{\alert{n}time}(f)$
      
      \vspace{-2mm}
      $$\structure{\textsc{\alert{co}-\alert{n}time}(f) \eqdef \left\{L \in \textsc{lang} \mid \overline{L} \in \textsc{\alert{n}time}(f)\right\}}$$

    \item[$\textsc{\alert{co}-\alert{n}space}(f)$ :] langages dont le complémentaire est dans $\textsc{\alert{n}space}(f)$
      
      \vspace{-2mm}
      $$\structure{\textsc{\alert{co}-\alert{n}space}(f) \eqdef \left\{L \in \textsc{lang} \mid \overline{L} \in \textsc{\alert{n}space}(f)\right\}}$$
    \end{description}
  \end{block}

\end{frame}

\endgroup
\endinput
