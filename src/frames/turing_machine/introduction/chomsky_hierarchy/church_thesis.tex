% SPDX-License-Identifier: CC-BY-SA-4.0
% Author: Matthieu Perrin
% Part: <Nom de la partie>
% Section: <Nom de la section>
% Sub-section: <Nom de la sous-section>  % (facultatif, laisser vide si non utilisé)
% Frame: <Titre de la slide>

\begingroup

\begin{frame}{Thèse de Church -- Turing}

  \onBlock[left=.75\textwidth, y=22mm]{Thèse de Church -- Turing (1936-1937)}{
    \centering
    \structure{La définition des \og fonctions calculables \fg par des \\
      \alert{Machines de Turing déterministes}  \\
      caractérise la notion intuitive de \og procédure effective \fg.}
  }
  
  \onBlock[y=-15mm]{Arguments en faveur de la thèse}{
    \begin{itemize}
    \item Équivalence entre formalismes
      \begin{itemize}
      \item Machines de Turing, $\lambda$-calcul, langages de programmation...
      \end{itemize}
    \item On ne connaît pas de machine plus puissante 
      \begin{itemize}
      \item Les formalismes plus expressifs necessitent des \og oracles \fg
        \begin{itemize}
        \item Par exemple, on peut \emph{définir} des objets mathématiques non-calculables
        \end{itemize}
      \item Les limites au formalisme sont internes
        \begin{itemize}
        \item L'indécidabilité du \structure{problème de l'arrêt} vient d'un paradoxe
        \end{itemize}
      \end{itemize}
    \item Possibilité de simuler l'univers...
      \begin{itemize}
      \item ... donc toute machine qui peut y être effectivement construite 
      \end{itemize}
    \end{itemize}
  }

  \onImage[x=42mm,y=15mm]{%
    width=2.3cm,
    title={Alonzo Church},
    license={$\copyright$ - Princeton University Library (voir \href{https://en.wikipedia.org/wiki/File:Alonzo_Church.jpg}{Wikimedia}). Utilisation non commerciale à des fins pédagogiques (fair use)},
    img={Church.jpg}
  }

\end{frame}



\endgroup
