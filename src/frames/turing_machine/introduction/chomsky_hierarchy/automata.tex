% SPDX-License-Identifier: CC-BY-SA-4.0
% Author: Matthieu Perrin
% Part: <Nom de la partie>
% Section: <Nom de la section>
% Sub-section: <Nom de la sous-section>  % (facultatif, laisser vide si non utilisé)
% Frame: <Titre de la slide>

\begingroup

\begin{frame}{Hiérarchie de chomsky et automates}

  \centering
  \begin{tikzpicture}[2Darray, x=25mm, y=17mm]
    \arrayColumn[width=8mm, header=Type]{
      \arrayLine{0}
      \arrayLine{1}
      \arrayLine{2}
      \arrayLine{3}
    }
    
    \arrayColumn[header=Grammaire]{
      \arrayLine{Non-restreinte\\ $\alpha A \beta \rightarrow \gamma$}
      \arrayLine{Contextuelle\\   $\alpha A \beta \rightarrow \alpha \gamma \beta$ \\ $\gamma\neq\varepsilon$}
      \arrayLine{Algébrique\\     $A \rightarrow \beta$}
      \arrayLine{Rationnelle\\    $A \rightarrow a B \mid b \mid \varepsilon$}
    }
    
    \arrayColumn[header=Langage]{
      \arrayLine{Récursivement\\ énumérable}
      \arrayLine{Contextuel}
      \arrayLine{Algébrique}
      \arrayLine{Rationnel}
    }
    
    \arrayColumn[header=Automate]{
      \arrayLine{Machine\\de Turing}
      \arrayLine{Automate\\linéairement\\ borné}
      \arrayLine{Automate à pile\\non déterministe}
      \arrayLine{Automate fini}
    }
    
    \arrayColumn[header=Déterminisable]{
      \arrayLine{Oui, mais \\complexité inconnue\\ \footnotesize $\textsc{p} \stackrel{?}{=} \textsc{np}$}
      \arrayLine{Inconnu\\ \footnotesize $\textsc{nspace}(\mathcal{O}(n))$ \\ $\stackrel{?}{=}$\\$\textsc{dspace}(\mathcal{O}(n))$}
      \arrayLine{Impossible}
      \arrayLine{Oui,\\ exponentiel\\ en nombre\\ d'états}
    }
  \end{tikzpicture}

  \begin{block}{Automates finis généralisés}
    Si un automate fini n'est pas suffisant, ajouter une structure de données
    \begin{itemize}
    \item Pile : \structure{automate à pile}
    \item Tableau de taille $|u|$ : \structure{automate linéairement bornée}
    \item Mémoire infini : \structure{machine de Turing}
    \begin{itemize}
    \item Quel modèle de mémoire permet de modéliser les algorithmes ? 
    \end{itemize}
    \end{itemize}
  \end{block}
  
\end{frame}

\endgroup
