% SPDX-License-Identifier: CC-BY-SA-4.0
% Author: Matthieu Perrin
% Part: 
% Section: 
% Sub-section: 
% Frame: 

\begingroup

\begin{frame}{Comment modéliser la mémoire ?}

  \begin{block}{Modèle de processus dans les OS}
    \begin{itemize}
    \item \structure{Mémoire disponible : }
      \begin{itemize}
      \item Un tableau de $2^N$ octets.
      \item Des pointeurs sur $N$ bits.
        \begin{itemize}
        \item Typiquement, $N=32$ ou $N=64$.
        \end{itemize}
      \end{itemize}
    \item \alert{Problème :} 
      \begin{itemize}
      \item En théorie, un automate fini à $2^{2^N}$ états
        \begin{itemize}
        \item En pratique, $N$ fini et $2^N$ infini
        \end{itemize}
      \end{itemize}
    \end{itemize}
  \end{block}
  \pause
  \begin{block}{Modèle RAM (\emph{Random access machine})}
    \begin{itemize}
    \item \structure{Mémoire disponible : }
      \begin{itemize}
      \item Un tableau infini d'entiers non-bornés.
      \end{itemize}
    \item \alert{Problème :} 
      \begin{itemize}
      \item Il faut des opérations sur les entiers dans le modèle
        \begin{itemize}
        \item Ces opérations ne sont-elles pas déjà des algorithmes ?
        \item On veut un \structure{ensemble de règles fini}.
        \end{itemize}
      \end{itemize}
    \end{itemize}
  \end{block}
\end{frame}

\endgroup
