% SPDX-License-Identifier: CC-BY-SA-4.0
% Author: Matthieu Perrin
% Part: <Nom de la partie>
% Section: <Nom de la section>
% Sub-section: <Nom de la sous-section>  % (facultatif, laisser vide si non utilisé)
% Frame: <Titre de la slide>

\begingroup

\begin{frame}{Le programme de Hilbert}

  \onBlock[top=-2mm]{Une ambition fondationnelle (début XX\ieme)}{
    Après les paradoxes en théorie des ensembles, l'objectif est de
    \begin{itemize}
    \item formaliser rigoureusement toutes les mathématiques
    \item garantir la \structure{cohérence} des théories formelles
    \item idéalement, assurer leur \structure{complétude}
    \end{itemize}
  }

  \onBlock[bottom=3mm]{Une exigence nouvelle}{
    Les raisonnements mathématiques doivent être
    \begin{itemize}
    \item finis
    \item formels
    \item et \structure{mécaniquement vérifiables}
    \end{itemize}
  }

\end{frame}

\begin{frame}{Le Problème de la Décision}

  \onBlock[left=.6\textwidth,y=1]{Der Entscheidungsproblem}{
    Existe-t-il une procédure mécanique permettant de déterminer, pour toute formule logique donnée, si celle-ci est logiquement valide ?
  }

  \onImage[x=3.5,y=1]{%
    width=2.5cm,
    title={David Hilbert},
    licenselogo={\ccPublicDomain},
    license={Domaine public (1912, \href{https://commons.wikimedia.org/wiki/File:Hilbert.jpg}{Wikimedia})},
    img={Hilbert.jpg}
  }

 
  \onAlertBlock[y=-2]{Questions}{
    \begin{itemize}
    \item Qu'est-ce qu'un \og \alert{problème de décision} \fg, en général ?
    \item Qu'est-ce qu'une \og \alert{procédure mécanique} \fg ?
    \end{itemize}
  }
  
\end{frame}

\endgroup
