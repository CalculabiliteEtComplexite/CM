% SPDX-License-Identifier: CC-BY-SA-4.0
% Author: Matthieu Perrin
% Part: <Nom de la partie>
% Section: <Nom de la section>
% Sub-section: <Nom de la sous-section>  % (facultatif, laisser vide si non utilisé)
% Frame: <Titre de la slide>

\begingroup

\begin{frame}{Exemple -- Système de preuves formel}

  \begin{block}{Rappel -- Deux problèmes mathématiques}

    Soit $\langle \Phi, \Pi, \valid \rangle$ un système de preuve. On s'intéresse aux problèmes : 

    \hspace\fill%
    \begin{minipage}[t]{.45\textwidth}
      \begin{problembox}
        \textsc{Vérification des preuves}\\
        \structure{Instance :} $\langle \pi, \varphi \rangle \in \Pi \times \Phi$ \\
        \structure{Question :} $\pi \valid \varphi$ ?
      \end{problembox}
    \end{minipage}%
    \hspace\fill%
    \begin{minipage}[t]{.5\textwidth}
      \begin{problembox}
        \textsc{Entscheidungsproblem}\\
        \structure{Instance :} $\varphi \in \Phi$ \\
        \structure{Question :} $\exists \pi \in \Pi, \pi \valid \varphi$ ?
      \end{problembox}
    \end{minipage}%
    \hspace\fill%
  \end{block}

  \begin{block}{Corollaire du théorème -- Système formel}
    Les deux énoncés suivants sont équivalents :
    \begin{enumerate}
    \item On peut énumérer les propositions démontrables
      $$\textsc{Entscheidungsproblem} \in \textsc{re}$$
    \item On peut vérifier qu'une démonstration valide une proposition
      $$\textsc{Vérification des preuves} \in \textsc{r}$$
    \end{enumerate}
    \begin{itemize}
    \item On dira alors que $\langle \Phi, \Pi, \valid \rangle$ est un système de preuves \structure{formel}.
    \end{itemize}
  \end{block}
  
\end{frame}

\endgroup
