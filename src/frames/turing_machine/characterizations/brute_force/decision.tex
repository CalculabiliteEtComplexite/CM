% SPDX-License-Identifier: CC-BY-SA-4.0
% Author: Matthieu Perrin
% Part: 
% Section: 
% Sub-section: 
% Frame: 

\begingroup

\begin{frame}{Décision de l'appartenance à un langage contextuel}

  \vspace{-1mm}
  \begin{block}{Problème de décision}
    Soit $L$ un langage de type 0 ou 1.
    
    \Probleme{\structure{Decision$_L$}}{
      Un mot \structure{$u \in \Sigma^\star$}
    }{
      Est-ce que \structure{$u \in L$} ?
    }
  \end{block}

  \vspace{-2mm}
  \begin{block}{Algorithme de recherche ascendante par force brute} 
  \vspace{-1mm}

    \begin{description}
    \item [Entrées :]
      \begin{itemize}
      \item Une grammaire $G$, si possible contextuelle
      \item Un mot $u$
      \end{itemize}
    \item [Sortie :] \example{$u \in \mathcal{L}(G)$}, \structure{$u \notin \mathcal{L}(G)$}, ou \alert{boucle infinie}
    \item [Teminaison :] 
      \begin{itemize}
      \item garantie si $G$ est contextuelle (décision)
      \item garantie si $G$ est générale et $u \in L$ (semi-décision)
      \end{itemize}
    \item [Complexité :] 
      \begin{itemize}
      \item exponentielle par rapport à $|u|$ si $G$ est contextuelle
      \item non bornée si $G$ est générale (même si $u \in L$)
      \end{itemize}
    \end{description}
  \end{block}
  
\end{frame}

\endgroup
\endinput
