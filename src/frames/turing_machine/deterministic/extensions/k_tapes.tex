% SPDX-License-Identifier: CC-BY-SA-4.0
% Author: Matthieu Perrin
% Part: 
% Section: 
% Sub-section: 
% Frame: 

\begingroup
\begin{frame}{Machine de Turing à $k$ rubans}
  \small  
  Une \structure{machine de Turing à $k$ rubans} est un tuple \alert{$\langle \Sigma, \Gamma, \langle\blank_1, ..., \blank_k\rangle, Q, q_0, F, \rightarrow \rangle$} :
  \begin{description}
  \item[\alert{$\Sigma$}, \alert{$\Gamma$}, \alert{$Q$}, \alert{$q_0$}, \alert{$F$}] jouent le même rôle qu'avec un seul ruban
  \item[\alert{$\blank_1, ..., \blank_k$}] $\in \Gamma\setminus\Sigma$ : un \structure{symbole vide} par ruban (en général le même)
  \item[\alert{$\rightarrow$}] $\subseteq \left(Q \times \Gamma^k\right) \times \left(Q \times \left(\Gamma \times \{ \triangleleft, \triangleright, \diamond \}\right)^k\right)$ : \structure{la relation de transition}
  \end{description}

  \vspace{2mm}
  Une \structure{transition} est une paire \alert{$\langle \langle q, \langle a_1, ..., a_k \rangle\rangle, \langle q', \langle a'_1, d_1 \rangle, ..., \langle a'_k, d_k \rangle \rangle \in \rightarrow$} :
  \begin{itemize}
  \item Représente une transition \alert{$\langle \langle q, a_i \rangle, \langle q', a'_i, d_i \rangle \rangle$} par ruban
  \item Graphiquement, les transitions non spécifiées laissent leur ruban inchangé.
  \begin{itemize}
  \item Par exemple, pour $\Gamma = \{0,1\}$ et $k=2$ :
    \begin{tikzpicture}[turingMachine]
      \node at (1.5,0) {$\equiv$}; 

      \state (0) at (0,0) {$q$}; 
      \state (1) at (1,0) {$q'$}; 
      \state (2) at (2,0) {$q$}; 
      \state (3) at (3,0) {$q'$}; 

      \path (0) edge node       {\smTMtransR[1]{0}{1}}                               (1);
      \path (2) edge node       {\smGroup{\smTMtransR[1]{0}{1}\smTMtransL[2]{0}{0}}} (3);
      \path (2) edge node[swap] {\smGroup{\smTMtransR[1]{0}{1}\smTMtransL[2]{1}{1}}} (3);
    \end{tikzpicture}
  \end{itemize}
  \item Une \structure{configuration} $\langle r_1, \ldots, r_k, q\rangle$ représente l'état $q$ et chaque ruban $r_i$
  \item La \structure{configuration initiale de $u$} est $\alert{C_{\mathit{init}}(u) \eqdef \langle \langle \varepsilon, u \rangle, \langle \varepsilon, \varepsilon \rangle, ..., \langle \varepsilon, \varepsilon \rangle, q_0 \rangle}$.
  \item Une \structure{action} met à jour les $k$ rubans en même temps. 
  \end{itemize}
\end{frame}

\endgroup
