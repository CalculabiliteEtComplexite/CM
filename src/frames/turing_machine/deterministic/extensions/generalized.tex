% SPDX-License-Identifier: CC-BY-SA-4.0
% Author: Matthieu Perrin
% Part: <Nom de la partie>
% Section: <Nom de la section>
% Sub-section: <Nom de la sous-section>  % (facultatif, laisser vide si non utilisé)
% Frame: <Titre de la slide>

\begingroup

%\begin{frame}{Généralisation des machines de Turing déterministes}
% 
%  Pour simplifier la conception de machines de Turing, on s'autorise parfois 
%  des transitions plus complexes
%  
%  \begin{block}{Opération sans déplacement}
%    \begin{itemize}
%    \item \structure{$q \xrightarrow{\smTMtransS{a}{b}} q'$} lit $a$, écrit $b$, ne déplace pas la tête de lecture%\hspace\fill\alert{$ \langle G, a D \rangle  \rightarrow \langle G, b D \rangle $}
%    \end{itemize}
%  \end{block}
% 
%%    \begin{description}
%%    \item<1->[$q \xrightarrow{\smTMtransS{a}{b}} q'$ :] lit $a$, écrit $b$, ne déplace pas la tête de lecture\hspace\fill\alert{$ \langle G, a D \rangle  \rightarrow \langle G, b D \rangle $}
%%    \item<4->[$q \xrightarrow{\smTMtransP{a}{b}} q'$ :] lit $a$, ajoute $b$ à gauche\hspace\fill\alert{$ \langle G, a D \rangle  \rightarrow \langle G b, a D \rangle $}
%%    \item<9->[$q \xrightarrow{\smTMtransM{a}} q'$ :] lit $a$, supprime $a$\hspace\fill\alert{$ \langle G, a D \rangle  \rightarrow \langle G, D \rangle $}
%%    \end{description}
% 
%  
%  \begin{block}{Opération sur des mots}
%    \item \structure{$q \xrightarrow{\smTMtrans{a}{b}{d}} q'$} a le   lit $a$, écrit $b$, ne déplace pas la tête de lecture%\hspace\fill\alert{$ \langle G, a D \rangle  \rightarrow \langle G, b D \rangle $}
%  \end{block}
% 
%  \begin{block}{Machines de Turing multi-rubans}
%  \end{block}
% 
%\end{frame}

\begin{frame}{Généralisation des machines de Turing}
  Pour simplifier la conception de machines de Turing, on s'autorise parfois des  \structure{transitions généralisées}
  \alert{$\langle \langle q, u_1, ..., u_k \rangle, \langle q', \langle v_1, d_1 \rangle, ..., \langle v_k, d_k \rangle \rangle \rangle$}, telles que :
  \begin{description}[-------]
  \item[\alert{$q$}] $\in Q$ : \structure{l'état de départ}
  \item[\alert{$u_i$}] $\in \Gamma\alert{^\star}$ : \structure{le mot lu sur le ruban \alert{$r_i$}}
  \item[\alert{$q'$}] $\in Q$ : \structure{l'état d'arrivée}
  \item[\alert{$v_i$}] $\in \Gamma\alert{^\star}$ : \structure{le mot écrit sur le ruban \alert{$r_i$}}
  \item[\alert{$d_i$}] $\in \{\triangleleft, \alert{\diamond}, \triangleright\}$ : \structure{le déplacement de la tête de \alert{$r_i$}}
  \end{description}
 
  \on[x=37.5mm,y=10mm]{
    \begin{tikzpicture}[turingMachine, x=25mm]
      \state (q)  at (0,0) {$q$}; 
      \state (q1) at (1,0) {$q'$}; 
      \path  (q) edge node {\smGroup{\smTMtrans[r_1]{u_1}{v_1}{d_1}...\\\smTMtrans[r_k]{u_k}{v_k}{d_k}}} (q1);
    \end{tikzpicture}
  }

  \begin{block}{Nouvelles fonctionnalités}
    \begin{description}[Multi-ruban :]
    \item[Réécriture :] lire un mot \(u\in\Gamma^\star\), écrire un mot \(v\in\Gamma^\star\)
      \begin{itemize}
      \item ajoute/supprime des cases si $|u| \neq |v|$
      \end{itemize}
    \item[Immobilité :] $\diamond$ ne déplace pas la tête de lecture
      \begin{itemize}
      \item $\varepsilon$-transitions possibles : $\smTMtransS{\varepsilon}{\varepsilon}$
      \end{itemize}
    \item[Multi-ruban :] \(k\) rubans parallèles, une tête/opération par ruban (simultané).
      \begin{itemize}
      \item choisir un ruban pour les entrées/sorties
      \end{itemize}
    \end{description}
  \end{block}

\end{frame}




\endgroup
