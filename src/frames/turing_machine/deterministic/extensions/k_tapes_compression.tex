% SPDX-License-Identifier: CC-BY-SA-4.0
% Author: Matthieu Perrin
% Part: 
% Section: 
% Sub-section: 
% Frame: 

\begingroup

\newcommand<>\cellTop[1]{%
  \uncover#2{%
    \smsave{x}%
    \begin{scope}[background]%
      \node[fill=structure!30, fit=(x.north west)(x.east), inner sep=0pt] {};%
      \node[fit=(x.north west)(x.east), anchor=mid] {\structure{#1}};%
    \end{scope}%
  }
}

\newcommand<>\cellBot[1]{%
  \uncover#2{%
  \smsave{x}%
  \begin{scope}[background]%
    \node[fill=alert!30, fit=(x.south west)(x.east), inner sep=0pt] {};%
    \node[fit=(x.south west)(x.east), anchor=mid] {\alert{#1}};%
  \end{scope}%
  }
}

\begin{frame}{Équivalence entre les deux modèles}

  \onBlock[top=-5mm]{Théorème -- Puissance des MT étendues}{
    Les MT étendues \structure{décident}, \structure{semi-décident}, 
    et \structure{énumèrent} les \alert{mêmes langages}, et \structure{calculent}
    les \alert{mêmes fonctions} que les MT strictes.
  }

  \onBlock<2>[y=5mm]{Démonstration}{
    Soit $M = \langle \Sigma, \Gamma, \blank, Q, q_0, F, \rightarrow \rangle$ une MT étendue.
    
    On construit $M' = \langle \Sigma, \Gamma', \blank', Q', q_0, F, \rightarrow' \rangle$ équivalente dans le modèle strict.

    \begin{itemize}
    \item \structure{Encodage des rubans}
      \begin{itemize}
      \item $\Gamma' = \Gamma^k \cup \{\text{\faMapMarker}\}$, $\blank' = \blank^k$
      \item \faMapMarker{} est à gauche de la position des têtes de lecture
      \item Les autres cases contiennent un symbole de chaque ruban
      \end{itemize}
    \item \example{Exemple avec $\Gamma = \{a, b, \blank\}$ et $k = 2$ :}
    \end{itemize}
  }

  \on<2>[y=-18mm, x=-27mm]{\footnotesize
    \begin{tikzpicture}[tape, size=3.5mm]
      \node{$1 :$};
      \cell{}
      \cell{}
      \cell[structure]{$a$}
      \cell[structure]{$b$}
      \cell[structure]{$b$}
      \cell[structure]{$a$} 
      \cell[structure]{$b$} \smhead[structure]
      \cell[structure]{$a$}
      \cell[structure]{$b$}
      \cell{}
      \cell{}
    \end{tikzpicture}
  }

  \on<2>[y=-18mm, x=27mm]{\footnotesize
    \begin{tikzpicture}[tape, size=3.5mm]
      \node{$2 :$};
      \cell{}
      \cell{}
      \cell[alert]{$b$}
      \cell[alert]{$a$}
      \cell[alert]{$a$}
      \cell[alert]{$a$} \smhead[alert]
      \cell[alert]{$a$}
      \cell[alert]{$b$}
      \cell[alert]{$b$}
      \cell{}
      \cell{}
    \end{tikzpicture}
  }

  \on<2>[y=-28mm]{\footnotesize
    \begin{tikzpicture}[tape, size=7mm]
      \cell{}
      \cell{} \cellTop{a}
      \cell{} \cellTop{b}\cellBot{b}
      \cell{} \cellTop{b}\cellBot{a}
      \cell{} \cellTop{a}\cellBot{a}
      \cell[example]{\faMapMarker} 
      \cell{} \cellTop{b}\cellBot{a} \smhead[example]
      \cell{} \cellTop{a}\cellBot{a}
      \cell{} \cellTop{b}\cellBot{b}
      \cell{}            \cellBot{b}
      \cell{}
    \end{tikzpicture}
  }

  \on<2>[text, bottom=-2mm]{
    \begin{itemize}
    \item \structure{Simulation des transitions}
      \begin{itemize}
      \item Problème : il faut garder les têtes de lecture au même endroit
      \item Au lieu de déplacer la tête de lecture, on déplace le ruban à l'envers
      \end{itemize}
    \end{itemize}
  }
\end{frame}

\endgroup
