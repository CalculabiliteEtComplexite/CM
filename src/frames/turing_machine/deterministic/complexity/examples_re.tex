% SPDX-License-Identifier: CC-BY-SA-4.0
% Author: Matthieu Perrin
% Part: <Nom de la partie>
% Section: <Nom de la section>
% Sub-section: <Nom de la sous-section>  % (facultatif, laisser vide si non utilisé)
% Frame: <Titre de la slide>

\begingroup

\begin{frame}{Exemples de problèmes semi-décidables}

  \vspace{-2mm}
  \begin{block}{Intersection de langages algébriques}

  \vspace{-2mm}
    \Probleme{Intersect-Alg}{
      Deux grammaires algébriques $G$ et $H$
    }{
      Existe-t-il un mot $w \in \mathcal{L}(G) \cap \mathcal{L}(H)$ ?
    }

  \vspace{-2mm}
    \begin{itemize}
    \item \structure{Vérifier} que $w \in \mathcal{L}(G) \cap \mathcal{L}(H)$ est \alert{décidable} par CYK
    \item \textsc{Intersect-Alg} est \alert{semi-décidable} (théorème précédent)
    \item \textsc{Intersect-Alg} est \alert{indécidable} (preuve en TD)
    \end{itemize}
  \end{block}

  \vspace{-2mm}
  \begin{block}{Équations Diophantiennes}

  \vspace{-2mm}
    \Probleme{Diophantine}{
      Un polynôme $P$ à coefficients entiers
    }{
      Existe-t-il un entier $x\in \mathbb{N}, P(x) = 0$ ?
    }

  \vspace{-2mm}
    \begin{itemize}
    \item \structure{Vérifier} que $P(x) = 0$ est \alert{décidable}
    \item \textsc{Diophantine} est \alert{semi-décidable}, mais \alert{indécidable} (admis)
    \end{itemize}
  \end{block}
  
\end{frame}

\endgroup
