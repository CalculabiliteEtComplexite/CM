% SPDX-License-Identifier: CC-BY-SA-4.0
% Author: Matthieu Perrin
% Part: <Nom de la partie>
% Section: <Nom de la section>
% Sub-section: <Nom de la sous-section>  % (facultatif, laisser vide si non utilisé)
% Frame: <Titre de la slide>

\begingroup

\SetKwFunction{IsPrime}{is\_prime}

\begin{frame}{Exemple : test de primalité}

  \on[top=-4mm, text]{
    \Probleme{Primes}{
      Un entier $N\in \mathbb{N}$
    }{
      $N$ est-il premier ? 
    }
  }
  
  \onBlock[left=.6\textwidth, y=-2mm]{Un problème non-trivial}{
    \vspace{-1mm}
    \begin{itemize}
    \item La complexité doit être exprimée en fonction du nombre de chiffres de :

      \vspace{-2mm}
      $$n = |(N)_b| = \lceil \log_b(N) \rceil$$

    \item L'algorithme de droite est exponentiel

      \vspace{-2mm}
      $$\Omega\left(\sqrt{N}\right) = \Omega\left(2^{\frac{n}{2}}\right)$$
    \end{itemize}
  }

  \on[right=.43\textwidth, y=-2mm]{
    \begin{algorithm}[H]
      \Fun{$\IsPrime\left(N\in \mathbb{N}\right)$}{
        \For{$k$ \From $2$ \To $\sqrt{N}$}{
          \If{$N \equiv 0 \pmod k$}{\Return \False;}
        }
        \Return \True;
      }
    \end{algorithm}
  }  

  \onBlock[bottom=2mm]{Algorithme d'Agrawal, Kayal et Saxena (2004)}{
    \vspace{-1mm}
    \begin{itemize}
    \item Décide \textsc{Prime} en $\mathcal{O}(n^{12})$ 
    \item Donc $\textsc{Prime} \in \textsc{p}$ 
    \end{itemize}
  }

  \footnoteref{M. Agrawal, N. Kayal, N. Saxena. \emph{\textsc{primes} is in \textsc{p}}. Annals of mathematics (2004)}
  
\end{frame}

\endgroup
