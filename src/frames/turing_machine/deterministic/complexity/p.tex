% SPDX-License-Identifier: CC-BY-SA-4.0
% Author: Matthieu Perrin
% Part: <Nom de la partie>
% Section: <Nom de la section>
% Sub-section: <Nom de la sous-section>  % (facultatif, laisser vide si non utilisé)
% Frame: <Titre de la slide>

\begingroup

\begin{frame}{Importance de la classe P}

  \onBlock[top=-4mm]{La classe des problèmes \og raisonnables \fg}{
    \begin{itemize}
    \item Souvent faisable en pratique pour les ordinateurs modernes
    \item Composer deux algorithmes polynomiaux reste polynomial
    \item Classe très stable vis-à-vis de variantes raisonnables du calcul
      \begin{itemize}
      \item langages de programmation, $\lambda$-calcul, extensions des MTD...
      \end{itemize}
    \end{itemize}
  }

  \onExampleBlock[bottom=2mm, left=.6\textwidth]{Illustration}{
    \begin{itemize}
    \item On suppose 1 ns par instruction
    \end{itemize}
    \footnotesize
    \begin{tabular}{c|c@{~~}c@{~~}c}
      \structure{$n$} 
      & \structure{$n^2$} 
      & \structure{$n^3$} 
      & \structure{$n^6$} \\
      \hline
      10
      & 100 ns
      & 1 µs
      & 1 ms \\

      20
      & 400 ns
      & 8 µs
      & 64 ms \\

      40
      & 1.6 µs
      & 64 µs
      & 4.1 s \\

      60
      & 3.6 µs
      & 216 µs
      & 46.7 s \\[2mm]

      \structure{$n$} 
      & \structure{$2^n$} 
      & \structure{$3^n$} 
      & \structure{$n^n$} \\
      \hline
      10
      & 1 µs
      & 59 µs
      & 10 s \\

      20
      & 1 ms
      & 3.48 s
      & $6.49 \cdot 10^{6}$ ans \\

      40
      & 18.3 min
      & 385 ans
      & $1.49 \cdot 10^{42}$ ans \\

      60
      & 36.5 ans
      & $1.34 \cdot 10^{12}$ ans
      & $1.54 \cdot 10^{83}$ ans \\
    \end{tabular}
  }
  
  \on[bottom, x=30mm]{
    \begin{tikzpicture}[x=1mm, y=3mm]
      
      \footnotesize
      \draw[->]             (0,0) -- (45,0) node[right]{$n$};
      \draw[->]             (0,0) -- (0,10) node[above]{$t {\color{black!40}(s)}$};

      \foreach \x in {0,5,...,45}{
        \draw[densely dotted, black!40] (\x,0) -- (\x,10);
        \node[below, black!40] at (\x,0) {\tiny\x};
      }
      \foreach \y in {0,1,...,10}{
        \draw[densely dotted, black!40] (0,\y) -- (45,\y);
        \node[left, black!40] at (0,\y) {\tiny\y};
      }

      \begin{scope}
        \clip (0,0) rectangle (45,10);
        \draw[domain=0.1:11, samples=100, smooth, black!40 ] plot (\x,{exp(\x*ln(\x)-9*ln(10))});
        \draw[domain=0:22,   samples=100, smooth, alert    ] plot (\x,{exp(\x*ln(3) -9*ln(10))});
        \draw[domain=0:34,   samples=100, smooth, alert    ] plot (\x,{exp(\x*ln(2) -9*ln(10))});
        \draw[domain=0.1:45, samples=100, smooth, structure] plot (\x,{exp(6*ln(\x) -9*ln(10))});
        \draw[domain=0.1:45, samples=100, smooth, structure] plot (\x,{exp(5*ln(\x) -9*ln(10))});
      \end{scope}
      
      \node[below right, black!40 ] at (10,10)  {$n^n$};
      \node[below right, alert    ] at (21,10)  {$3^n$};
      \node[below right, alert    ] at (33,10)  {$2^n$};
      \node[above left,  structure] at (45,3.5) {$n^6$};
      \node[above left,  structure] at (45,0)   {$n^5$};

    \end{tikzpicture}
  }

\end{frame}

\endgroup
