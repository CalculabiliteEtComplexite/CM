% SPDX-License-Identifier: CC-BY-SA-4.0
% Author: Matthieu Perrin
% Part: <Nom de la partie>
% Section: <Nom de la section>
% Sub-section: <Nom de la sous-section>  % (facultatif, laisser vide si non utilisé)
% Frame: <Titre de la slide>

\begingroup

\begin{frame}{Suite à croissance polynomiale}

  \begin{block}{Définition -- la classe \textsc{poly}}
    Soit $f$ une suite de $\mathbb{N}$ dans $\mathbb{R}^+$.

    On dit que $f$ est \structure{à croissance polynomiale} s'il existe \structure{$k\in \mathbb{N}$}, tel que \structure{$f \in \mathcal{O}(n^k)$}.

    $$\alert{\textsc{poly} = \bigcup_{k\in \mathbb{N}} \mathcal{O}\left(n^k\right)}$$

    \vspace{-3mm}
    \structure{Remarque : }
    \begin{itemize}
    \item $f\in \textsc{poly}$ ssi $f\in \mathcal{O}(P)$ pour un polynôme $P$ à coefficients positifs
    \item On définit les suites \structure{à croissance exponentielle} : $\alert{\textsc{exp} = \bigcup_{k\in \mathbb{N}} \mathcal{O}\left(k^n\right)}$
    \end{itemize}
  \end{block}
  
  \begin{block}{Stabilité de la classe \textsc{poly}}
    \begin{description}[Par composition :] 
    \item[Par somme :] 
      $\begin{array}[t]{@{}l@{,~}l@{,\quad}l@{}}
      \forall f \in \mathcal{O}\left(n^k\right) & \forall g \in \mathcal{O}\left(n^l\right) & \alert{f+g \in \mathcal{O}\left(n^{\max(k,l)}\right)}\\
      \forall f \in \textsc{poly}               & \forall g \in \textsc{poly}               & \alert{f+g \in \textsc{poly}}
    \end{array}$
    \item[Par produit :]\vspace{1mm}
      $\begin{array}[t]{@{}l@{,~}l@{,\quad}l@{}}
      \forall f \in \mathcal{O}\left(n^k\right) & \forall g \in \mathcal{O}\left(n^l\right) & \alert{f\times g \in \mathcal{O}\left(n^{k + l}\right)}\\
      \forall f \in \textsc{poly}               & \forall g \in \textsc{poly}               & \alert{f\times g \in \textsc{poly}}
    \end{array}$
    \item[Par composition :]\vspace{1mm}
      $\begin{array}[t]{@{}l@{,~}l@{,\quad}l@{}}
      \forall f \in \mathcal{O}\left(n^k\right) & \forall g \in \mathcal{O}\left(n^l\right) & \alert{f \circ g \in \mathcal{O}\left(n^{k \times l}\right)}\\
      \forall f \in \textsc{poly}               & \forall g \in \textsc{poly}               & \alert{f \circ g \in \textsc{poly}}
    \end{array}$
    \end{description}
  \end{block}

\end{frame}

\endgroup
