% SPDX-License-Identifier: CC-BY-SA-4.0
% Author: Matthieu Perrin
% Part: <Nom de la partie>
% Section: <Nom de la section>
% Sub-section: <Nom de la sous-section>  % (facultatif, laisser vide si non utilisé)
% Frame: <Titre de la slide>

\begingroup

\begin{frame}{Machine de Turing déterministe (MTD)}

  Soit $M = \langle \Sigma, \Gamma, \blank, Q, q_0, F, \rightarrow \rangle$ une MTND.
  \begin{block}{Définition -- Machine de Turing déterministe}
    On dit que $M$ est \structure{déterministe}
    si sa relation de transition $\rightarrow$ est fonctionnelle :

    $$\forall \alert{q}\in Q, \forall \alert{a}\in \Gamma,
    \left|\left\{
    \left\langle \structure{q'}, \structure{b}, \structure{d}\right\rangle \in Q \times \Gamma \times \{\triangleleft, \triangleright\}
    \;\middle\mid\;
    \alert{q} \xrightarrow{\smTMtrans{\alert{a}}{\structure{b}}{\structure{d}}} \structure{q'}
    \right\}\right|
    \le 1$$
  \end{block}
  
  \begin{block}{Remarques -- Notion de machine de Turing non-déterministe}
    \begin{itemize}
    \item Toute machine de Turing déterministe est une MTND
    \item \alert{``Non-déterministe''} signifie \alert{``pas forcément déterministe''}
    \end{itemize}
  \end{block}

\end{frame}

\endgroup
