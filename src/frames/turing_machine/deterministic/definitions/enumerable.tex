% SPDX-License-Identifier: CC-BY-SA-4.0
% Author: Matthieu Perrin
% Part: <Nom de la partie>
% Section: <Nom de la section>
% Sub-section: <Nom de la sous-section>  % (facultatif, laisser vide si non utilisé)
% Frame: <Titre de la slide>

\begingroup

\begin{frame}{Langage récursivement énumérable}

  \onBlock[top=-5mm]{Définition -- Énumération d'un langage}{
    Soit $M=\langle \Sigma, \Gamma, \blank, Q, q_0, F, \rightarrow \rangle$ une MTD.
    \begin{itemize}
    \item Le \structure{langage énuméré par $M$} est l'ensemble des mots écrits sur le ruban quand $M$ passe par un état accepteur à partir du mot vide:
      $$\alert{\mathcal{L}_E(M) = \left\{G\cdot D \in \Sigma^\star \;\middle\mid\; \exists q\in F,~ C_{\mathit{init}}(\varepsilon) \leadsto_M^\star \langle G, q, D \rangle \right\}}$$
    \item \structure{$L$ est récursivement énumérable} s'il existe une MTD qui énumère $L$
    \item On note \alert{\textsc{re}} la classe des langages récursivement énumérables
    \end{itemize}
  }

  \onExampleBlock[y=-10mm]{Exemple -- Énumération de $\{a^n b^n \mid n\in \mathbb{N}\}$}{}
  
  \on[x=-20mm,y=-22mm]{
    \begin{tikzpicture}[tape, x=7mm, y=7mm]
      \cell{}                              \smheadfromb<7>{5} 
      \cell[alert ob=<6>]{\alt<-5>{}{$a$}} \smheadfromb<5>{3} 
      \cell[alert ob=<4>]{\alt<-3>{}{$a$}} \smheadfromb<3>{1} 
      \cell[alert ob=<2>]{\alt<-1>{}{$a$}} \smhead     <1>   
      \cell[alert ob=<3>]{\alt<-2>{}{$b$}} \smheadb    <2> 
      \cell[alert ob=<5>]{\alt<-4>{}{$b$}} \smheadfromb<4>{-2} 
      \cell[alert ob=<7>]{\alt<-6>{}{$b$}} \smheadfromb<6>{-4}   
      \cell{}                                                  
    \end{tikzpicture}
  }

  \on[x=-20mm, bottom=5mm]{
    $\mathcal{L}_E(M) = \{\varepsilon\uncover<3->{, ab}\uncover<5->{, aabb}\uncover<7->{, aaabbb, ...}\}$
  }

  \on[x=35mm, bottom=2mm]{
    \begin{tikzpicture}[turingMachine]
      \state[alert ob=<{1,3,5,7}>, initial, accepting] (0) at (0,1) {0}; 
      \state[alert ob=<{2,4,6}>                      ] (1) at (0,0) {1}; 
      
      \path (0) edge[bend left]  node {\smTMtransR{\blank}{a}}                       (1);
      \path (1) edge[bend left]  node {\smTMtransL{\blank}{b}}                       (0);
      \path (0) edge[loop right]  node {\smGroup{\smTMtransL{a}{a}\smTMtransL{b}{b}}} (0);
      \path (1) edge[loop right] node {\smGroup{\smTMtransR{a}{a}\smTMtransR{b}{b}}} (1);
    \end{tikzpicture}
  }

\end{frame}

\endgroup
