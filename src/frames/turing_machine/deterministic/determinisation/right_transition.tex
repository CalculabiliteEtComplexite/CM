% SPDX-License-Identifier: CC-BY-SA-4.0
% Author: Matthieu Perrin
% Part: <Nom de la partie>
% Section: <Nom de la section>
% Sub-section: <Nom de la sous-section>  % (facultatif, laisser vide si non utilisé)
% Frame: <Titre de la slide>

\begingroup

\begin{frame}{Simulation de la transition $t = \langle \langle q, a \rangle, \langle q', a', \triangleright \rangle \rangle$}

  \onBlock[top]{Transition avec déplacement vers la droite}{
    \begin{itemize}
    \item Vérifier si la transition peut être appliquée sur $c$
      \begin{itemize}
      \item Si oui, ajouter la nouvelle configuration à la fin de $f$ 
      \item Sinon, aller directement à l'état $f_t$
      \end{itemize}
    \item Si l'état $q'$ est accepteur dans $M$, bloquer dans l'état accepteur $2_t$ 
      \begin{itemize}
      \item On suppose un unique état accepteur $q'$ sans transition sortante
      \end{itemize}
    \end{itemize}
  }

  \on[x=-25mm, y=-20mm]{
    \begin{tikzpicture}[tape, x=5mm, y=5mm]
      \node{$c$ :};
      \cell                  {$\blank$}
      \cell                  {$g_1$}
      \cell                  {$g_2$}
      \cell[structure on=<1>]{$q$}
      \cell[structure on=<2>]{$a$}
      \cell                  {$d_1$}
      \cell                  {$d_2$}
      \cell                  {$\blank$}
 
      \fill<1> [alert] (3.0,0) +(.1,-.3) -- +(.25,0) -- +(.4,-.3);
      
      \fill<2> [alert] (3.5,0) +(.1,-.3) -- +(.25,0) -- +(.4,-.3);
      
      \fill<3>[alert!20] (3.0,0) +(.1,-.3) -- +(.25,0) -- +(.4,-.3);
      \draw<3>[alert, -latex] (3.0,-.15) -- (2.0, -.15);
      \fill<3> [alert] (1.5,0) +(.1,-.3) -- +(.25,0) -- +(.4,-.3);
      
      \fill<4>[alert!20] (2.0,0) +(.1,-.3) -- +(.25,0) -- +(.4,-.3);
      \draw<4>[alert, -latex] (2.5,-.15) -- (3.0, -.15);
      \fill<4> [alert] (3.0,0) +(.1,-.3) -- +(.25,0) -- +(.4,-.3);
      
      \fill<5> [alert] (3.5,0) +(.1,-.3) -- +(.25,0) -- +(.4,-.3);
      
      \fill<6>[alert!20] (4.0,0) +(.1,-.3) -- +(.25,0) -- +(.4,-.3);
      \draw<6>[alert, -latex] (4.5,-.15) -- (5.0, -.15);
      \fill<6> [alert] (5.0,0) +(.1,-.3) -- +(.25,0) -- +(.4,-.3);
      
      \fill<7>[alert!20] (4.5,0) +(.1,-.3) -- +(.25,0) -- +(.4,-.3);
      \draw<7>[alert, -latex] (4.5,-.15) -- (3.5, -.15);
      \fill<7> [alert] (3.0,0) +(.1,-.3) -- +(.25,0) -- +(.4,-.3);
 
    \end{tikzpicture}
  }
 
  \on[x=-25mm, y=-30mm]{
    \begin{tikzpicture}[tape, x=5mm, y=5mm]
      \draw (0.5,0.25) node{$f$ :}; 
      \cell{example on=<4>,\alt<-3>{$\sharp$}{$|$}}
      \cell{example on=<4>,\alt<-3>{$\sharp$}{$g_1$}}
      \cell{example on=<4>,\alt<-3>{$\sharp$}{$g_2$}}
      \cell{example on=<5>,\alt<-4>{$\sharp$}{$a'$}}
      \cell{example on=<6>,\alt<-5>{$\sharp$}{$q'$}}
      \cell{example on=<6>,\alt<-5>{$\sharp$}{$d_1$}}
      \cell{example on=<6>,\alt<-5>{$\sharp$}{$d_2$}}
      \cell{$\sharp$}
 
      \fill<-3> [alert] (1.5,0) +(.1,-.3) -- +(.25,0) -- +(.4,-.3);
 
      \fill<4>[alert!20] (2.0,0) +(.1,-.3) -- +(.25,0) -- +(.4,-.3);
      \draw<4>[alert, -latex] (2.5,-.15) -- (3.0, -.15);
      \fill<4> [alert] (3.0,0) +(.1,-.3) -- +(.25,0) -- +(.4,-.3);
 
      \fill<5> [alert] (3.5,0) +(.1,-.3) -- +(.25,0) -- +(.4,-.3);
 
      \fill<6>[alert!20] (4.0,0) +(.1,-.3) -- +(.25,0) -- +(.4,-.3);
      \draw<6>[alert, -latex] (4.5,-.15) -- (5.0, -.15);
      \fill<6-> [alert] (5.0,0) +(.1,-.3) -- +(.25,0) -- +(.4,-.3);
 
    \end{tikzpicture}
  }

  \on[y=-20mm]{
    \begin{tikzpicture}[turingMachine, x=15mm]

%      \draw [rounded corners, fill=structure!10] (0,-1.5) rectangle (14.4,1.5); 
%      \draw[structure] (0, 1.5) node[below right] { $\langle \langle q, a \rangle, \langle q', a', \triangleright \rangle \rangle$}; 
% 
%      \draw[structure] (14.4, -1.5) node[above left]{$\forall x\in \Gamma$};
      
      \state[structure, alert on=<1>, initial] (0) at (0,0) {$0_t$}; 
      \state[structure, alert on=<2>]          (1) at (1,0) {$1_t$}; 
      \state[structure, alert on=<3>]          (2) at (2,0) {$2_t$}; 
      \state[structure, alert on=<4>]          (3) at (3,0) {$3_t$}; 
      \state[structure, alert on=<5>]          (4) at (4,0) {$4_t$}; 
      \state[structure, alert on=<6>]          (5) at (5,0) {$5_t$}; 
      \state[structure, alert on=<7>]          (6) at (6,0) {$f_t$}; 

      \begin{scope}[font=\tiny]
        \path (0) edge                node {\smTMtransR[c]{q}{q}}                                               (1);
        \path (1) edge                node {\smTMtransL[c]{a}{a}}                                               (2);
        \path (2) edge[loop above]    node {\smTMtransL[c]{x}{x}}                                               (2);
        \path (2) edge                node {\smGroup{\smTMtransR[f]{\sharp}{|} \smTMtransR[c]{\blank}{\blank}}} (3);
        \path (3) edge[loop above]    node {\smGroup{\smTMtransR[f]{\sharp}{x} \smTMtransR[c]{x}{x}}}           (3);
        \path (3) edge                node {\smGroup{\smTMtransR[f]{\sharp}{a'}\smTMtransR[c]{q}{q}}}           (4);
        \path (4) edge                node {\smGroup{\smTMtransR[f]{\sharp}{q'}\smTMtransR[c]{a}{a}}}           (5);
        \path (5) edge[loop above]    node {\smGroup{\smTMtransR[f]{\sharp}{x} \smTMtransR[c]{x}{x}}}           (5);
        \path (0) edge[bend right=17] node {\smTMtransR[c]{q'}{q'}, $\forall q' \neq q$}                        (6);
        \path (1) edge[bend right=12] node {\smTMtransL[c]{y}{y}, $\forall y \neq a$}                           (6);
        \path (5) edge                node {\smTMtransL[c]{\blank}{\blank}}                                     (6);
        \path (6) edge[loop, out=150, in=120] node[above left] {\smTMtransL[c]{x}{x}}                           (6);
      \end{scope}
      
    \end{tikzpicture}
  }

\end{frame}

\endgroup
