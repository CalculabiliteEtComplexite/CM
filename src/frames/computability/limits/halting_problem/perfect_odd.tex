% SPDX-License-Identifier: CC-BY-SA-4.0
% Author: Matthieu Perrin
% Part: <Nom de la partie>
% Section: <Nom de la section>
% Sub-section: <Nom de la sous-section>  % (facultatif, laisser vide si non utilisé)
% Frame: <Titre de la slide>

\begingroup

\SetKwData{X}{x}
\SetKwData{I}{i}
\SetKwFunction{searchPerfect}{search\_perfect}
\SetKwFunction{perfect}{perfect}
\SetKwData{S}{somme}

\begin{frame}{Un exemple plus compliqué}
  
  \onExampleBlock[top=-2mm,left=.66\textwidth]{Ces instances sont-elles positives ?}{
    \begin{itemize}
    \item $\mathit{algo} = \searchPerfect$
    \item $x = 2$ puis $x = 3$
    \end{itemize}
    \vspace{1mm}
    \begin{algorithm}[H]
      \Proc{$\searchPerfect(\X \in \mathbb{N})$}{
        \lWhile{$\lnot \perfect(\X)$}{
          $\X \leftarrow \X+2$%
        }
      }
      \Fun{$\perfect(\X \in \mathbb{N}) \in \mathbb{B}$}{
        $\S\leftarrow 0$\;
        \For{$\I$ \From $1$ \To $\X-1$}{
          \If{$\X \!\!\mod \I = 0$}{$\S \leftarrow \S + \I$}
        }
        \Return $\S = \X$\;
      }
    \end{algorithm}
  }
  
  \onBlock<2-|handout>[top=-2mm,right=.33\textwidth]{Nombre parfait}{
    Nombre égal à la somme de ses diviseurs propres.
  }

  \onExampleBlock<2-|handout>[right=.33\textwidth,y=2mm]{Exemples}{
    \begin{description}[45 :]
    \item[$4$ :]  $1 + 2 = 3$
    \item[$6$ :]  $1 + 2 + 3 = 6$
    \item[$45$ :] $1 + 3 + 5 + 9 + 15 = 33$
    \end{description}
  }

  \onAlertBlock<3-|handout>[bottom=-9mm]{Conjecture -- Il n'existe pas de nombre parfait impair}{
    \myquote[alert]{Euler}{
      Il reste la question de savoir si un nombre parfait impair peut exister ; c'est une question assurément très difficile, jamais résolue ni par les Anciens ni par les Modernes.
    }
  }

\end{frame}

\endgroup
