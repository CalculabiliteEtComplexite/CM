% SPDX-License-Identifier: CC-BY-SA-4.0
% Author: Matthieu Perrin
% Part: <Nom de la partie>
% Section: <Nom de la section>
% Sub-section: <Nom de la sous-section>  % (facultatif, laisser vide si non utilisé)
% Frame: <Titre de la slide>

\begingroup

\begin{frame}{Classes de calculabilité}
  \small 
  Soient $\Sigma$ un alphabet, et $L_A$ et $L_B$ deux langages de $\Sigma^\star$. 
  \begin{block}{Propriété : $\leq_M$ est un préordre sur $\mathscr{P}(\Sigma^\star)$}
      \begin{itemize}
      \item $\leq_M$ est réflexive
      \item $\leq_M$ est transitive
      \item $\leq_M$ n'est ni symétrique ni antisymétrique. 
      \end{itemize}
      $L_A$ et $L_B$ sont \structure{équivalents par mappage}, noté \alert{$L_A \equiv_m L_B$}, si $L_A \leq_m L_B$ et $L_B \leq_m L_A$.
  \end{block}
  
  \begin{exampleblock}{Notion de préordre}
    \begin{itemize}
    \item Un \example{préordre} est une relation $\sqsubseteq$ réflexive et transitive. 
      \begin{itemize}
      \item On définit $\equiv$ par $x \equiv y$ si $x \sqsubseteq y$ et $y \sqsubseteq x$
      \begin{itemize}
      \item $\equiv$ est une relation d'équivalence
      \item $\sqsubseteq$ est une relation d'ordre sur les classes d'équivalence de $\equiv$
      \end{itemize}
      \end{itemize}
    \item Par exemple, sur $\{a, b\}^\star$, $u \sqsubseteq v$ si toutes les lettres de $u$ sont dans $v$. 
      \begin{itemize}
      \item $u \equiv v$ si $u$ et $v$ utilisent les mêmes lettres. 
      \item Quatre classes d'équivalence : \example{$\{\varepsilon\}$}, \example{$\mathcal{L}(a^\star)$}, \example{$\mathcal{L}(b^\star)$} et \example{$\mathcal{L}(\Sigma^\star (ab|ba) \Sigma^\star)$}
      \end{itemize}
    \end{itemize}

  \end{exampleblock}
\end{frame}
\endgroup
