% SPDX-License-Identifier: CC-BY-SA-4.0
% Author: Matthieu Perrin
% Part: <Nom de la partie>
% Section: <Nom de la section>
% Sub-section: <Nom de la sous-section>  % (facultatif, laisser vide si non utilisé)
% Frame: <Titre de la slide>

\begingroup

\begin{frame}{Généralisation : le théorème de Rice}
  ~
  
 \vspace{-9mm}
  \begin{tikzpicture}
  \hspace{-3mm}

    \draw (5,5) node[above]{\begin{minipage}{1.1\textwidth}
        \begin{block}{Théorème -- Théorème de Rice}
          Toute propriété sémantique non triviale \\d'un programme est indécidable.

          \vspace{2mm}
          \begin{description}
            \item[\og Sémantique \fg : ] prédicat sur les exécutions
            \item[\og Non-triviale \fg : ] vraie sur certains programmes, \\\hspace{5mm}fausse sur d'autres
          \end{description}
        \end{block}  
    \end{minipage}};

    \draw (5,5) node[below]{\begin{minipage}{1.1\textwidth}
        \begin{exampleblock}{Exemples de propriétés indécidables}
          \begin{itemize}
          \item Le programme s'arrête sur une certaine entrée
          \item Le programme s'arrête sur toutes ses entrées
          \item Le programme retourne le même résultat qu'un autre programme
          \item Le programme retourne un résultat correct par rapport à sa spécification
          \item Le programme ne déréférence pas le pointeur nul
          \item Le programme ne lève pas d'exception
          \item Le programme est un virus
          \end{itemize}
        \end{exampleblock}  
    \end{minipage}};

%    \draw (8.5,6) node{\includegraphics[width=2.5cm]{Rice}};
%    \draw[structure] (8.5,4.4) node{Gordon Henry Rice};
  \end{tikzpicture}
\end{frame}

\endgroup
