% SPDX-License-Identifier: CC-BY-SA-4.0
% Author: Matthieu Perrin
% Part: <Nom de la partie>
% Section: <Nom de la section>
% Sub-section: <Nom de la sous-section>  % (facultatif, laisser vide si non utilisé)
% Frame: <Titre de la slide>

\begingroup

\SetKwFunction{Decide}{decide}

\begin{frame}{Notion de réduction}
  
  \begin{tikzpicture}
    \draw[white] (5,9) -- (5,1);
    \draw (5,8.9) node{\begin{minipage}{\textwidth}
        Soient $\Sigma_A$ et $\Sigma_B$ deux alphabets, et $L_A \subseteq \Sigma_A^\star$ et $L_B \subseteq \Sigma_B^\star$ deux langages.
    \end{minipage}};

    \draw (5,5) node[above]{\begin{minipage}{\textwidth}
        \begin{block}{Définition -- Réduction de Mappage}
          \begin{itemize}
          \item Une \structure{réduction de $L_A$ vers $L_B$} est une fonction
            $f : \Sigma_A^\star \rightarrow \Sigma_B^\star$ telle que
            \begin{itemize}
            \item $f$ est calculable ;
            \item $\forall u \in \Sigma_A^\star, u\in L_A \Leftrightarrow f(u)\in L_B$.
            \end{itemize}
          \item On note $\structure{L_A \leq_m L_B}$, s'il existe une réduction de $L_A$ vers $L_B$. 
            \begin{itemize}
            \item Lire \structure{\og $L_A$ se réduit à $L_B$ \fg}.
            \item Si $L_A \leq_m L_B$, le problème $u \in L_A$ est \og plus facile que \fg $u \in L_B$.
            \end{itemize}
          \end{itemize}
        \end{block}
    \end{minipage}};

    \draw<3> (5,5) node[below]{\begin{minipage}{\textwidth}
        \begin{block}{Théorème -- Preuve d'indécidabilité par réduction}
          Si $L_A \leq_m L_B$ et $L_A$ est indécidable, alors $L_B$ est indécidable.

          \vspace{2mm}
          \structure{Démonstration : }
          \begin{enumerate}
          \item Supposons (par l'absurde) que $L_B$ est décidable par $\structure{\Decide_{L_B}}$.
          \item Alors $\structure{u \rightarrow \Decide_{L_B}(f(u))}$ décide $L_A$. 
          \item Absurde, car $L_A$ est indécidable. Donc $L_B$ est indécidable. 
          \end{enumerate}
        \end{block}
    \end{minipage}};

    \draw<1-2> (5,5) node[below]{\begin{minipage}{\textwidth}
        \begin{exampleblock}{Exemple -- Problème de l'arrêt sur chaîne vide}
          \begin{itemize}
          \item La fonction $\mathit{initialiseur}$ est une réduction de $L_{\mathit{halt}}$ vers $L_{\mathit{empty\_halt}}$ :
          \begin{itemize}
          \item $f$ est calculable ;
          \item $\langle M, u \rangle \in L_{\mathit{halt}} \Leftrightarrow \mathit{initialiseur}(M, u)\in L_\mathit{empty\_halt}$.
          \end{itemize}
          \item Donc $\alert{L_{\mathit{halt}}\leq_m L_{\mathit{empty\_halt}}}$
          \item<2> On en a déduit l'indécidabilité de $L_{\mathit{empty\_halt}}$ de celle de $L_{\mathit{halt}}$. 
          \end{itemize}
        \end{exampleblock}
    \end{minipage}};

  \end{tikzpicture}
\end{frame}
\endgroup
