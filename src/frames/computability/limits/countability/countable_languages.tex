% SPDX-License-Identifier: CC-BY-SA-4.0
% Author: Matthieu Perrin
% Part: <Nom de la partie>
% Section: <Nom de la section>
% Sub-section: <Nom de la sous-section>  % (facultatif, laisser vide si non utilisé)
% Frame: <Titre de la slide>

\begingroup


\begin{frame}{Dénombrabilité des langages}

  Soient $\Sigma$ un alphabet et $f$ une bijection de $\Sigma$ dans $\{1, ..., |\Sigma|\}$.

  \begin{block}{Propriété -- Dénombrabilité de chaque langage}
    Tout langage $L \subseteq \Sigma^\star$ est dénombrable.
  \end{block}

  \vspace{-1mm}
  \begin{block}{Démonstration}
    On interprète les mots de $L$ comme des écritures en base $|\Sigma| + 1$.
    \begin{itemize}
    \item La fonction suivante est injective :
      \vspace{-2mm}
      $$
      g = \left\{\begin{array}{ccc}
      L &\rightarrow& \mathbb{N}\\
      u_{n-1}...u_{0} &\mapsto&  \displaystyle \sum_{i=0}^{n-1} f(u_i) \times (|\Sigma| + 1)^{i}  \\ 
      \end{array}\right.
      $$
    \item Exemples : 
    \begin{itemize}
    \item Si $\Sigma = \{1, ..., 9\}$ et $f : x \mapsto x$, on a $\example{g(12345) = 12345}$
    \item Si $\Sigma = \{a, b\}$, avec $f(a) = 1$ et $f(b) = 2$, on a $\example{g(abab) = 50}$
      \begin{itemize}
      \item car $50 = 2\times 3^3 + 1\times 3^2 + 2\times 3^1 + 1\times 3^0$
      \end{itemize}
    \item Si $|\Sigma| = 1$, on a $\example{g(u) = 2^{|u| -1}}$
    \end{itemize}
    \end{itemize}
  \end{block}
\end{frame}
\endgroup
