% SPDX-License-Identifier: CC-BY-SA-4.0
% Author: Matthieu Perrin
% Part: <Nom de la partie>
% Section: <Nom de la section>
% Sub-section: <Nom de la sous-section>  % (facultatif, laisser vide si non utilisé)
% Frame: <Titre de la slide>

\begingroup

\begin{frame}{Dénombrabilité}
  Soient $E$ un ensemble et $n\in \mathbb{N}$.
  
  \begin{block}{Définition -- Ensemble fini}
    \begin{itemize}
    \item On dit que $E$ est \structure{de cardinal $n$} s'il est équipotent à $\{1, ..., n\}$.
      \begin{itemize}
      \item \example{Exemple : $\{a, b, c\}$ est cardinal $3$}.
      \end{itemize}
    \item On dit que $E$ est \structure{fini} s'il existe $m\in \mathbb{N}$ tel que $E$ est de cardinal $m$.
      \begin{itemize}
      \item \example{Exemple : $\{a, b, c\}$ est cardinal $3$}.
      \end{itemize}
    \end{itemize}
  \end{block}

  \begin{block}{Définition -- Ensemble dénombrable}
    \begin{itemize}
    \item On dit que $E$ est \structure{infini dénombrable} s'il est équipotent à $\mathbb{N}$.
      \begin{itemize}
      \item \example{Exemples : $\mathbb{N}$, $\mathbb{N}\setminus \{0\}$, $\mathbb{Z}$ et $\mathbb{Q}$ sont infinis dénombrables}.
      \end{itemize}
    \item On dit que $E$ est \structure{dénombrable} s'il est fini ou infini dénombrable.
      \begin{itemize}
      \item S'il existe une fonction injective de $E$ dans $\mathbb{N}$, alors $E$ est dénombrable.
      \end{itemize}
    \item On dit que $E$ est \structure{indénombrable} s'il n'est pas dénombrable.
      \begin{itemize}
      \item S'il existe une fonction injective de $\mathbb{R}$ dans $E$, alors $E$ est indénombrable.
      \item \example{Exemples : $\mathbb{R}$, $[0, 1[$ et $\mathbb{C}$ sont indénombrables}.
      \end{itemize}
    \end{itemize}
  \end{block}
\end{frame}

\endgroup
