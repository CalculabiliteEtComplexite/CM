% SPDX-License-Identifier: CC-BY-SA-4.0
% Author: Matthieu Perrin
% Part: <Nom de la partie>
% Section: <Nom de la section>
% Sub-section: <Nom de la sous-section>  % (facultatif, laisser vide si non utilisé)
% Frame: <Titre de la slide>

\begingroup

\begin{frame}{Classes de décidabilité}
  \begin{block}{Rappel}
    On a deux notions de décidabilité par les machines de Turing :
    \begin{description}
    \item[$\textsc{r}$] est la classe des langages \alert{décidables} ;
    \item[$\textsc{re}$] est la classe des langages \alert{semi-décidables}.
    \end{description}
  \end{block}
  
  \begin{block}{Théorème -- Inclusion des classes de calculabilité}
    On a : \alert{$$\textsc{r} \subsetneq \textsc{re} \subsetneq \mathscr{P}(\Sigma^\star).$$}
  \end{block}
  
  \begin{block}{Démonstration}
    \begin{enumerate}
    \item<2-> \structure{$\textsc{r} \subseteq \textsc{re} \subseteq \mathscr{P}(\Sigma^\star)$ :} déjà connu.
    \item<3-> \structure{$\textsc{re} \neq \mathscr{P}(\Sigma^\star)$ :} preuve par dénombrement.
    \item<4-> \structure{$\textsc{r} \neq \textsc{re}$ :} preuve par exemple (le problème de l'arrêt).
    \end{enumerate}
  \end{block}
\end{frame}
\endgroup
