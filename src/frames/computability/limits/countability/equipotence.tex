% SPDX-License-Identifier: CC-BY-SA-4.0
% Author: Matthieu Perrin
% Part: <Nom de la partie>
% Section: <Nom de la section>
% Sub-section: <Nom de la sous-section>  % (facultatif, laisser vide si non utilisé)
% Frame: <Titre de la slide>

\begingroup


\begin{frame}{Rappel : Equipotence}

  \small
  
  \begin{block}{Définition -- Equipotence}
    Soient $E$ et $F$ deux ensembles.
    \begin{itemize}
    \item On dit que $E$ et $F$ sont \structure{équipotents} s'il existe une \alert{bijection entre $E$ et $F$}. 
    \item De manière équivalente, $E$ et $F$ sont équipotents s'il existe
    \begin{itemize}
    \item une fonction injective de $E$ dans $F$, et 
    \item une fonction injective de $F$ dans $E$
    \end{itemize}
    \item On dit de deux ensembles équipotents qu'ils ont \alert{\og le même cardinal \fg}.
    \item L'équipotence est \structure{réflexive}, \structure{symétrique} et \structure{transitive}.
    \begin{itemize}
    \item L'équipotence définit une \structure{relation d'équivalence} sur $\mathcal{P}(E)$.
    \end{itemize}
    \end{itemize}
  \end{block}

  \begin{exampleblock}{Exemple : $\{a, b, c\}$ et $\{1, 2, 3\}$ sont équipotents}
    \vspace{-3mm}
    $$
    \left\{\begin{array}{ccl}
    \{1, 2, 3\} &\rightarrow& \{a, b, c\}\\
    1 &\mapsto& a\\ 
    2 &\mapsto& b\\ 
    3 &\mapsto& c\\ 
    \end{array}\right.
    $$
  \end{exampleblock}
\end{frame}

\endgroup
