% SPDX-License-Identifier: CC-BY-SA-4.0
% Author: Matthieu Perrin
% Part: <Nom de la partie>
% Section: <Nom de la section>
% Sub-section: <Nom de la sous-section>  % (facultatif, laisser vide si non utilisé)
% Frame: <Titre de la slide>

\begingroup

\begin{frame}{Problème de Correspondance de Post Initialisé}

  \on[text,top]{
    Soit $\Sigma$ un alphabet contenant au moins deux symboles. 
    
    \begin{block}{Le problème PCPI}
      \begin{description}
      \item[Entrées :]
        \begin{itemize}
        \item Un ensemble fini de \structure{sortes de dominos} $D \in \mathscr{P}(\Sigma^+ \times \Sigma^+)$.
        \item Un \structure{domino initial} $\VDomino{\gamma}{\delta} \in D$
        \end{itemize}
      \item[Solution :] Un mot non-vide
        $\VDomino{\alpha_1}{\beta_1} \cdots \VDomino{\alpha_k}{\beta_k} \in D^+$ de dominos tel que :
        \begin{itemize}
        \item $\alpha_{1} \cdots \alpha_{k} = \beta_{1} \cdots \beta_{k}$ et $\VDomino{\alpha_1}{\beta_1} = \VDomino{\gamma}{\delta}$
        \end{itemize}
      \item[Sortie :] \alert{\og oui \fg} s'il existe une solution, \alert{\og non \fg} sinon.
      \end{description}
    \end{block}
  }
  
  \onExampleBlock<2-|handout> [bottom=2mm] {Exemple} {
    \begin{itemize}
    \item $D=\left\{
      \VDomino{a}{ab}, 
      \VDomino{baba}{a}, 
      \VDomino{bb}{ba}
      \right\}, d_0 = \VDomino{a}{ab}$\\
      \uncover<3-|handout> {a pour solution
        \VDomino{a}{ab}\, 
        \VDomino{bb}{ba}\,
        \VDomino{a}{ab}\,
        \VDomino{a}{ab}\,
        \VDomino{baba}{a}.}
    \end{itemize}
  }

  
\end{frame}
\endgroup
