% SPDX-License-Identifier: CC-BY-SA-4.0
% Author: Matthieu Perrin
% Part: <Nom de la partie>
% Section: <Nom de la section>
% Sub-section: <Nom de la sous-section>  % (facultatif, laisser vide si non utilisé)
% Frame: <Titre de la slide>

\begingroup

\begin{frame}{Démonstration de l'indécidabilité de PCP}
  \newcommand\sep{\bullet}
  \small
    \begin{tikzpicture}

      \draw (8, 3.6) node{\begin{minipage}{\textwidth}
          Réduction du problème de correspondance de Post initialisé.
          \begin{itemize}
          \item Soit $\structure{\left\langle D, \HDomino[1.1]{\gamma_1\cdots \gamma_{|\gamma|}}{\rightarrow_1\cdots \rightarrow_{|\rightarrow|}} \right\rangle}$ une instance du PCPI sur $\Sigma$
          \item<1,7->  $\left\langle D, \HDomino{\gamma}{\rightarrow} \right\rangle$ a une solution ssi l'instance $\alert{D'}$ du PCP sur $\alert{\Sigma'}$ a une solution :

            \uncover<7->{
              $$\begin{array}{rcl}
                \Sigma' &=&\Sigma\cup \{\sep\}
                \vspace{1mm}\\
                D' &=&
                \left\{\VDomino[2]{~\alert{\sep}~\gamma_1\sep ...\sep\gamma_{|\gamma|}~~~~}{\alert{\sep}~~\rightarrow_1\sep ... \sep\rightarrow_{|\rightarrow|}~~\alert{\sep}}\right\}
                \vspace{1mm}\\
                &\cup&
                \left\{ \VDomino[2]{\alert{\sep}~~\alpha_1 \sep ... \sep\alpha_{|\alpha|}~~~~}{~~~~\beta_1\sep ... \sep\beta_{|\beta|}~~\alert{\sep}} \,\middle|\,
                \VDomino[1.1]{\alpha_1 ... \alpha_{|\alpha|}}{\beta_1 ... \beta_{|\beta|}} \in D
                \right\}
                \vspace{1mm}\\
                &\cup&
                \left\{
                \VDomino{\alert{\sep~\sep}}{\phantom{\sep}~\alert{\sep}}
                \right\}
              \end{array}
              $$
            }
          \item<1,8-> \vspace{-1mm}Où la réduction $\left\langle D, \HDomino{\gamma}{\rightarrow} \right\rangle \mapsto D'$  est calculable.
          \item<1,8-> Or le problème de correspondance de Post initialisé est indécidable.
          \item<1,8-> Donc le problème de correspondance de Post est indécidable.
          \end{itemize}
      \end{minipage}};

      \draw<1> (8, 3.6) node{$\begin{array}{ccc}\Sigma' &=& ???\\ D' &=& ???\end{array}$};

      
      \draw<2-6> (8, 3) node{\begin{minipage}{\textwidth}
          \begin{exampleblock}{Exemple : $\color{black}\left\langle D=\left\{
              \HDomino{a}{baa},
              \HDomino{ab}{aa},
              \HDomino{bba}{bb}
              \right\},
              d_0 = \HDomino{bba}{bb}
              \right\rangle$}
            \begin{enumerate}
            \item<3-> Nouveau caractère $\sep$ à gauche pour forcer l'initialisation avec $d_0$
            \item<4-> Intercaller $\sep$ entre toutes les lettres pour permettre le découpage
            \item<5-> Ajouter un domino final pour terminer par $\sep\sep$
            \end{enumerate}

            \uncover<3-> {
            $$D' = \left\{
            \only<-3>{
              \VDomino[1.1]{\alert{\sep} b b a}{\alert{\sep} b b},
              \VDomino[1.1]{\alert{\sep} a}{b a a},
              \VDomino[1.1]{\alert{\sep} a b}{a a},
              \VDomino[1.1]{\alert{\sep} b b a}{b b}
            }
            \only<4>{
              \VDomino[1.1]{\sep b \alert{\sep} b \alert{\sep} a} {\sep b \alert{\sep} b \,\alert{\sep}},
              \VDomino[1.1]{\sep a}                               {b \alert{\sep} a \alert{\sep} a \,\alert{\sep}},
              \VDomino[1.1]{\sep a \alert{\sep} b}                {a \alert{\sep} a \,\alert{\sep}},
              \VDomino[1.1]{\sep b \alert{\sep} b \alert{\sep} a} {b \alert{\sep} b \,\alert{\sep}}
            }
            \only<5->{
              \VDomino[1.1]{\sep b \sep b\sep a}{\sep b\sep b\,\sep},
              \VDomino[1.1]{\sep a}{b\sep a\sep a\,\sep},
              \VDomino[1.1]{\sep a\sep b}{a\sep a\,\sep },
              \VDomino[1.1]{\sep b\sep b\sep a}{b\sep b\,\sep }
            }
            \uncover<5->{, \VDomino{\alert<5>{\sep\sep}}{\alert<5>{\sep}}}
            \right\}$$
            }

            \uncover<6-> {
            \begin{tabular}{ll}
              \example{Solution pour $D'$ :}&  \scalebox{.9}{
                $\VDomino[1.1]{\sep b \sep b\sep a}{\sep b\sep b\sep}\,
                \VDomino[1.1]{\sep a\sep b}{a\sep a\sep}\,
                \VDomino[1.1]{\sep b\sep b\sep a}{b\sep b\sep}\,
                \VDomino[1.1]{\sep a}{b\sep a\sep a\sep}\,
                \VDomino{\sep\sep}{\sep}$
              }
            \vspace{1mm}\\
            \example{Solution pour $D$ :}&
            \scalebox{.9}{
              $\VDomino[1.1]{b b a}{ b b}\,
              \VDomino[1.1]{ab}{a a}\,
              \VDomino[1.1]{bba}{b b}\,
              \VDomino[1.1]{a}{baa}$
            }
            \end{tabular}
            }

          \end{exampleblock}
      \end{minipage}};
    \end{tikzpicture}
\end{frame}

\endgroup
