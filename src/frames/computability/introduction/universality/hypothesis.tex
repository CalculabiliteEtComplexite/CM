% SPDX-License-Identifier: CC-BY-SA-4.0
% Author: Matthieu Perrin
% Part: <Nom de la partie>
% Section: <Nom de la section>
% Sub-section: <Nom de la sous-section>  % (facultatif, laisser vide si non utilisé)
% Frame: <Titre de la slide>

\begingroup

\begin{frame}{Hypothèses pour la suite}

  \begin{block}{Dans toute la suite, on supposera :}
    \begin{itemize}
    \item L'alphabet considéré sera toujours $\Sigma = \{0,1\}$
      \begin{itemize}
      \item On peut toujours encoder n'importe quel alphabet dans $\{0,1\}^\star$
      \end{itemize}
    \item L'encodage des MT sera un langage sur $\{0,1\}$ : $$\forall M, \mathit{encode}(M) \in \{0,1\}^\star$$ 
    \item La machine universelle $M_{univ}$ :
      \begin{itemize}
      \item sera déterministe
      \item pourra simuler des machines des machines de Turing non-déterministes
      \item utilise (au moins) deux rubans : 
        \begin{itemize}
        \item \textsc{m} initialisé à $\mathit{encode}(M)$ 
        \item \textsc{u} initialisé à $u$ 
        \end{itemize}
      \end{itemize}
    \end{itemize}
  \end{block}

\end{frame}


\endgroup
