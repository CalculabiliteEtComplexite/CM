% SPDX-License-Identifier: CC-BY-SA-4.0
% Author: Matthieu Perrin
% Part: <Nom de la partie>
% Section: <Nom de la section>
% Sub-section: <Nom de la sous-section>  % (facultatif, laisser vide si non utilisé)
% Frame: <Titre de la slide>

\begingroup

\begin{frame}{Machine de Turing universelle}
  \small
  
  \begin{block}{Remarque}
    \begin{itemize}
    \item Il existe un algorithme qui simule toutes les MTD.
    \item Tout algorithme peut être transformé en une MTD.
    \item Il existe une MTD qui simule toutes les MTD.
    \end{itemize}
  \end{block}

  \pause

  \begin{block}{Théorème -- Machine de Turing universelle}
    Soit $\Sigma$ un alphabet.
    Il existe une MTD $M_{univ}$ qui prend en entrée :
    \begin{itemize}
    \item un encodage $\mathit{encode}(M)$ d'une MTD $M$,
    \item un mot $u \in \Sigma^\star$,
    \end{itemize}
    et telle que :
    \begin{itemize}
    \item \structure{$M_{univ}$ termine sur $\langle \mathit{encode}(M), u \rangle$} si, et seulement si, \structure{$M$ termine sur $u$};
    \item \structure{$M_{univ}$ accepte $\langle \mathit{encode}(M), u \rangle$} si, et seulement si, \structure{$M$ accepte $u$}. 
    \end{itemize}

    \structure{Conséquence : } \alert{$L_{univ} \eqdef \{\langle \mathit{encode}(M), u \rangle | M \text{ décide } u \}$ est semi-décidable}. 
  \end{block}
  
\end{frame}


\endgroup
