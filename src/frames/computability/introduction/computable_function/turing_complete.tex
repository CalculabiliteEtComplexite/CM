% SPDX-License-Identifier: CC-BY-SA-4.0
% Author: Matthieu Perrin
% Part: <Nom de la partie>
% Section: <Nom de la section>
% Sub-section: <Nom de la sous-section>  % (facultatif, laisser vide si non utilisé)
% Frame: <Titre de la slide>

\begingroup

\begin{frame}{Turing--complet}

  \begin{block}{Rappel -- Thèse de Church--Turing}
    \begin{center}
      \structure{La définition des \og \alert{fonctions calculables} \fg par des \\
        Machines de Turing déterministes  \\
        caractérise la notion intuitive de \og \alert{procédure effective} \fg.}
    \end{center}
  \end{block}

  \begin{block}{Définition -- Complet au sens de Turing}
    Un système formel est dit \structure{Turing-complet} s'il peut \og simuler \fg toute MTD.
  \end{block}

  \begin{exampleblock}{Exemples}
    \begin{itemize}
    \item Le $\lambda$-calcul est équivalent aux machines de Turing
    \item Les langages de programmation généralistes
    \item Minecraft, le jeu de la vie...
    \end{itemize}
  \end{exampleblock}

  \begin{alertblock}{Contre-exemples}
    \begin{itemize}
    \item Automates finis et à pile, HTML, CSS, SQL non récursif, ...
    \end{itemize}
  \end{alertblock}
  
\end{frame}

\endgroup
