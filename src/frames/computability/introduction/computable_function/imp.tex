% SPDX-License-Identifier: CC-BY-SA-4.0
% Author: Matthieu Perrin
% Part: <Nom de la partie>
% Section: <Nom de la section>
% Sub-section: <Nom de la sous-section>  % (facultatif, laisser vide si non utilisé)
% Frame: <Titre de la slide>

\begingroup

\begin{frame}{Le langage IMP}
  \small
  \DontPrintSemicolon
  On se donne :
  \begin{itemize}
  \item Un alphabet $\Sigma$ pour écrire les identifiants de variables
  \item Un alphabet $\Gamma$ pour encoder les valeurs, contenant $T$ (vrai) et $F$ (faux)
  \item Un ensemble $\{ f_1, ..., f_n \}$ de fonctions calculables d'arité $\{ a_1, ..., a_n \}$. 
  \begin{itemize}
  \item $f_i : (\Sigma^\star)^n \rightarrow \Sigma^\star$
  \end{itemize}
  \end{itemize}
  $$
  \left\{\begin{array}{rcl}
  P &\rightarrow& ID \leftarrow \Gamma^+ \\
  & | & ID \leftarrow f_i(ID_1, ..., ID_{a_i}) \\
  & | & P; P \\
  & | & \leSi{ID}{~P~}{P} \\
  & | & \lTantque{ID}{P} \\
  \end{array}\right.
  $$
 
  \begin{block}{Remarque}
    On peut ajouter : 
    \begin{itemize}
    \item d'autres structures de contrôle (boucle pour, etc.)
    \item les fonctions (nécessité d'ajouter une pile)
    \end{itemize}
  \end{block}
 
\end{frame}
\endgroup
