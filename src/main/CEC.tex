% SPDX-License-Identifier: CC-BY-SA-4.0
% Author: Matthieu Perrin

\ifdefined\HANDOUT
  \documentclass[9pt, handout]{beamer}
  \usetheme{Nantes}
\else
  \documentclass[9pt]{beamer}
  \usetheme[sectionpage]{Nantes}
\fi

\usepackage{src/sty/config}

\title[Calculabilité et Complexité]{Calculabilité et Complexité}

\author[Matthieu Perrin]{
  Matthieu Perrin\\
  Laboratoire des Sciences du Numérique de Nantes \\
  UMR CNRS 6004\\
  Bureau 410, bâtiment 34 \\
  \url{matthieu.perrin@univ-nantes.fr}\\
}

\date{
  Nantes Université\\
  Licence d'Informatique, troisième année\\
  2025-2026
}

\hypersetup{
  pdftitle  = {\insertshorttitle},
  pdfauthor = {\insertshortauthor},
  pdfsubject  = {Cours de L3 sur les limites de l'algorithmique en calculabilité et en complexité},
  pdfkeywords = {machines de Turing, décidabilité, semi-décidabilité, calculabilité, complexité, P versus NP}
}

\begin{document}

\begin{frame}[plain,noframenumbering]
  \titlepage
  \on[text, bottom=-5mm]{\scriptsize
    \begin{description}
    \item[\href{https://github.com/CalculabiliteEtComplexite/.github/blob/main/LICENSE.md}{Licence :}] \href{https://creativecommons.org/licenses/by-sa/4.0/}{CC BY-SA 4.0} (\ccbysa{}) — Matthieu Perrin \textit{(sauf mention contraire)}
    \item[Code source :]\vspace{-.5mm}  \url{https://github.com/CalculabiliteEtComplexite}
    \end{description}
  }
\end{frame}


\part{Langages et problèmes de décision}
 
 
\section{Organisation du module}
 
\subsection{Généralités}
% SPDX-License-Identifier: CC-BY-SA-4.0
% Author: Matthieu Perrin
% Part: <Nom de la partie>
% Section: <Nom de la section>
% Sub-section: <Nom de la sous-section>  % (facultatif, laisser vide si non utilisé)
% Frame: <Titre de la slide>

\begingroup

\begin{frame}{Généralités}

  \begin{block}{Prérequis}
    \begin{description}[Mathématiques : ]
    \item[Mathématiques : ] logique, théorie des ensembles
    \item[Informatique : ] algorithmique, théorie des langages formels
    \end{description}
  \end{block}

  \begin{block}{Objectifs du cours}
    \begin{itemize}
    \item Comprendre les \structure{fondements de la théorie de la calculabilité}
    \item Modéliser des algorithmes simples à l’aide de \structure{machines de Turing}
    \item Comprendre et manipuler les classes de complexité \structure{P} et \structure{NP}
    \item Appliquer des \structure{techniques de réduction} de calculabilité et de complexité
    \end{itemize}
  \end{block}

  \begin{block}{Bibliographie}
    \begin{itemize}
    \item P. Wolper. \structure{\emph{Introduction à la calculabilité}} -- 3e Ed., Dunod 2006
    \item O. Carton. \structure{\emph{Langages Formels -- Calculabilité et complexité}}, Vuibert 2008
    \end{itemize}
  \end{block}
  
\end{frame}

\endgroup

 
\section{De la crise des fondements à l'informatique moderne}
 
\subsection{Contexte historique}
% SPDX-License-Identifier: CC-BY-SA-4.0
% Author: Matthieu Perrin
% Part: 
% Section: 
% Sub-section: 
% Frame: 

\begingroup

\begin{frame}{Langage engendré par une grammaire non-restreinte}
  Soient \alert{$\langle \Sigma, \Gamma, S, \rightarrow \rangle$} une grammaire non-restreinte, $u, v \in (\Sigma \cup \Gamma)^\star$ et $w \in \Sigma^\star$.

  \begin{block}{Définitions -- Dérivation}

    \begin{itemize}
    \item On dit que \structure{$u$ se dérive directement en $v$}, noté $\alert{u \vdash v}$, si :

      \vspace{-3mm}
      $$\alert{\exists x, y, \alpha, \beta \in (\Sigma \cup \Gamma)^\star,\quad
        \example{\alpha \rightarrow \beta} \quad\land\quad u = \structure{x} \cdot \example{\alpha} \cdot \structure{y} \quad\land\quad v = \structure{x} \cdot \example{\beta} \cdot \structure{y}}$$

    \item On dit que \structure{$u$ se dérive en $v$} si \alert{$u \vdash^\star v$},\\
      où \alert{$\vdash^\star$ est la fermeture transitive et réflexive de $\vdash$}.

    \end{itemize}
  \end{block}

  \begin{block}{Définitions -- Génération et langage engendré}
    \begin{itemize}
    \item Une \structure{génération de $w$ par $G$} est une dérivation \alert{$S \vdash^\star w$} à partir de $S$.
    \item On dit que \structure{$w$ est généré par $G$} s'il existe une génération de $w$ par $G$. 
    \item Le \structure{langage engendré} par $G$, $\alert{\mathcal{L}(G)}$, contient les mots générés par $G$ :
      $$\alert{\mathcal{L}(G) = \left\{w\in\Sigma^\star \,\middle\mid\, S \vdash^\star w \right\}}.$$
    \end{itemize}
  \end{block}

\end{frame}

\endgroup

% SPDX-License-Identifier: CC-BY-SA-4.0
% Author: Matthieu Perrin
% Part: 
% Section: 
% Sub-section: 
% Frame: 

\begingroup

\begin{frame}{Cadre formel et conventions\footnotemark[1]}

  \vspace{-2mm}
  \begin{block}{Univers des symboles}
    On fixe un ensemble \alert{$\Omega$} \structure{infini dénombrable}.
    \begin{itemize}
    \item Contient symboles de mots, états, et autres marqueurs techniques
    \item On note \alert{$\Omega_f$} l'ensemble des sous-ensembles \structure{finis} de $\Omega$
    \end{itemize}
  \end{block}

  \vspace{-1mm}
  \begin{block}{Alphabets, mots et langages}
    \begin{itemize}
    \item Un \structure{alphabet} est un ensemble \alert{$\Sigma \in \Omega_f$}
    \item Un \structure{mot} sur $\Sigma$ est une suite finie de symbole de $\Sigma$
    \item L'\structure{ensemble des mots} sur $\Sigma$ est noté \alert{$\Sigma^\star$}
    \item L'\structure{ensemble des langages} sur $\Sigma$ est noté \alert{$\textsc{lang}_\Sigma \eqdef \mathscr{P}\left(\Sigma^\star\right)$}
    \item On définit la \structure{classe\footnotemark[2] des mots} comme \alert{$\Omega^\star \eqdef \bigcup_{\Sigma \in \Omega_f} \Sigma^\star$}
    \item On définit la \structure{classe des langages} comme \alert{$\textsc{lang} \eqdef \bigcup_{\Sigma \in \Omega_f} \textsc{lang}_\Sigma$}
    \item Pour un langage $L\in \textsc{lang}$, on définit son alphabet \alert{$\Sigma(L)$} comme l'ensemble des symboles qu'il utilise :
      $\alert{\Sigma(L) \eqdef \bigcap \{\Sigma \in \Omega_f \mid L\subseteq \Sigma^\star\}}$
    \end{itemize}
  \end{block}

  \footnotetext[1]{Voir le cours de \structure{Langages et Automates} (\url{https://github.com/LangagesEtAutomates})}
  \footnotetext[2]{Dans ce cours, on considère que \structure{ensemble} et \structure{classe} sont synonymes.}
  
\end{frame}

\endgroup
\endinput


% SPDX-License-Identifier: CC-BY-SA-4.0
% Author: Matthieu Perrin
% Part: 
% Section: 
% Sub-section: 
% Frame: 

\begingroup

\begin{frame}{L'homme vit dans un univers symbolique\footnotemark[1]}

  \footnotetext[1]{Ernst Cassirer. \emph{Essai sur l'Homme}. 1944}

  \on[top]{
    \myquote{Michael Tomasello}{
      La caractéristique la plus distinctive de la cognition humaine est la capacité à créer et utiliser des représentations symboliques partagées.
    }
  }

  \onBlock[y=13mm]{L'invention de l'écriture et du calcul (vers -3300)}{}
  
  \on[left=.45\textwidth, bottom=7mm]{
    \myquote{Denise Schmandt-Besserat}{
      Les premières formes d’écriture se sont développées à partir de systèmes de comptabilité.
    }
    
    \myquote{Jack Goody}{
      Dès lors que l’écriture existe, la pensée n’est plus limitée par la mémoire humaine.
    }
  }

  \onImage[x=.3\textwidth,bottom=12mm]{%
    height=3cm,
    title={Décompte de malt et d'orge},
    licenselogo={\ccPublicDomain},
    license={Domaine public (Uruk, Irak, vers -3000). Source : \href{https://commons.wikimedia.org/wiki/File:Cuneiform_tablet-_administrative_account_with_entries_concerning_malt_and_barley_groats_MET_DP293245.jpg}{Metropolitan Museum of Art}},
    img={Tablet.jpg}
  }
  
\end{frame}

\endgroup
\endinput


% SPDX-License-Identifier: CC-BY-SA-4.0
% Author: Matthieu Perrin
% Part: 
% Section: 
% Sub-section: 
% Frame: 

\begingroup

\begin{frame}{Mathématiques de l'âge du bronze}
 
  \onBlock[left=.7\textwidth, top=-2mm]{Mathématiques mésopotamiennes}{
    Approche procédurale de calculs avancés
    \begin{itemize}
    \item Multiplication, inverse, racine...
    \item Triplets pythagoriciens
    \item Numération en base 60
    \end{itemize}
  }
 
  \onBlock<2->[right=.6\textwidth, bottom=6mm]{Géométrie égyptienne}{
    Procédures systématiques de construction
    \begin{itemize}
    \item Problèmes de géométrie pratique résolus
    \item Calcul d’aires et de volumes
    \item Numération en base 10
    \item Procédures pour approcher l'aire du cercle par celui d'un carré
    \end{itemize}
  }
 
  \onImage[x=.3\textwidth,top]{%
    height=25mm,
    title={Tablette Plimpton 322},
    licenselogo={\ccPublicDomain},
    license*={Domaine public (Irak, vers -1700 : Photo : \href{https://www.sciencesetavenir.fr/archeo-paleo/archeologie/la-trigonometrie-version-babylonienne_116888}{Andrew Kelly})},
    img={plimpton.jpg}
  }
 
  \onImage<2->[x=-.3\textwidth,bottom=5mm]{%
    height=3.5cm,
    title={Papyrus Rhind},
    licenselogo={\ccPublicDomain},
    license*={Domaine public (Égypte, vers -1650. Photo : \href{https://old.maa.org/press/periodicals/convergence/mathematical-treasure-the-rhind-and-moscow-mathematical-papyri}{Frank J. Swetz})},
    img={rhind.png}
  }
  
\end{frame}

\endgroup
\endinput


% SPDX-License-Identifier: CC-BY-SA-4.0
% Author: Matthieu Perrin
% Part: 
% Section: 
% Sub-section: 
% Frame: 

\begingroup
 
\begin{frame}{Mathématiques grecques}
 
  \onBlock[right=.75\textwidth, top=-2mm]{Logique aristotélicienne}{
    Fondements du raisonnement déductif
    \begin{itemize}
    \item Logique formelle et syllogismes : $\alert{(A \land (A \Rightarrow B)) \Rightarrow B}$
    \item Raisonnement à partir de principes
      \begin{itemize}
      \item connaissance préalable de \structure{$A$} et \structure{$A\Rightarrow B$}
      \end{itemize}
    \end{itemize}
  }
 
  \onBlock<2->[left=.6\textwidth, bottom=6mm]{Géométrie euclidienne}{
    Mathématiques axiomatiques
    \begin{itemize}
    \item Preuves générales et abstraites
    \begin{itemize}
    \item Définitions, axiomes, théorèmes
    \end{itemize}
    \item Approche géométrique de l'algèbre
      \begin{itemize}
      \item Tout est points, lignes, et cercles
      \end{itemize}
    \item Problème de la quadrature du cercle
      \begin{itemize}
      \item Construire un carré de l'aire d'un cercle ?
      \end{itemize}
    \end{itemize}
  }
 
  \onImage[x=-.4\textwidth,top]{%
    height=25mm,
    title={Buste d'Aristote},
    licenselogo={\ccbysa},
    license*={{\href{https://creativecommons.org/licenses/by-sa/2.0/}{CC BY-SA-2.0}} -- Marie-Lan Nguyen, 2010 (\href{https://commons.wikimedia.org/wiki/File:Buste_Aristote_Bibliotheque_Mazarine_Paris.jpg}{Wikimedia})},
    img={Aristote.jpg}
  }
 
  \onImage<2->[x=.3\textwidth,bottom=10mm]{%
    height=27mm,
    title={Les Éléments de Géométrie},
    licenselogo={\ccPublicDomain},
    license*={Domaine public (Oxyrhynque, Égypte, -125 à -75. \href{https://commons.wikimedia.org/wiki/File:P._Oxy._I_29.jpg}{Wikimedia})},
    img={Elements.jpg}
  }
  
\end{frame}

\endgroup

% SPDX-License-Identifier: CC-BY-SA-4.0
% Author: Matthieu Perrin
% Part: 
% Section: 
% Sub-section: 
% Frame: 

\begingroup

\begin{frame}{Des fondements qui se fissurent}

  \onBlock[left=.6\textwidth, top=-3mm]{Coordonnées cartésiennes}{
    Approche algébrique de la géométrie
    \begin{itemize}
    \item Rupture avec la géométrie constructive
    \item Les figures ne sont plus le centre
      \begin{itemize}
      \item Nombres, tuples, ensembles
      \end{itemize}
    \item Pas d'axiomatisation formelle
    \end{itemize}
  }

  \onBlock<2->[right=.65\textwidth, bottom=5mm]{Limites de l'algèbre géométrique}{
    Construction à la règle et au compas : 
    \begin{itemize}
    \item Caractérisation des nombres \structure{constructibles}
      \begin{itemize}
      \item Théorème de Wantzel (1837)
      \end{itemize}
    \item Le nombre $\pi$ est \structure{transcendant}
      \begin{itemize}
      \item Théorème de Lindemann–Weierstrass (1882)
      \end{itemize}
    \item Donc la quadrature du cercle est impossible
    \end{itemize}
  }

  \onImage[x=.33\textwidth,top]{%
    height=2.8cm,
    title={René Descartes},
    licenselogo={\ccPublicDomain},
    license*={Domaine public (Frans Hals. \emph{Portret van René Descartes}. 1649. \href{https://commons.wikimedia.org/wiki/File:Frans_Hals_-_Portret_van_Ren\%C3\%A9_Descartes.jpg}{Wikimedia})},
    img={Descartes.jpg}
  }

  \onImage<2->[x=-.33\textwidth,bottom=5mm]{%
    height=2.8cm,
    title={Ferdinand von Lindemann},
    licenselogo={\ccPublicDomain},
    license*={Domaine public (1852-1939. \href{https://commons.wikimedia.org/wiki/File:Carl_Louis_Ferdinand_von_Lindemann.jpg}{Wikimedia})},
    img={Lindemann.jpg}
  }
  
\end{frame}

\endgroup

 
\subsection{La crise des fondements}
% SPDX-License-Identifier: CC-BY-SA-4.0
% Author: Matthieu Perrin
% Part: 
% Section: 
% Sub-section: 
% Frame: 

\begingroup

\begin{frame}{Vers une théorie ensembliste de l'arithmétique}

  \onBlock[top=-4mm, left=.62\textwidth]{Théorie (naïve) des ensembles (Frege)}{      
    \begin{itemize}
    \item Tout objet mathématique est un ensemble 
    \item Ensemble = prédicat logique
      \begin{itemize}
      \item Par exemple, $\{1, 2\} = \{x \mid \alert{x=1 \lor x=2}\}$
      \end{itemize}
    \end{itemize}
    \myquote{Gottlob Frege}{Les lois de l'arithmétique doivent être déduites des lois de la logique}
  }

  \onBlock<2->[bottom=6mm, right=.7\textwidth]{L'infini comme objet mathématique (Cantor)}{
    \begin{itemize}
    \item Étude systématique des ensembles infinis
    \item Théorème de Cantor :
      \begin{itemize}
      \item $\forall E, | \mathscr{P}(E) | > | E |$
      \item En particulier, $| \mathbb{R} | > | \mathbb{N} |$
      \item Argument de diagonalisation
      \end{itemize}
    \item Infini non unique : \structure{hiérarchie des infinis}
    \end{itemize}
  }
  
  \onImage[x=.37\textwidth,top]{%
    height=28mm,
    title={Gottlob Frege},
    licenselogo={\ccPublicDomain},
    license*={Domaine public (vers 1879. \href{https://commons.wikimedia.org/wiki/File:Young_frege.jpg}{Wikimedia})},
    img={Frege.jpg}
  }

  \onImage<2->[x=-.37\textwidth,bottom=7mm]{%
    height=28mm,
    title={Georg Cantor},
    licenselogo={\ccPublicDomain},
    license*={Domaine public (vers 1900. \href{https://commons.wikimedia.org/wiki/File:Georg_Cantor2.jpg}{Wikimedia})},
    img={Cantor.jpg}
  }
  
\end{frame}

\endgroup

% SPDX-License-Identifier: CC-BY-SA-4.0
% Author: Matthieu Perrin
% Part: 
% Section: 
% Sub-section: 
% Frame: 

\begingroup

\begin{frame}{Le paradoxe du barbier}

  \on[y=-5mm]{
    \begin{tikzpicture}
      \node [faded background picture=Village,    text width=\paperwidth/2] (A) at (-\paperwidth/4,0) {};
      \node [faded background picture=Barbershop, text width=\paperwidth/2] (B) at ( \paperwidth/4,0) {};

      \node[anchor=south, outer sep=0pt, inner sep=0pt] at (A.south) {\includegraphics[height=30mm]{shaving}};
      \node[anchor=south, outer sep=0pt, inner sep=0pt] at (B.south) {\includegraphics[height=30mm]{barber}};
    \end{tikzpicture}
  }

  \on[x=-.25\paperwidth, y=-5mm]{
    \chatBubble[color=example]{Je me rase moi-même}
  }

  \on[x=.25\paperwidth, y=-5mm]{
    \chatBubble[color=alert]{Je rase ceux qui ne se rasent pas}
  }

  \on[y=22mm]{
    \begin{tikzpicture}
      \node[
        every chat bubble,
        draw=structure,
        top color=structure!10,
        bottom color=structure!30,
      ]{
        \begin{minipage}{.95\textwidth}
        Dans un petit village suisse, vivait un barbier.

        Tous les villageois étaient bien rasés.
        \vspace{-2mm}
        \uncover<2->{%
          $$\forall x,~ x\ \structure{\text{est~rasé~par}}\  \mathit{le~barbier} \Leftrightarrow \lnot (x\ \structure{\text{est~rasé~par}}\ x)$$%
        }%
        \vspace{-6mm}%
        \uncover<3->{%
          \begin{center}%
            \emph{\Large\structure{Qui rase le barbier ?}}%
          \end{center}%
        }%
        \vspace{2mm}
        \end{minipage}%
      };
    \end{tikzpicture}
  }
  
\end{frame}

\endgroup

% SPDX-License-Identifier: CC-BY-SA-4.0
% Author: Matthieu Perrin
% Part: 
% Section: 
% Sub-section: 
% Frame: 

\begingroup

\begin{frame}{Le paradoxe de Russell}

  \onBlock[top=-2mm]{Conséquence : un tel barbier n'existe pas}{
    On généralise à n'importe quelle \structure{relation homogène} $\bowtie$ :
    $$\structure{\nexists y, \forall x, x \bowtie y \Leftrightarrow \lnot (x \bowtie x)}$$
    
    Dans le paradoxe du barbier, on a :
    \begin{itemize}
    \item $y$ est le barbier
    \item $\bowtie$ est la relation ``est rasé par''
    \end{itemize}
  }  

  \onAlertBlock<2->[bottom=3mm]{L'argument de Russell}{
    \begin{itemize}
    \item Application à l'appartenance ensembliste $\in$ :
      \begin{itemize}
      \item $\nexists y = \{ x \mid x \notin x \}, \quad \forall x, x \in y \Leftrightarrow x \notin x$
      \end{itemize}
    \item Définition des ensembles par compréhension :
      \begin{itemize}
      \item $\exists y = \{ x \mid x \notin x \}, \quad \forall x, x \in y \Leftrightarrow x \notin x$
      \end{itemize}
    \end{itemize}
    \begin{center}
      \alert{La théorie naïve des ensembles est \emph{incohérente}}
    \end{center}
  }  

  \onImage<2->[x=.35\textwidth, bottom=10mm]{%
    height=4cm,
    title={Bertrand Russell},
    licenselogo={\ccPublicDomain},
    license={Domaine public (Bassano Ltd, 1936. \href{https://commons.wikimedia.org/wiki/File:Bertrand_Russell_photo.jpg?uselang=fr}{Wikimedia})},
    img={Russell.jpg}
  }
  
\end{frame}

\endgroup
\endinput

% SPDX-License-Identifier: CC-BY-SA-4.0
% Author: Matthieu Perrin
% Part: 
% Section: 
% Sub-section: 
% Frame: 

\begingroup

\begin{frame}{Cohérence et complétude d'un système de preuves}

  \begin{block}{Système de preuves}
    Un système de preuves est un triplet \structure{$S = \langle \Phi, \Pi, \valid \rangle$}, constitué de :
    \begin{description}[Complétude :]
    \item[$\Phi\in\textsc{lang}$ :] un langage de \structure{propositions} \hspace{\fill} \example{exemple : formules logiques}
      \begin{itemize}
      \item On suppose que $\Phi$ possède un opérateur de négation $\lnot$
      \end{itemize}
      \item[$\Pi\in\textsc{lang}$ :] un langage de \structure{preuves} \hspace{\fill} \example{exemple : arbres d'inférence}
      \item[$\valid \subseteq \Pi \times \Phi$ :] une relation entre les preuves et les propositions
      \begin{itemize}
      \item ``\structure{$\pi \valid \varphi$}'' signifie ``$\pi$ est une preuve valide de $\varphi$''
      \end{itemize}
    \end{description}
    Une proposition $\varphi \in \Phi$ est dite \structure{démontrable dans $S$} si \alert{$\exists \pi\in \Pi, \pi\valid\varphi$}.
  \end{block}

  \pause
  \begin{block}{Deux propriétés fondamentales}
    Un système de preuves $S=\langle \Phi,\Pi,\valid\rangle$ est dit :
    \begin{description}[Complétude :]
    \item[Complet] si pour toute proposition $\varphi \in \Phi$, \structure{$\varphi$ ou $\lnot \varphi$ est démontrable}
      \begin{itemize}
      \item sinon, \alert{certaines propositions restent indécidables}
      \end{itemize}
    \item[Cohérent] si pour toute proposition $\varphi \in \Phi$, \structure{$\varphi$ ou $\lnot \varphi$ est indémontrable}
      \begin{itemize}
      \item sinon, \alert{toute proposition est démontrable} : $(\varphi\land \lnot \varphi) \Rightarrow \psi$
      \item donc le système logique devient inutile
      \end{itemize}
    \end{description}
  \end{block}

\end{frame}

\endgroup

% SPDX-License-Identifier: CC-BY-SA-4.0
% Author: Matthieu Perrin
% Part: 
% Section: 
% Sub-section: 
% Frame: 

\begingroup

\begin{frame}{La crise des fondements}

  \on[top]{
    Sur quoi repose la légitimité des objets mathématiques ?
  }

  \onBlock[bottom=-3mm]{Une crise existentielle : réalistes\footnote{\structure{Réalisme :} Le monde a une existence \emph{ontologique}, c'est-à-dire indépendante de notre représentation} contre idéalistes\footnote{\structure{Idéalisme :} La réalité n'existe qu'à travers des représentations de l'esprit.}}{

    Des positions radicalement opposées se dégagent :

    \begin{description}
    \item[Platonicisme :] existence ontologique des objets mathématiques.
    \end{description}
    \myquote{Charles Hermite}{
      Il existe (...) un monde tout entier, qui est la totalité des vérités mathématiques, (...) comme il existe un monde de réalités physiques.
    }
    
    \vspace{-7mm}
    \begin{description}
    \item[Finitisme :] seuls les objets finis ont une existence légitime.
    \end{description}
    \myquote{Leopold Kronecker}{
      Dieu a créé les entiers, tout le reste est \oe uvre humaine.
    }

    \vspace{-7mm}
    \begin{description}
    \item[Formalisme :] les objets existent dans un cadre formel, s'il est cohérent.
    \end{description}
    \myquote{David Hilbert}{
      Les mathématiques sont un jeu joué selon certaines règles simples avec des symboles dénués de signification.
    }
  }

\end{frame}

\endgroup
\endinput

% SPDX-License-Identifier: CC-BY-SA-4.0
% Author: Matthieu Perrin
% Part: <Nom de la partie>
% Section: <Nom de la section>
% Sub-section: <Nom de la sous-section>  % (facultatif, laisser vide si non utilisé)
% Frame: <Titre de la slide>

\begingroup

\begin{frame}{Assurer les fondements des mathématiques}
  
  \onBlock[left=.7\textwidth, top=9mm]{Le programme de Hilbert (d'après Frege)}{
    \begin{itemize}
    \item Formaliser rigoureusement les mathématiques
    \item Garantir la \structure{cohérence} des théories formelles
    \item Idéalement, assurer leur \structure{complétude}
    \item Rendre les preuves \structure{mécaniquement vérifiables}
    \end{itemize}
  }

  \on[left=.6\textwidth, top=-1mm]{
    \myquote{David Hilbert\footnote{En réponse à Emil du Bois-Reymond : ``Nous ne savons pas, nous ne saurons pas''}}{
      Nous devons savoir, nous saurons.
    }
  }
  
  \onAlertBlock[bottom=7mm]{Der Entscheidungsproblem (Le Problème de la Décision)}{
    Existe-t-il une procédure mécanique permettant de déterminer, pour toute formule logique donnée, si celle-ci est logiquement valide ?
    \begin{itemize}
    \item Qu'est-ce qu'un \og \alert{problème de décision} \fg, en général ?
    \item Qu'est-ce qu'une \og \alert{procédure mécanique} \fg ?
    \end{itemize}
  }

  \onImage[x=40mm,top]{%
    width=2.5cm,
    title={David Hilbert},
    licenselogo={\ccPublicDomain},
    license={Domaine public (Göttingen, 1912, \href{https://commons.wikimedia.org/wiki/File:Hilbert.jpg}{Wikimedia})},
    img={Hilbert.jpg}
  }
  
\end{frame}

\endgroup

 
\subsection{Le problème de la décision}
% SPDX-License-Identifier: CC-BY-SA-4.0
% Author: Matthieu Perrin
% Part: <Nom de la partie>
% Section: <Nom de la section>
% Sub-section: <Nom de la sous-section>  % (facultatif, laisser vide si non utilisé)
% Frame: <Titre de la slide>

\begingroup

\begin{frame}{Problème binaire et problème de décision}

  Un \structure{problème binaire} est une question dont la réponse est \alert{oui} ou \alert{non} en fonction de ses \structure{entrées}.

  \begin{exampleblock}{Exemples}
    \begin{itemize}
    \item \example{Primalité :} un entier est-il premier ?  
    \item \example{Classification :} une image donnée contient-elle un chat ? 
    \item \example{Base de données :} une requête donnée retourne-t-elle des résultats ?  
    \item \example{Sudoku :} une grille donnée peut-elle être remplie ? 
    \item \example{SAT :} une formule en forme normale conjonctive est-elle satisfiable ? 
    \item \example{Preuve :} un énoncé de théorème est-il démontrable dans une théorie ? 
    \item \example{Décision :} étant donnés un mot $u$ et un langage $L$, a-t-on $u\in L$ ? 
    \end{itemize}
  \end{exampleblock}
  
  \begin{alertblock}{Observations}
    \begin{itemize}
    \item En informatique, les entrées sont toujours encodables par des mots
    \item \alert{Les problèmes binaires sont encodés par des langages}
    \end{itemize}
  \end{alertblock}
  
\end{frame}

\endgroup
\endinput

% SPDX-License-Identifier: CC-BY-SA-4.0
% Author: Matthieu Perrin
% Part: <Nom de la partie>
% Section: <Nom de la section>
% Sub-section: <Nom de la sous-section>  % (facultatif, laisser vide si non utilisé)
% Frame: <Titre de la slide>

\begingroup

\newcommand\PROBLEM{\textsc{problem}}

\begin{frame}{Description informelle d'un problème de décision}

  \on[top, text]{
    Un problème de décision \PROBLEM{} est décrit par :
    \begin{description}
    \item[Instance :] une entrée possible de \PROBLEM
    \item[Question :] une question fermée (oui/non) sur les instances
      \begin{itemize}
      \item Une instance est \alert{positive} si la réponse est ``\structure{oui}''
      \item Une instance est \alert{négative} si la réponse est ``\structure{non}''
      \end{itemize}
    \end{description}
  }

  \onExampleBlock[left=.5\textwidth, y=10mm, anchor=north]{Exemple : langage}{
    Soit $L = a^+ b^+$
    
    \Probleme{\example{Decision$_L$}}{
      Un mot $u \in \{a, b\}^\star$.
    }{
      Est-ce que $u \in L$ ?
    }
    
    \begin{itemize}
    \item Instances positives : \example{$aab, abb$}
    \item Instances négatives : \example{$bab, a$}
    \end{itemize}
  }

  \onExampleBlock[right=.5\textwidth, y=10mm, anchor=north]{Exemple : test de primalité}{
    ~
    
    \Probleme{\example{Primalité}}{
      Un entier $n \in \mathbb{N}$.
    }{
      $n$ est-il premier ?
    }
    
    \begin{itemize}
    \item Instances positives : \example{$2, 7, 13$}
    \item Instances négatives : \example{$1, 9, 14$}
    \end{itemize}
  }

\end{frame}

\endgroup
\endinput

% SPDX-License-Identifier: CC-BY-SA-4.0
% Author: Matthieu Perrin
% Part: <Nom de la partie>
% Section: <Nom de la section>
% Sub-section: <Nom de la sous-section>  % (facultatif, laisser vide si non utilisé)
% Frame: <Titre de la slide>

\begingroup

\newcommand\PROBLEM{\textsc{problem}}

\begin{frame}{Problème de décision}

  \begin{block}{Définition -- Problème de décision formel}
    Un \structure{problème de décision formel} est un langage \alert{$\PROBLEM{} \in \textsc{lang}$}. 
  \end{block}

  \begin{block}{Définition -- Problème de décision (à domaine) contraint}
    Un \structure{problème de décision contraint} est un couple \alert{$\PROBLEM{} = \langle I, P \rangle$}, où :
    \begin{description}
    \item[$I \in \textsc{lang}$] est le langage des \structure{instances} de \PROBLEM{}
    \item[$P \subseteq I$] est le langage des \structure{instances positives} de \PROBLEM{}
    \end{description}
  \end{block}
  
  \begin{block}{Notations}
    \begin{tikzpicture}[2Darray, x=25mm, y=6mm]
      \arrayColumn[header=\structure{Notation}]{
        \arrayLine{\structure{$\Sigma(\PROBLEM)$}}
        \arrayLine{\structure{$\mathcal{I}(\PROBLEM)$}}
        \arrayLine{\structure{$\textsc{pos}(\PROBLEM)$}}
        \arrayLine{\structure{$\textsc{neg}(\PROBLEM)$}}
      }
      \arrayColumn[header=\structure{$L \in \textsc{lang}$}]{
        \arrayLine{$\Sigma(L)$}
        \arrayLine{$\Sigma(L)^\star$}
        \arrayLine{$L$}
        \arrayLine{$\Sigma(L)^\star \setminus L$}
      }
      \arrayColumn[header=\structure{$\langle I, P\rangle \in \textsc{lang}^2$}]{
        \arrayLine{$\Sigma(I)$}
        \arrayLine{$I$}
        \arrayLine{$P$}
        \arrayLine{$I \setminus P$}
      }
      \arrayColumn[width=30mm, header=\structure{Notion}]{
        \arrayLine{Alphabet}
        \arrayLine{Instances}
        \arrayLine{Instances positives}
        \arrayLine{Instances négatives}
      }
    \end{tikzpicture}
  \end{block}

\end{frame}

\endgroup
\endinput

% SPDX-License-Identifier: CC-BY-SA-4.0
% Author: Matthieu Perrin
% Part: <Nom de la partie>
% Section: <Nom de la section>
% Sub-section: <Nom de la sous-section>  % (facultatif, laisser vide si non utilisé)
% Frame: <Titre de la slide>

\begingroup

\begin{frame}{Comment décrire les problèmes de décision ?}

  \begin{alertblock}{Remarque}
    \begin{itemize}
    \item Un problème doit être décrit dans un formalisme centré sur un langage
    \end{itemize}
  \end{alertblock}
  
  \begin{exampleblock}{Exemples : langage composé des seuls mots $aa$ et $ab$}
    \begin{description}[Grammaires algébriques :]
    \item[Mathématiques :]
      $\{aa, ab\}$
    \item[Expressions rationnelles :]
      $a (a\mid b)$
    \item[Grammaires algébriques :]
      $\{S \rightarrow aa \mid ab\}$
    \item[Langage C++ :]
      \lstinline0bool decide(string s) \{return s=="aa" || s=="ab";\}0
    \end{description}
  \end{exampleblock}

  \begin{block}{Définition -- Langage de descriptions}
    Un \structure{langage de descriptions} $\langle \mathcal{L}, \llbracket \cdot \rrbracket \rangle$ est un langage muni d'une sémantique :
    \begin{description}[$\llbracket \cdot \rrbracket : \mathcal{L} \rightarrow \textsc{lang}$]
    \item[$\mathcal{L} \in \textsc{lang}$] est la \alert{syntaxe} du langage de descriptions
    \item[$\llbracket \cdot \rrbracket : \mathcal{L} \rightarrow \textsc{lang}$] est la \alert{sémantique} du langage de descriptions
    \end{description}
    \begin{itemize}
    \item Un mot \structure{$u \in \mathcal{L}$} est la représentation du problème formel \structure{$\llbracket u \rrbracket \in \textsc{lang}$}.
    \end{itemize}
  \end{block}
  
\end{frame}

\endgroup
\endinput

% SPDX-License-Identifier: CC-BY-SA-4.0
% Author: Matthieu Perrin
% Part: <Nom de la partie>
% Section: <Nom de la section>
% Sub-section: <Nom de la sous-section>  % (facultatif, laisser vide si non utilisé)
% Frame: <Titre de la slide>

\begingroup

\tikzset{
  language/.style={
    rounded corners,
    fill=structure!10,
    draw=structure,
    text=structure,
    align=center,
  },
  word/.style={
    {Circle[length=2.4pt,width=2.4pt]}-{Stealth[length=5pt]},
    shorten <=-1.2pt,
    auto,
  },
  noword/.style={
    Rays-{Stealth[length=5pt]},
    shorten <=-1.2pt,
    auto,
  },
}

\begin{frame}{Mots, langages et descriptions}

  \on[top]{
    \begin{tikzpicture}\scriptsize

      \node[language, text width=105mm, minimum height=70mm, fill=structure!5] (all) at (0,0) {};
      \node[structure, anchor=north west] at (all.north west) {$\Omega^\star$};

      \uncover<2->{
        \node[language, text width=50mm, minimum height=60mm, anchor=south west] (reg) at ($(all.south west)+(5mm,5mm)$) {};
        \node[structure, anchor=north west] at (reg.north west) {Expressions rationnelles};
      }
      
      \node[language, text width=30mm, minimum height=50mm, anchor=south east, fill=structure!15] (ab) at ($(reg.south east)+(-5mm,5mm)$) {};
      \node[structure, anchor=north west] at (ab.north west) {$\{a, b\}^\star$};
      
      \node[language, text width=25mm, minimum height=10mm, anchor=north] (anbnan) at ($(ab.north)+(0mm, -5mm)$) {};
      \node[structure, anchor=north west] at (anbnan.north west) {$\{a^n b^n a^n \mid n>0\}$};

      \node[language, text width=25mm, minimum height=5mm, anchor=north] (ba) at ($(ab)+(0mm, 7.5mm)$) {};
      \node[structure, anchor=north west] at (ba.north west) {$\{ba\}$};
      
      \node[language, text width=25mm, minimum height=20mm, anchor=south] (asbs) at ($(ab.south)+(0mm, 5mm)$) {};
      \node[structure, anchor=south west] at (asbs.south west) {$\{a^m b^n \mid m, n>0\}$};

      \node[language, text width=20mm, minimum height=10mm, anchor=south, fill=structure!15] (anbn) at ($(asbs.south)+(0mm, 5mm)$) {};
      \node[structure, anchor=south west] at (anbn.south west) {$\{a^n b^n \mid n>0\}$};

      \uncover<3->{
        \node[language, text width=40mm, minimum height=27.5mm, anchor=north east] (gram) at ($(all.north east)+(-5mm,-5mm)$) {};
        \node[structure, anchor=north east] at (gram.north east) {Grammaires algébriques};
      }

      \uncover<4->{
        \node[language, text width=40mm, minimum height=27.5mm, anchor=south east] (cpp) at ($(all.south east)+(-5mm,5mm)$) {};
        \node[structure, anchor=south east] at (cpp.south east) {C++};
      }
      
      \draw[example] (all.north) node[below]{$abc$};
      \draw[example] (ab.north) node[below]{$baba$};
      \draw[example] ($(anbnan.south west)+(5mm,5mm)$)  node{$aba$};
      \draw[example] ($(anbnan.south east)+(-7mm,5mm)$) node{$aabbaa$};
      \draw[example] ($(asbs.north west)+(5mm,-2mm)$) node{$aab$};
      \draw[example] ($(anbn.north west)+(5mm,-2mm)$) node{$ab$};
      \draw[example] ($(anbn.north east)+(-5mm,-2mm)$) node{$aabb$};
      \draw[example] (ba.center) node[left]{$ba$};

      \uncover<2->{
        \draw[example] (ba.center) edge[word, out=100, in=50, distance=12pt] (ba);
        \draw[example] ($(reg.west)+(5mm, 5mm)$) node[above]{$(a\mid b)^\star$} +(0,0) edge[word] (ab);
        \draw[example] ($(reg.west)+(5mm,-5mm)$) node[above]{$a^\star b^\star$} +(0,0) edge[word] (asbs);

        \draw[alert] ($(reg.west)+(5mm,-15mm)$) edge[noword] node[sloped]{lemme de} node[sloped, swap]{l'étoile} (anbn);
        \draw[alert] ($(reg.west)+(5mm, 15mm)$) edge[noword] node[sloped]{lemme de} node[sloped, swap]{l'étoile} (anbnan);
      }

      \uncover<3->{
        \draw[example] ($(gram.south west)+(5mm,20mm)$) node[right]{$S \rightarrow a\mid b \mid SS \mid S^\star \mid...$} +(0,0) edge[word] (reg);
        \draw[example] ($(gram.south west)+(5mm,15mm)$) node[right]{$S \rightarrow aS\mid bS \mid \varepsilon$} +(0,0) edge[word] (ab);
        \draw[example] ($(gram.south west)+(5mm,10mm)$) node[right]{$S \rightarrow aS\mid Sb \mid \varepsilon$} +(0,0) edge[word] (asbs);
        \draw[example] ($(gram.south west)+(5mm,5mm)$) node[right]{$S \rightarrow aSb\mid \varepsilon$} +(0,0) edge[word] (anbn);
        \draw[example] ($(gram.east)+(-10mm,0mm)$) node[above]{$S \rightarrow...$} +(0,0) edge[word] (gram);
        \draw[alert] ($(gram.south west)+(5mm,25mm)$) edge[noword] node[sloped]{lemme de} node[sloped, swap]{pompage} (anbnan);
      }
      
      \uncover<4->{
        \draw[example] ($(cpp.north west)+(5mm,-4mm)$)  node[right]{flex} +(0,0) edge[word] (reg.east);
        \draw[example] ($(cpp.north west)+(5mm,-8mm)$)  node[right]{...}  +(0,0) edge[word] (anbnan);
        \draw[example] ($(cpp.north west)+(5mm,-12mm)$) node[right]{bool decide(string s) \{return s=="ba";\}}  +(0,0) edge[word] (ba);
        \draw[example] ($(cpp.north west)+(5mm,-16mm)$) node[right]{...}  +(0,0) edge[word] (ab);
        \draw[example] ($(cpp.north west)+(5mm,-20mm)$) node[right]{...} +(0,0) edge[word] (asbs);
        \draw[example] ($(cpp.north west)+(5mm,-24mm)$) node[right]{...} +(0,0) edge[word] (anbn);
        \draw[example] ($(cpp.south)+(0mm,5mm)$) node[above]{gcc} edge[word] (cpp);
        
        \draw[example] ($(gram.south east)+(-10mm,5mm)$) node[above]{$S \rightarrow \texttt{if}...$} +(0,0) edge[bend right, word] ($(cpp.north east)+(-10mm,0mm)$);
        \draw[example] ($(cpp.north east)+(-10mm,-5mm)$) node[below]{bison} +(0,0) edge[bend right, word] ($(gram.south east)+(-10mm,0mm)$);
      }
      
    \end{tikzpicture}
  }
  
  \on<4->[bottom]{Le langage \structure{C++} permet-il de décrire tous les langages ?}

\end{frame}

\endgroup
\endinput

% SPDX-License-Identifier: CC-BY-SA-4.0
% Author: Matthieu Perrin
% Part: <Nom de la partie>
% Section: <Nom de la section>
% Sub-section: <Nom de la sous-section>  % (facultatif, laisser vide si non utilisé)
% Frame: <Titre de la slide>

\begingroup

\newcommand\PROBLEM{\textsc{problem}}

\begin{frame}{Expressivité des langages de description}

  On veut un langage $\langle \mathcal{L}, \llbracket \cdot \rrbracket \rangle$ où tout problème de décision est représentable :

  \vspace{-2mm}
  $$\alert{\forall \PROBLEM \in \textsc{lang},~ \exists u \in \mathcal{L},~ \PROBLEM = \llbracket u \rrbracket}$$
  
  \begin{block}{Théorème -- Corollaire du théorème de Cantor}
    Il n'existe pas de langage de description $\langle \mathcal{L}, \llbracket \cdot \rrbracket \rangle$
    où $\llbracket \cdot \rrbracket$ est surjective.
  \end{block}

  \vspace{-1mm}
  \begin{alertblock}{Démonstration}
    \begin{itemize}
    \item Supposons (par contradiction) qu'il existe un tel $\langle \mathcal{L}, \llbracket \cdot \rrbracket \rangle$.
    \item Posons $E = \{ x \in \mathcal{L} \mid x \notin \llbracket x \rrbracket\}$.
    \item Par surjectivité, il existe $y \in \mathcal{L}$ tel que $E = \llbracket y \rrbracket$.
    \item A-t-on $y \in \llbracket y \rrbracket$ ?
    \end{itemize}
  \end{alertblock}

  \vspace{-1mm}
  \begin{block}{Remarques}
    C'est encore le paradoxe du barbier, avec :
    \begin{itemize}
    \item La relation $x \bowtie y \eqdef x \in \llbracket y \rrbracket$ joue le rôle de ``est rasé par''.
    \item $y$ joue le rôle du barbier.
    \end{itemize}
    $\mathcal{L}$ est \structure{dénombrable} et $\textsc{lang}$ est \structure{indénombrable}.
  \end{block}
  
\end{frame}

\endgroup
\endinput

% SPDX-License-Identifier: CC-BY-SA-4.0
% Author: Matthieu Perrin
% Part: <Nom de la partie>
% Section: <Nom de la section>
% Sub-section: <Nom de la sous-section>  % (facultatif, laisser vide si non utilisé)
% Frame: <Titre de la slide>

\begingroup


\begin{frame}{Conséquence}

  \begin{block}{Tout langage de formalisme a des limites}
  Pour tout langage de formalisme $\langle \mathcal{L}, \llbracket \cdot \rrbracket \rangle$, il existe des problèmes de décision qui ne sont représentés par aucun mot de $\mathcal{L}$.
  \end{block}
  
  \begin{exampleblock}{Exemples}
    \begin{itemize}
    \item Si $L$ est le langage des \structure{expressions rationnelles} (flex)
      \begin{itemize}
      \item Il existe des langages qui ne sont \alert{pas rationnels}
      \end{itemize}
    \item Si $L$ est le langage des \structure{grammaires algébriques} (bison)
      \begin{itemize}
      \item Il existe des langages qui ne sont \alert{pas algébriques}
      \end{itemize}
    \item Si $L$ est la théorie de la \structure{géométrie euclidienne}
      \begin{itemize}
      \item Il existe des nombres qui ne sont \alert{pas constructibles}
      \end{itemize}
    \item Si $L$ est le langage des \structure{mathématiques}
      \begin{itemize}
      \item Il existe des problèmes qui ne sont \alert{pas définissables}
      \end{itemize}
    \item Si $L$ est un \structure{langage de programmation}
      \begin{itemize}
      \item Il existe des problèmes qui ne sont \alert{pas décidables} dans ce langage
      \end{itemize}
    \end{itemize}
  \end{exampleblock}

  \begin{center}
    \alert{Quels problèmes ne sont pas décidables par des algorithmes ?}
  \end{center}

  
\end{frame}

\endgroup
\endinput

% SPDX-License-Identifier: CC-BY-SA-4.0
% Author: Matthieu Perrin
% Part: <Nom de la partie>
% Section: <Nom de la section>
% Sub-section: <Nom de la sous-section>  % (facultatif, laisser vide si non utilisé)
% Frame: <Titre de la slide>

\begingroup

\begin{frame}{Reformulation du programme de Hilbert}

  \begin{block}{Le programme de Hilbert}
    Trouver un système de preuve $\langle \Phi, \Pi, \valid \rangle$ :
    \begin{itemize}
    \item cohérent et complet ; 
    \item suffisamment expressif pour formaliser les mathématiques ;
    \item tel que ces problèmes soient \alert{décidables par une procédure mécanique} :
    \end{itemize}

    \hspace\fill
    \begin{minipage}[t]{.45\textwidth}
      \begin{problembox}
        \textsc{Vérification des preuves}\\
        \structure{Instance :}
        \begin{itemize}
        \item Une preuve $\pi \in \Pi$ 
        \item Une proposition $\varphi \in \Phi$ 
        \end{itemize}
        \structure{Question :}
        \begin{itemize}
        \item Est-ce que $\pi \valid \varphi$ ?
        \end{itemize}
      \end{problembox}
    \end{minipage}
    \hspace\fill
    \begin{minipage}[t]{.5\textwidth}
      \begin{problembox}
        \textsc{Entscheidungsproblem}\\
        \structure{Instance :}
        \begin{itemize}
        \item Une proposition $\varphi \in \Phi$ 
        \end{itemize}
        \structure{Question :}
        \begin{itemize}
        \item Est-ce que $\exists \pi \in \Pi,   \pi \valid \varphi$ ?
        \end{itemize}
      \end{problembox}
    \end{minipage}
    \hspace\fill
  \end{block}
  
  \begin{center}
    \alert{Que signifie ``décidable par une procédure mécanique'' ?} 
  \end{center}

\end{frame}

\endgroup
\endinput

 
\subsection{Résolution de la crise}
% SPDX-License-Identifier: CC-BY-SA-4.0
% Author: Matthieu Perrin
% Part: <Nom de la partie>
% Section: <Nom de la section>
% Sub-section: <Nom de la sous-section>  % (facultatif, laisser vide si non utilisé)
% Frame: <Titre de la slide>

\begingroup

\begin{frame}{Vers une axiomatisation nécessairement incomplète}
 
  \onBlock[right=.75\textwidth, top=-2mm]{Théorème d'incomplétude de Gödel (1931)}{
    Il n'existe pas de système de preuve :
    \begin{itemize}
    \item formel (\textsc{Vérification des preuves} est décidable)
    \item suffisamment expressif pour l'arithmétique entière
    \item à la fois complet et cohérent
    \end{itemize}
  }
 
  \onBlock<2->[left=.71\textwidth, bottom=-3mm]{Vers une théorie des ensembles cohérente}{
    \begin{itemize}
    \item Schéma de compréhension restreint
      $\structure{\{x \alert{\in E} \mid P(x) \}}$
    \item La base de l'axiomatisation moderne (ZFC)
      \begin{itemize}
      \item Complétée par Fraenkel, Skolem, Von Neumann...
      \end{itemize}
    \item Cohérence toujours indémontrée
    \end{itemize}
    \begin{minipage}{.9\textwidth}
      \myquote{David Hilbert}{Personne ne doit pouvoir nous chasser du paradis que Cantor nous a créé}
    \end{minipage}
  }
  
  \onImage[x=-.4\textwidth,top]{%
    height=25mm,
    title={Kurt Gödel},
    licenselogo={\ccPublicDomain},
    license*={Domaine public (Vienne, 1925. \href{https://commons.wikimedia.org/wiki/File:Young_Kurt_G\%C3\%B6del_as_a_student_in_1925.jpg}{Wikimedia})},
    img={Godel.jpg}
  }
 
  \onImage<2->[x=.4\textwidth,bottom=5mm]{%
    height=30mm,
    title={Ernst Zermelo},
    licenselogo={\ccPublicDomain},
    license*={Domaine public (Italie, vers 1900. \href{https://commons.wikimedia.org/wiki/File:Ernst_Zermelo_1900s.jpg}{Wikimedia})},
    img={Zermelo.jpg}
  }
  
\end{frame}




%\begin{frame}{Thèse de Church -- Turing}
% 
%  \onBlock[left=.75\textwidth, y=22mm]{Thèse de Church -- Turing (1936-1937)}{
%    \centering
%    \structure{La définition des \og fonctions calculables \fg par des \\
%      \alert{Machines de Turing déterministes}  \\
%      caractérise la notion intuitive de \og procédure effective \fg.}
%  }
%  
%  \onBlock[y=-15mm]{Arguments en faveur de la thèse}{
%    \begin{itemize}
%    \item Équivalence entre formalismes
%      \begin{itemize}
%      \item Machines de Turing, $\lambda$-calcul, langages de programmation...
%      \end{itemize}
%    \item On ne connaît pas de machine plus puissante 
%      \begin{itemize}
%      \item Les formalismes plus expressifs necessitent des \og oracles \fg
%        \begin{itemize}
%        \item Par exemple, on peut \emph{définir} des objets mathématiques non-calculables
%        \end{itemize}
%      \item Les limites au formalisme sont internes
%        \begin{itemize}
%        \item L'indécidabilité du \structure{problème de l'arrêt} vient d'un paradoxe
%        \end{itemize}
%      \end{itemize}
%    \item Possibilité de simuler l'univers...
%      \begin{itemize}
%      \item ... donc toute machine qui peut y être effectivement construite 
%      \end{itemize}
%    \end{itemize}
%  }
% 
%  \onImage[x=42mm,y=15mm]{%
%    width=2.3cm,
%    title={Alonzo Church},
%    license={$\copyright$ - Princeton University Library (voir \href{https://en.wikipedia.org/wiki/File:Alonzo_Church.jpg}{Wikimedia}). Utilisation non commerciale à des fins pédagogiques (fair use)},
%    img={Church.jpg}
%  }
% 
% 
%%  Constructivisme mathématique
%%  
%%  L.E.J. Brouwer, Intuitionism and Formalism (1908) :
%%  
%%  « A mathematical assertion is true only if we have a construction for it. »
%  
%\end{frame}



\endgroup

% SPDX-License-Identifier: CC-BY-SA-4.0
% Author: Matthieu Perrin
% Part: <Nom de la partie>
% Section: <Nom de la section>
% Sub-section: <Nom de la sous-section>  % (facultatif, laisser vide si non utilisé)
% Frame: <Titre de la slide>

\begingroup

\begin{frame}{Qu'est-ce qu'une procédure mécanique ?}
 
  \onBlock[right=.7\textwidth, top=-2mm]{Développement du $\lambda$-calcul (années 1930)}{
    \begin{itemize}
    \item Tout est fonction
    \item Le calcul est une réécriture symbolique
    \item Formalise une notion abstraite de calcul effectif
    \item Modèle de la \structure{programmation fonctionnelle}
    \end{itemize}
  }
 
  \onBlock[left=.7\textwidth, bottom=10mm]{La machine de Turing (1936)}{
    \begin{itemize}
    \item Modèle mécanique et opérationnel
    \item Formalisme proche des automates
    \item Définit une notion de fonction calculable
    \item Modèle de la \structure{programmation impérative}
    \end{itemize}
  }

  \on[bottom=6mm]{\alert{Les deux modèles définissent les mêmes fonctions}}

  
  \onImage[x=-.35\textwidth,top]{%
    height=3cm,
    title={Alonzo Church},
    licenselogo={$\copyright$},
    license*={$\copyright$ - Princeton University Library (voir \href{https://en.wikipedia.org/wiki/File:Alonzo_Church.jpg}{Wikimedia}). Utilisation non commerciale à des fins pédagogiques (fair use)},
    img={Church.jpg}
  }
 
  \onImage[x=.35\textwidth,bottom=10mm]{%
    height=3cm,
    title={Alan Turing},
    licenselogo={\ccPublicDomain},
    license*={Domaine public UK (1951, \href{https://commons.wikimedia.org/wiki/File:Alan_Turing_(1951).jpg}{Wikimedia})},
    img={Turing.jpg}
  }
  
\end{frame}

\endgroup

% SPDX-License-Identifier: CC-BY-SA-4.0
% Author: Matthieu Perrin
% Part: <Nom de la partie>
% Section: <Nom de la section>
% Sub-section: <Nom de la sous-section>  % (facultatif, laisser vide si non utilisé)
% Frame: <Titre de la slide>

\begingroup

\begin{frame}{Thèse de Church -- Turing}

  \onBlock[top=-5mm]{Thèse de Church -- Turing (1936-1937)}{
    \centering
    \structure{La définition des \og fonctions calculables \fg par des \\
      \alert{machines de Turing déterministes}  \\
      caractérise la notion intuitive de \og procédure effective \fg.}
  }
  
  \onBlock[y=-5mm]{Arguments en faveur de la thèse}{
    \begin{itemize}
    \item Équivalence entre formalismes
      \begin{itemize}
      \item Machines de Turing, $\lambda$-calcul, langages de programmation...
      \end{itemize}
    \item On ne connaît pas de machine plus puissante 
      \begin{itemize}
      \item Les formalismes plus expressifs necessitent des \og oracles \fg
        \begin{itemize}
        \item Par exemple, on peut \emph{définir} des objets mathématiques non-calculables
        \end{itemize}
      \item Les limites au formalisme sont internes
        \begin{itemize}
        \item Résultats d'indécidabilité par diagonalisation
        \end{itemize}
      \end{itemize}
    \item Possibilité de simuler l'univers...
      \begin{itemize}
      \item ... donc toute machine qui peut y être effectivement construite 
      \end{itemize}
    \end{itemize}
  }

  \onBlock[bottom=-9mm]{Constructivisme mathématique}{
    \vspace{-2mm}
    \myquote{Luitzen Egbertus Jan Brouwer}{Une assertion mathématique n'est vraie que si elle est construite.}
  }

\end{frame}

\endgroup

% SPDX-License-Identifier: CC-BY-SA-4.0
% Author: Matthieu Perrin
% Part: <Nom de la partie>
% Section: <Nom de la section>
% Sub-section: <Nom de la sous-section>  % (facultatif, laisser vide si non utilisé)
% Frame: <Titre de la slide>

\begingroup

\begin{frame}{Le jeu de l'imitation}

  \on[y=-5mm]{
    \begin{tikzpicture}
      \node [faded background picture=Village,    text width=\paperwidth/3] (A) at (-\paperwidth/3,0) {};
      \node [faded background picture=Salon, text width=\paperwidth/3]      (B) at ( 0            ,0) {};
      \node [faded background picture=Machineroom, text width=\paperwidth/3] (C) at ( \paperwidth/3,0) {};

      \node[anchor=south, outer sep=0pt, inner sep=0pt] at (A.south) {\includegraphics[height=22mm]{Bob}};
      \node[anchor=south, outer sep=0pt, inner sep=0pt] at (B.south) {\includegraphics[height=22mm]{Carole}};
      \node[anchor=south, outer sep=0pt, inner sep=0pt] at (C.south) {\includegraphics[height=22mm]{Robot}};
    \end{tikzpicture}
  }

  \on<2>[x=-\paperwidth/3, y=-15mm]{
    \chatBubble[color=example]{Thé, mails, métro}
  }
  \on<3>[x=-\paperwidth/3, y=-15mm]{
    \chatBubble[color=example]{Raté !}
  }

  \on<1>[x=0, y=-15mm]{
    \chatBubble[color=alert]{Décris ta matinée en trois mots}
  }
  \on<3>[x=0, y=-15mm]{
    \chatBubble[color=alert]{Humain ?}
  }

  \on<2>[x=\paperwidth/3, y=-15mm]{
    \chatBubble[color=structure]{Café, mails, métro}
  }
  \on<3>[x=\paperwidth/3, y=-15mm]{
    \chatBubble[color=structure]{Raté !}
  }

  \on[x=-\paperwidth/3, y=15mm]{
    \begin{chat}[color=example]{Carole}
      \chatRecv[color=alert]{Décris ta matinée en trois mots}
      \only<2->{\chatSend{Thé, mails, métro}}
    \end{chat}
  }

  \on[x=0, y=15mm]{
    \begin{chat}[color=alert]{Mystère}
      \chatSend{Décris ta matinée en trois mots}
      \only<2->{\chatRecv[color=structure]{Café, mails, métro}}
    \end{chat}
  }

  \on[x=\paperwidth/3, y=15mm]{
    \begin{chat}[color=structure]{Carole}
      \chatRecv[color=alert]{Décris ta matinée en trois mots}
      \only<2->{\chatSend{Café, mails, métro}}
    \end{chat}
  }

\end{frame}

\endgroup

% SPDX-License-Identifier: CC-BY-SA-4.0
% Author: Matthieu Perrin
% Part: <Nom de la partie>
% Section: <Nom de la section>
% Sub-section: <Nom de la sous-section>  % (facultatif, laisser vide si non utilisé)
% Frame: <Titre de la slide>

\begingroup

\begin{frame}{Une machine peut-elle penser ?}
  
  \begin{block}{L'intelligence humaine comme fonction entre des langages}
    Dans le test de Turing, l'homme est vu comme une \structure{relation} : 
    \begin{itemize}
    \item on lui donne un texte en entrée
    \item il répond un texte en sortie
    \end{itemize}
    La \structure{manière} dont la réponse est produite est hors de portée de l'observateur :
  \myquote{Alan Turing}{La seule manière d’être certain qu'une machine pense serait d’être cette machine et de se sentir penser.}
  \end{block}

  \vspace{-5mm}
  \begin{block}{Quelle est la puissance de calcul de l'intelligence humaine ?}
    \begin{itemize}
    \item si l'humain est observé par ses seules productions symboliques,
    \item en quoi serait-il calculatoirement plus puissant qu'une machine ?
    \end{itemize}
  \myquote{Alan Turing}{On peut espérer que les machines finiront par rivaliser avec l'homme dans tous les domaines purement intellectuels.}
  \end{block}

\end{frame}

\endgroup

 
\section{Langages et grammaires}
 
\subsection{Grammaires générales}
% SPDX-License-Identifier: CC-BY-SA-4.0
% Author: Matthieu Perrin
% Part: 
% Section: 
% Sub-section: 
% Frame: 

\begingroup

\begin{frame}{Comment décrire un problème de décision ?}

  \onAlertBlock[top=-5mm]{Problème -- Définition d'un langage}{
    On cherche un langage de description \alert{$\langle\mathcal{L}, \llbracket \cdot \rrbracket \rangle$}
    le plus expressif possible pour décrire des langages sur un alphabet $\Sigma$
  }

  \onBlock[y=12mm]{Approche générative}{
    \vspace{-2mm}
    \begin{itemize}
    \item Donner une \structure{description syntaxique} de la forme des mots de $L$
    \item Une \structure{grammaire} est un ensemble de règles qui \structure{génèrent} les mots de $L$
    \end{itemize}
  }
  
  \onBlock[y=-7mm]{Approche algorithmique}{
    \vspace{-2mm}
    \begin{itemize}
    \item Décrire une \structure{procédure de décision} pour l'appartenance à $L$
    \item Un \structure{automate} spécifie un calcul effectif sur les mots
    \end{itemize}
  }

  \on[bottom, width=.53\textwidth, x=-26mm]{
    \Probleme{\structure{Decision$_L$} \normalfont{(pour $L \in \mathcal{L}$)}}{
      \vspace{2.3mm}Un mot \structure{$u \in \Sigma^\star$}
    }{
      \vspace{2mm}Est-ce que \structure{$u \in \llbracket L \rrbracket$} ?
    }
  }
  \on[bottom, width=.55\textwidth, x=27mm]{
    \Probleme{\structure{Universal$_{\langle\mathcal{L}, \llbracket \cdot \rrbracket \rangle}$}}{
      \begin{itemize}
      \item Un langage \structure{$L \in \mathcal{L}$}
      \item Un mot \structure{$u \in \Sigma^\star$}
      \end{itemize}
    }{
      Est-ce que \structure{$u \in \llbracket L \rrbracket$} ?
    }
  }

\end{frame}

\endgroup
\endinput

% SPDX-License-Identifier: CC-BY-SA-4.0
% Author: Matthieu Perrin
% Part: 
% Section: 
% Sub-section: 
% Frame: 

\begingroup

\begin{frame}{Notion de grammaire}
  
  \begin{block}{Définition -- Grammaire non-restreinte}
    Une \structure{grammaire non-restreinte} est un quadruplet \alert{$\langle \Sigma, \Gamma, S, \rightarrow \rangle$} tel que :
    \begin{description}[xxxx]
    \item[\alert{$\Sigma$}] $\in \Omega_f$ l'alphabet des \structure{terminaux}
    \item[\alert{$\Gamma$}] $\in \Omega_f$ : l'alphabet des \structure{non-terminaux}; tel que $\Sigma \cap \Gamma = \emptyset$
    \item[\alert{$S$}] $\in \Gamma$ : l'\structure{axiome}
    \item[\alert{$\rightarrow$}] $\subseteq ((\Sigma \cup \Gamma)^\star \cdot \Gamma \cdot (\Sigma \cup \Gamma)^\star) \times (\Sigma \cup \Gamma)^\star$ : l'ensemble des \structure{règles de production}
    \end{description}

    \vspace{1mm}
    Une \structure{règle de production} est un couple \alert{$\langle \alpha, \beta \rangle \in \rightarrow$}, noté $\alert{\alpha \rightarrow \beta}$ tel que :
    \begin{description}[xxxx]
    \item[\alert{$\alpha$}] $\in (\Sigma \cup \Gamma)^\star \cdot \Gamma \cdot (\Sigma \cup \Gamma)^\star$ : \structure{membre gauche}
    \item[\alert{$\beta$}] $\in (\Sigma \cup \Gamma)^\star$ : \structure{membre droit}
    \end{description}
  \end{block}

  \begin{block}{Remarques}
    Soient $G = \langle \Sigma, \Gamma, S, \rightarrow \rangle$ une grammaire et $\langle \alpha, \beta \rangle \in \rightarrow$ une règle de production. 
    \begin{itemize}
    \item On n'impose pas que $|\alpha| = 1$ 
    \item On impose que $\alpha$ contienne au moins un non-terminal 
    \item On dit que $G$ est \structure{algébrique} si $\forall \langle \alpha, \beta \rangle \in \rightarrow, |\alpha| = 1$
    \end{itemize}
  \end{block}

\end{frame}

\endgroup
\endinput

% SPDX-License-Identifier: CC-BY-SA-4.0
% Author: Matthieu Perrin
% Part: 
% Section: 
% Sub-section: 
% Frame: 

\begingroup

\begin{frame}{Langage engendré par une grammaire non-restreinte}
  Soient \alert{$\langle \Sigma, \Gamma, S, \rightarrow \rangle$} une grammaire non-restreinte, $u, v \in (\Sigma \cup \Gamma)^\star$ et $w \in \Sigma^\star$.

  \begin{block}{Définitions -- Dérivation}

    \begin{itemize}
    \item On dit que \structure{$u$ se dérive directement en $v$}, noté $\alert{u \vdash v}$, si :

      \vspace{-3mm}
      $$\alert{\exists x, y, \alpha, \beta \in (\Sigma \cup \Gamma)^\star,\quad
        \example{\alpha \rightarrow \beta} \quad\land\quad u = \structure{x} \cdot \example{\alpha} \cdot \structure{y} \quad\land\quad v = \structure{x} \cdot \example{\beta} \cdot \structure{y}}$$

    \item On dit que \structure{$u$ se dérive en $v$} si \alert{$u \vdash^\star v$},\\
      où \alert{$\vdash^\star$ est la fermeture transitive et réflexive de $\vdash$}.

    \end{itemize}
  \end{block}

  \begin{block}{Définitions -- Génération et langage engendré}
    \begin{itemize}
    \item Une \structure{génération de $w$ par $G$} est une dérivation \alert{$S \vdash^\star w$} à partir de $S$.
    \item On dit que \structure{$w$ est généré par $G$} s'il existe une génération de $w$ par $G$. 
    \item Le \structure{langage engendré} par $G$, $\alert{\mathcal{L}(G)}$, contient les mots générés par $G$ :
      $$\alert{\mathcal{L}(G) = \left\{w\in\Sigma^\star \,\middle\mid\, S \vdash^\star w \right\}}.$$
    \end{itemize}
  \end{block}

\end{frame}

\endgroup

\input{languages/chomsky/grammars/graph}
 
\subsection{Restrictions sur les grammaires}
% SPDX-License-Identifier: CC-BY-SA-4.0
% Author: Matthieu Perrin
% Part: 
% Section: 
% Sub-section: 
% Frame: 

\begingroup

\begin{frame}{Grammaires et langages algébriques}

  \begin{block}{Définition -- Grammaire algébrique}
    Une grammaire $G = \langle \Sigma, \Gamma, S, \rightarrow \rangle$ est dite \structure{algébrique} si les membres gauches de toutes ses règles sont un non-terminal :

    $$\forall \langle \alpha, \beta\rangle \in \rightarrow, \quad \alert{\alpha \in \Gamma}$$
  \end{block}

  \vspace{-2mm}
  \begin{block}{Définition -- Langage algébrique}
    Soient $\Sigma$ un alphabet et $L$ un langage sur $\Sigma$.
    \begin{itemize}
    \item $L$ est dit \structure{algébrique} s'il est engendré par une grammaire algébrique
    \item L'\structure{ensemble des langages algébrique} sur $\Sigma$ est noté \alert{$\textsc{alg}_\Sigma$}
    \item La \structure{classe des langages algébrique} est noté \alert{$\textsc{alg}$} (indépendant de $\Sigma$)
    \end{itemize}
  \end{block}

  \begin{exampleblock}{Exemple -- Langage $\{a^n b^n \mid n\in \mathbb{N}\}$}
    \begin{itemize}
    \item $G = \langle \{a, b\}, \{S\}, S, \left\{S \rightarrow a S b \mid \varepsilon \right\} \rangle $
    \item Génération de $aabb$ : $ S \vdash aSb \vdash aaSbb \vdash aabb $
    \end{itemize}
  \end{exampleblock}
  
\end{frame}

\endgroup
\endinput

% SPDX-License-Identifier: CC-BY-SA-4.0
% Author: Matthieu Perrin
% Part: 
% Section: 
% Sub-section: 
% Frame: 

\begingroup

\begin{frame}{Grammaires et langages rationnels}

  \vspace{-2mm}
  \begin{block}{Définition -- Grammaire rationnelle}
    Soit \alert{$G = \langle \Sigma, \Gamma, S, \rightarrow \rangle$} une grammaire algébrique. 
    \begin{itemize}
    \item $G$ est dite \structure{rationnelle à droite} si ses règles sont de la forme :
      \begin{center}
        $\alert{A \rightarrow a \cdot B}$ \quad ou\quad $\alert{A \rightarrow a}$ \quad ou\quad $\alert{A \rightarrow \varepsilon}$
        \quad avec \structure{$A, B \in \Gamma$} et \structure{$a\in \Sigma$}
      \end{center}
    \item $G$ est dite \structure{rationnelle à gauche} si ses règles sont de la forme :
      \begin{center}
        $\alert{A \rightarrow B \cdot a}$ \quad ou\quad $\alert{A \rightarrow a}$ \quad ou\quad $\alert{A \rightarrow \varepsilon}$
        \quad avec \structure{$A, B \in \Gamma$} et \structure{$a\in \Sigma$}
      \end{center}
    \item $G$ est dite \structure{rationnelle} si elle est \structure{rationnelle à droite ou à gauche}
    \end{itemize}
  \end{block}

  \vspace{-1mm}
  \begin{block}{Définition -- Langage rationnel}
    \vspace{-1mm}
    Soient $\Sigma$ un alphabet et $L$ un langage sur $\Sigma$.
    \begin{itemize}
    \item On dit que $L$ est \structure{rationnel} s'il est engendré par une grammaire rationnelle
    \item L'\structure{ensemble des langages rationnels} sur $\Sigma$ est noté \alert{$\textsc{rat}_\Sigma$}
    \item La \structure{classe des langages rationnels} est noté \alert{$\textsc{rat}$} (indépendant de $\Sigma$)
    \end{itemize}
  \end{block}

  \vspace{-1mm}
  \begin{exampleblock}{Exemple -- Langage $\{a^m b^n \mid m, n > 0\}$}
  \vspace{-1mm}
    \begin{itemize}
    \item $G = \langle \{a, b\}, \{S, T\}, S, \left\{S \rightarrow a S \mid a T ; T \rightarrow b T \mid b \right\} \rangle $
    \end{itemize}
  \end{exampleblock}

\end{frame}

\endgroup
\endinput

% SPDX-License-Identifier: CC-BY-SA-4.0
% Author: Matthieu Perrin
% Part: 
% Section: 
% Sub-section: 
% Frame: 

\begingroup

\begin{frame}{Grammaires et langages contextuels}

  \begin{block}{Définition -- Grammaire contextuelle}
    Une grammaire \alert{$\langle \Sigma, \Gamma, S, \rightarrow \rangle$} est dite \structure{contextuelle} si toutes ses règles sont :
    
    \begin{itemize}
    \item soit de la forme $\alert{g \cdot A \cdot d \rightarrow g \cdot \alpha \cdot d}$, avec

      \begin{center}
      \begin{tabular}{ccll}
      \alert{$g$}      & $\in$ & $(\Sigma \cup \Gamma)^\star$ & \structure{le contexte gauche}, \\
      \alert{$d$}      & $\in$ & $(\Sigma \cup \Gamma)^\star$ & \structure{le contexte droit}, \\
      \alert{$A$}      & $\in$ & $\Gamma$                    & un non-terminal, \\
      \alert{$\alpha$} & $\in$ & $(\Sigma \cup \Gamma)^+$    & un mot \alert{non-vide}. \\
      \end{tabular}
      \end{center}

      Lire : \og{} $A$ peut se réécrire en $\alpha$, à condition d'être entre $g$ et $d$ \fg.
    \item soit la règle $S \rightarrow \varepsilon$, si $S$ n'apparaît jamais à droite d'une règle
    \end{itemize}
  \end{block}

  \begin{block}{Définition -- Langage contextuel (\emph{context-sensitive})}
    Soient $\Sigma$ un alphabet et $L$ un langage sur $\Sigma$.
    \begin{itemize}
    \item On dit que $L$ est \structure{contextuel} s'il est engendré par une grammaire contextuelle
    \item L'\structure{ensemble des langages contextuels} sur $\Sigma$ est noté \alert{$\textsc{cs}_\Sigma$}
    \item La \structure{classe des langages contextuels} est noté \alert{$\textsc{cs}$} (indépendant de $\Sigma$)
    \end{itemize}
  \end{block}
  
\end{frame}

\endgroup
\endinput

% SPDX-License-Identifier: CC-BY-SA-4.0
% Author: Matthieu Perrin
% Part: <Nom de la partie>
% Section: <Nom de la section>
% Sub-section: <Nom de la sous-section>  % (facultatif, laisser vide si non utilisé)
% Frame: <Titre de la slide>

\begingroup

\begin{frame}{Exemple : $aab \in \mathcal{L}(ab|aab)$ ?}

  \begin{tikzpicture}

    \draw (5,10) node[below]{\begin{minipage}{\textwidth}
        \begin{block}{Machine non-déterministe $M$ :}
          \scalebox{.8}{\begin{tikzpicture}[shorten >=1pt, node distance=1.5cm, on grid, auto]
              \node (nq0)                   {};
              \node (nq1) [above right of=nq0] {};
              \node (nq2) [right of=nq1]    {};
              \node (nq3) [below right of=nq0] {};
              \node (nq4) [right of=nq3]    {};
              \node (nq5) [right of=nq4]    {};

              \node[state, initial, initial text=] (q0) at (nq0) {$q_0$};
              \node[state]                         (q1) at (nq1) {$q_1$};
              \node[state, accepting]              (q2) at (nq2) {$q_2$};
              \node[state]                         (q3) at (nq3) {$q_3$};
              \node[state]                         (q4) at (nq4) {$q_4$};
              \node[state, accepting]              (q5) at (nq5) {$q_5$};

              \node<2-7>[fill=alert!20, state, initial, initial text=] (q0) at (nq0) {$q_0$};
              \node<10>[fill=alert!20, state]                          (q1) at (nq1) {$q_1$};
              \node<8-9>[fill=alert!20, state]                       (q3) at (nq3) {$q_3$};
              \node<11>[fill=alert!20, state]                          (q4) at (nq4) {$q_4$};
              \node<11>[fill=structure!20, state, accepting]                (q5) at (nq5) {$q_5$};


              \path<-3,5-> [->] (q0)       edge[bend left] node {$\smTMtransR{a}{a}$} (q1);
              \path<-4,6-> [->] (q1)       edge[bend left] node {$\smTMtransR{b}{b}$} (q2);
              %              \path<-5,7-> [->] (q2)       edge[bend left] node {$\smTMtransR{a}{a}$} (q1);
              \path<-2,4-> [->] (q0)       edge[bend right] node[swap] {$\smTMtransR{a}{a}$} (q3);
              \path<-5,7-8,10-> [->] (q3) edge[bend left] node {$\smTMtransR{a}{a}$} (q4);
              \path<-6,8-10> [->]       (q4)       edge[bend left] node {$\smTMtransR{b}{b}$} (q5);
              %              \path<-8,10-> [->]      (q5)      edge[bend left] node {$\smTMtransR{a}{a}$} (q3);


              \path<4> [alert, ->] (q0)   edge[bend left] node {$\smTMtransR{a}{a}$} (q1);
              \path<5> [alert, ->] (q1)   edge[bend left] node {$\smTMtransR{b}{b}$} (q2);
              %              \path<6> [alert, ->] (q2)   edge[bend left] node {$\smTMtransR{a}{a}$} (q1);
              \path<3> [alert, ->] (q0)   edge[bend right] node[swap] {$\smTMtransR{a}{a}$} (q3);
              \path<6,9> [alert, ->] (q3)   edge[bend left] node {$\smTMtransR{a}{a}$} (q4);
              \path<7,11> [alert, ->] (q4)   edge[bend left] node {$\smTMtransR{b}{b}$} (q5);
              %              \path<9> [alert, ->] (q5)   edge[bend left] node {$\smTMtransR{a}{a}$} (q3);

          \end{tikzpicture}}
        \end{block}
    \end{minipage}};


    \draw (7.5,10) node[below]{\begin{minipage}{.3\textwidth}\begin{block}{Configuration :}\end{block}\end{minipage}};
    \draw (6,8.5) node[right]{\structure{$f$ : }~
      \only<-2>{$\varepsilon$}%
      \only<3-7>{$\mid a\, q_3\, ab$}%
      \only<4-9>{$\mid a\, q_1\, ab$}%
      \only<9-10>{$\mid aa\, q_4\, b$}%
      \only<11>{$\varepsilon$}%
      %      \only<11>{$\mid aab\, q_5$}%
    };
    \draw (6,8) node[right]{\structure{$c$ : }~
      \only<1>{$aab$}%
      \only<2-7>{$\alert{q_0\, a}ab$}%
      \only<8-9>{$a\, \alert{q_3\, a}b$}%
      \only<10>{$a\, \alert{q_1\, a}b$}%
      \only<11>{$aa\, \alert{q_4\, b}$}%
    };
    \draw<11> (6,7.5) node[right]{\structure{Le mot $aab$ est accepté}};

    \draw (5,4) node{\begin{minipage}{\textwidth}
        \begin{block}{Machine déterminisée $M_D$ :}
          \scalebox{.8}{\begin{tikzpicture}[shorten >=1pt, node distance=3cm, on grid, auto]
              \tikzset{mynode/.style={draw, rounded corners=8, align=center, minimum height=0.6cm, minimum width=1cm}}

              \node (nt0)                       {};
              \node (ninit)   [left of=nt0]     {};
              \node (nt1)     [right of=nt0]    {};
              \node (nt2)     [right of=nt1]    {};
              \node (nt3)     [right of=nt2]    {};
              \node (nreinit) at (0,-1.5)       {};
              \node (nt6)     [right of=nreinit]{};
              \node (nt5)     [right of=nt6]    {};
              \node (nt4)     [right of=nt5]    {};

              \node[fill=example!20, mynode]         (init)   at (ninit)    {\footnotesize init};
              \node[fill=structure!20, mynode]            (t0)     at (nt0)      {\footnotesize $q_0 \xrightarrow{\smTMtransR{a}{a}} q_3$};
              \node[fill=structure!20, mynode]            (t1)     at (nt1)      {\footnotesize $q_0 \xrightarrow{\smTMtransR{a}{a}} q_1$};
              \node[fill=structure!20, mynode, accepting] (t2)     at (nt2)      {\footnotesize $q_1 \xrightarrow{\smTMtransR{b}{b}} q_2$};
              \node[fill=structure!20, mynode]            (t4)     at (nt5)      {\footnotesize $q_3 \xrightarrow{\smTMtransR{a}{a}} q_4$};
              \node[fill=structure!20, mynode, accepting] (t5)     at (nt6)      {\footnotesize $q_4 \xrightarrow{\smTMtransR{b}{b}} q_5$};
              \node[fill=example!20, mynode]         (reinit) at (nreinit)  {\footnotesize reinit};

              \node<2>[fill=alert!20, mynode]         (init)   at (ninit)       {\footnotesize init};
              \node<3>[fill=alert!20, mynode]            (t0)     at (nt0)      {\footnotesize $q_0 \xrightarrow{\smTMtransR{a}{a}} q_3$};
              \node<4>[fill=alert!20, mynode]            (t1)     at (nt1)      {\footnotesize $q_0 \xrightarrow{\smTMtransR{a}{a}} q_1$};
              \node<5>[fill=alert!20, mynode, accepting] (t2)     at (nt2)      {\footnotesize $q_1 \xrightarrow{\smTMtransR{b}{b}} q_2$};
              \node<6,9>[fill=alert!20, mynode]            (t4)     at (nt5)   {\footnotesize $q_3 \xrightarrow{\smTMtransR{a}{a}} q_4$};
              \node<7,11>[fill=alert!20, mynode, accepting] (t5)     at (nt6)      {\footnotesize $q_4 \xrightarrow{\smTMtransR{b}{b}} q_5$};
              \node<8,10>[fill=alert!20, mynode]         (reinit) at (nreinit) {\footnotesize reinit};

              \path [->] (init)   edge node[swap] {\scriptsize$c:\smTMtransR{q}{q}$} (t0);
              \path [->] (t0)     edge node[swap] {\scriptsize$c:\smTMtransR{q}{q}$} (t1);
              \path [->] (t1)     edge node[swap] {\scriptsize$c:\smTMtransR{q}{q}$} (t2);
              \path [->] (t2)     edge node[swap] {\scriptsize$c:\smTMtransR{q}{q}$} (t4);
              \path [->] (t4)     edge node[swap] {\scriptsize$c:\smTMtransR{q}{q}$} (t5);
              \path [->] (t5)     edge node[swap] {\scriptsize$c:\smTMtransR{q}{q}$} (reinit);
              \path [->] (reinit) edge node[swap] {\scriptsize$c:\smTMtransR{q}{q}$} (t0);
          \end{tikzpicture}}
        \end{block}
    \end{minipage}};

  \end{tikzpicture}

\end{frame}

\endgroup

 
\subsection{La hiérarchie de Chomsky}
% SPDX-License-Identifier: CC-BY-SA-4.0
% Author: Matthieu Perrin
% Part: 
% Section: 
% Sub-section: 
% Frame: 

\begingroup

\begin{frame}{Classification des grammaires}

  On dit qu'une grammaire $G = \langle \Sigma, \Gamma, S, \rightarrow \rangle$ \structure{est de type} $i \in \{0, 1, 2, 3\}$ si :
  \begin{description}[xType 0 :]
  \item[Type 0 :] \vspace{2mm} $G$ est grammaire \alert{non-containte} si pour toute règle $\structure{\alpha \rightarrow \beta}$ \\
    \begin{itemize}
    \item \structure{$\alpha$} contient au moins un non-terminal de $\Gamma$
    \item[\example{Exemple :}] $a B c \rightarrow B a$
    \end{itemize}
  \item[Type 1 :] \vspace{2mm}$G$ est grammaire \alert{contextuelle} si pour toute règle $\structure{\alpha \rightarrow \beta}$ \\
    \begin{itemize}
    \item \structure{$\alpha = g A d$} et \structure{$\beta = g \gamma d$} avec \structure{$A \in \Gamma$} et \structure{$\gamma \neq \varepsilon$}
    \item[\example{Exemple :}] $a B c \rightarrow ac B ac \mid acac$ 
    \end{itemize}
  \item[Type 2 :] \vspace{2mm}$G$ est grammaire \alert{algébrique} si pour toute règle $\structure{\alpha \rightarrow \beta}$ \\
    \begin{itemize}
    \item \structure{$\alpha \in \Gamma$}
    \item[\example{Exemple :}] $S \rightarrow a S b \mid \varepsilon$ 
    \end{itemize}

  \item[Type 3 :] \vspace{2mm}$G$ est grammaire \alert{rationnelle} si $G$ est dans l'un des deux cas :
    \begin{itemize}
    \item \alert{rationnelle droite :} $\forall \structure{\alpha \rightarrow \beta}$,  \structure{$\alpha \in \Gamma$} et \structure{$\beta \in (\Sigma \cdot \Gamma^?)^?$}
    \item[\example{Exemple :}] $S \rightarrow aS \mid b \mid \varepsilon$ 
    \end{itemize}
    \begin{itemize}
    \item \alert{rationnelle gauche :} $\forall \structure{\alpha \rightarrow \beta}$,  \structure{$\alpha \in \Gamma$} et \structure{$\beta \in (\Gamma^? \cdot \Sigma)^?$}
    \item[\example{Exemple :}] $S \rightarrow Sa \mid b \mid \varepsilon$ 
    \end{itemize}
  \end{description}

\end{frame}

\endgroup

% SPDX-License-Identifier: CC-BY-SA-4.0
% Author: Matthieu Perrin
% Part: 
% Section: 
% Sub-section: 
% Frame: 

\begingroup

\begin{frame}{Lien entre grammaires algébriques et contextuelles}

  \begin{block}{Remarque -- Non-inclusion des grammaires}
    Il existe des grammaires algébriques qui ne sont pas contextuelles. \\
    \example{Contre-exemple} : $\left\langle \{a\}, \{S, A\}, S, \left\{\begin{array}{rcl} S &\rightarrow & aA \\ A &\rightarrow & b\alert{S} \,|\, \alert{\varepsilon} \end{array}\right\} \right\rangle$.
    \begin{itemize}
    \item Grammaire algébrique : le membre gauche des règles est $S$ ou $A$
    \item Grammaire non contextuelle à cause de la règle $A \rightarrow \varepsilon$
    \end{itemize}
  \end{block}

  \begin{block}{Théorème -- Inclusion des langages}
    Tout langage algébrique est contextuel.
  \end{block}
  
  \begin{block}{Remarque}
    \begin{itemize}
    \item Une règle $g \cdot A \cdot d \rightarrow g \cdot \alpha \cdot d$ peut être interprétée comme :\\ \og \structure{$A \rightarrow \alpha$, sous la restriction d'un contexte gauche et/ou droit} \fg
    \item Les grammaires algébriques sont souvent appelées \structure{hors contexte} (\textit{context-free}), car $g = d = \varepsilon$ pour toute règle
    \end{itemize}
  \end{block}
  
\end{frame}

\endgroup

\input{languages/chomsky/hierarchy/languages}
% SPDX-License-Identifier: CC-BY-SA-4.0
% Author: Matthieu Perrin
% Part: 
% Section: 
% Sub-section: 
% Frame: 

\begingroup

\begin{frame}{Limite des langages rationnels et algébriques}
  \small

  \onAlertBlock[top=-5mm, left=.5\textwidth]{Lemme -- Lemme de l'étoile}{
    \vspace{-3mm}
    $$
    \begin{array}{l}
      \forall \Sigma, \forall L\in \textsc{rat}_{\Sigma}, \exists N\in \mathbb{N},\\
      \forall u\in L,|u| \ge N \Rightarrow\\
      \exists x, y, z\in \Sigma^\star,\\
      \begin{array}{rl}
        &u = x\cdot y \cdot z \\
        \land& y\neq \varepsilon \\
        \land& |xy| \le N\\
        \land& \forall i \in \mathbb{N}, x\cdot y^i \cdot z \in L
      \end{array}
    \end{array}
    $$
  }

  \onAlertBlock[top=-5mm, right=.5\textwidth]{Lemme -- Lemme de pompage}{
    \vspace{-6mm}
    $$
    \begin{array}{l}
      \forall \Sigma, \forall L\in \textsc{alg}_{\Sigma}, \exists N\in \mathbb{N},\\
      \forall u\in L,|u| \ge N \Rightarrow\\
      \exists v, w, x, y, z\in \Sigma^\star\\
      \begin{array}{rl}
        &u = v \cdot w \cdot x\cdot y \cdot z \\
        \land& wy\neq \varepsilon \\
        \land& |wxy| \le N\\
        \land& \forall i \in \mathbb{N}, v \cdot w^i \cdot x\cdot y^i \cdot z \in L
      \end{array}
    \end{array}
    $$
  }

  \onBlock[y=-9mm, left=.5\textwidth]{Démonstration}{
    Pour un \emph{grand} mot, on a $T \vdash^\star T y $

    \vspace{-2mm}
    \begin{center}
    \begin{tikzpicture}[x=3mm, y=7mm]
      \fill[structure!20] (0,3) -- (0,0) -- (6,0);
      \fill[alert!20]     (0,2) -- (0,0) -- (4,0);
      \fill[structure!20] (0,1) -- (0,0) -- (2,0);

      \node (S)  at (0,3) {$S$};
      \node (T1) at (0,2) {$T$};
      \node (T2) at (0,1) {$T$};

      \node (x) at ($(1,0)+(-2mm,2mm)$) {$x$};
      \node (y) at ($(3,0)+(-2mm,2mm)$) {$y$};
      \node (z) at ($(5,0)+(-2mm,2mm)$) {$z$};
 
      \draw[dashed] (S)  -- (6,0);
      \draw[dashed] (S)  -- (T1);
      \draw[dashed] (T1) -- (4,0);
      \draw[dashed] (T1) -- (T2);
      \draw[dashed] (T2) -- (2,0);
      \draw[dashed] (T2) -- (0,0);
    \end{tikzpicture}
    \end{center}
  }
  
  \onBlock[y=-9mm, right=.5\textwidth]{Démonstration}{
    Pour un \emph{grand} mot, on a $T \vdash^\star w T y $

    \vspace{-2mm}
    \begin{center}
    \begin{tikzpicture}[x=3mm, y=7mm]
      \fill[structure!20] (-6,0) -- (0,3) -- (6,0);
      \fill[alert!20]     (-4,0) -- (0,2) -- (4,0);
      \fill[structure!20] (-2,0) -- (0,1) -- (2,0);

      \node (S)  at (0,3) {$S$};
      \node (T1) at (0,2) {$T$};
      \node (T2) at (0,1) {$T$};

      \node (v) at ($(-5,0)+(2mm,2mm)$) {$v$};
      \node (w) at ($(-3,0)+(2mm,2mm)$) {$w$};
      \node (x) at ($(0,0)+(-0mm,2mm)$) {$x$};
      \node (y) at ($(3,0)+(-2mm,2mm)$) {$y$};
      \node (z) at ($(5,0)+(-2mm,2mm)$) {$z$};
 
      \draw[dashed] (S)  -- (-6,0);
      \draw[dashed] (S)  -- (T1);
      \draw[dashed] (S)  -- (6,0);
      \draw[dashed] (T1) -- (-4,0);
      \draw[dashed] (T1) -- (T2);
      \draw[dashed] (T1) -- (4,0);
      \draw[dashed] (T2) -- (-2,0);
      \draw[dashed] (T2) -- (2,0);
    \end{tikzpicture}
    \end{center}
  }
  
  \onExampleBlock[bottom=4mm, left=.5\textwidth]{Illustration}{
    \vspace{-1mm}
    \begin{itemize}
    \item Montrer que $\{a^n b^n \mid n\in \mathbb{N}\} \notin \textsc{rat}$.
    \end{itemize}
  }
 
  \onExampleBlock[bottom=4mm, right=.5\textwidth]{Illustration}{
    \vspace{-1mm}
    \begin{itemize}
    \item Montrer que $\{a^n b^n c^n \mid n\in \mathbb{N}\} \notin \textsc{alg}$.
    \end{itemize}
  }

  \footnoteref{M. O. Rabin, D. Scott. \textit{Finite automata and their decision problems.} (1959)}
  \footnoteref{Y. Bar-Hillel, M. Perles, E. Shamir. \textit{On formal properties of simple phrase structure grammars.} (1961)}
\end{frame}

\endgroup
\endinput

 
\section{Décision des langages rationnels}
 
\subsection{Notion d'automate fini}
% SPDX-License-Identifier: CC-BY-SA-4.0
% Author: Matthieu Perrin
% Part: 
% Section: 
% Sub-section: 
% Frame: 

\begingroup

\begin{frame}{Décision de l'appartenance à un langage contextuel}

  \begin{block}{Problème de décision}
    Soit $L$ un langage contextuel
    
    \Probleme{\structure{Decision$_L$}}{
      Un mot \structure{$u \in \Sigma^\star$}
    }{
      Est-ce que \structure{$u \in L$} ?
    }
  \end{block}

  \begin{block}{Algorithme de recherche ascendante par force brute} 

    \begin{description}
    \item [Entrées :]
      \begin{itemize}
      \item Une grammaire $G$, si possible contextuelle
      \item Un mot $u$
      \end{itemize}

    \item [Sortie :] une réponse booléenne sur \structure{$u \in \mathcal{L}(G)$}
    \item [Teminaison :] garantie si $G$ est contextuelle
    \item [Complexité :] exponentielle par rapport à $|u|$
    \end{description}
  \end{block}
  
\end{frame}

\endgroup
\endinput

\input{languages/automata/definition/lexer}
% SPDX-License-Identifier: CC-BY-SA-4.0
% Author: Matthieu Perrin
% Part: 
% Section: 
% Sub-section: 
% Frame: 

\begingroup

\begin{frame}{Modélisation mathématique}

  \begin{block}{Définition -- Automate fini non-déterministe (AFN)}
    \vspace{1mm}
    Un \structure{automate fini} est un quintuplet \alert{$\langle \Sigma, Q, q_0, F, \rightarrow \rangle$} tel que :
    \begin{description}[xxxxx]
    \item[\alert{$\Sigma$}] $\in \Omega_f$ ensemble fini non vide : \structure{l'alphabet}
    \item[\alert{$Q$}]  $\in \Omega_f$ ensemble fini non vide : \structure{les états}
    \item[\alert{$q_0$}] $\in Q$ : \structure{l'état initial}
    \item[\alert{$F$}] $\subseteq Q$ : \structure{les états finaux (ou accepteurs)}
    \item[\alert{$\rightarrow$}] $\subseteq  Q \times (\Sigma \cup \{\varepsilon\}) \times Q$ : \structure{la relation de transition}
    \end{description}

    \vspace{1mm}
    Une \structure{transition} est un triplet \alert{$\langle q, a, q' \rangle \in \rightarrow$}, que l'on note \alert{$q\xrightarrow{a} q'$}, tel que :
    \begin{description}[xxxxx]
    \item[\alert{$q$}] $\in Q$ : \structure{l'état de départ}
    \item[\alert{$a$}] $\in \Sigma \cup \{\varepsilon\}$ : \structure{l'étiquette}
    \item[\alert{$q'$}] $\in Q$ : \structure{l'état d'arrivée}
    \end{description}
    
    \vspace{1mm}
    \structure{Remarques : }
    \begin{itemize}
    \item On ne considère que des automates \structure{unitaires} (un seul état initial)
    \item Une $\varepsilon$-transition peut être empruntée sans lire de symbole du mot
    \end{itemize}
  \end{block}
  
  \on[y=-18mm, x=20mm]{
    \begin{tikzpicture}[automaton]
      \state (q)  at (0,0) {$q$}; 
      \state (q1) at (1,0) {$q'$}; 
      \path  (q) edge node {$a$} (q1);
    \end{tikzpicture}
  }
  
\end{frame}

\endgroup
\endinput

% SPDX-License-Identifier: CC-BY-SA-4.0
% Author: Matthieu Perrin
% Part: 
% Section: 
% Sub-section: 
% Frame: 

\begingroup

\begin{frame}{Configurations d'une machine de Turing}

  \onBlock[top=-5mm]{Définition -- Configuration d'une machine de Turing}{
    Soit $M=\langle \Sigma, \Gamma, \blank , Q, q_0, F, \rightarrow \rangle$ une machine de Turing.\\
    Une \structure{configuration} de $M$ est représentée par un triplet \alert{$\langle G, q, D \rangle$} tel que :
    \begin{description}[xxxx]
    \item[\alert{$G$}] $\in \Gamma^\star$ : le mot à \structure{gauche} de la tête (excluant la case sous la tête)
    \item[\alert{$q$}] $\in Q$ : \structure{l'état courant dans la simulation}
    \item[\alert{$D$}] $\in \Gamma^\star$ : le mot à \structure{droite} de la tête (incluant la case sous la tête)
    \end{description}
  }
  
  \onExampleBlock{Exemple -- La configuration {\color{black}$\langle \structure{ba},\alert{1},\example{ab}\rangle $}}{}

  \on[x=-29mm,y=-11mm] {
    \begin{tikzpicture}[tape, x=7mm, y=7mm]
      \cell{}
      \cell[structure]{$b$}  
      \cell[structure]{$a$}  
      \cell[example]  {$a$} \smhead[example]
      \cell[example]  {$b$}  
      \cell{} 
    \end{tikzpicture}
  }
 
  \on[x=29mm,y=-10mm] {
    \begin{tikzpicture}[turingMachine]
      \state[initial  ] (0) at (0,0) {0}; 
      \state[alert    ] (1) at (1,0) {1}; 
      \state[accepting] (2) at (2,0) {2}; 
      
      \path (0) edge[bend left] node {\smTMtransR{a}{b}} (1);
      \path (1) edge[bend left] node {\smTMtransR{a}{b}} (0);
      \path (1) edge            node {\smTMtransL{b}{a}} (2);
    \end{tikzpicture}
  }

  \onBlock[bottom=-1mm]{Remarque -- Modélisation des symboles blancs}{
    L'infinité de symboles blancs est modélisée par la \structure{relation d'équivalence} :\\[-2mm]
    $$\forall n, m\in \mathbb{N},\quad\structure{\langle G, q, D \rangle \alert{\,\simeq\,} \langle \alert{\blank ^n} G, q, D \alert{\blank ^m} \rangle}$$
    On note {\small\alert{$\mathcal{C}_M \eqdef (\Gamma^\star \times Q \times \Gamma^\star)/_\simeq$}} l'\structure{ensemble des configurations de $M$}
  }

  
\end{frame}

\endgroup

% SPDX-License-Identifier: CC-BY-SA-4.0
% Author: Matthieu Perrin
% Part: 
% Section: 
% Sub-section: 
% Frame: 

\begingroup

\begin{frame}{Actions d'une machine de Turing}
  
  \onBlock[top=-4mm]{Définition -- Action}{
    Soit $M=\langle \Sigma, \Gamma, \blank, Q, q_0, F, \rightarrow \rangle$ une machine de Turing. \\
    Les \structure{actions} de $M$ composent une relation binaire \alert{$\leadsto_M$}\footnote{On note \alert{$C \leadsto C'$} si $M$ est clair d'après le contexte.} sur $\mathcal{C}_M$\\
    Pour tous $q, q' \in Q$, $a, b, c \in \Gamma$, et $G, D \in \Gamma^\star$, on a :

    \vspace{-3mm}
    $$
    \begin{array}{@{\langle\,}l@{,\,}c@{,\,}r@{\,\rangle \leadsto_M \langle\,}l@{,\,}c@{,\,}r@{\,\rangle\quad\text{si}\quad}l}
      G           & \structure{q} & \alert{a} D   & G \alert{b} & \structure{q'} &  D   & \structure{q \xrightarrow{\alert{\smTMtransR{a}{b}}} q'}\\
      G \alert{c} & \structure{q} & \alert{a} D   & G           & \structure{q'} & \alert{cb} D   & \structure{q \xrightarrow{\alert{\smTMtransL{a}{b}}} q'}\\
    \end{array}
    $$
    
    On note \alert{$\leadsto_M^\star$} \structure{la fermeture transitive et réflexive de $\leadsto_M$}.
  }

  \onExampleBlock[text,y=-9mm] {Exemple -- $\structure{\langle b,1,aab\rangle} \leadsto \alert{\langle bb,0,ab\rangle}$} {}

  \on[x=-29mm,y=-19mm] {
    \begin{tikzpicture}[tape, x=6mm, y=6mm]
      \cell{}
      \cell{$b$}  
      \cell[structure]{$a$} \smhead[structure] 
      \cell{$a$}
      \cell{$b$} 
      \cell{}
    \end{tikzpicture}
  }

  \on[x=29mm,y=-19mm] {
    \begin{tikzpicture}[tape, x=6mm, y=6mm]
      \cell{}
      \cell{$b$}  
      \cell[alert]{$b$}
      \cell{$a$} \smheadfrom[example]{-1} \smhead[alert]
      \cell{$b$} 
      \cell{}
    \end{tikzpicture}
  }
  
  \on[x=-29mm,y=-32mm] {
    \begin{tikzpicture}[turingMachine]
      \state[initial  ] (0) at (0,0) {0}; 
      \state[structure] (1) at (1,0) {1}; 
      \state[accepting] (2) at (2,0) {2}; 
      
      \path          (0) edge[bend left=15] node {\smTMtransR{a}{b}} (1);
      \path[example] (1) edge[bend left=15] node {\smTMtransR{a}{b}} (0);
      \path          (1) edge               node {\smTMtransL{b}{a}} (2);
    \end{tikzpicture}
  }
  
  \on[x=29mm,y=-32mm] {
    \begin{tikzpicture}[turingMachine]
      \state[initial,alert] (0) at (0,0) {0}; 
      \state[             ] (1) at (1,0) {1}; 
      \state[accepting    ] (2) at (2,0) {2}; 
      
      \path          (0) edge[bend left=15] node {\smTMtransR{a}{b}} (1);
      \path[example] (1) edge[bend left=15] node {\smTMtransR{a}{b}} (0);
      \path          (1) edge               node {\smTMtransL{b}{a}} (2);
    \end{tikzpicture}
  }

  
\end{frame}

\endgroup

% SPDX-License-Identifier: CC-BY-SA-4.0
% Author: Matthieu Perrin
% Part: 
% Section: 
% Sub-section: 
% Frame: 

\begingroup

\begin{frame}{Langage engendré par une grammaire non-restreinte}
  Soient \alert{$\langle \Sigma, \Gamma, S, \rightarrow \rangle$} une grammaire non-restreinte, $u, v \in (\Sigma \cup \Gamma)^\star$ et $w \in \Sigma^\star$.

  \begin{block}{Définitions -- Dérivation}

    \begin{itemize}
    \item On dit que \structure{$u$ se dérive directement en $v$}, noté $\alert{u \vdash v}$, si :

      \vspace{-3mm}
      $$\alert{\exists x, y, \alpha, \beta \in (\Sigma \cup \Gamma)^\star,\quad
        \example{\alpha \rightarrow \beta} \quad\land\quad u = \structure{x} \cdot \example{\alpha} \cdot \structure{y} \quad\land\quad v = \structure{x} \cdot \example{\beta} \cdot \structure{y}}$$

    \item On dit que \structure{$u$ se dérive en $v$} si \alert{$u \vdash^\star v$},\\
      où \alert{$\vdash^\star$ est la fermeture transitive et réflexive de $\vdash$}.

    \end{itemize}
  \end{block}

  \begin{block}{Définitions -- Génération et langage engendré}
    \begin{itemize}
    \item Une \structure{génération de $w$ par $G$} est une dérivation \alert{$S \vdash^\star w$} à partir de $S$.
    \item On dit que \structure{$w$ est généré par $G$} s'il existe une génération de $w$ par $G$. 
    \item Le \structure{langage engendré} par $G$, $\alert{\mathcal{L}(G)}$, contient les mots générés par $G$ :
      $$\alert{\mathcal{L}(G) = \left\{w\in\Sigma^\star \,\middle\mid\, S \vdash^\star w \right\}}.$$
    \end{itemize}
  \end{block}

\end{frame}

\endgroup

 
\subsection{Automates complets et déterministes}
% SPDX-License-Identifier: CC-BY-SA-4.0
% Author: Matthieu Perrin
% Part: 
% Section: 
% Sub-section: 
% Frame: 

\begingroup

\begin{frame}{Automates déterministes et complets}

  Soit $A = \langle \Sigma, Q, q_0, F, \rightarrow \rangle$ un AFN.
  
  \begin{block}{Définition -- Automate fini déterministe (AFD)}
    On dit que $A$ est \structure{déterministe} si toutes les conditions sont vérifiées
    \begin{description}[relation fonctionnelle :]
    \item[$\varepsilon$-liberté :] $A$ ne possède pas d'$\varepsilon$-transition

      \vspace{-2mm}
      $$\alert{\rightarrow\, \subseteq Q \times \Sigma \times Q}$$

    \item[relation fonctionnelle :] pour chaque état $q$ et chaque symbole $a$, il existe \alert{au plus} une transition sortant de $q$ étiquetée $a$

      \vspace{-2mm}
      $$\alert{\forall q, q_1, q_2\in Q,  \forall a\in \Sigma, \left(\structure{q\xrightarrow{a} q_1} \land \structure{q\xrightarrow{a} q_2}\right) \structure{\Rightarrow q_1 = q_2}}$$
    \end{description}
  \end{block}

  \vspace{-2mm}
  \begin{block}{Définition -- Automate fini complet}
    On dit que $A$ est \structure{complet} si la condition suivante est vérifiée
    \begin{description}[relation fonctionnelle :]
    \item[relation totale :] pour chaque état $q$ et chaque symbole $a$, il existe \alert{au moins} une transition sortant de $q$ étiquetée $a$

      \vspace{-2mm}
      $$\alert{\forall q\in Q,  \forall a\in \Sigma, \structure{\exists q' \in Q,~q \xrightarrow{a} q'}}$$
    \end{description}
  \end{block}
  
\end{frame}

\endgroup
\endinput

% SPDX-License-Identifier: CC-BY-SA-4.0
% Author: Matthieu Perrin
% Part: 
% Section: 
% Sub-section: 
% Frame: 

\begingroup

\begin{frame}{Interprétation du non-déterminisme}

  \begin{block}{Plusieurs formes de non-déterminisme}
    \begin{description}
    \item[Aléatoire :] assigner \structure{une probabilité} à chaque transition possible
      \begin{itemize}
      \item Arrivée dans un état final \structure{avec grande probabilité}
      \end{itemize}
    \item[Adversaire :] la transition à emprunter est décidée par \structure{l'environnement}
      \begin{itemize}
      \item Arrivée dans un état final \structure{quel que soit le chemin}
      \end{itemize}
    \item[\alert{Ubiquitaire :}] la transition à emprunter est décidée par \alert{un oracle}
      \begin{itemize}
      \item Arrivée dans un état final \alert{pour au moins un chemin}
      \end{itemize}
    \end{description}
  \end{block}
  
  \begin{block}{Le non-déterminisme comme de l'ubiquité}
    \begin{itemize}
    \item L'automate se trouve dans un sous-ensemble des états
    \item Le mot est reconnu si l'un des états du sous-ensemble est final
    \item Ces sous-ensembles forment un nouvel automate, qui est déterministe
    \end{itemize}
  \end{block}

\end{frame}

\endgroup
\endinput

% SPDX-License-Identifier: CC-BY-SA-4.0
% Author: Matthieu Perrin
% Part: 
% Section: 
% Sub-section: 
% Frame: 

\begingroup

\begin{frame}{Équivalence entre automates à pile et grammaires}

  \onBlock[top=-2mm, left=.6\textwidth]{Théorème de Chomsky et Schützenberger}{
    Un langage est \structure{algébrique} si et seulement s’il est reconnu par un \structure{automate à pile non déterministe}.
  }

  \onImage[top, x=.35\textwidth]{%
    height=25mm,
    title={Marcel-Paul Schützenberger},
    licenselogo={\ccby},
    license={{\href{https://creativecommons.org/licenses/by/2.0/}{CC BY-2.0}} -- Konrad Jacobs, 1972 (\href{https://commons.wikimedia.org/wiki/File:Sch\%C3\%BCtzenberger.jpeg?uselang=fr}{Wikimedia})},
    img={Schutzenberger.jpeg}
  }

 \onExampleBlock<2->[y=0mm]{Exemple : $\{ a^n c b^n \mid n\in \mathbb{N} \}$}{}
 
 \on<2->[bottom=9mm,x=-.45\textwidth]{
   \begin{tikzpicture}[stack, x=7mm, y=7mm]
     \cell[alert ob=<2>]{\oneof[$S$]{\on<3->{$T$}\on<5->{$B$}\on<7->{}}}
     \cell              {\oneof[]{\on<3>{$A$}\on<5>{$S$}}}
   \end{tikzpicture}
 }
 
 \on<2->[bottom=23mm,x=-.3\textwidth]{
   \begin{tikzpicture}[word, x=5mm, y=5mm]
     \cell[alert ob=<4->]{$a$} \smhead[on=<-3>]
     \cell[alert ob=<6->]{$c$} \smhead[ob=<4-5>]
     \cell[alert ob=<7->]{$b$} \smhead[ob=<6>]
   \end{tikzpicture}
 }
 
 \on<2->[bottom=10mm,x=-.15\textwidth]{\small
   \begin{tikzpicture}[pushdown]
     \state[initial above, accepting, initial text={\alertb<2>{$S$}}] (q) {};
     \path (q) edge [loop left] node {
       $\begin{array}{r}
         \alertb<4>{\smPAtrans{a}{A}{\varepsilon}} \\
         \alertb<7>{\smPAtrans{b}{B}{\varepsilon}} \\
         \alertb<6>{\smPAtrans{c}{S}{\varepsilon}} \\
       \end{array}$
     } (q);
     \path (q) edge [loop right] node {
       $\begin{array}{l}
         \alertb<3>{\smPAtrans{\varepsilon}{S}{TA}} \\
         \alertb<5>{\smPAtrans{\varepsilon}{T}{BS}} \\
       \end{array}$
     } (q);
   \end{tikzpicture}
 }
 
 \on<2->[bottom=15mm, x=.2\textwidth, width=2.5cm]{
   \example{Grammaire}\\\vspace{2mm}
   $\left\{\begin{array}{@{\,}r@{~\rightarrow~}l@{\,}}
   S & \alertb<3>{AT} \mid \alertb<6>{c}\\
   T & \alertb<5>{SB}\\
   A & \alertb<4>{a}\\
   B & \alertb<7>{b}\\
   \end{array}\right.$
 }
 
 \on<2->[bottom=10mm,x=.4\textwidth]{
   \begin{tikzpicture}[tree, x=7mm,y=7mm, tree node internal/.append style={structure,}, tree node leave/.append style={example,},]
     \tree{\alertb<2>{$S$}}{
       \tree[on=<3->]{\alertb<3>{$A$}}{
         \tree[on=<4->,yshift=-1]{\alertb<4>{$a$}}{}
       }
       \tree[on=<3->]{\alertb<3>{$T$}}{
         \tree[on=<5->]{\alertb<5>{$S$}}{
           \tree[on=<6->]{\alertb<6>{$c$}}{}
         }
         \tree[on=<5->]{\alertb<5>{$B$}}{
           \tree[on=<7->]{\alertb<7>{$b$}}{}
         }
       }
     }
   \end{tikzpicture}
 }

  \footnoteref{Chomsky, Schützenberger. \emph{The algebraic theory of context-free languages.} SLFM. (1959)}
  
\end{frame}

\endgroup

\input{languages/automata/determinism/algorithm}
 
\subsection{Expressivité des automates finis}
\input{languages/automata/expressiveness/kleene}
\input{languages/automata/expressiveness/formalisms}
 
\section{Décision des langages algébriques}
 
\subsection{Notion d'automate à pile}
% SPDX-License-Identifier: CC-BY-SA-4.0
% Author: Matthieu Perrin
% Part: 
% Section: 
% Sub-section: 
% Frame: 

\begingroup

\begin{frame}{Décision de l'appartenance à un langage contextuel}

  \begin{block}{Problème de décision}
    Soit $L$ un langage contextuel
    
    \Probleme{\structure{Decision$_L$}}{
      Un mot \structure{$u \in \Sigma^\star$}
    }{
      Est-ce que \structure{$u \in L$} ?
    }
  \end{block}

  \begin{block}{Algorithme de recherche ascendante par force brute} 

    \begin{description}
    \item [Entrées :]
      \begin{itemize}
      \item Une grammaire $G$, si possible contextuelle
      \item Un mot $u$
      \end{itemize}

    \item [Sortie :] une réponse booléenne sur \structure{$u \in \mathcal{L}(G)$}
    \item [Teminaison :] garantie si $G$ est contextuelle
    \item [Complexité :] exponentielle par rapport à $|u|$
    \end{description}
  \end{block}
  
\end{frame}

\endgroup
\endinput

% SPDX-License-Identifier: CC-BY-SA-4.0
% Author: Matthieu Perrin
% Part: <Nom de la partie>
% Section: <Nom de la section>
% Sub-section: <Nom de la sous-section>  % (facultatif, laisser vide si non utilisé)
% Frame: <Titre de la slide>

\begingroup

%\begin{frame}{Généralisation des machines de Turing déterministes}
% 
%  Pour simplifier la conception de machines de Turing, on s'autorise parfois 
%  des transitions plus complexes
%  
%  \begin{block}{Opération sans déplacement}
%    \begin{itemize}
%    \item \structure{$q \xrightarrow{\smTMtransS{a}{b}} q'$} lit $a$, écrit $b$, ne déplace pas la tête de lecture%\hspace\fill\alert{$ \langle G, a D \rangle  \rightarrow \langle G, b D \rangle $}
%    \end{itemize}
%  \end{block}
% 
%%    \begin{description}
%%    \item<1->[$q \xrightarrow{\smTMtransS{a}{b}} q'$ :] lit $a$, écrit $b$, ne déplace pas la tête de lecture\hspace\fill\alert{$ \langle G, a D \rangle  \rightarrow \langle G, b D \rangle $}
%%    \item<4->[$q \xrightarrow{\smTMtransP{a}{b}} q'$ :] lit $a$, ajoute $b$ à gauche\hspace\fill\alert{$ \langle G, a D \rangle  \rightarrow \langle G b, a D \rangle $}
%%    \item<9->[$q \xrightarrow{\smTMtransM{a}} q'$ :] lit $a$, supprime $a$\hspace\fill\alert{$ \langle G, a D \rangle  \rightarrow \langle G, D \rangle $}
%%    \end{description}
% 
%  
%  \begin{block}{Opération sur des mots}
%    \item \structure{$q \xrightarrow{\smTMtrans{a}{b}{d}} q'$} a le   lit $a$, écrit $b$, ne déplace pas la tête de lecture%\hspace\fill\alert{$ \langle G, a D \rangle  \rightarrow \langle G, b D \rangle $}
%  \end{block}
% 
%  \begin{block}{Machines de Turing multi-rubans}
%  \end{block}
% 
%\end{frame}

\begin{frame}{Généralisation des machines de Turing}
  Pour simplifier la conception de machines de Turing, on s'autorise parfois des  \structure{transitions généralisées}
  \alert{$\langle \langle q, u_1, ..., u_k \rangle, \langle q', \langle v_1, d_1 \rangle, ..., \langle v_k, d_k \rangle \rangle \rangle$}, telles que :
  \begin{description}[-------]
  \item[\alert{$q$}] $\in Q$ : \structure{l'état de départ}
  \item[\alert{$u_i$}] $\in \Gamma\alert{^\star}$ : \structure{le mot lu sur le ruban \alert{$r_i$}}
  \item[\alert{$q'$}] $\in Q$ : \structure{l'état d'arrivée}
  \item[\alert{$v_i$}] $\in \Gamma\alert{^\star}$ : \structure{le mot écrit sur le ruban \alert{$r_i$}}
  \item[\alert{$d_i$}] $\in \{\triangleleft, \alert{\diamond}, \triangleright\}$ : \structure{le déplacement de la tête de \alert{$r_i$}}
  \end{description}
 
  \on[x=37.5mm,y=10mm]{
    \begin{tikzpicture}[turingMachine, x=25mm]
      \state (q)  at (0,0) {$q$}; 
      \state (q1) at (1,0) {$q'$}; 
      \path  (q) edge node {\smGroup{\smTMtrans[r_1]{u_1}{v_1}{d_1}...\\\smTMtrans[r_k]{u_k}{v_k}{d_k}}} (q1);
    \end{tikzpicture}
  }

  \begin{block}{Nouvelles fonctionnalités}
    \begin{description}[Multi-ruban :]
    \item[Réécriture :] lire un mot \(u\in\Gamma^\star\), écrire un mot \(v\in\Gamma^\star\)
      \begin{itemize}
      \item ajoute/supprime des cases si $|u| \neq |v|$
      \end{itemize}
    \item[Immobilité :] $\diamond$ ne déplace pas la tête de lecture
      \begin{itemize}
      \item $\varepsilon$-transitions possibles : $\smTMtransS{\varepsilon}{\varepsilon}$
      \end{itemize}
    \item[Multi-ruban :] \(k\) rubans parallèles, une tête/opération par ruban (simultané).
      \begin{itemize}
      \item choisir un ruban pour les entrées/sorties
      \end{itemize}
    \end{description}
  \end{block}

\end{frame}




\endgroup

\input{languages/pushdown/intuition/recursion}
 
\subsection{Formalisme}
% SPDX-License-Identifier: CC-BY-SA-4.0
% Author: Matthieu Perrin
% Part: 
% Section: 
% Sub-section: 
% Frame: 

\begingroup

\begin{frame}{Modélisation mathématique}

  \begin{block}{Définition -- Automate fini non-déterministe (AFN)}
    \vspace{1mm}
    Un \structure{automate fini} est un quintuplet \alert{$\langle \Sigma, Q, q_0, F, \rightarrow \rangle$} tel que :
    \begin{description}[xxxxx]
    \item[\alert{$\Sigma$}] $\in \Omega_f$ ensemble fini non vide : \structure{l'alphabet}
    \item[\alert{$Q$}]  $\in \Omega_f$ ensemble fini non vide : \structure{les états}
    \item[\alert{$q_0$}] $\in Q$ : \structure{l'état initial}
    \item[\alert{$F$}] $\subseteq Q$ : \structure{les états finaux (ou accepteurs)}
    \item[\alert{$\rightarrow$}] $\subseteq  Q \times (\Sigma \cup \{\varepsilon\}) \times Q$ : \structure{la relation de transition}
    \end{description}

    \vspace{1mm}
    Une \structure{transition} est un triplet \alert{$\langle q, a, q' \rangle \in \rightarrow$}, que l'on note \alert{$q\xrightarrow{a} q'$}, tel que :
    \begin{description}[xxxxx]
    \item[\alert{$q$}] $\in Q$ : \structure{l'état de départ}
    \item[\alert{$a$}] $\in \Sigma \cup \{\varepsilon\}$ : \structure{l'étiquette}
    \item[\alert{$q'$}] $\in Q$ : \structure{l'état d'arrivée}
    \end{description}
    
    \vspace{1mm}
    \structure{Remarques : }
    \begin{itemize}
    \item On ne considère que des automates \structure{unitaires} (un seul état initial)
    \item Une $\varepsilon$-transition peut être empruntée sans lire de symbole du mot
    \end{itemize}
  \end{block}
  
  \on[y=-18mm, x=20mm]{
    \begin{tikzpicture}[automaton]
      \state (q)  at (0,0) {$q$}; 
      \state (q1) at (1,0) {$q'$}; 
      \path  (q) edge node {$a$} (q1);
    \end{tikzpicture}
  }
  
\end{frame}

\endgroup
\endinput

% SPDX-License-Identifier: CC-BY-SA-4.0
% Author: Matthieu Perrin
% Part: 
% Section: 
% Sub-section: 
% Frame: 

\begingroup

\begin{frame}{Configurations d'une machine de Turing}

  \onBlock[top=-5mm]{Définition -- Configuration d'une machine de Turing}{
    Soit $M=\langle \Sigma, \Gamma, \blank , Q, q_0, F, \rightarrow \rangle$ une machine de Turing.\\
    Une \structure{configuration} de $M$ est représentée par un triplet \alert{$\langle G, q, D \rangle$} tel que :
    \begin{description}[xxxx]
    \item[\alert{$G$}] $\in \Gamma^\star$ : le mot à \structure{gauche} de la tête (excluant la case sous la tête)
    \item[\alert{$q$}] $\in Q$ : \structure{l'état courant dans la simulation}
    \item[\alert{$D$}] $\in \Gamma^\star$ : le mot à \structure{droite} de la tête (incluant la case sous la tête)
    \end{description}
  }
  
  \onExampleBlock{Exemple -- La configuration {\color{black}$\langle \structure{ba},\alert{1},\example{ab}\rangle $}}{}

  \on[x=-29mm,y=-11mm] {
    \begin{tikzpicture}[tape, x=7mm, y=7mm]
      \cell{}
      \cell[structure]{$b$}  
      \cell[structure]{$a$}  
      \cell[example]  {$a$} \smhead[example]
      \cell[example]  {$b$}  
      \cell{} 
    \end{tikzpicture}
  }
 
  \on[x=29mm,y=-10mm] {
    \begin{tikzpicture}[turingMachine]
      \state[initial  ] (0) at (0,0) {0}; 
      \state[alert    ] (1) at (1,0) {1}; 
      \state[accepting] (2) at (2,0) {2}; 
      
      \path (0) edge[bend left] node {\smTMtransR{a}{b}} (1);
      \path (1) edge[bend left] node {\smTMtransR{a}{b}} (0);
      \path (1) edge            node {\smTMtransL{b}{a}} (2);
    \end{tikzpicture}
  }

  \onBlock[bottom=-1mm]{Remarque -- Modélisation des symboles blancs}{
    L'infinité de symboles blancs est modélisée par la \structure{relation d'équivalence} :\\[-2mm]
    $$\forall n, m\in \mathbb{N},\quad\structure{\langle G, q, D \rangle \alert{\,\simeq\,} \langle \alert{\blank ^n} G, q, D \alert{\blank ^m} \rangle}$$
    On note {\small\alert{$\mathcal{C}_M \eqdef (\Gamma^\star \times Q \times \Gamma^\star)/_\simeq$}} l'\structure{ensemble des configurations de $M$}
  }

  
\end{frame}

\endgroup

% SPDX-License-Identifier: CC-BY-SA-4.0
% Author: Matthieu Perrin
% Part: 
% Section: 
% Sub-section: 
% Frame: 

\begingroup

\begin{frame}{Actions d'une machine de Turing}
  
  \onBlock[top=-4mm]{Définition -- Action}{
    Soit $M=\langle \Sigma, \Gamma, \blank, Q, q_0, F, \rightarrow \rangle$ une machine de Turing. \\
    Les \structure{actions} de $M$ composent une relation binaire \alert{$\leadsto_M$}\footnote{On note \alert{$C \leadsto C'$} si $M$ est clair d'après le contexte.} sur $\mathcal{C}_M$\\
    Pour tous $q, q' \in Q$, $a, b, c \in \Gamma$, et $G, D \in \Gamma^\star$, on a :

    \vspace{-3mm}
    $$
    \begin{array}{@{\langle\,}l@{,\,}c@{,\,}r@{\,\rangle \leadsto_M \langle\,}l@{,\,}c@{,\,}r@{\,\rangle\quad\text{si}\quad}l}
      G           & \structure{q} & \alert{a} D   & G \alert{b} & \structure{q'} &  D   & \structure{q \xrightarrow{\alert{\smTMtransR{a}{b}}} q'}\\
      G \alert{c} & \structure{q} & \alert{a} D   & G           & \structure{q'} & \alert{cb} D   & \structure{q \xrightarrow{\alert{\smTMtransL{a}{b}}} q'}\\
    \end{array}
    $$
    
    On note \alert{$\leadsto_M^\star$} \structure{la fermeture transitive et réflexive de $\leadsto_M$}.
  }

  \onExampleBlock[text,y=-9mm] {Exemple -- $\structure{\langle b,1,aab\rangle} \leadsto \alert{\langle bb,0,ab\rangle}$} {}

  \on[x=-29mm,y=-19mm] {
    \begin{tikzpicture}[tape, x=6mm, y=6mm]
      \cell{}
      \cell{$b$}  
      \cell[structure]{$a$} \smhead[structure] 
      \cell{$a$}
      \cell{$b$} 
      \cell{}
    \end{tikzpicture}
  }

  \on[x=29mm,y=-19mm] {
    \begin{tikzpicture}[tape, x=6mm, y=6mm]
      \cell{}
      \cell{$b$}  
      \cell[alert]{$b$}
      \cell{$a$} \smheadfrom[example]{-1} \smhead[alert]
      \cell{$b$} 
      \cell{}
    \end{tikzpicture}
  }
  
  \on[x=-29mm,y=-32mm] {
    \begin{tikzpicture}[turingMachine]
      \state[initial  ] (0) at (0,0) {0}; 
      \state[structure] (1) at (1,0) {1}; 
      \state[accepting] (2) at (2,0) {2}; 
      
      \path          (0) edge[bend left=15] node {\smTMtransR{a}{b}} (1);
      \path[example] (1) edge[bend left=15] node {\smTMtransR{a}{b}} (0);
      \path          (1) edge               node {\smTMtransL{b}{a}} (2);
    \end{tikzpicture}
  }
  
  \on[x=29mm,y=-32mm] {
    \begin{tikzpicture}[turingMachine]
      \state[initial,alert] (0) at (0,0) {0}; 
      \state[             ] (1) at (1,0) {1}; 
      \state[accepting    ] (2) at (2,0) {2}; 
      
      \path          (0) edge[bend left=15] node {\smTMtransR{a}{b}} (1);
      \path[example] (1) edge[bend left=15] node {\smTMtransR{a}{b}} (0);
      \path          (1) edge               node {\smTMtransL{b}{a}} (2);
    \end{tikzpicture}
  }

  
\end{frame}

\endgroup

% SPDX-License-Identifier: CC-BY-SA-4.0
% Author: Matthieu Perrin
% Part: 
% Section: 
% Sub-section: 
% Frame: 

\begingroup

\begin{frame}{Langage engendré par une grammaire non-restreinte}
  Soient \alert{$\langle \Sigma, \Gamma, S, \rightarrow \rangle$} une grammaire non-restreinte, $u, v \in (\Sigma \cup \Gamma)^\star$ et $w \in \Sigma^\star$.

  \begin{block}{Définitions -- Dérivation}

    \begin{itemize}
    \item On dit que \structure{$u$ se dérive directement en $v$}, noté $\alert{u \vdash v}$, si :

      \vspace{-3mm}
      $$\alert{\exists x, y, \alpha, \beta \in (\Sigma \cup \Gamma)^\star,\quad
        \example{\alpha \rightarrow \beta} \quad\land\quad u = \structure{x} \cdot \example{\alpha} \cdot \structure{y} \quad\land\quad v = \structure{x} \cdot \example{\beta} \cdot \structure{y}}$$

    \item On dit que \structure{$u$ se dérive en $v$} si \alert{$u \vdash^\star v$},\\
      où \alert{$\vdash^\star$ est la fermeture transitive et réflexive de $\vdash$}.

    \end{itemize}
  \end{block}

  \begin{block}{Définitions -- Génération et langage engendré}
    \begin{itemize}
    \item Une \structure{génération de $w$ par $G$} est une dérivation \alert{$S \vdash^\star w$} à partir de $S$.
    \item On dit que \structure{$w$ est généré par $G$} s'il existe une génération de $w$ par $G$. 
    \item Le \structure{langage engendré} par $G$, $\alert{\mathcal{L}(G)}$, contient les mots générés par $G$ :
      $$\alert{\mathcal{L}(G) = \left\{w\in\Sigma^\star \,\middle\mid\, S \vdash^\star w \right\}}.$$
    \end{itemize}
  \end{block}

\end{frame}

\endgroup

% SPDX-License-Identifier: CC-BY-SA-4.0
% Author: Matthieu Perrin
% Part: <Nom de la partie>
% Section: <Nom de la section>
% Sub-section: <Nom de la sous-section>  % (facultatif, laisser vide si non utilisé)
% Frame: <Titre de la slide>

\begingroup

\begin{frame}{Exemple : $aab \in \mathcal{L}(ab|aab)$ ?}

  \begin{tikzpicture}

    \draw (5,10) node[below]{\begin{minipage}{\textwidth}
        \begin{block}{Machine non-déterministe $M$ :}
          \scalebox{.8}{\begin{tikzpicture}[shorten >=1pt, node distance=1.5cm, on grid, auto]
              \node (nq0)                   {};
              \node (nq1) [above right of=nq0] {};
              \node (nq2) [right of=nq1]    {};
              \node (nq3) [below right of=nq0] {};
              \node (nq4) [right of=nq3]    {};
              \node (nq5) [right of=nq4]    {};

              \node[state, initial, initial text=] (q0) at (nq0) {$q_0$};
              \node[state]                         (q1) at (nq1) {$q_1$};
              \node[state, accepting]              (q2) at (nq2) {$q_2$};
              \node[state]                         (q3) at (nq3) {$q_3$};
              \node[state]                         (q4) at (nq4) {$q_4$};
              \node[state, accepting]              (q5) at (nq5) {$q_5$};

              \node<2-7>[fill=alert!20, state, initial, initial text=] (q0) at (nq0) {$q_0$};
              \node<10>[fill=alert!20, state]                          (q1) at (nq1) {$q_1$};
              \node<8-9>[fill=alert!20, state]                       (q3) at (nq3) {$q_3$};
              \node<11>[fill=alert!20, state]                          (q4) at (nq4) {$q_4$};
              \node<11>[fill=structure!20, state, accepting]                (q5) at (nq5) {$q_5$};


              \path<-3,5-> [->] (q0)       edge[bend left] node {$\smTMtransR{a}{a}$} (q1);
              \path<-4,6-> [->] (q1)       edge[bend left] node {$\smTMtransR{b}{b}$} (q2);
              %              \path<-5,7-> [->] (q2)       edge[bend left] node {$\smTMtransR{a}{a}$} (q1);
              \path<-2,4-> [->] (q0)       edge[bend right] node[swap] {$\smTMtransR{a}{a}$} (q3);
              \path<-5,7-8,10-> [->] (q3) edge[bend left] node {$\smTMtransR{a}{a}$} (q4);
              \path<-6,8-10> [->]       (q4)       edge[bend left] node {$\smTMtransR{b}{b}$} (q5);
              %              \path<-8,10-> [->]      (q5)      edge[bend left] node {$\smTMtransR{a}{a}$} (q3);


              \path<4> [alert, ->] (q0)   edge[bend left] node {$\smTMtransR{a}{a}$} (q1);
              \path<5> [alert, ->] (q1)   edge[bend left] node {$\smTMtransR{b}{b}$} (q2);
              %              \path<6> [alert, ->] (q2)   edge[bend left] node {$\smTMtransR{a}{a}$} (q1);
              \path<3> [alert, ->] (q0)   edge[bend right] node[swap] {$\smTMtransR{a}{a}$} (q3);
              \path<6,9> [alert, ->] (q3)   edge[bend left] node {$\smTMtransR{a}{a}$} (q4);
              \path<7,11> [alert, ->] (q4)   edge[bend left] node {$\smTMtransR{b}{b}$} (q5);
              %              \path<9> [alert, ->] (q5)   edge[bend left] node {$\smTMtransR{a}{a}$} (q3);

          \end{tikzpicture}}
        \end{block}
    \end{minipage}};


    \draw (7.5,10) node[below]{\begin{minipage}{.3\textwidth}\begin{block}{Configuration :}\end{block}\end{minipage}};
    \draw (6,8.5) node[right]{\structure{$f$ : }~
      \only<-2>{$\varepsilon$}%
      \only<3-7>{$\mid a\, q_3\, ab$}%
      \only<4-9>{$\mid a\, q_1\, ab$}%
      \only<9-10>{$\mid aa\, q_4\, b$}%
      \only<11>{$\varepsilon$}%
      %      \only<11>{$\mid aab\, q_5$}%
    };
    \draw (6,8) node[right]{\structure{$c$ : }~
      \only<1>{$aab$}%
      \only<2-7>{$\alert{q_0\, a}ab$}%
      \only<8-9>{$a\, \alert{q_3\, a}b$}%
      \only<10>{$a\, \alert{q_1\, a}b$}%
      \only<11>{$aa\, \alert{q_4\, b}$}%
    };
    \draw<11> (6,7.5) node[right]{\structure{Le mot $aab$ est accepté}};

    \draw (5,4) node{\begin{minipage}{\textwidth}
        \begin{block}{Machine déterminisée $M_D$ :}
          \scalebox{.8}{\begin{tikzpicture}[shorten >=1pt, node distance=3cm, on grid, auto]
              \tikzset{mynode/.style={draw, rounded corners=8, align=center, minimum height=0.6cm, minimum width=1cm}}

              \node (nt0)                       {};
              \node (ninit)   [left of=nt0]     {};
              \node (nt1)     [right of=nt0]    {};
              \node (nt2)     [right of=nt1]    {};
              \node (nt3)     [right of=nt2]    {};
              \node (nreinit) at (0,-1.5)       {};
              \node (nt6)     [right of=nreinit]{};
              \node (nt5)     [right of=nt6]    {};
              \node (nt4)     [right of=nt5]    {};

              \node[fill=example!20, mynode]         (init)   at (ninit)    {\footnotesize init};
              \node[fill=structure!20, mynode]            (t0)     at (nt0)      {\footnotesize $q_0 \xrightarrow{\smTMtransR{a}{a}} q_3$};
              \node[fill=structure!20, mynode]            (t1)     at (nt1)      {\footnotesize $q_0 \xrightarrow{\smTMtransR{a}{a}} q_1$};
              \node[fill=structure!20, mynode, accepting] (t2)     at (nt2)      {\footnotesize $q_1 \xrightarrow{\smTMtransR{b}{b}} q_2$};
              \node[fill=structure!20, mynode]            (t4)     at (nt5)      {\footnotesize $q_3 \xrightarrow{\smTMtransR{a}{a}} q_4$};
              \node[fill=structure!20, mynode, accepting] (t5)     at (nt6)      {\footnotesize $q_4 \xrightarrow{\smTMtransR{b}{b}} q_5$};
              \node[fill=example!20, mynode]         (reinit) at (nreinit)  {\footnotesize reinit};

              \node<2>[fill=alert!20, mynode]         (init)   at (ninit)       {\footnotesize init};
              \node<3>[fill=alert!20, mynode]            (t0)     at (nt0)      {\footnotesize $q_0 \xrightarrow{\smTMtransR{a}{a}} q_3$};
              \node<4>[fill=alert!20, mynode]            (t1)     at (nt1)      {\footnotesize $q_0 \xrightarrow{\smTMtransR{a}{a}} q_1$};
              \node<5>[fill=alert!20, mynode, accepting] (t2)     at (nt2)      {\footnotesize $q_1 \xrightarrow{\smTMtransR{b}{b}} q_2$};
              \node<6,9>[fill=alert!20, mynode]            (t4)     at (nt5)   {\footnotesize $q_3 \xrightarrow{\smTMtransR{a}{a}} q_4$};
              \node<7,11>[fill=alert!20, mynode, accepting] (t5)     at (nt6)      {\footnotesize $q_4 \xrightarrow{\smTMtransR{b}{b}} q_5$};
              \node<8,10>[fill=alert!20, mynode]         (reinit) at (nreinit) {\footnotesize reinit};

              \path [->] (init)   edge node[swap] {\scriptsize$c:\smTMtransR{q}{q}$} (t0);
              \path [->] (t0)     edge node[swap] {\scriptsize$c:\smTMtransR{q}{q}$} (t1);
              \path [->] (t1)     edge node[swap] {\scriptsize$c:\smTMtransR{q}{q}$} (t2);
              \path [->] (t2)     edge node[swap] {\scriptsize$c:\smTMtransR{q}{q}$} (t4);
              \path [->] (t4)     edge node[swap] {\scriptsize$c:\smTMtransR{q}{q}$} (t5);
              \path [->] (t5)     edge node[swap] {\scriptsize$c:\smTMtransR{q}{q}$} (reinit);
              \path [->] (reinit) edge node[swap] {\scriptsize$c:\smTMtransR{q}{q}$} (t0);
          \end{tikzpicture}}
        \end{block}
    \end{minipage}};

  \end{tikzpicture}

\end{frame}

\endgroup

% SPDX-License-Identifier: CC-BY-SA-4.0
% Author: Matthieu Perrin
% Part: 
% Section: 
% Sub-section: 
% Frame: 

\begingroup

\begin{frame}{Équivalence entre automates à pile et grammaires}

  \onBlock[top=-2mm, left=.6\textwidth]{Théorème de Chomsky et Schützenberger}{
    Un langage est \structure{algébrique} si et seulement s’il est reconnu par un \structure{automate à pile non déterministe}.
  }

  \onImage[top, x=.35\textwidth]{%
    height=25mm,
    title={Marcel-Paul Schützenberger},
    licenselogo={\ccby},
    license={{\href{https://creativecommons.org/licenses/by/2.0/}{CC BY-2.0}} -- Konrad Jacobs, 1972 (\href{https://commons.wikimedia.org/wiki/File:Sch\%C3\%BCtzenberger.jpeg?uselang=fr}{Wikimedia})},
    img={Schutzenberger.jpeg}
  }

 \onExampleBlock<2->[y=0mm]{Exemple : $\{ a^n c b^n \mid n\in \mathbb{N} \}$}{}
 
 \on<2->[bottom=9mm,x=-.45\textwidth]{
   \begin{tikzpicture}[stack, x=7mm, y=7mm]
     \cell[alert ob=<2>]{\oneof[$S$]{\on<3->{$T$}\on<5->{$B$}\on<7->{}}}
     \cell              {\oneof[]{\on<3>{$A$}\on<5>{$S$}}}
   \end{tikzpicture}
 }
 
 \on<2->[bottom=23mm,x=-.3\textwidth]{
   \begin{tikzpicture}[word, x=5mm, y=5mm]
     \cell[alert ob=<4->]{$a$} \smhead[on=<-3>]
     \cell[alert ob=<6->]{$c$} \smhead[ob=<4-5>]
     \cell[alert ob=<7->]{$b$} \smhead[ob=<6>]
   \end{tikzpicture}
 }
 
 \on<2->[bottom=10mm,x=-.15\textwidth]{\small
   \begin{tikzpicture}[pushdown]
     \state[initial above, accepting, initial text={\alertb<2>{$S$}}] (q) {};
     \path (q) edge [loop left] node {
       $\begin{array}{r}
         \alertb<4>{\smPAtrans{a}{A}{\varepsilon}} \\
         \alertb<7>{\smPAtrans{b}{B}{\varepsilon}} \\
         \alertb<6>{\smPAtrans{c}{S}{\varepsilon}} \\
       \end{array}$
     } (q);
     \path (q) edge [loop right] node {
       $\begin{array}{l}
         \alertb<3>{\smPAtrans{\varepsilon}{S}{TA}} \\
         \alertb<5>{\smPAtrans{\varepsilon}{T}{BS}} \\
       \end{array}$
     } (q);
   \end{tikzpicture}
 }
 
 \on<2->[bottom=15mm, x=.2\textwidth, width=2.5cm]{
   \example{Grammaire}\\\vspace{2mm}
   $\left\{\begin{array}{@{\,}r@{~\rightarrow~}l@{\,}}
   S & \alertb<3>{AT} \mid \alertb<6>{c}\\
   T & \alertb<5>{SB}\\
   A & \alertb<4>{a}\\
   B & \alertb<7>{b}\\
   \end{array}\right.$
 }
 
 \on<2->[bottom=10mm,x=.4\textwidth]{
   \begin{tikzpicture}[tree, x=7mm,y=7mm, tree node internal/.append style={structure,}, tree node leave/.append style={example,},]
     \tree{\alertb<2>{$S$}}{
       \tree[on=<3->]{\alertb<3>{$A$}}{
         \tree[on=<4->,yshift=-1]{\alertb<4>{$a$}}{}
       }
       \tree[on=<3->]{\alertb<3>{$T$}}{
         \tree[on=<5->]{\alertb<5>{$S$}}{
           \tree[on=<6->]{\alertb<6>{$c$}}{}
         }
         \tree[on=<5->]{\alertb<5>{$B$}}{
           \tree[on=<7->]{\alertb<7>{$b$}}{}
         }
       }
     }
   \end{tikzpicture}
 }

  \footnoteref{Chomsky, Schützenberger. \emph{The algebraic theory of context-free languages.} SLFM. (1959)}
  
\end{frame}

\endgroup

 
\subsection{Automates à pile déterministes}
% SPDX-License-Identifier: CC-BY-SA-4.0
% Author: Matthieu Perrin
% Part: 
% Section: 
% Sub-section: 
% Frame: 

\begingroup

\begin{frame}{Automates déterministes et complets}

  Soit $A = \langle \Sigma, Q, q_0, F, \rightarrow \rangle$ un AFN.
  
  \begin{block}{Définition -- Automate fini déterministe (AFD)}
    On dit que $A$ est \structure{déterministe} si toutes les conditions sont vérifiées
    \begin{description}[relation fonctionnelle :]
    \item[$\varepsilon$-liberté :] $A$ ne possède pas d'$\varepsilon$-transition

      \vspace{-2mm}
      $$\alert{\rightarrow\, \subseteq Q \times \Sigma \times Q}$$

    \item[relation fonctionnelle :] pour chaque état $q$ et chaque symbole $a$, il existe \alert{au plus} une transition sortant de $q$ étiquetée $a$

      \vspace{-2mm}
      $$\alert{\forall q, q_1, q_2\in Q,  \forall a\in \Sigma, \left(\structure{q\xrightarrow{a} q_1} \land \structure{q\xrightarrow{a} q_2}\right) \structure{\Rightarrow q_1 = q_2}}$$
    \end{description}
  \end{block}

  \vspace{-2mm}
  \begin{block}{Définition -- Automate fini complet}
    On dit que $A$ est \structure{complet} si la condition suivante est vérifiée
    \begin{description}[relation fonctionnelle :]
    \item[relation totale :] pour chaque état $q$ et chaque symbole $a$, il existe \alert{au moins} une transition sortant de $q$ étiquetée $a$

      \vspace{-2mm}
      $$\alert{\forall q\in Q,  \forall a\in \Sigma, \structure{\exists q' \in Q,~q \xrightarrow{a} q'}}$$
    \end{description}
  \end{block}
  
\end{frame}

\endgroup
\endinput

% SPDX-License-Identifier: CC-BY-SA-4.0
% Author: Matthieu Perrin
% Part: <Nom de la partie>
% Section: <Nom de la section>
% Sub-section: <Nom de la sous-section>  % (facultatif, laisser vide si non utilisé)
% Frame: <Titre de la slide>

\begingroup

\begin{frame}{Déterminisation d'une machine de Turing}
  
  \onBlock[top=-3mm]{Théorème -- Équivalence entre MTND et MTD}{
    Soit $\Sigma$ un alphabet, et $L \subseteq \Sigma^\star$ un langage sur $\Sigma$. 
    \begin{itemize}
    \item Si $L$ est reconnaissable, alors $L$ est semi-décidable.
    \item Si $L$ est généré par une MTND, alors $L$ est récursivement énumérable.
    \end{itemize}
  }

  \obBlock<1>[anchor=north, y=12mm]{Conséquence}{
    Les cinq notions suivantes sont équivalentes :
    \begin{itemize}
    \item $L$ est reconnaissable
    \item $L$ est semi-décidable
    \item $L$ est généré par une MTND
    \item $L$ est engendré par une grammaire
    \item $L$ est récursivement énumérable
    \end{itemize}
  }

  \onBlock<2>[anchor=north, y=12mm]{Démonstration}{
    Soit $M$ une MTND reconnaissant $L$.\\
    On construit une MTD $M_D$ reconnaissant $L$.
    \begin{itemize}
    \item On considère l'arbre d'exécution pour un mot $u$
      \begin{itemize}
      \item La racine est la configuration initiale $C_{\mathit{init}}(u)$
      \item Les fils de $c$ sont les $c'$ telles que $c\leadsto_M c'$
        \begin{itemize}
        \item Le nombre de fils de $c$ est borné par $|\rightarrow|$
        \end{itemize}
      \end{itemize}
    \item $M_D$  cherche une configuration acceptante
      \begin{itemize}
      \item Une exploration en profondeur ne fonctionne pas :
        \begin{itemize}
        \item certaines branches peuvent être infinies
        \end{itemize}
      \item $M_D$ exécute une exploration en largeur
        \begin{itemize}
        \item Nécessite une file (FIFO) de configurations
        \end{itemize}
      \end{itemize}
    \end{itemize}
  }

  \on<2>[x=40mm,y=5mm] {
    \begin{tikzpicture}[turingMachine, example, x=15mm]
      \state[accepting]     (2) at (0,0) {$2$}; 
      \state[initial above] (0) at (1,0) {$0$}; 
      \state                (1) at (2,0) {$1$}; 

      \path (0) edge[bend left=5mm] node       {$\smTMtransR{a}{a}$} (1);
      \path (1) edge[bend left=5mm] node       {$\smTMtransL{a}{a}$} (0);
      \path (0) edge                node[swap] {$\smTMtransR{a}{a}$} (2);
    \end{tikzpicture}
  }
  
  \on<2>[x=42mm,y=-23mm] {
    \begin{tikzpicture}[tree, example, x=17mm, y=10mm]\small
      \tree[edges=leadsto, xshift=-.5]{$\langle \langle \varepsilon, aa \rangle, 0 \rangle$}{
        \tree{$\langle \langle a, a \rangle, 1 \rangle$}{
          \tree[xshift=-.5]{$\langle \langle \varepsilon, aa \rangle, 0 \rangle$}{
            \tree{$\langle \langle a, a \rangle, 1 \rangle$}{}           
            \tree{$\langle \langle a, a \rangle, 2 \rangle$}{}            
          }
        }           
        \tree[xshift=-1]{$\langle \langle a, a \rangle, 2 \rangle$}{}
      }
    \end{tikzpicture}
  }

\end{frame}

\endgroup

\input{languages/pushdown/determinism/nondeterministic}


\part{Machines de Turing}


\section{Notion de machine de Turing}
 
\subsection{Introduction aux machines de Turing}
% SPDX-License-Identifier: CC-BY-SA-4.0
% Author: Matthieu Perrin
% Part: 
% Section: 
% Sub-section: 
% Frame: 

\begingroup

\begin{frame}{Décision de l'appartenance à un langage contextuel}

  \begin{block}{Problème de décision}
    Soit $L$ un langage contextuel
    
    \Probleme{\structure{Decision$_L$}}{
      Un mot \structure{$u \in \Sigma^\star$}
    }{
      Est-ce que \structure{$u \in L$} ?
    }
  \end{block}

  \begin{block}{Algorithme de recherche ascendante par force brute} 

    \begin{description}
    \item [Entrées :]
      \begin{itemize}
      \item Une grammaire $G$, si possible contextuelle
      \item Un mot $u$
      \end{itemize}

    \item [Sortie :] une réponse booléenne sur \structure{$u \in \mathcal{L}(G)$}
    \item [Teminaison :] garantie si $G$ est contextuelle
    \item [Complexité :] exponentielle par rapport à $|u|$
    \end{description}
  \end{block}
  
\end{frame}

\endgroup
\endinput

% SPDX-License-Identifier: CC-BY-SA-4.0
% Author: Matthieu Perrin
% Part: 
% Section: 
% Sub-section: 
% Frame: 

\begingroup

\begin{frame}{Comment modéliser la mémoire ?}

  \begin{block}{Modèle de processus dans les OS}
    \begin{itemize}
    \item \structure{Mémoire disponible : }
      \begin{itemize}
      \item Un tableau de $2^N$ octets.
      \item Des pointeurs sur $N$ bits.
        \begin{itemize}
        \item Typiquement, $N=32$ ou $N=64$.
        \end{itemize}
      \end{itemize}
    \item \alert{Problème :} 
      \begin{itemize}
      \item En théorie, un automate fini à $2^{2^N}$ états
        \begin{itemize}
        \item En pratique, $N$ fini et $2^N$ infini
        \end{itemize}
      \end{itemize}
    \end{itemize}
  \end{block}
  \pause
  \begin{block}{Modèle RAM (\emph{Random access machine})}
    \begin{itemize}
    \item \structure{Mémoire disponible : }
      \begin{itemize}
      \item Un tableau infini d'entiers non-bornés.
      \end{itemize}
    \item \alert{Problème :} 
      \begin{itemize}
      \item Il faut des opérations sur les entiers dans le modèle
        \begin{itemize}
        \item Ces opérations ne sont-elles pas déjà des algorithmes ?
        \item On veut un \structure{ensemble de règles fini}.
        \end{itemize}
      \end{itemize}
    \end{itemize}
  \end{block}
\end{frame}

\endgroup

% SPDX-License-Identifier: CC-BY-SA-4.0
% Author: Matthieu Perrin
% Part: 
% Section: 
% Sub-section: 
% Frame: 

\begingroup

\begin{frame}{Un automate à ruban}

  \on[text,top]{
    \begin{itemize}
    \item Une \structure{machine de Turing} est un \alert{automate} à \alert{ruban}.
      \begin{itemize}
      \item en : ``\structure{\textit{Infinite tape}}'' = fr : \og \structure{\textit{ruban infini}} \fg
      \item en : ``\structure{\textit{Magnetic tape}}'' = fr : \og \structure{\textit{bande magnétique}} \fg
      \end{itemize}
    \end{itemize}
  }

  \onImage[y=4mm]{%
    height=2.2cm,
    title={Cassette VHS},
    license={{}\ccbysa{} \href{https://creativecommons.org/licenses/by-sa/4.0/}{CC BY-SA 4.0} -- Toby Hudson 2012 -- \href{https://commons.wikimedia.org/wiki/File:VHS_cassette_tape_12.JPG}{Wikimedia}},
    img={vhs.jpg}
  }
  
  \on[text,bottom=3mm]{
    \begin{itemize}
    \item Exécution du \structure{contrôleur} :
      \begin{itemize}
      \item Un \structure{automate fini} décide des opérations sur la mémoire
      \item L'exécution dure aussi longtemps qu'il y a des actions possibles
      \end{itemize}
    \item Opérations sur le \structure{ruban} :
      \begin{itemize}
      \item \alert{Lire/écrire} à la position de la tête de lecture/écriture
      \item \alert{Déplacer} la tête de lecture/écriture vers la gauche ou vers la droite
      \item Pour simplifier : une lecture, une écriture, et un déplacement par étape
      \end{itemize}
    \end{itemize}
  }
  
\end{frame}

\endgroup

% SPDX-License-Identifier: CC-BY-SA-4.0
% Author: Matthieu Perrin
% Part: 
% Section: 
% Sub-section: 
% Frame: 

\begingroup

\begin{frame}{Exemple : le \og busy beaver \fg}
  
  \on[top=4mm] {
    \begin{tikzpicture}[tape]
      \cell{0}
      \cell[alert ob=<{6}>       ]{\alt<-6> {0}{1}} \smheadb<6>
      \cell[alert ob=<{5,7}>     ]{\alt<-5> {0}{1}} \smheadb<5,7>
      \cell[alert ob=<{4,8}>     ]{\alt<-4> {0}{1}} \smheadb<4,8>
      \cell[alert on=<{1,3,9,13}>]{\alt<-1> {0}{1}} \smhead<1,3,9,13>
      \cell[alert ob=<{2,10,12}> ]{\alt<-2> {0}{1}} \smheadb<2,10,12>
      \cell[alert ob=<{11}>      ]{\alt<-11>{0}{1}} \smheadb<11>
      \cell{0}
    \end{tikzpicture}
  }
  
  \on[y=-6mm]{
    \begin{tikzpicture}[turingMachine, x=13mm, y=20mm]
      \state[alert on=<{1,3,6,12}>, initial] (a) at (0, 1) {$A$};
      \state[alert ob=<{2,5,7-11}>         ] (b) at (2, 1) {$B$};
      \state[alert ob=<{4,13}>,   accepting] (c) at (1, 0) {$C$};
      
      \path[alert on=<{1,6}>   ] (a) edge[bend left]  node       {\smTMtransR{0}{1}} (b);
      \path[alert ob=<{3,12}>  ] (a) edge[bend right] node[swap] {\smTMtransL{1}{1}} (c);
      \path[alert ob=<{2,5,11}>] (b) edge[bend left]  node       {\smTMtransL{0}{1}} (a);
      \path[alert ob=<{7-10}>  ] (b) edge[loop right] node       {\smTMtransR{1}{1}} (b);
      \path[alert ob=<{4}>     ] (c) edge[bend right] node[swap] {\smTMtransL{0}{1}} (b);
    \end{tikzpicture}
  }
 
  \on[text,bottom] {\footnotesize
    \begin{description}
    \item[$A \xrightarrow{\smTMtransR{0}{1}} B$ :] si on lit $0$ dans l'état $A$, écrire $1$, aller en $B$ et se déplacer à droite\vspace{-2mm}
    \item[$B \xrightarrow{\smTMtransL{0}{1}} A$ :] si on lit $0$ dans l'état $B$, écrire $1$, aller en $A$ et se déplacer à gauche
    \end{description}
  }

\end{frame}

\endgroup

 
\subsection{Définition des machines de Turing}
\input{turing_machine/introduction/definition/ndtm}
% SPDX-License-Identifier: CC-BY-SA-4.0
% Author: Matthieu Perrin
% Part: 
% Section: 
% Sub-section: 
% Frame: 

\begingroup

\begin{frame}{Configurations d'une machine de Turing}

  \onBlock[top=-5mm]{Définition -- Configuration d'une machine de Turing}{
    Soit $M=\langle \Sigma, \Gamma, \blank , Q, q_0, F, \rightarrow \rangle$ une machine de Turing.\\
    Une \structure{configuration} de $M$ est représentée par un triplet \alert{$\langle G, q, D \rangle$} tel que :
    \begin{description}[xxxx]
    \item[\alert{$G$}] $\in \Gamma^\star$ : le mot à \structure{gauche} de la tête (excluant la case sous la tête)
    \item[\alert{$q$}] $\in Q$ : \structure{l'état courant dans la simulation}
    \item[\alert{$D$}] $\in \Gamma^\star$ : le mot à \structure{droite} de la tête (incluant la case sous la tête)
    \end{description}
  }
  
  \onExampleBlock{Exemple -- La configuration {\color{black}$\langle \structure{ba},\alert{1},\example{ab}\rangle $}}{}

  \on[x=-29mm,y=-11mm] {
    \begin{tikzpicture}[tape, x=7mm, y=7mm]
      \cell{}
      \cell[structure]{$b$}  
      \cell[structure]{$a$}  
      \cell[example]  {$a$} \smhead[example]
      \cell[example]  {$b$}  
      \cell{} 
    \end{tikzpicture}
  }
 
  \on[x=29mm,y=-10mm] {
    \begin{tikzpicture}[turingMachine]
      \state[initial  ] (0) at (0,0) {0}; 
      \state[alert    ] (1) at (1,0) {1}; 
      \state[accepting] (2) at (2,0) {2}; 
      
      \path (0) edge[bend left] node {\smTMtransR{a}{b}} (1);
      \path (1) edge[bend left] node {\smTMtransR{a}{b}} (0);
      \path (1) edge            node {\smTMtransL{b}{a}} (2);
    \end{tikzpicture}
  }

  \onBlock[bottom=-1mm]{Remarque -- Modélisation des symboles blancs}{
    L'infinité de symboles blancs est modélisée par la \structure{relation d'équivalence} :\\[-2mm]
    $$\forall n, m\in \mathbb{N},\quad\structure{\langle G, q, D \rangle \alert{\,\simeq\,} \langle \alert{\blank ^n} G, q, D \alert{\blank ^m} \rangle}$$
    On note {\small\alert{$\mathcal{C}_M \eqdef (\Gamma^\star \times Q \times \Gamma^\star)/_\simeq$}} l'\structure{ensemble des configurations de $M$}
  }

  
\end{frame}

\endgroup

% SPDX-License-Identifier: CC-BY-SA-4.0
% Author: Matthieu Perrin
% Part: 
% Section: 
% Sub-section: 
% Frame: 

\begingroup

\begin{frame}{Actions d'une machine de Turing}
  
  \onBlock[top=-4mm]{Définition -- Action}{
    Soit $M=\langle \Sigma, \Gamma, \blank, Q, q_0, F, \rightarrow \rangle$ une machine de Turing. \\
    Les \structure{actions} de $M$ composent une relation binaire \alert{$\leadsto_M$}\footnote{On note \alert{$C \leadsto C'$} si $M$ est clair d'après le contexte.} sur $\mathcal{C}_M$\\
    Pour tous $q, q' \in Q$, $a, b, c \in \Gamma$, et $G, D \in \Gamma^\star$, on a :

    \vspace{-3mm}
    $$
    \begin{array}{@{\langle\,}l@{,\,}c@{,\,}r@{\,\rangle \leadsto_M \langle\,}l@{,\,}c@{,\,}r@{\,\rangle\quad\text{si}\quad}l}
      G           & \structure{q} & \alert{a} D   & G \alert{b} & \structure{q'} &  D   & \structure{q \xrightarrow{\alert{\smTMtransR{a}{b}}} q'}\\
      G \alert{c} & \structure{q} & \alert{a} D   & G           & \structure{q'} & \alert{cb} D   & \structure{q \xrightarrow{\alert{\smTMtransL{a}{b}}} q'}\\
    \end{array}
    $$
    
    On note \alert{$\leadsto_M^\star$} \structure{la fermeture transitive et réflexive de $\leadsto_M$}.
  }

  \onExampleBlock[text,y=-9mm] {Exemple -- $\structure{\langle b,1,aab\rangle} \leadsto \alert{\langle bb,0,ab\rangle}$} {}

  \on[x=-29mm,y=-19mm] {
    \begin{tikzpicture}[tape, x=6mm, y=6mm]
      \cell{}
      \cell{$b$}  
      \cell[structure]{$a$} \smhead[structure] 
      \cell{$a$}
      \cell{$b$} 
      \cell{}
    \end{tikzpicture}
  }

  \on[x=29mm,y=-19mm] {
    \begin{tikzpicture}[tape, x=6mm, y=6mm]
      \cell{}
      \cell{$b$}  
      \cell[alert]{$b$}
      \cell{$a$} \smheadfrom[example]{-1} \smhead[alert]
      \cell{$b$} 
      \cell{}
    \end{tikzpicture}
  }
  
  \on[x=-29mm,y=-32mm] {
    \begin{tikzpicture}[turingMachine]
      \state[initial  ] (0) at (0,0) {0}; 
      \state[structure] (1) at (1,0) {1}; 
      \state[accepting] (2) at (2,0) {2}; 
      
      \path          (0) edge[bend left=15] node {\smTMtransR{a}{b}} (1);
      \path[example] (1) edge[bend left=15] node {\smTMtransR{a}{b}} (0);
      \path          (1) edge               node {\smTMtransL{b}{a}} (2);
    \end{tikzpicture}
  }
  
  \on[x=29mm,y=-32mm] {
    \begin{tikzpicture}[turingMachine]
      \state[initial,alert] (0) at (0,0) {0}; 
      \state[             ] (1) at (1,0) {1}; 
      \state[accepting    ] (2) at (2,0) {2}; 
      
      \path          (0) edge[bend left=15] node {\smTMtransR{a}{b}} (1);
      \path[example] (1) edge[bend left=15] node {\smTMtransR{a}{b}} (0);
      \path          (1) edge               node {\smTMtransL{b}{a}} (2);
    \end{tikzpicture}
  }

  
\end{frame}

\endgroup

% SPDX-License-Identifier: CC-BY-SA-4.0
% Author: Matthieu Perrin
% Part: 
% Section: 
% Sub-section: 
% Frame: 

\begingroup

\begin{frame}{Exécution d'une machine de Turing}
  
  \vspace{-1mm}Soient $M=\langle \Sigma, \Gamma, \blank, Q, q_0, F, \rightarrow \rangle$ une machine de Turing, $u\in \Sigma^\star$.
 
  \begin{block}{Définitions -- Configurations remarquables}
    Une configuration $c = \langle G, q, D \rangle \in \mathcal{C}_M$ est dite :
    \begin{description}[s'arrête sur $u$ :]
    \item[initiale :] si $G = \varepsilon$ et $q = q_0$
      \hspace\fill{$\alert{C_{\mathit{init}}(u) \eqdef \langle \varepsilon, q_0, u \rangle}$}
    \item[d'arrêt :] si elle ne permet aucune action
      \hspace\fill{$\alert{\mathcal{C}_M^{\mathit{halt}} \eqdef \left\{c\in \mathcal{C}_M \,\middle\mid\, \nexists c', c\leadsto_M c'\right\}}$}
    \item[acceptante :] si $c\in \mathcal{C}_M^\mathit{halt}$ et $q \in F$
      \hspace\fill{$\alert{\mathcal{C}_M^+ \eqdef \left\{\langle G, q, D \rangle \in \mathcal{C}_M^\mathit{halt} \,\middle\mid\, q\in F\right\}}$}
    \end{description}
  \end{block}
 
  \begin{block}{Définitions -- Exécution de $M$ sur $u$}
    On dit que $M$ :
    \begin{description}[s'arrête sur $u$ :]
    \item[s'arrête sur $u$ :] si $M$ atteint une configuration d'arrêt
      \hspace\fill{$\alert{\exists c\in \mathcal{C}_M^\mathit{halt}, C_{\mathit{init}}(u) \leadsto_M^\star c}$}
    \item[accepte $u$ :] si $M$ atteint une config. acceptante
      \hspace\fill{$\alert{\exists c\in \mathcal{C}_M^+, C_{\mathit{init}}(u) \leadsto_M^\star c}$}
    \item[s'arrête :] si $M$ s'arrête sur toute entrée
      \hspace\fill{$\alert{\forall u\in\Sigma^\star, \exists c\in \mathcal{C}_M^\mathit{halt}, C_{\mathit{init}}(u) \leadsto_M^\star c}$}
    \end{description}

    Le \structure{langage accepté par $M$} est l'ensemble des mots acceptés par $M$\\[-2mm]
    $$\alert{\mathcal{L}(M) \eqdef \left\{u \in \Sigma^\star \;\middle\mid\; \exists c \in \mathcal{C}_M^+, C_{\mathit{init}}(u) \leadsto_M^\star c\right\}}$$
  \end{block}

\end{frame}

\endgroup

 
\section{Machines de Turing Déterministes}
 
\subsection{Machines de Turing déterministes}
% SPDX-License-Identifier: CC-BY-SA-4.0
% Author: Matthieu Perrin
% Part: <Nom de la partie>
% Section: <Nom de la section>
% Sub-section: <Nom de la sous-section>  % (facultatif, laisser vide si non utilisé)
% Frame: <Titre de la slide>

\begingroup

\begin{frame}{Machine de Turing déterministe (MTD)}

  Soit $M = \langle \Sigma, \Gamma, \blank, Q, q_0, F, \rightarrow \rangle$ une MTND.
  \begin{block}{Définition -- Machine de Turing déterministe}
    On dit que $M$ est \structure{déterministe}
    si sa relation de transition $\rightarrow$ est fonctionnelle :

    $$\forall \alert{q}\in Q, \forall \alert{a}\in \Gamma,
    \left|\left\{
    \left\langle \structure{q'}, \structure{b}, \structure{d}\right\rangle \in Q \times \Gamma \times \{\triangleleft, \triangleright\}
    \;\middle\mid\;
    \alert{q} \xrightarrow{\smTMtrans{\alert{a}}{\structure{b}}{\structure{d}}} \structure{q'}
    \right\}\right|
    \le 1$$
  \end{block}
  
  \begin{block}{Remarques -- Notion de machine de Turing non-déterministe}
    \begin{itemize}
    \item Toute machine de Turing déterministe est une MTND
    \item \alert{``Non-déterministe''} signifie \alert{``pas forcément déterministe''}
    \end{itemize}
  \end{block}

\end{frame}

\endgroup

% SPDX-License-Identifier: CC-BY-SA-4.0
% Author: Matthieu Perrin
% Part: 
% Section: 
% Sub-section: 
% Frame: 

\begingroup

\begin{frame}{Machine de Turing non-déterministe}
  
  \onBlock[top=-4mm]{Rappel -- Machine de Turing déterministe}{
    Soit $M = \langle \Sigma, \Gamma, \blank, Q, q_0, F, \rightarrow \rangle$ une machine de Turing.
    \begin{itemize}
    \item $M$ est \structure{déterministe} si sa relation de transition $\rightarrow$ est fonctionnelle.
    \item \alert{``Non-déterministe''} signifie \alert{``pas forcément déterministe''}.
    \end{itemize}
    \begin{description}[Ubiquité :]
    \item[Modèle :] on autorise le modèle étendu (multi-ruban, immobilité, réécriture)
    \item[Ubiquité :] si non-déterminisme, un \structure{oracle} \alert{\emph{devine}} la meilleure transition
    \end{description}
  }
  
  \onBlock<2->[y=-9mm]{Comment utiliser le non-déterminisme ?}{
    \begin{itemize}
    \item Déplacement à un endroit \alert{arbitraire} du ruban
    \item Écriture d'un mot \example{arbitraire} sur le ruban
    \item Choix \structure{arbitraire} d'une sous-machine à exécuter
    \end{itemize}
  }

  \on<2->[x=40mm, y=-10mm] {
    \begin{tikzpicture}[turingMachine, x=15mm, y=8mm]\scriptsize
      \state[alert]     (01) at (0,1) {\faRandom}; 
      \state[example]   (11) at (1,1) {\faRandom}; 
      \state[example]   (21) at (2,1) {\faCheck}; 
      \state[structure] (00) at (0,0) {$A$}; 
      \state[structure] (10) at (1,0) {\faRandom}; 
      \state[structure] (20) at (2,0) {$B$}; 

      \path (01) edge[loop above] node  {\smAlign{\smTMtransR{x}{x}\smTMtransL{x}{x}}} (01);
      \path (11) edge[loop above] node  {\smAlign{\smTMtransL{\blank}{a}\smTMtransL{\blank}{b}}} (11);
      \path (11) edge node  {\smTMtransR{\blank}{\blank}} (21);

      \path (10) edge node  {\smTMtransS{\varepsilon}{\varepsilon}} (20);
      \path (10) edge node[swap] {\smTMtransS{\varepsilon}{\varepsilon}} (00);

      \node at (2,2) {$\forall x\in \Sigma$}; 
    \end{tikzpicture}
  }
  
  \onBlock<2->[y=-20mm, left=.48\textwidth, anchor=north]{Reconnaître un langage $L$}{
    \begin{description}[Entrée :]
    \item[Entrée :] $u\in \Sigma^\star$ 
    \item[Sortie :] \alert{une} exécution $\cmark{}$ si $u\in L$
    \end{description}
  }
  
  \onBlock<2->[y=-20mm, right=.48\textwidth, anchor=north]{Générer un langage $L$}{
    \begin{description}[Entrée :]
    \item[Entrée :] le mot vide $\varepsilon$ 
    \item[Sortie :] \alert{un} mot arbitraire de $L$
    \end{description}
  }

\end{frame}

\endgroup

% SPDX-License-Identifier: CC-BY-SA-4.0
% Author: Matthieu Perrin
% Part: <Nom de la partie>
% Section: <Nom de la section>
% Sub-section: <Nom de la sous-section>  % (facultatif, laisser vide si non utilisé)
% Frame: <Titre de la slide>

\begingroup

\begin{frame}{Langage décidable}

  \onBlock[top=-4mm]{Définition -- Décision d'un langage}{
    Soient $M=\langle \Sigma, \Gamma, \blank, Q, q_0, F, \rightarrow \rangle$ une MTD, $u\in \Sigma^\star$ et $L\subseteq \Sigma^\star$.
    \begin{itemize}
    \item On dit que \structure{$M$ décide $L$} si \alert{$L = \mathcal{L}(M)$} et \alert{$M$ s'arrête sur toutes ses entrées}
    \item On dit que \structure{$L$ est décidable} s'il existe une MTD qui décide $L$
    \item On note \alert{\textsc{r}} (pour \structure{récursif}) la classe des langages décidables
    \end{itemize}
  }

  \onExampleBlock[y=5mm]{Exemple -- Décision de $\{a^n b^n c^n \mid n > 0 \}$}

  \on[y=-4mm] {
    \begin{tikzpicture}[tape, x=7mm, y=7mm]%
      \cell{}
      \cell{}
      \cell{\alt<-1>{$a$}{$A$}} \smhead[alert]<1>      \smheadfromb[alert]<4>{3}  
      \cell{\alt<-5>{$a$}{$A$}} \smheadb[alert]<5>     \smheadfromb[alert]<8>{3}  
      \cell{\alt<-2>{$b$}{$B$}} \smheadb[alert]<9>     \smheadfromb[alert]<2>{-1} 
      \cell{\alt<-6>{$b$}{$B$}}                        \smheadfromb[alert]<6>{-1} 
      \cell{\alt<-3>{$c$}{$C$}}                        \smheadfromb[alert]<3>{-1}
      \cell{\alt<-7>{$c$}{$C$}} \smheadb[alert]<11>    \smheadfromb[alert]<7>{-1}
      \cell{}                                          \smheadfromb[alert]<10>{-3}
      \cell{}
    \end{tikzpicture}%
  }

  \on[bottom] {
    \begin{tikzpicture}[turingMachine]
      \state[alert ob=<{1,5,9}>, initial above] (0) at (1,1) {0};
      \state[alert ob=<{2,6}>                 ] (1) at (1,0) {1};
      \state[alert ob=<{3,7}>                 ] (2) at (0,0) {2};
      \state[alert ob=<{4,8}>                 ] (3) at (0,1) {3};
      \state[alert ob=<{10}>                  ] (4) at (2,1) {4};
      \state[alert ob=<{11}>, accepting       ] (5) at (3,1) {5};
      
      \path (0) edge             node[sloped,swap] {\smTMtransR{a}{A}} (1);
      \path (1) edge             node[sloped     ] {\smTMtransR{b}{B}} (2);
      \path (2) edge             node[sloped,swap] {\smTMtransL{c}{C}} (3);
      \path (3) edge             node[sloped,swap] {\smTMtransR{A}{A}} (0);
      \path (0) edge             node              {\smTMtransR{B}{B}} (4);
      \path (4) edge             node              {\smTMtransL{\blank}{\blank}} (5);
      \path (1) edge[loop right] node              {\smAlign{\smTMtransR{a}{a}\smTMtransR{B}{B}}} (1);
      \path (2) edge[loop left ] node              {\smAlign{\smTMtransR{b}{b}\smTMtransR{C}{C}}} (2);
      \path (3) edge[loop left ] node              {\smAlign{\smTMtransL{a}{a}\smTMtransL{b}{b}\smTMtransL{B}{B}\smTMtransL{C}{C}}} (3);
      \path (4) edge[loop below] node              {\smAlign{\smTMtransR{B}{B}\smTMtransR{C}{C}}} (4);
    \end{tikzpicture}
  }

\end{frame}

\endgroup

% SPDX-License-Identifier: CC-BY-SA-4.0
% Author: Matthieu Perrin
% Part: 
% Section: 
% Sub-section: 
% Frame: 

\begingroup

\SetKwFunction{Syracuse}{Syracuse}

\begin{frame}{Langage semi-décidable}

  \onBlock[top=-4mm]{Définition -- Semi-décision d'un langage}{
    Soient $M=\langle \Sigma, \Gamma, \blank, Q, q_0, F, \rightarrow \rangle$ une MTD, $u\in \Sigma^\star$ et $L\subseteq \Sigma^\star$.
    \begin{itemize}
    \item On dit que \structure{$M$ semi-décide $L$} si \alert{$L = \mathcal{L}(M)$}
    \item On dit que \structure{$L$ est semi-décidable} s'il existe une MTD qui semi-décide $L$
    \end{itemize}
  }

  \onExampleBlock[y=7mm]{Exemple -- Semi-décision de $\{(n)_2 \mid \Syracuse(n) \}$}{}

  \on[left=.5\textwidth, y=-10mm]{
    \begin{algorithm}[H]\small
      \Algo{$\Syracuse(n\in \mathbb{N}) \in \mathbb{B}$}{
        \While{\,\Example<1,22-23>{$n\neq 1$}}{\vspace{.1mm}
          \Structure<1-5,18-21>{\lIf{$n \equiv 0 \pmod 2$}{$n\leftarrow \frac{n}{2}$;}}\vspace{.5mm}
          \Alert<1,6-16>{\lElse{$n\leftarrow \frac{3n+1}{2}$;}}
        }
        \Return \True\;
      }
    \end{algorithm}
  }

  \on[left=.5\textwidth, y=-35mm]{
    \example{Pour $n = 6$ :}\\[1mm]
    $\begin{array}{r@{~}c@{~}l@{~}c@{~}l@{~}c@{~}l}
      \structure{6} &
      \uncover<1,6->{\rightarrow}    &
      \uncover<1,6->{\alert{3}}      &
      \uncover<1,10->{\rightarrow}   &
      \uncover<1,10->{\alert{5}}     &
      \uncover<1,17->{\rightarrow}   &
      \uncover<1,17->{\structure{8}} \\
      &
      \uncover<1,18->{\rightarrow}   &
      \uncover<1,18->{\structure{4}} &
      \uncover<1,19->{\rightarrow}   &
      \uncover<1,19->{\structure{2}} &
      \uncover<1,20->{\rightarrow}   &
      \uncover<1,20->{\example{1}}
    \end{array}$
  }
  
  \on[x=25mm, y=-3mm]{
    \begin{tikzpicture}[tape, x=5mm, y=5mm]
      \cell{}                                                                                 \smheadb<17,23>
      \cell[structure ob=<17-21>,example ob=<22>     ]{\oneof[ ]{\on<17-22>{1}}}              \smheadb<16,22>
      \cell[structure ob=<17-21>                     ]{\oneof[ ]{\on<16-21>{0}}}              \smheadb<10,15,21>
      \cell[structure ob=<17-20>,alert ob=<{10-12}>  ]{\oneof[ ]{\on<10-14>{1}\on<15-20>{0}}} \smheadb<9,14,20>
      \cell[structure ob=<17-19>,alert ob=<{10-12}>  ]{\oneof[ ]{\on<9-19>{0}}}               \smheadb<8,13,19>
      \cell[structure on=<{2-5}>,alert ob=<{6,10-12}>]{\oneof[1]{\on<13->{}}}                 \smhead<1,7,12>  \smheadfromb<18>{-4}
      \cell[structure on=<{2-5}>,alert ob=<{6}>      ]{\oneof[1]{\on<7->{}}}                  \smheadb<2,6>    \smheadfromb<11>{-3}
      \cell[structure on=<{2-5}>                     ]{\oneof[0]{\on<6->{}}}                  \smheadb<3,5>
      \cell{}                                                                                 \smheadb<4>
    \end{tikzpicture}
  }

  \on[x=25mm,bottom=-2mm]{
    \begin{tikzpicture}[turingMachine,x=20mm, y=20mm]\small
      \state[example on=<1>,alert ob=<{11}>,structure ob=<{2-4,18}>,initial above] (s) at (0,1) {\faForward};
      \state[alert on=<{1,10}>,structure ob=<{17}>                               ] (0) at (1,1) {0};
      \state[alert on=<{1,9,14,16}>                                              ] (1) at (2,1) {1};
      \state[structure on=<{1,5,19-21}>,alert ob=<{6,12}>,example ob=<{22}>      ] (d) at (0,0) {\textdiv};
      \state[example on=<{1,23}>,alert ob=<{7,13}>, accepting                    ] (f) at (1,0) {\cmark};
      \state[alert on=<{1,8,15}>                                                 ] (2) at (2,0) {2};

      \tiny
      \path (s) edge[loop left]      node[swap]   {\smAlign{\smTMtransR{0}{0}\smTMtransR{1}{1}}}       (s);
      \path (s) edge                 node[swap]   {\smTMtransL{\blank}{\blank}}                        (d);
      \path (d) edge[loop left]      node[swap]   {\smTMtransL{0}{\blank}}                             (d);
      \path (d) edge                 node[swap]   {\smTMtransL{1}{\blank}}                             (f);
      \path (f) edge                 node[sloped] {\smTMtransL{0}{0}}                                  (1);
      \path (f) edge                 node[swap]   {\smTMtransL{1}{1}}                                  (2);
      \path (0) edge[loop below]     node[swap]   {\smTMtransL{0}{0}}                                  (0);
      \path (0) edge                 node[swap]   {\smTMtransL{1}{1}}                                  (1);
      \path (0) edge                 node[swap]   {\smTMtransR{\blank}{\blank}}                        (s);
      \path (1) edge[bend right=5mm] node[swap]   {\smAlign{\smTMtransL{0}{1}\smTMtransL{\blank}{1}}} (0);
      \path (1) edge                 node[swap]   {\smTMtransL{1}{0}}                                  (2);
      \path (2) edge[bend right]     node[swap]   {\smAlign{\smTMtransL{0}{0}\smTMtransL{\blank}{0}}}  (1);
      \path (2) edge[loop right]     node[swap]   {\smTMtransL{1}{1}}                                  (2);
    \end{tikzpicture}
  }

\end{frame}

\endgroup



% SPDX-License-Identifier: CC-BY-SA-4.0
% Author: Matthieu Perrin
% Part: <Nom de la partie>
% Section: <Nom de la section>
% Sub-section: <Nom de la sous-section>  % (facultatif, laisser vide si non utilisé)
% Frame: <Titre de la slide>

\begingroup

\begin{frame}{Langage récursivement énumérable}

  \onBlock[top=-5mm]{Définition -- Énumération d'un langage}{
    Soit $M=\langle \Sigma, \Gamma, \blank, Q, q_0, F, \rightarrow \rangle$ une MTD.
    \begin{itemize}
    \item Le \structure{langage énuméré par $M$} est l'ensemble des mots écrits sur le ruban quand $M$ passe par un état accepteur à partir du mot vide:
      $$\alert{\mathcal{L}_E(M) = \left\{G\cdot D \in \Sigma^\star \;\middle\mid\; \exists q\in F,~ C_{\mathit{init}}(\varepsilon) \leadsto_M^\star \langle G, q, D \rangle \right\}}$$
    \item \structure{$L$ est récursivement énumérable} s'il existe une MTD qui énumère $L$
    \item On note \alert{\textsc{re}} la classe des langages récursivement énumérables
    \end{itemize}
  }

  \onExampleBlock[y=-10mm]{Exemple -- Énumération de $\{a^n b^n \mid n\in \mathbb{N}\}$}{}
  
  \on[x=-20mm,y=-22mm]{
    \begin{tikzpicture}[tape, x=7mm, y=7mm]
      \cell{}                              \smheadfromb<7>{5} 
      \cell[alert ob=<6>]{\alt<-5>{}{$a$}} \smheadfromb<5>{3} 
      \cell[alert ob=<4>]{\alt<-3>{}{$a$}} \smheadfromb<3>{1} 
      \cell[alert ob=<2>]{\alt<-1>{}{$a$}} \smhead     <1>   
      \cell[alert ob=<3>]{\alt<-2>{}{$b$}} \smheadb    <2> 
      \cell[alert ob=<5>]{\alt<-4>{}{$b$}} \smheadfromb<4>{-2} 
      \cell[alert ob=<7>]{\alt<-6>{}{$b$}} \smheadfromb<6>{-4}   
      \cell{}                                                  
    \end{tikzpicture}
  }

  \on[x=-20mm, bottom=5mm]{
    $\mathcal{L}_E(M) = \{\varepsilon\uncover<3->{, ab}\uncover<5->{, aabb}\uncover<7->{, aaabbb, ...}\}$
  }

  \on[x=35mm, bottom=2mm]{
    \begin{tikzpicture}[turingMachine]
      \state[alert ob=<{1,3,5,7}>, initial, accepting] (0) at (0,1) {0}; 
      \state[alert ob=<{2,4,6}>                      ] (1) at (0,0) {1}; 
      
      \path (0) edge[bend left]  node {\smTMtransR{\blank}{a}}                       (1);
      \path (1) edge[bend left]  node {\smTMtransL{\blank}{b}}                       (0);
      \path (0) edge[loop right]  node {\smGroup{\smTMtransL{a}{a}\smTMtransL{b}{b}}} (0);
      \path (1) edge[loop right] node {\smGroup{\smTMtransR{a}{a}\smTMtransR{b}{b}}} (1);
    \end{tikzpicture}
  }

\end{frame}

\endgroup

% SPDX-License-Identifier: CC-BY-SA-4.0
% Author: Matthieu Perrin
% Part: <Nom de la partie>
% Section: <Nom de la section>
% Sub-section: <Nom de la sous-section>  % (facultatif, laisser vide si non utilisé)
% Frame: <Titre de la slide>

\begingroup

\begin{frame}{Fonction calculable}

  \onBlock[top=-3mm]{Définition -- Calcul d'une fonction}{
    Soient $M=\langle \Sigma, \Gamma, \blank, Q, q_0, F, \rightarrow \rangle$ une MTD, $\mathcal{I} \subseteq \Sigma^\star$, et $f : \mathcal{I} \rightarrow \Sigma^\star$. On dit que :
    \begin{itemize}
    \item \structure{$M$ calcule $f$} si, pour tout $x\in \mathcal{I}$, $M$ écrit $f(x)$ sur le ruban puis termine
      $$\alert{ \forall x\in \mathcal{I}, \exists c=\langle G, q, D \rangle \in \mathcal{C}_M^\mathit{halt}, C_{\mathit{init}}(x) \leadsto_M^\star c \land f(x)=G\cdot D}$$
    \item \structure{$f$ est calculable} s'il existe une MTD qui calcule $f$
    \end{itemize}
  }

  \onExampleBlock[y=-6mm]{Exemple -- Incrémentation de 11 écrit en binaire}{}

  \on[y=-17mm]{
    \begin{tikzpicture}[tape, x=7mm, y=7mm]
      \cell{}
      \cell{$1$}                \smhead<1>
      \cell{\alt<-8>{$0$}{$1$}} \smheadb<2,8>
      \cell{\alt<-7>{$1$}{$0$}} \smheadb<3,7,9> 
      \cell{\alt<-6>{$1$}{$0$}} \smheadb<4,6>
      \cell{}                   \smheadb<5>
    \end{tikzpicture}
  }

  \on[bottom=2mm]{
    \begin{tikzpicture}[turingMachine]
      \state[alert ob=<-5>, initial ] (0) at (0,0) {$0$}; 
      \state[alert ob=<6-8>         ] (1) at (1,0) {$1$}; 
      \state[alert ob=<9>, accepting] (2) at (2,0) {$2$}; 

      \path (0) edge[loop above] node {\smAlign{\smTMtransR{0}{0}\smTMtransR{1}{1}}} (0);
      \path (1) edge[loop above] node {\smTMtransL{1}{0}} (1);
      \path (0) edge             node {\smTMtransL{\blank}{\blank}} (1);
      \path (1) edge             node {\smAlign{\smTMtransR{0}{1}\smTMtransR{\blank}{1}}} (2);
    \end{tikzpicture}
  }

\end{frame}

\endgroup

 
\subsection{Introduction à la Complexité}
% SPDX-License-Identifier: CC-BY-SA-4.0
% Author: Matthieu Perrin
% Part: <Nom de la partie>
% Section: <Nom de la section>
% Sub-section: <Nom de la sous-section>  % (facultatif, laisser vide si non utilisé)
% Frame: <Titre de la slide>

\begingroup

\begin{frame}{Complexité d'une machine de Turing déterministe}
  Soient $M=\langle \Sigma, \Gamma, \blank, Q, q_0, F, \rightarrow \rangle$ une MTD qui termine, $u \in \Sigma^\star$ un mot et $n\in \mathbb{N}$.

  \begin{block}{Complexité temporelle déterministe}
    \begin{itemize}
    \item\vspace{-1mm} La \structure{complexité temporelle} de $M$ sur $u$ est le \alert{nombre d'actions} lors de l'exécution de $u$ par $M$ :

      \vspace{-2mm}
      $$\alert{T_M(u) \eqdef \operatorname{unique}\left\{n \in \mathbb{N} \mid \exists c_f \in \mathcal{C}_M^\mathit{halt}, C_{\mathit{init}}(u) \leadsto^n c_f\right\}}.$$ 

    \item\vspace{-1mm} La \structure{complexité temporelle dans le pire cas} de $M$, pour $n\in \mathbb{N}$, est : \\

      \vspace{-2mm}
      $$\alert{T_M(n) \eqdef \max \left\{T_M(u) \mid u \in \Sigma^n \right\}}.$$ 
    \end{itemize}
  \end{block}

  \vspace{-2mm}
  \begin{block}{Complexité spatiale déterministe}
    \begin{itemize}
    \item\vspace{-1mm} La \structure{complexité spatiale} de $M$ sur $u$ est le \alert{nombre maximal de cases du ruban utilisées} lors de l'exécution de $u$ par $M$ :

      \vspace{-2mm}
      $$\alert{S_M(u) \eqdef \max\left\{|GD| \mid \exists q\in Q, C_{\mathit{init}}(u) \leadsto^\star \langle G, q, D \rangle \right\}}.$$ 

      
    \item\vspace{-1mm} La \structure{complexité spatiale dans le pire cas} de $M$, pour $n\in \mathbb{N}$, est : \\

      \vspace{-2mm}
      $$\alert{S_M(n) \eqdef \max \{S_M(u) \mid u \in \Sigma^n \}}.$$ 
    \end{itemize}
  \end{block}

\end{frame}

\endgroup

% SPDX-License-Identifier: CC-BY-SA-4.0
% Author: Matthieu Perrin
% Part: <Nom de la partie>
% Section: <Nom de la section>
% Sub-section: <Nom de la sous-section>  % (facultatif, laisser vide si non utilisé)
% Frame: <Titre de la slide>

\begingroup

\begin{frame}{Exemple : Le langage $\{a^p b^p \mid p\in \mathbb{N}\}$}

  \on<-20>[x=23mm, y=25mm]{
    \begin{tikzpicture}[tape, x=6mm, y=6mm]
      \cell{}
      \cell[structure ob=<1>,  alert ob=<2> ]{\alt<1>{$a$}{}}   \smhead<1>           \smheadfromb<10>{4}
      \cell[structure ob=<11>, alert ob=<12>]{\alt<-11>{$a$}{}} \smheadb<2,11>       \smheadfromb<14>{2}  
      \cell[structure ob=<15>, alert ob=<16>]{\alt<-15>{$a$}{}} \smheadb<3,15,18>   
      \cell[structure ob=<17>, alert ob=<18>]{\alt<-17>{$b$}{}} \smheadb<4,17,19> 
      \cell[structure ob=<13>, alert ob=<14>]{\alt<-13>{$b$}{}} \smheadb<5,9,13,20>  \smheadfromb<16>{-1}
      \cell[structure ob=<8>,  alert ob=<9> ]{\alt<-8>{$b$}{}}  \smheadb<6,8>        \smheadfromb<12>{-3}  
      \cell{}                                                   \smheadb<7>            
    \end{tikzpicture}
  }

  \ob<21->[x=23mm, y=25mm]{
    \begin{tikzpicture}[tape, x=6mm, y=6mm]
      \cell                                 {}
      \cell[structure ob=<21>,alert ob=<22>]{\alt<-21>{$a$}{}}  \smheadb<21>          \smheadfromb<24>{4}
      \cell[alert ob=<25->]                 {$b$}               \smheadb<25->
      \cell                                 {$a$}
      \cell                                 {$b$}
      \cell                                 {$a$}
      \cell[structure ob=<23>,alert ob=<24>]{\alt<-23>{$b$}{}}  \smheadb<23>
      \cell                                 {}                                        \smheadfromb<22>{-5}
    \end{tikzpicture}
  }

  \on[x=23mm, y=3mm]{
    \begin{tikzpicture}[turingMachine]
      \state [alert ob=<{1,11,15,19,21,25-}>, initial above] (A) at (1,1) {\textbf{A}}; 
      \state [alert ob=<{2-7,12,16,22}>                    ] (B) at (2,1) {\faForward}; 
      \state [alert ob=<{8,13,17,23}>                      ] (C) at (2,0) {\textbf{B}}; 
      \state [alert ob=<{9,10,14,18,24}>                   ] (D) at (1,0) {\faBackward}; 
      \state [alert ob=<{20}>, accepting                   ] (E) at (0,1) {\faCheck}; 

      \path (A) edge             node {\smTMtransR{a}{\blank}} (B);
      \path (B) edge[loop right] node {\smGroup{\smTMtransR{a}{a}\smTMtransR{b}{b}}} (B);
      \path (B) edge             node {\smTMtransL{\blank}{\blank}} (C);
      \path (C) edge             node {\smTMtransL{b}{\blank}} (D);
      \path (D) edge[loop left]  node {\smGroup{\smTMtransL{a}{a}\smTMtransL{b}{b}}} (D);
      \path (D) edge             node {\smTMtransR{\blank}{\blank}} (A);
      \path (A) edge             node {\smTMtransR{\blank}{\blank}} (E);
    \end{tikzpicture}
  }

  \onBlock[left=40mm, y=25mm]{Exécution de $aaabbb$}{
    \begin{itemize}
    \item $T_M(aaabbb) \alt<-19>{\ge}{=} \oneof[28]{%
      \on<1>{0}%
      \on<2>{1}%
      \on<3>{2}% 
      \on<4>{3}% 
      \on<5>{4}% 
      \on<6>{5}% 
      \on<7>{6}% 
      \on<8>{7}% 
      \on<9>{8}% 
      \on<10>{12}% 
      \on<11>{13}% 
      \on<12>{17}% 
      \on<13>{18}% 
      \on<14>{21}% 
      \on<15>{22}% 
      \on<16>{24}% 
      \on<17>{25}% 
      \on<18>{26}% 
      \on<19>{27}% 
    }$
    \item $S_M(aaabbb) = 6$
    \item<20-> $aaabbb$ est \alert{accepté}
    \end{itemize}
  }

  \onBlock<21->[left=40mm]{Exécution de $ababab$}{
    \begin{itemize}
    \item $T_M(ababab) \alt<-24>{\ge}{=} \oneof[13]{%
      \on<21>{0}%
      \on<22>{6}%
      \on<23>{7}% 
      \on<24>{12}% 
    }$
    \item $S_M(ababab) = 6$
    \item<25-> $ababab$ est \alert{rejeté}
    \end{itemize}
  }

  \onBlock<26->[bottom=2mm]{Exécution de $M$ dans le pire cas}{
    Si le mot est rejeté, l'exécution s'arrêtera plus rapidement que s'il est accepté.
    
    \vspace{1mm}
    \begin{minipage}{.5\textwidth}
      \begin{itemize}
      \item $T_M(6) = 28$
      \item $S_M(6) = 6$
      \end{itemize}
    \end{minipage}%
    \begin{minipage}{.5\textwidth}
      \begin{itemize}
      \item $T_M(n) = \frac{(n+1)(n+2)}{2} \in \mathcal{O}\left(n^2\right)$
      \item $S_M(n) = n$
      \end{itemize}
    \end{minipage}%
  }
  
\end{frame}

\endgroup

% SPDX-License-Identifier: CC-BY-SA-4.0
% Author: Matthieu Perrin
% Part: <Nom de la partie>
% Section: <Nom de la section>
% Sub-section: <Nom de la sous-section>  % (facultatif, laisser vide si non utilisé)
% Frame: <Titre de la slide>

\begingroup

\begin{frame}{Comparaisons asymptotiques de fonctions}

  \onBlock[top=-4mm]{Notations de Landau}{
    Soient $f$ et $g$ deux suites de $\mathbb{N}$ dans $\mathbb{R}^+$. On dit que :
    \begin{itemize}
    \item $f$ est \structure{dominé} par $g$, noté \alert{$f \in \mathcal{O}(g)$}, si \alert{$\exists n_0, c \in \mathbb{R}^+, \forall n > n_0, f(n) \le c \times g(n)$}
    \item<2-> $f$ est \structure{minorée} par $g$, noté \alert{$f \in \Omega(g)$}, si \alert{$g \in \mathcal{O}(f)$}
    \item<2-> $f$ est \structure{du même ordre de grandeur} que $g$, noté \alert{$f \!\in\! \Theta(g)$}, si \alert{$f \!\in\! \mathcal{O}(g) \!\cap\! \Omega(g)$}
    \end{itemize}
    \begin{center}
      \alert{$\mathcal{O}\left(1\right)
        \subsetneq \mathcal{O}\left(\log(n)\right)
        \subsetneq \mathcal{O}\left(n\right)
        \subsetneq \mathcal{O}\left(n\log(n)\right)
        \subsetneq \mathcal{O}\left(n^2\right)
        \subsetneq \mathcal{O}\left(n^3\right)
        \subsetneq \mathcal{O}\left(2^n\right)
        \subsetneq \mathcal{O}\left(n!\right)$}
    \end{center}
  }

  \onExampleBlock[y=-2mm]{Exemple : $f(n) \eqdef n\left(n + 2\sqrt{n} + 5\sin(3n)\right)$}{}

  \on[bottom, x=-15mm]{
    \begin{tikzpicture}[x=3.5mm, y=3.2mm]
      
      \footnotesize
      \draw[->]             (0,0) -- (20,0) node[right]{$n$};
      \draw[->]             (0,0) -- (0,8.5);

      \draw[domain=0:20, samples=100, smooth, alert    ] plot (\x,{0.01*2*\x*\x}) node[right]{$2 n^2$};
      \draw[domain=0:20, samples=100, smooth, example  ] plot (\x,{0.01*\x*(\x + 2*sqrt(\x) + 5*sin(1.5/3.14*\x*180))}) node[right]{$f(n)$};
      \draw[domain=0:20, samples=100, smooth, structure] plot (\x,{0.01*\x*\x}) node[right]{$n^2$};

      \uncover<2->{
      \draw[domain=0:16, samples=100, smooth, black!50 ] plot (\x,{0.01*.2*\x*\x*\x}) node[right]{$\frac{n^3}{5}$};
      \draw[domain=0:20, samples=100, smooth, black!50 ] plot (\x,{0.01*10*\x}) node[right]{$10n$};
      }
      
      \draw[example!50] (11,0) -- (11,8.5) node[below right]{$\begin{array}{@{}l@{\,}l@{\,}l@{}}n_0 &=& 11\\c&=&2\end{array}$};
      
      \foreach \x in {0,1,...,16}{
        \fill<2->[black!50]  (\x, 0.01*.2*\x*\x*\x)                                     circle[radius=1pt];
      }
      \foreach \x in {0,1,...,20}{
        \fill<2->[black!50]  (\x, 0.01*10*\x)                                           circle[radius=1pt];
        \draw[densely dotted, black!40] (\x,0) -- (\x,8.5);
        \node[below, black!40] at (\x,0) {\tiny\x};
        \fill[alert]     (\x, 0.01*2*\x*\x)                                         circle[radius=1pt];
        \fill[example]   (\x, {0.01*\x*(\x + 2*sqrt(\x) + 5*sin(1.5/3.14*\x*180))}) circle[radius=1pt];
        \fill[structure] (\x, 0.01*\x*\x)                                           circle[radius=1pt];
      }
    \end{tikzpicture}
  }

  \on<2->[text, y=-20mm, x=3mm]{
    \begin{itemize}
    \item $f(n) \in \mathcal{O}\left(n^3\right)$
    \item $f(n) \in \Theta\left(n^2\right)$
    \item $f(n) \in \Omega\left(n\right)$
    \end{itemize}
  }

  \on<2->[bottom, x=40mm]{
    \begin{tikzpicture}[y=4mm]
      \draw[latex-latex] (0,0) -- (0,6);
      \draw              (-.1,4) -- (.1,4);
      \draw              (-.1,3) -- (.1,3);
      \draw              (-.1,2) -- (.1,2);
      \node[above, align=center] at (0,6) {Borne supérieure};
      \node[example, left]  at (0,3) {$g$};
      \node[below, align=center] at (0,0) {Borne inférieure};

      \draw[alert,     fill=alert!25]     (.4, 2)  rectangle (.5, 6);
      \draw[example,   fill=example!25]   (.6, 2)  rectangle (.7, 4);
      \draw[structure, fill=structure!25] (.2, 0)  rectangle (.3, 4);
      \draw[alert,     fill=alert!25]     (-.2, 4) rectangle (-.3, 6);
      \draw[structure, fill=structure!25] (-.2, 0) rectangle (-.3, 2);

      \node[alert, left]      at (-.3,5) {$\omega(g)$};
      \node[alert, right]     at ( .7,5) {$\Omega(g)$};
      \node[example, right]   at ( .7,3) {$\Theta(g)$};
      \node[structure, right] at ( .7,1) {$\mathcal{O}(g)$};
      \node[structure, left]  at (-.3,1) {$o(g)$};
    \end{tikzpicture}
  }
  
\end{frame}

\endgroup

% SPDX-License-Identifier: CC-BY-SA-4.0
% Author: Matthieu Perrin
% Part: <Nom de la partie>
% Section: <Nom de la section>
% Sub-section: <Nom de la sous-section>  % (facultatif, laisser vide si non utilisé)
% Frame: <Titre de la slide>

\begingroup

\begin{frame}{Suite à croissance polynomiale}

  \begin{block}{Définition -- la classe \textsc{poly}}
    Soit $f$ une suite de $\mathbb{N}$ dans $\mathbb{R}^+$.

    On dit que $f$ est \structure{à croissance polynomiale} s'il existe \structure{$k\in \mathbb{N}$}, tel que \structure{$f \in \mathcal{O}(n^k)$}.

    $$\alert{\textsc{poly} = \bigcup_{k\in \mathbb{N}} \mathcal{O}\left(n^k\right)}$$

    \vspace{-3mm}
    \structure{Remarque : }
    \begin{itemize}
    \item $f\in \textsc{poly}$ ssi $f\in \mathcal{O}(P)$ pour un polynôme $P$ à coefficients positifs
    \item On définit les suites \structure{à croissance exponentielle} : $\alert{\textsc{exp} = \bigcup_{k\in \mathbb{N}} \mathcal{O}\left(k^n\right)}$
    \end{itemize}
  \end{block}
  
  \begin{block}{Stabilité de la classe \textsc{poly}}
    \begin{description}[Par composition :] 
    \item[Par somme :] 
      $\begin{array}[t]{@{}l@{,~}l@{,\quad}l@{}}
      \forall f \in \mathcal{O}\left(n^k\right) & \forall g \in \mathcal{O}\left(n^l\right) & \alert{f+g \in \mathcal{O}\left(n^{\max(k,l)}\right)}\\
      \forall f \in \textsc{poly}               & \forall g \in \textsc{poly}               & \alert{f+g \in \textsc{poly}}
    \end{array}$
    \item[Par produit :]\vspace{1mm}
      $\begin{array}[t]{@{}l@{,~}l@{,\quad}l@{}}
      \forall f \in \mathcal{O}\left(n^k\right) & \forall g \in \mathcal{O}\left(n^l\right) & \alert{f\times g \in \mathcal{O}\left(n^{k + l}\right)}\\
      \forall f \in \textsc{poly}               & \forall g \in \textsc{poly}               & \alert{f\times g \in \textsc{poly}}
    \end{array}$
    \item[Par composition :]\vspace{1mm}
      $\begin{array}[t]{@{}l@{,~}l@{,\quad}l@{}}
      \forall f \in \mathcal{O}\left(n^k\right) & \forall g \in \mathcal{O}\left(n^l\right) & \alert{f \circ g \in \mathcal{O}\left(n^{k \times l}\right)}\\
      \forall f \in \textsc{poly}               & \forall g \in \textsc{poly}               & \alert{f \circ g \in \textsc{poly}}
    \end{array}$
    \end{description}
  \end{block}

\end{frame}

\endgroup

% SPDX-License-Identifier: CC-BY-SA-4.0
% Author: Matthieu Perrin
% Part: <Nom de la partie>
% Section: <Nom de la section>
% Sub-section: <Nom de la sous-section>  % (facultatif, laisser vide si non utilisé)
% Frame: <Titre de la slide>

\begingroup

\begin{frame}{Complexité d'un problème}

  \begin{block}{Classes de complexité d'un problème}
    Soit $f$ une suite de $\mathbb{N}$ dans $\mathbb{R}^+$. 
    On définit les \structure{classes de complexité} : 
    \begin{description}[$\textsc{dspace}(f)$ :]
    \item[$\textsc{dtime}(f)$ :] les problèmes décidables en temps $\mathcal{O}(f)$ \structure{par une MTD} :
      
      \vspace{-2mm}
      $$\alert{\textsc{dtime}(f) \eqdef \{\structure{\alert{L} \in \textsc{lang} \mid \exists M, L = \mathcal{L}(M) \land \alert{T_M \in \mathcal{O}(f)}} \}}$$
    \item[$\textsc{dspace}(f)$ :] les problèmes décidables en espace $\mathcal{O}(f)$ \structure{par une MTD} :

      \vspace{-2mm}
      $$\alert{\textsc{dspace}(f) \eqdef \{\structure{\alert{L} \in \textsc{lang} \mid \exists M, L = \mathcal{L}(M) \land \alert{S_M \in \mathcal{O}(f)}} \}}$$
    \end{description}
  \end{block}

  \pause
  \vspace{-1mm}
  \begin{block}{Classes de complexité importantes}

    \begin{tikzpicture}[y=4mm, x=21mm]
      \draw (0,0) -- (4,0);
      \draw (0,2) -- (4,2);
      \draw (0,4) -- (4,4);
      \draw (0, 0) -- (0, 4);
      \draw (2, 0) -- (2, 4);
      \draw (4, 0) -- (4, 4);
      
      \node[align=center] at (1,3) {$\alert{\displaystyle \textsc{p} = \bigcup_{k\in \mathbb{N}} \textsc{dtime}\left(n^k\right)}$};
      \node[align=center] at (3,3) {$\alert{\displaystyle \textsc{exptime} = \bigcup_{k\in \mathbb{N}} \textsc{dtime}\left(2^{n^k}\right)}$};
      \node[align=center] at (1,1) {$\alert{\displaystyle \textsc{pspace} = \bigcup_{k\in \mathbb{N}} \textsc{dspace}\left(n^k\right)}$};
      \node[align=center] at (3,1) {$\alert{\displaystyle \textsc{expspace} = \bigcup_{k\in \mathbb{N}} \textsc{dspace}\left(2^{n^k}\right)}$};

      \node[align=center, above] at (1,4) {Complexité \structure{polynomiale}};
      \node[align=center, above] at (3,4) {Complexité \structure{exponentielle}};
      \node[align=right,  left ] at (0,1) {Complexité\\[-2pt]\structure{spatiale}};
      \node[align=right,  left ] at (0,3) {Complexité\\[-2pt]\structure{temporelle}};
    \end{tikzpicture}
    
    $$ \textsc{dtime}\left(\log(n)\right) \subsetneq \alert{\textsc{p}} \subsetneq \textsc{dtime}\left(n^{\log(n)}\right) \subsetneq \alert{\textsc{exptime}} \subsetneq \textsc{dtime}\left(n!\right) $$
  \end{block}
  
\end{frame}

\endgroup
\endinput

% SPDX-License-Identifier: CC-BY-SA-4.0
% Author: Matthieu Perrin
% Part: <Nom de la partie>
% Section: <Nom de la section>
% Sub-section: <Nom de la sous-section>  % (facultatif, laisser vide si non utilisé)
% Frame: <Titre de la slide>

\begingroup

\begin{frame}{Importance de la classe P}

  \onBlock[top=-4mm]{La classe des problèmes \og raisonnables \fg}{
    \begin{itemize}
    \item Souvent faisable en pratique pour les ordinateurs modernes
    \item Composer deux algorithmes polynomiaux reste polynomial
    \item Classe très stable vis-à-vis de variantes raisonnables du calcul
      \begin{itemize}
      \item langages de programmation, $\lambda$-calcul, extensions des MTD...
      \end{itemize}
    \end{itemize}
  }

  \onExampleBlock[bottom=2mm, left=.6\textwidth]{Illustration}{
    \begin{itemize}
    \item On suppose 1 ns par instruction
    \end{itemize}
    \footnotesize
    \begin{tabular}{c|c@{~~}c@{~~}c}
      \structure{$n$} 
      & \structure{$n^2$} 
      & \structure{$n^3$} 
      & \structure{$n^6$} \\
      \hline
      10
      & 100 ns
      & 1 µs
      & 1 ms \\

      20
      & 400 ns
      & 8 µs
      & 64 ms \\

      40
      & 1.6 µs
      & 64 µs
      & 4.1 s \\

      60
      & 3.6 µs
      & 216 µs
      & 46.7 s \\[2mm]

      \structure{$n$} 
      & \structure{$2^n$} 
      & \structure{$3^n$} 
      & \structure{$n^n$} \\
      \hline
      10
      & 1 µs
      & 59 µs
      & 10 s \\

      20
      & 1 ms
      & 3.48 s
      & $6.49 \cdot 10^{6}$ ans \\

      40
      & 18.3 min
      & 385 ans
      & $1.49 \cdot 10^{42}$ ans \\

      60
      & 36.5 ans
      & $1.34 \cdot 10^{12}$ ans
      & $1.54 \cdot 10^{83}$ ans \\
    \end{tabular}
  }
  
  \on[bottom, x=30mm]{
    \begin{tikzpicture}[x=1mm, y=3mm]
      
      \footnotesize
      \draw[->]             (0,0) -- (45,0) node[right]{$n$};
      \draw[->]             (0,0) -- (0,10) node[above]{$t {\color{black!40}(s)}$};

      \foreach \x in {0,5,...,45}{
        \draw[densely dotted, black!40] (\x,0) -- (\x,10);
        \node[below, black!40] at (\x,0) {\tiny\x};
      }
      \foreach \y in {0,1,...,10}{
        \draw[densely dotted, black!40] (0,\y) -- (45,\y);
        \node[left, black!40] at (0,\y) {\tiny\y};
      }

      \begin{scope}
        \clip (0,0) rectangle (45,10);
        \draw[domain=0.1:11, samples=100, smooth, black!40 ] plot (\x,{exp(\x*ln(\x)-9*ln(10))});
        \draw[domain=0:22,   samples=100, smooth, alert    ] plot (\x,{exp(\x*ln(3) -9*ln(10))});
        \draw[domain=0:34,   samples=100, smooth, alert    ] plot (\x,{exp(\x*ln(2) -9*ln(10))});
        \draw[domain=0.1:45, samples=100, smooth, structure] plot (\x,{exp(6*ln(\x) -9*ln(10))});
        \draw[domain=0.1:45, samples=100, smooth, structure] plot (\x,{exp(5*ln(\x) -9*ln(10))});
      \end{scope}
      
      \node[below right, black!40 ] at (10,10)  {$n^n$};
      \node[below right, alert    ] at (21,10)  {$3^n$};
      \node[below right, alert    ] at (33,10)  {$2^n$};
      \node[above left,  structure] at (45,3.5) {$n^6$};
      \node[above left,  structure] at (45,0)   {$n^5$};

    \end{tikzpicture}
  }

\end{frame}

\endgroup

% SPDX-License-Identifier: CC-BY-SA-4.0
% Author: Matthieu Perrin
% Part: <Nom de la partie>
% Section: <Nom de la section>
% Sub-section: <Nom de la sous-section>  % (facultatif, laisser vide si non utilisé)
% Frame: <Titre de la slide>

\begingroup

\SetKwFunction{IsPrime}{is\_prime}

\begin{frame}{Exemple : test de primalité}

  \on[top=-4mm, text]{
    \Probleme{Primes}{
      Un entier $N\in \mathbb{N}$
    }{
      $N$ est-il premier ? 
    }
  }
  
  \onBlock[left=.6\textwidth, y=-2mm]{Un problème non-trivial}{
    \vspace{-1mm}
    \begin{itemize}
    \item La complexité doit être exprimée en fonction du nombre de chiffres de :

      \vspace{-2mm}
      $$n = |(N)_b| = \lceil \log_b(N) \rceil$$

    \item L'algorithme de droite est exponentiel

      \vspace{-2mm}
      $$\Omega\left(\sqrt{N}\right) = \Omega\left(2^{\frac{n}{2}}\right)$$
    \end{itemize}
  }

  \on[right=.43\textwidth, y=-2mm]{
    \begin{algorithm}[H]
      \Fun{$\IsPrime\left(N\in \mathbb{N}\right)$}{
        \For{$k$ \From $2$ \To $\sqrt{N}$}{
          \If{$N \equiv 0 \pmod k$}{\Return \False;}
        }
        \Return \True;
      }
    \end{algorithm}
  }  

  \onBlock[bottom=2mm]{Algorithme d'Agrawal, Kayal et Saxena (2004)}{
    \vspace{-1mm}
    \begin{itemize}
    \item Décide \textsc{Prime} en $\mathcal{O}(n^{12})$ 
    \item Donc $\textsc{Prime} \in \textsc{p}$ 
    \end{itemize}
  }

  \footnoteref{M. Agrawal, N. Kayal, N. Saxena. \emph{\textsc{primes} is in \textsc{p}}. Annals of mathematics (2004)}
  
\end{frame}

\endgroup

 
\subsection{Extensions du Modèle}
% SPDX-License-Identifier: CC-BY-SA-4.0
% Author: Matthieu Perrin
% Part: <Nom de la partie>
% Section: <Nom de la section>
% Sub-section: <Nom de la sous-section>  % (facultatif, laisser vide si non utilisé)
% Frame: <Titre de la slide>

\begingroup

%\begin{frame}{Généralisation des machines de Turing déterministes}
% 
%  Pour simplifier la conception de machines de Turing, on s'autorise parfois 
%  des transitions plus complexes
%  
%  \begin{block}{Opération sans déplacement}
%    \begin{itemize}
%    \item \structure{$q \xrightarrow{\smTMtransS{a}{b}} q'$} lit $a$, écrit $b$, ne déplace pas la tête de lecture%\hspace\fill\alert{$ \langle G, a D \rangle  \rightarrow \langle G, b D \rangle $}
%    \end{itemize}
%  \end{block}
% 
%%    \begin{description}
%%    \item<1->[$q \xrightarrow{\smTMtransS{a}{b}} q'$ :] lit $a$, écrit $b$, ne déplace pas la tête de lecture\hspace\fill\alert{$ \langle G, a D \rangle  \rightarrow \langle G, b D \rangle $}
%%    \item<4->[$q \xrightarrow{\smTMtransP{a}{b}} q'$ :] lit $a$, ajoute $b$ à gauche\hspace\fill\alert{$ \langle G, a D \rangle  \rightarrow \langle G b, a D \rangle $}
%%    \item<9->[$q \xrightarrow{\smTMtransM{a}} q'$ :] lit $a$, supprime $a$\hspace\fill\alert{$ \langle G, a D \rangle  \rightarrow \langle G, D \rangle $}
%%    \end{description}
% 
%  
%  \begin{block}{Opération sur des mots}
%    \item \structure{$q \xrightarrow{\smTMtrans{a}{b}{d}} q'$} a le   lit $a$, écrit $b$, ne déplace pas la tête de lecture%\hspace\fill\alert{$ \langle G, a D \rangle  \rightarrow \langle G, b D \rangle $}
%  \end{block}
% 
%  \begin{block}{Machines de Turing multi-rubans}
%  \end{block}
% 
%\end{frame}

\begin{frame}{Généralisation des machines de Turing}
  Pour simplifier la conception de machines de Turing, on s'autorise parfois des  \structure{transitions généralisées}
  \alert{$\langle \langle q, u_1, ..., u_k \rangle, \langle q', \langle v_1, d_1 \rangle, ..., \langle v_k, d_k \rangle \rangle \rangle$}, telles que :
  \begin{description}[-------]
  \item[\alert{$q$}] $\in Q$ : \structure{l'état de départ}
  \item[\alert{$u_i$}] $\in \Gamma\alert{^\star}$ : \structure{le mot lu sur le ruban \alert{$r_i$}}
  \item[\alert{$q'$}] $\in Q$ : \structure{l'état d'arrivée}
  \item[\alert{$v_i$}] $\in \Gamma\alert{^\star}$ : \structure{le mot écrit sur le ruban \alert{$r_i$}}
  \item[\alert{$d_i$}] $\in \{\triangleleft, \alert{\diamond}, \triangleright\}$ : \structure{le déplacement de la tête de \alert{$r_i$}}
  \end{description}
 
  \on[x=37.5mm,y=10mm]{
    \begin{tikzpicture}[turingMachine, x=25mm]
      \state (q)  at (0,0) {$q$}; 
      \state (q1) at (1,0) {$q'$}; 
      \path  (q) edge node {\smGroup{\smTMtrans[r_1]{u_1}{v_1}{d_1}...\\\smTMtrans[r_k]{u_k}{v_k}{d_k}}} (q1);
    \end{tikzpicture}
  }

  \begin{block}{Nouvelles fonctionnalités}
    \begin{description}[Multi-ruban :]
    \item[Réécriture :] lire un mot \(u\in\Gamma^\star\), écrire un mot \(v\in\Gamma^\star\)
      \begin{itemize}
      \item ajoute/supprime des cases si $|u| \neq |v|$
      \end{itemize}
    \item[Immobilité :] $\diamond$ ne déplace pas la tête de lecture
      \begin{itemize}
      \item $\varepsilon$-transitions possibles : $\smTMtransS{\varepsilon}{\varepsilon}$
      \end{itemize}
    \item[Multi-ruban :] \(k\) rubans parallèles, une tête/opération par ruban (simultané).
      \begin{itemize}
      \item choisir un ruban pour les entrées/sorties
      \end{itemize}
    \end{description}
  \end{block}

\end{frame}




\endgroup

% SPDX-License-Identifier: CC-BY-SA-4.0
% Author: Matthieu Perrin
% Part: <Nom de la partie>
% Section: <Nom de la section>
% Sub-section: <Nom de la sous-section>  % (facultatif, laisser vide si non utilisé)
% Frame: <Titre de la slide>

\begingroup

\begin{frame}{Modélisation des machines de Turing étendues}

  \on[text, y=-2mm]{
    \begin{itemize}

    \item Généralisation des \structure{configurations} : 
      \alert{$\langle \langle G_1, ..., G_k \rangle, q, \langle D_1, ..., D_k \rangle \rangle$}
      \begin{description}[-------]
      \item[\alert{$G_i$}] $\in \Gamma\alert{^\star}$ : le mot à gauche de la tête (exclue)  sur le ruban \alert{$r_i$}
      \item[\alert{$D_i$}] $\in \Gamma\alert{^\star}$ : le mot à droite de la tête (incluse) sur le ruban \alert{$r_i$}
      \end{description}

    \item\vspace{1mm} Généralisation des \structure{transitions} : 
      \alert{$\langle \langle q, u_1, ..., u_k \rangle, \langle q', \langle v_1, d_1 \rangle, ..., \langle v_k, d_k \rangle \rangle \rangle$}
      \begin{description}[-------]
      \item[\alert{$q$}] $\in Q$ : \structure{l'état de départ}
      \item[\alert{$u_i$}] $\in \Gamma\alert{^\star}$ : \structure{le mot lu sur le ruban \alert{$r_i$}}
      \item[\alert{$q'$}] $\in Q$ : \structure{l'état d'arrivée}
      \item[\alert{$v_i$}] $\in \Gamma\alert{^\star}$ : \structure{le mot écrit sur le ruban \alert{$r_i$}}
      \item[\alert{$d_i$}] $\in \{\triangleleft, \alert{\diamond}, \triangleright\}$ : \structure{le déplacement de la tête de \alert{$r_i$}}
      \end{description}

    \item\vspace{2mm} Généralisation des \structure{actions} :

      \vspace{-4mm}{\footnotesize$$
        \begin{array}{@{\langle\,\langle...,\,}l@{, ...\rangle,\,}c@{,\,\langle...,}r@{, ...\rangle\rangle\,\structure{\leadsto}\,
              \langle\,\langle...,\,}l@{, ...\rangle\,}c@{,\,\langle...,}r@{, ...\rangle\rangle\quad\text{si}\quad}l}
          %
          \structure{G_i}             & \structure{q}  & \alert{u_i} \structure{D_i} &
          \structure{G_i} \alert{v_i} & \structure{q'} & \structure{D_i} &
          \structure{q \xrightarrow{\alert{\smTMtransR[r_i]{u_i}{v_i}}} q'}\\
          %
          \structure{G_i} \alert{c}   & \structure{q}  & \alert{u_i}   \structure{D_i} &
          \structure{G_i}             & \structure{q'} & \alert{c v_i} \structure{D_i} &
          \structure{q \xrightarrow{\alert{\smTMtransL[r_i]{u_i}{v_i}}} q'}\\
          %
          \structure{G_i}             & \structure{q}  & \alert{u_i} \structure{D_i} &
          \structure{G_i}             & \structure{q'} & \alert{v_i} \structure{D_i} &
          \structure{q \xrightarrow{\alert{\smTMtransS[r_i]{u_i}{v_i}}} q'}\\
          %
        \end{array}
        $$}
      
    \item\vspace{2mm} Généralisation du \structure{non-déterministe} :\\
      pour toutes transitions $\langle \langle \alert{q}, \alert{u_1}, ..., \alert{u_k} \rangle, ... \rangle \,\alert{\neq}\, \langle \langle \alert{q}, \alert{u'_1}, ..., \alert{u'_k} \rangle, ... \rangle$,
      $$\exists i \in \{1, ..., k\}, \alert{u_1 \notin \mathit{Prefix}(u'_i)} \land \alert{u'_1 \notin \mathit{Prefix}(u_i)}$$
      
    \end{itemize}
  }
  
  \on[x=37.5mm,y=8mm]{
    \begin{tikzpicture}[turingMachine, x=25mm]
      \state (q)  at (0,0) {$q$}; 
      \state (q1) at (1,0) {$q'$}; 
      \path  (q) edge node {\smGroup{\smTMtrans[r_1]{u_1}{v_1}{d_1}...\\\smTMtrans[r_k]{u_k}{v_k}{d_k}}} (q1);
      \path  (q) edge node[below] {\begin{tabular}{c}ou \smTMtransS[r_i]{\varepsilon}{\varepsilon}\\si non spécifié\end{tabular}} (q1);
    \end{tikzpicture}
  }

\end{frame}

\endgroup
\endinput

% SPDX-License-Identifier: CC-BY-SA-4.0
% Author: Matthieu Perrin
% Part: <Nom de la partie>
% Section: <Nom de la section>
% Sub-section: <Nom de la sous-section>  % (facultatif, laisser vide si non utilisé)
% Frame: <Titre de la slide>

\begingroup

\begin{frame}{Exemple -- Fonction calculable à plusieurs arguments}

  \onBlock[top=-2mm]{Addition de deux nombres entiers}{
    \begin{itemize}
    \item Entrées : $a$ et $b$, sur leur ruban respectif
    \item Sortie : $a + b$, sur le ruban $c$
    \end{itemize}
  }

  \on[x=-23mm, y=5mm]{
    \begin{tikzpicture}[tape, x=5mm, y=5mm]
      \node{a :};
      \cell{}    \smheadb<11>
      \cell{$0$} \smhead<1,10,12>
      \cell{$1$} \smheadb<9>
      \cell{$0$} \smheadb<8>
      \cell{$0$} \smheadb<7>
      \cell{$1$} \smheadb<6>
      \cell{$0$} \smheadb<5>
      \cell{$1$} \smheadb<4>
      \cell{$0$} \smheadb<3>
      \cell{}    \smheadfromb<2>{-8}
    \end{tikzpicture}
  }
  \on[x=-23mm, y=-5mm]{
    \begin{tikzpicture}[tape, x=5mm, y=5mm]
      \node{b :};
      \cell{}    \smheadb<11>
      \cell{$0$} \smhead<1,10,12>
      \cell{$0$} \smheadb<9>
      \cell{$0$} \smheadb<8>
      \cell{$0$} \smheadb<7>
      \cell{$1$} \smheadb<6>
      \cell{$1$} \smheadb<5>
      \cell{$0$} \smheadb<4>
      \cell{$0$} \smheadb<3>
      \cell{}    \smheadfromb<2>{-8}
    \end{tikzpicture}
  }

  \on[x=-23mm, y=-15mm]{
    \begin{tikzpicture}[tape, x=5mm, y=5mm]
      \node{c :};
      \cell{}                 \smheadb<11>
      \cell{$\onlyb<11->{0}$} \smheadb<10,12>
      \cell{$\onlyb<10->{1}$} \smheadb<9>
      \cell{$\onlyb<9->{0}$}  \smheadb<8>
      \cell{$\onlyb<8->{1}$}  \smheadb<7>
      \cell{$\onlyb<7->{0}$}  \smheadb<6>
      \cell{$\onlyb<6->{1}$}  \smheadb<5>
      \cell{$\onlyb<5->{1}$}  \smheadb<4>
      \cell{$\onlyb<4->{0}$}  \smhead<-3>
      \cell{}
    \end{tikzpicture}
  }

  \on[x=35mm, y=0mm]{
    \begin{tikzpicture}[turingMachine, x=18mm, y=25mm]
      \state[alert ob=<{-2}>,initial above]   (a) at (1,2) {a}; 
      \state[alert ob=<{3-6,8-11}>]     (0) at (1,1) {0}; 
      \state[alert ob=<{7}>]            (1) at (0,1) {1}; 
      \state[alert ob=<{12}>,accepting] (f) at (1,0) {f}; 
      
      \scriptsize
      \path (a) edge[loop left] node {\begin{tabular}{@{}c@{}}\smGroup{\smTMtransR[a]{x}{x}\smTMtransR[b]{y}{y}}\\$\forall x, y \in \{0,1\}$\end{tabular}} (a);
      \path (0) edge[loop right] node {$\delta_0$} (0);
      \path (1) edge[loop left]  node {$\delta_1$} (1);
      \path (0) edge[bend left]  node {\smGroup{\smTMtransL[a]{1}{1}\smTMtransL[b]{1}{1}\smTMtransL[c]{\blank}{0}}} (1);
      \path (1) edge[bend left]  node {\smGroup{\smTMtransL[a]{0}{0}\smTMtransL[b]{0}{0}\smTMtransL[c]{\blank}{1}}} (0);
      \path (a) edge             node {\smGroup{\smTMtransL[a]{\blank}{\blank}\smTMtransL[b]{\blank}{\blank}}} (0);
      \path (0) edge             node {\smGroup{\smTMtransR[a]{\blank}{\blank}\smTMtransR[b]{\blank}{\blank}\smTMtransR[c]{\blank}{\blank}}} (f);
    \end{tikzpicture}
  }

  \on[text, bottom]{\scriptsize
    \begin{itemize}
    \item $\delta_0 =
      \left\{\begin{array}{c} a : \smTMtransL{0}{0} \\ b : \smTMtransL{0}{0} \\ c : \smTMtransL{\blank}{0} \end{array}\right.
      \left\{\begin{array}{c} a : \smTMtransL{1}{1} \\ b : \smTMtransL{0}{0} \\ c : \smTMtransL{\blank}{1} \end{array}\right.
      \left\{\begin{array}{c} a : \smTMtransL{0}{0} \\ b : \smTMtransL{1}{1} \\ c : \smTMtransL{\blank}{1} \end{array}\right.
      $
    \item $\delta_1 =
      \left\{\begin{array}{c} a : \smTMtransL{0}{0} \\ b : \smTMtransL{1}{1} \\ c : \smTMtransL{\blank}{0} \end{array}\right.
      \left\{\begin{array}{c} a : \smTMtransL{1}{1} \\ b : \smTMtransL{0}{0} \\ c : \smTMtransL{\blank}{0} \end{array}\right.
      \left\{\begin{array}{c} a : \smTMtransL{1}{1} \\ b : \smTMtransL{1}{1} \\ c : \smTMtransL{\blank}{1} \end{array}\right.
      $
    \end{itemize}
  }
  
\end{frame}

\endgroup

\input{turing_machine/deterministic/extensions/compression}
% SPDX-License-Identifier: CC-BY-SA-4.0
% Author: Matthieu Perrin
% Part: <Nom de la partie>
% Section: <Nom de la section>
% Sub-section: <Nom de la sous-section>  % (facultatif, laisser vide si non utilisé)
% Frame: <Titre de la slide>

\begingroup

\begin{frame}[fragile]{Simulation d'une machine de Turing déterministe}
  \small
  
  \begin{block}{Théorèmes : équivalence entre langage IMP et MTD}
    \begin{itemize}
    \item Toute MTD peut être \structure{simulée} par un algorithme en IMP
      
      \begin{algorithm}[H]
        \SetKwFunction{Forcebrute}{force\_brute}
        \SetKwData{A}{lu}
        \SetKwData{Etat}{etat}
        \SetKwData{Tete}{tete}
        \SetKwData{Ruban}{ruban}
        \SetKwFunction{Ecrire}{ecrit}
        \SetKwFunction{Deplace}{deplace}
        \SetKwFunction{Trans}{transition}
        \Fn{simule($M$ : MTD, $u$ : mot) : booléen}{
          $\Ruban \leftarrow u$;
          $\Tete \leftarrow 1$;
          $\Etat \leftarrow q_0$\;
          \Tantque{$\Etat\neq q_a \land \Etat\neq q_r$}{
            $\A \leftarrow \Ruban[\Tete]$\;
            $\Ruban[\Tete] \leftarrow \Ecrire[\Etat][\A]$\;
            $\Tete \leftarrow \Tete+\Deplace[\Etat][\A]$\;
            $\Etat \leftarrow \Trans[\Etat][\A]$\;
            
            \uSi{$\Tete=0$} {
              $\Ruban \leftarrow \text{``$\blank$''} + \Ruban$; $\Tete \leftarrow 1$
            }
            \SinonSi{$\Tete>|\Ruban|$} {
              $\Ruban \leftarrow \Ruban + \text{``$\blank$''} $
            }
          }
          \Retourner $\Etat=q_a$\;
        }
      \end{algorithm}
    \item<2-> Réciproquement, tout algorithme IMP peut être \structure{compilé} en une MTD.
    \end{itemize}
  \end{block}
  
\end{frame}
\endgroup

 
\subsection{Lien entre décidable et semi-décidable}
% SPDX-License-Identifier: CC-BY-SA-4.0
% Author: Matthieu Perrin
% Part: <Nom de la partie>
% Section: <Nom de la section>
% Sub-section: <Nom de la sous-section>  % (facultatif, laisser vide si non utilisé)
% Frame: <Titre de la slide>

\begingroup

\begin{frame}{Caractérisation des langages décidables}
  
  Les langages décidables sont les langages semi-décidables dont le complémentaire est également semi-décidable. 
  
  \vspace{-1mm}
  \begin{block}{Théorème : Clôture par complémentaire}
  \vspace{-1mm}
    Soient $\Sigma$ un alphabet, $L \subseteq \Sigma^\star$ un langage et $\overline{L} = \Sigma^\star \setminus L$.\\
    Les trois conditions suivantes sont équivalentes :
    \begin{enumerate}
    \item $L$ est décidable
    \item $\overline{L}$ est décidable
    \item $L$ et $\overline{L}$ sont tous les deux semi-décidables
    \end{enumerate}
  \end{block}
  
  \pause
  \vspace{-1mm}
  \begin{block}{Démonstration}
  \vspace{-1mm}
    \begin{description}
    \item<2->[$1 \Rightarrow 2$ :]
      Soit $M = \langle \Sigma, \Gamma, \blank, Q, q_0, \alert{F}, \rightarrow \rangle$ une MTD qui décide $L$.\\
      $\overline{M} = \langle \Sigma, \Gamma, \blank, Q, q_0, \alert{Q \setminus F}, \rightarrow \rangle$ est une MTD qui décide $\overline{L}$.\\
    \item<3->[$2 \Rightarrow 3$ :] Décidable implique semi-décidable par définition.
      %$\overline{L}$ est semi-décidable par définition. \\
      %$L = \overline{\overline{L}}$ est décidable par ($1 \Rightarrow 2$), donc semi-décidable.
    \item<4->[$3 \Rightarrow 1$ :] Il faut exécuter les deux machines de Turing en parallèle.\\
      L'une finit par s'arrêter en donnant la réponse.
    \end{description}
  \end{block}
\end{frame}

\endgroup


\section{Machines de Turing Non-Déterministes}

\subsection{Notion de non-déterminisme}
% SPDX-License-Identifier: CC-BY-SA-4.0
% Author: Matthieu Perrin
% Part: 
% Section: 
% Sub-section: 
% Frame: 

\begingroup

\begin{frame}{Machine de Turing non-déterministe}
  
  \onBlock[top=-4mm]{Rappel -- Machine de Turing déterministe}{
    Soit $M = \langle \Sigma, \Gamma, \blank, Q, q_0, F, \rightarrow \rangle$ une machine de Turing.
    \begin{itemize}
    \item $M$ est \structure{déterministe} si sa relation de transition $\rightarrow$ est fonctionnelle.
    \item \alert{``Non-déterministe''} signifie \alert{``pas forcément déterministe''}.
    \end{itemize}
    \begin{description}[Ubiquité :]
    \item[Modèle :] on autorise le modèle étendu (multi-ruban, immobilité, réécriture)
    \item[Ubiquité :] si non-déterminisme, un \structure{oracle} \alert{\emph{devine}} la meilleure transition
    \end{description}
  }
  
  \onBlock<2->[y=-9mm]{Comment utiliser le non-déterminisme ?}{
    \begin{itemize}
    \item Déplacement à un endroit \alert{arbitraire} du ruban
    \item Écriture d'un mot \example{arbitraire} sur le ruban
    \item Choix \structure{arbitraire} d'une sous-machine à exécuter
    \end{itemize}
  }

  \on<2->[x=40mm, y=-10mm] {
    \begin{tikzpicture}[turingMachine, x=15mm, y=8mm]\scriptsize
      \state[alert]     (01) at (0,1) {\faRandom}; 
      \state[example]   (11) at (1,1) {\faRandom}; 
      \state[example]   (21) at (2,1) {\faCheck}; 
      \state[structure] (00) at (0,0) {$A$}; 
      \state[structure] (10) at (1,0) {\faRandom}; 
      \state[structure] (20) at (2,0) {$B$}; 

      \path (01) edge[loop above] node  {\smAlign{\smTMtransR{x}{x}\smTMtransL{x}{x}}} (01);
      \path (11) edge[loop above] node  {\smAlign{\smTMtransL{\blank}{a}\smTMtransL{\blank}{b}}} (11);
      \path (11) edge node  {\smTMtransR{\blank}{\blank}} (21);

      \path (10) edge node  {\smTMtransS{\varepsilon}{\varepsilon}} (20);
      \path (10) edge node[swap] {\smTMtransS{\varepsilon}{\varepsilon}} (00);

      \node at (2,2) {$\forall x\in \Sigma$}; 
    \end{tikzpicture}
  }
  
  \onBlock<2->[y=-20mm, left=.48\textwidth, anchor=north]{Reconnaître un langage $L$}{
    \begin{description}[Entrée :]
    \item[Entrée :] $u\in \Sigma^\star$ 
    \item[Sortie :] \alert{une} exécution $\cmark{}$ si $u\in L$
    \end{description}
  }
  
  \onBlock<2->[y=-20mm, right=.48\textwidth, anchor=north]{Générer un langage $L$}{
    \begin{description}[Entrée :]
    \item[Entrée :] le mot vide $\varepsilon$ 
    \item[Sortie :] \alert{un} mot arbitraire de $L$
    \end{description}
  }

\end{frame}

\endgroup

% SPDX-License-Identifier: CC-BY-SA-4.0
% Author: Matthieu Perrin
% Part: 
% Section: 
% Sub-section: 
% Frame: 

\begingroup

\begin{frame}{Reconnaissance par une MTND}

  \onBlock[top=-4mm]{Définition -- Langage reconnu (ou accepté)}{
    Soient $M=\langle \Sigma, \Gamma, \blank, Q, q_0, F, \rightarrow \rangle$ une MTND et $L\subseteq \Sigma^\star$.
    \begin{itemize}
    \item On dit que \structure{$M$ reconnaît $L$} si \alert{$L = \mathcal{L}(M)$}
    \item \structure{Rappel :} $\mathcal{L}(M) \eqdef \left\{u \in \Sigma^\star \;\middle\mid\; \alert{\exists c \in \mathcal{C}_M^+, C_{\mathit{init}}(u) \leadsto_M^\star c} \right\}$
    \item On dit que \structure{$L$ est reconnaissable} s'il existe une MTND qui reconnaît $L$
    \item \structure{Rappel :} Si $M$ est déterministe, alors $\mathcal{L}(M)$ est semi-décidé par $M$
    \end{itemize}
  }

  \onExampleBlock[y=-2mm]{Exemple -- Reconnaissance de $\left\{u \cdot u \mid u \in \{a, b\}^\star \right\}$}{}

  \on[x=-25mm, y=-11mm]{
    \begin{tikzpicture}[tape, x=5mm, y=5mm]
      \node{\textsc{i}};
      \cell{}    
      \cell[structure on=<1>]{a} \smhead<1>
      \cell[structure ob=<2>]{b} \smheadb<2>
      \cell[structure ob=<3>]{a} \smheadb<3,4>
      \cell{b} \smheadb<6>  
      \cell{}  \smheadfromb<5>{-2}
    \end{tikzpicture}
  }

  \on[x=25mm, y=-11mm]{
    \begin{tikzpicture}[tape, x=5mm, y=5mm]
      \node{\textsc{u}};
      \cell{}                           \smheadfromb<4>{2}
      \cell[alert on=<1>]{\only<2->{a}} \smhead<1>
      \cell[alert ob=<2>]{\only<3->{b}} \smheadb<2,6>
      \cell[alert ob=<3>]{}             \smheadb<3>\smheadfromb<5>{-2}
      \cell{}    
      \cell{}  
    \end{tikzpicture}
  }

  \on[y=-30mm]{
    \begin{tikzpicture}[turingMachine,x=30mm]
      \state[initial, structure on=<1-3>] (0) at (0,0) {\faRandom};
      \state[structure ob=<4>]            (1) at (1,0) {\faFastBackward};
      \state[structure ob=<5>]            (2) at (2,0) {\faForward};
      \state[accepting, structure ob=<6>] (3) at (3,0) {\cmark};

      \path[structure on=<{1,3}>] (0) edge[loop above] node {\smGroup{\smTMtransR[\textsc{i}]{a}{a}\smTMtransR[\textsc{u}]{\alert<1,3>{\blank}}{a}}}   (0);
      \path[structure ob=<2>]     (0) edge[loop below] node {\smGroup{\smTMtransR[\textsc{i}]{b}{b}\smTMtransR[\textsc{u}]{\alertb<2>{\blank}}{b}}}    (0);
      \path[structure on=<1-3>]   (0) edge             node {\smTMtransL[\textsc{u}]{\alert<1-3>{\blank}}{\blank}}                                     (1);
      \path                       (1) edge[loop above] node {\smTMtransL[\textsc{u}]{a}{a}}                                                            (1);
      \path                       (1) edge[loop below] node {\smTMtransL[\textsc{u}]{b}{b}}                                                            (1);
      \path                       (1) edge             node {\smTMtransR[\textsc{u}]{\blank}{\blank}}                                                  (2);
      \path                       (2) edge[loop above] node {\smGroup{\smTMtransR[\textsc{i}]{a}{a}\smTMtransR[\textsc{u}]{a}{a}}}                     (2);
      \path                       (2) edge[loop below] node {\smGroup{\smTMtransR[\textsc{i}]{b}{b}\smTMtransR[\textsc{u}]{b}{b}}}                     (2);
      \path                       (2) edge             node {\smGroup{\smTMtransL[\textsc{i}]{\blank}{\blank}\smTMtransL[\textsc{u}]{\blank}{\blank}}} (3);
    \end{tikzpicture}
  }

\end{frame}

\endgroup



% SPDX-License-Identifier: CC-BY-SA-4.0
% Author: Matthieu Perrin
% Part: <Nom de la partie>
% Section: <Nom de la section>
% Sub-section: <Nom de la sous-section>  % (facultatif, laisser vide si non utilisé)
% Frame: <Titre de la slide>

\begingroup

\begin{frame}{Un exemple : le \og Subset-Sum Problem\fg}

  \Probleme{Subset-Sum}{
    Un ensemble fini d'entiers $E \subset \mathbb{Z}$
  }{
    Existe-t-il $S \subseteq E$ non vide tel que $\displaystyle\alert{\sum_{n\in S} n = 0}$ ?
  }

  \begin{exampleblock}{Exemple}
    \centering
    \begin{tikzpicture}[y=8mm]\large
      \node at (5,3)   {$31$};
      \node at (2,1)   {$27$};
      \node at (3,2)   {$-11$};
      \node at (4.5,0) {$81$};
      \node at (5,1)   {$22$};
      \node at (6.5,2) {$-15$};
      \node at (7.5,0) {$99$};
      \node at (8,1)   {$77$};
      \node at (6,3)   {$52$};
      \node at (7.5,3) {\alert<2->{$47$}};
      \node at (9,1)   {\alert<2->{$-82$}};
      \node at (3,0)   {\alert<2->{$58$}};
      \node at (3.5,1) {\alert<2->{$43$}};
      \node at (4.5,2) {\alert<2->{$-19$}};
      \node at (6,0)   {\alert<2->{$75$}};
      \node at (6.5,1) {\alert<2->{$-54$}};
      \node at (8,2)   {\alert<2->{$-33$}};
      \node at (9,3)   {\alert<2->{$14$}};
      \node at (3.5,3) {\alert<2->{$-49$}};
    \end{tikzpicture}
  \end{exampleblock}
  
  \pause
  \begin{alertblock}{Réponse}
    \begin{itemize}
    \item $47-82+58+43-19+75-54-33 + 14 -49 = 0$
    \end{itemize}
  \end{alertblock}

\end{frame}

\endgroup

% SPDX-License-Identifier: CC-BY-SA-4.0
% Author: Matthieu Perrin
% Part: 
% Section: 
% Sub-section: 
% Frame: 

\begingroup

\SetKwFunction{SubsetSum}{subset\_sum}
\newcommand\somme{\textsc{s}}
\newcommand\E{\textsc{e}}
\newcommand\separator{\bullet}

\tikzset{thebrace/.style={decorate, decoration={brace, amplitude=5pt, raise=0pt, mirror}},}

\begin{frame}{Exemple pour $\{4, 3, -4\}$}
  
  \on[left=.5\textwidth, top] {
    \begin{algorithm}[H]
      \Fun{$\SubsetSum(\E \in \mathbb{Z}[])$}{
        $\somme \leftarrow 0$;\\
        \For{$x \in \E$}{
          \If{choix nondéterministe}{
            $\somme \leftarrow \somme + x$;
          }
        }
        \Return $\somme = 0$;
      }
    \end{algorithm}
  }

  \on[x=-30mm, y=-5mm] {
    \begin{tikzpicture}[tree, x=5mm, y=7mm]
      \tree{\alert{\example<1>{$4$}}}{
        \tree[edge={structure ob=<2->}, node={structure ob=<2->}]{$3$}{
          \tree{$-4$}{
            \tree{\xmark}{}
            \tree{\xmark}{}
          }
          \tree[edge={structure ob=<2->}, node={structure ob=<2->}]{$-4$}{
            \tree{\xmark}{}
            \tree[edge={structure ob=<2->}]{\xmark}{}
          }
        }
        \tree[edge={example}, node={example}]{$3$}{
          \tree[edge={example}, node={example}]{$-4$}{
            \tree{\xmark}{}
            \tree[edge={example}]{\cmark}{}
          }
          \tree{$-4$}{
            \tree{\xmark}{}
            \tree{\xmark}{}
          }
        }
      }
    \end{tikzpicture}
  }

  \on<2->[x=25mm, y=10mm] {
    \begin{tikzpicture}[turingMachine, x=10mm, y=22mm]
      \state[alert ob=<2>,                                  initial] (0) at (1,0) {\faMapMarker}; 
      \state[alert ob=<{3,5,7,9}>, structure ob=<4>, example ob=<6>] (1) at (2,1) {\faRandom}; 
      \state[example ob=<10>,                             accepting] (2) at (3,0) {\faCheck}; 
      \state[structure ob=<6>, example ob=<4>                      ] (3) at (0,1) {+}; 
      \state[alert ob=<8>                                          ] (4) at (4,1) {-}; 
      
      \path                                       (0) edge                  node[left]  {\smTMtransS[\somme]{\blank}{\text{\tiny\faMapMarker}}} (1);
      \path                                       (1) edge                  node[right] {\smGroup{\smTMtransL[\E]{\text{\blank}}{\blank}\smTMtransS[\somme]{\text{\tiny\faMapMarker}}{\blank}}} (2);
      \path[example ob=<3>, structure ob=<5>]     (1) edge                  node        {\smTMtransR[\E]{+}{+}} (3);
      \path                                       (1) edge                  node[swap]  {\smTMtransR[\E]{-}{-}} (4);
      \path                                       (3) edge[bend left =15pt] node        {\smTMtransR[\E]{\separator}{\separator}} (1);
      \path                                       (4) edge[bend right=15pt] node[swap]  {\smTMtransR[\E]{\separator}{\separator}} (1);
      \path[structure ob=<3-4>, example ob=<5-6>] (1) edge[loop above]      node        {\smTMtransR[\E]{x}{x}} (1);
      \path[example ob=<4>, structure ob=<6>]     (3) edge[loop above]      node        {\smGroup{\smTMtransR[\E]{1}{1}\smTMtransR[\somme]{x}{x}}} (3);
      \path                                       (4) edge[loop above]      node        {\smGroup{\smTMtransR[\E]{1}{1}\smTMtransL[\somme]{x}{x}}} (4);

      \node[structure] at (2,1.6) {$\forall x \in \Gamma$};
    \end{tikzpicture}
  }
  
  \on[y=-27mm] {
    \begin{tikzpicture}[tape, x=5mm, y=5mm]
      \node{\E:};
      \cell{}
      \cell[alert on=<3>]{+} \smheadb<2-3>        \smsave{AD}
      \cell{1}               
      \cell{1}               
      \cell{1}               
      \cell{1}                                    \smsave{AF}
      \cell{$\separator$}    \smheadfromb<4>{-4}
      \cell[alert ob=<5>]{+} \smheadb<5>          \smsave{BD}
      \cell{1}               
      \cell{1}               
      \cell{1}                                    \smsave{BF}
      \cell{$\separator$}    \smheadfromb<6>{-3}
      \cell[alert ob=<7>]{-} \smheadb<7>          \smsave{CD}
      \cell{1}               
      \cell{1}               
      \cell{1}               
      \cell{1}                                    \smsave{CF}
      \cell{$\separator$}   \smheadfromb<8>{-4}   \smheadb<10>[example]
      \cell{}               \smheadb<9>

      \uncover<1>{
        \path (AD.south west) edge[thebrace] node[yshift=-4mm]{$4$}  (AF.south east);
        \path (BD.south west) edge[thebrace] node[yshift=-4mm]{$3$}  (BF.south east);
        \path (CD.south west) edge[thebrace] node[yshift=-4mm]{$-4$} (CF.south east);
      }
    \end{tikzpicture}
  }

  \on<1>[text,y=-38mm] {
    \begin{itemize}
    \item Encodage de $E$ comme une concaténation de valeurs suivies de $\bullet$
    \item Encodage des valeurs en unaire
    \end{itemize}
  }

  
  \ob<2->[y=-38mm] {
    \begin{tikzpicture}[tape, x=5mm, y=5mm]
      \node{\somme:};
      \cell{}
      \cell{}
      \cell{}
      \cell{}                        \smheadfrom[ob=<8>, structure]{4} \smhead[ob=<9>, structure]
      \cell{\only<3->{\faMapMarker}} \smhead[on=<-3>] \smhead[ob=<4-5>, structure]\smheadfrom[ob=<8>, example]{4}\smhead[ob=<9->, example]
      \cell{}
      \cell{}
      \cell{}                        \smheadfrom[ob=<6>, structure]{-3}\smhead[ob=<7>, structure]
      \cell{}                        \smheadfrom[ob=<4>, example]{-4}\smhead[ob=<5-7>, example]
      \cell{}
      \cell{}
      \cell{}
      \cell{}
      \cell{}
      \cell{}
      \cell{}
      \cell{}
      \cell{}
      \cell{}
    \end{tikzpicture}
  }
  
\end{frame}

\endgroup

% SPDX-License-Identifier: CC-BY-SA-4.0
% Author: Matthieu Perrin
% Part: 
% Section: 
% Sub-section: 
% Frame: 

\begingroup

\begin{frame}{Générateur d'un langage}
  
  \onBlock[top=-2mm]{Langage généré par $M$}{
    Soit $M=\langle \Sigma, \Gamma, \blank, Q, q_0, F, \rightarrow \rangle$ une machine de Turing. 
 
    Le langage \structure{généré} par $M$ est l'ensemble $\alert{\mathcal{L}_G(M)}$ des mots écrits sur le ruban quand $M$ atteint un état accepteur à partir de $C_{\mathit{init}}(\varepsilon)$ :
    $$
    \alert{\mathcal{L}_G(M) \eqdef \left\{ \mathit{mot}(r) \in \Sigma^\star \middle| \exists q_f\in F, C_{\mathit{init}}(\varepsilon) \leadsto_M^\star \langle r, q_f \rangle \right\}}.
    $$
 
    \begin{itemize}
    \item $\langle r, q_f \rangle$ n'est pas forcément une configuration d'arrêt
    \item $M$ est appelée un \structure{générateur} du langage $\mathcal{L}_G(M)$
    \end{itemize}
  }
  
  \onExampleBlock[y=-15mm]{Exemple : générateur du langage $\mathcal{L}(ba^\star b)$}{}
 
  \on[y=-30mm, x=-30mm]{
    \begin{tikzpicture}[tape, x=7mm, y=7mm]
      \cell{$\blank$}
      \cell{\alt<-1>{$\blank$}{$b$}} \smheadb<1>  
      \cell{\alt<-2>{$\blank$}{$a$}} \smheadb<2> 
      \cell{\alt<-2>{$\blank$}{$a$}} 
      \cell{\alt<-3>{$\blank$}{$b$}} \smheadfromb<3>{-2}
      \cell{$\blank$}                \smheadb<4> 
    \end{tikzpicture}
  }
  
  \on[y=-30mm, x=30mm]{
    \begin{tikzpicture}[turingMachine]
      \state[alert on=<{1}>,   initial] (0) at (0,0) {$0$}; 
      \state[alert ob=<{2,3}>         ] (1) at (1,0) {$1$}; 
      \state[alert ob=<{4}>, accepting] (2) at (2,0) {$2$}; 
      
      \path (0) edge             node {\smTMtransR{\blank}{b}} (1);
      \path (1) edge[loop below] node {\smTMtransR{\blank}{a}} (1);
      \path (1) edge             node {\smTMtransR{\blank}{b}} (2);
    \end{tikzpicture}
  }

\end{frame}

\endgroup

% SPDX-License-Identifier: CC-BY-SA-4.0
% Author: Matthieu Perrin
% Part: 
% Section: 
% Sub-section: 
% Frame: 

\begingroup

\begin{frame}{Équivalence entre automates à pile et grammaires}

  \onBlock[top=-2mm, left=.6\textwidth]{Théorème de Chomsky et Schützenberger}{
    Un langage est \structure{algébrique} si et seulement s’il est reconnu par un \structure{automate à pile non déterministe}.
  }

  \onImage[top, x=.35\textwidth]{%
    height=25mm,
    title={Marcel-Paul Schützenberger},
    licenselogo={\ccby},
    license={{\href{https://creativecommons.org/licenses/by/2.0/}{CC BY-2.0}} -- Konrad Jacobs, 1972 (\href{https://commons.wikimedia.org/wiki/File:Sch\%C3\%BCtzenberger.jpeg?uselang=fr}{Wikimedia})},
    img={Schutzenberger.jpeg}
  }

 \onExampleBlock<2->[y=0mm]{Exemple : $\{ a^n c b^n \mid n\in \mathbb{N} \}$}{}
 
 \on<2->[bottom=9mm,x=-.45\textwidth]{
   \begin{tikzpicture}[stack, x=7mm, y=7mm]
     \cell[alert ob=<2>]{\oneof[$S$]{\on<3->{$T$}\on<5->{$B$}\on<7->{}}}
     \cell              {\oneof[]{\on<3>{$A$}\on<5>{$S$}}}
   \end{tikzpicture}
 }
 
 \on<2->[bottom=23mm,x=-.3\textwidth]{
   \begin{tikzpicture}[word, x=5mm, y=5mm]
     \cell[alert ob=<4->]{$a$} \smhead[on=<-3>]
     \cell[alert ob=<6->]{$c$} \smhead[ob=<4-5>]
     \cell[alert ob=<7->]{$b$} \smhead[ob=<6>]
   \end{tikzpicture}
 }
 
 \on<2->[bottom=10mm,x=-.15\textwidth]{\small
   \begin{tikzpicture}[pushdown]
     \state[initial above, accepting, initial text={\alertb<2>{$S$}}] (q) {};
     \path (q) edge [loop left] node {
       $\begin{array}{r}
         \alertb<4>{\smPAtrans{a}{A}{\varepsilon}} \\
         \alertb<7>{\smPAtrans{b}{B}{\varepsilon}} \\
         \alertb<6>{\smPAtrans{c}{S}{\varepsilon}} \\
       \end{array}$
     } (q);
     \path (q) edge [loop right] node {
       $\begin{array}{l}
         \alertb<3>{\smPAtrans{\varepsilon}{S}{TA}} \\
         \alertb<5>{\smPAtrans{\varepsilon}{T}{BS}} \\
       \end{array}$
     } (q);
   \end{tikzpicture}
 }
 
 \on<2->[bottom=15mm, x=.2\textwidth, width=2.5cm]{
   \example{Grammaire}\\\vspace{2mm}
   $\left\{\begin{array}{@{\,}r@{~\rightarrow~}l@{\,}}
   S & \alertb<3>{AT} \mid \alertb<6>{c}\\
   T & \alertb<5>{SB}\\
   A & \alertb<4>{a}\\
   B & \alertb<7>{b}\\
   \end{array}\right.$
 }
 
 \on<2->[bottom=10mm,x=.4\textwidth]{
   \begin{tikzpicture}[tree, x=7mm,y=7mm, tree node internal/.append style={structure,}, tree node leave/.append style={example,},]
     \tree{\alertb<2>{$S$}}{
       \tree[on=<3->]{\alertb<3>{$A$}}{
         \tree[on=<4->,yshift=-1]{\alertb<4>{$a$}}{}
       }
       \tree[on=<3->]{\alertb<3>{$T$}}{
         \tree[on=<5->]{\alertb<5>{$S$}}{
           \tree[on=<6->]{\alertb<6>{$c$}}{}
         }
         \tree[on=<5->]{\alertb<5>{$B$}}{
           \tree[on=<7->]{\alertb<7>{$b$}}{}
         }
       }
     }
   \end{tikzpicture}
 }

  \footnoteref{Chomsky, Schützenberger. \emph{The algebraic theory of context-free languages.} SLFM. (1959)}
  
\end{frame}

\endgroup

 
\subsection{Complexité non-déterministe}
% SPDX-License-Identifier: CC-BY-SA-4.0
% Author: Matthieu Perrin
% Part: <Nom de la partie>
% Section: <Nom de la section>
% Sub-section: <Nom de la sous-section>  % (facultatif, laisser vide si non utilisé)
% Frame: <Titre de la slide>

\begingroup

\begin{frame}{Complexité d'une machine de Turing déterministe}
  Soient $M=\langle \Sigma, \Gamma, \blank, Q, q_0, F, \rightarrow \rangle$ une MTD qui termine, $u \in \Sigma^\star$ un mot et $n\in \mathbb{N}$.

  \begin{block}{Complexité temporelle déterministe}
    \begin{itemize}
    \item\vspace{-1mm} La \structure{complexité temporelle} de $M$ sur $u$ est le \alert{nombre d'actions} lors de l'exécution de $u$ par $M$ :

      \vspace{-2mm}
      $$\alert{T_M(u) \eqdef \operatorname{unique}\left\{n \in \mathbb{N} \mid \exists c_f \in \mathcal{C}_M^\mathit{halt}, C_{\mathit{init}}(u) \leadsto^n c_f\right\}}.$$ 

    \item\vspace{-1mm} La \structure{complexité temporelle dans le pire cas} de $M$, pour $n\in \mathbb{N}$, est : \\

      \vspace{-2mm}
      $$\alert{T_M(n) \eqdef \max \left\{T_M(u) \mid u \in \Sigma^n \right\}}.$$ 
    \end{itemize}
  \end{block}

  \vspace{-2mm}
  \begin{block}{Complexité spatiale déterministe}
    \begin{itemize}
    \item\vspace{-1mm} La \structure{complexité spatiale} de $M$ sur $u$ est le \alert{nombre maximal de cases du ruban utilisées} lors de l'exécution de $u$ par $M$ :

      \vspace{-2mm}
      $$\alert{S_M(u) \eqdef \max\left\{|GD| \mid \exists q\in Q, C_{\mathit{init}}(u) \leadsto^\star \langle G, q, D \rangle \right\}}.$$ 

      
    \item\vspace{-1mm} La \structure{complexité spatiale dans le pire cas} de $M$, pour $n\in \mathbb{N}$, est : \\

      \vspace{-2mm}
      $$\alert{S_M(n) \eqdef \max \{S_M(u) \mid u \in \Sigma^n \}}.$$ 
    \end{itemize}
  \end{block}

\end{frame}

\endgroup

% SPDX-License-Identifier: CC-BY-SA-4.0
% Author: Matthieu Perrin
% Part: <Nom de la partie>
% Section: <Nom de la section>
% Sub-section: <Nom de la sous-section>  % (facultatif, laisser vide si non utilisé)
% Frame: <Titre de la slide>

\begingroup

\begin{frame}{Complexité d'un problème}

  \begin{block}{Classes de complexité d'un problème}
    Soit $f$ une suite de $\mathbb{N}$ dans $\mathbb{R}^+$. 
    On définit les \structure{classes de complexité} : 
    \begin{description}[$\textsc{dspace}(f)$ :]
    \item[$\textsc{dtime}(f)$ :] les problèmes décidables en temps $\mathcal{O}(f)$ \structure{par une MTD} :
      
      \vspace{-2mm}
      $$\alert{\textsc{dtime}(f) \eqdef \{\structure{\alert{L} \in \textsc{lang} \mid \exists M, L = \mathcal{L}(M) \land \alert{T_M \in \mathcal{O}(f)}} \}}$$
    \item[$\textsc{dspace}(f)$ :] les problèmes décidables en espace $\mathcal{O}(f)$ \structure{par une MTD} :

      \vspace{-2mm}
      $$\alert{\textsc{dspace}(f) \eqdef \{\structure{\alert{L} \in \textsc{lang} \mid \exists M, L = \mathcal{L}(M) \land \alert{S_M \in \mathcal{O}(f)}} \}}$$
    \end{description}
  \end{block}

  \pause
  \vspace{-1mm}
  \begin{block}{Classes de complexité importantes}

    \begin{tikzpicture}[y=4mm, x=21mm]
      \draw (0,0) -- (4,0);
      \draw (0,2) -- (4,2);
      \draw (0,4) -- (4,4);
      \draw (0, 0) -- (0, 4);
      \draw (2, 0) -- (2, 4);
      \draw (4, 0) -- (4, 4);
      
      \node[align=center] at (1,3) {$\alert{\displaystyle \textsc{p} = \bigcup_{k\in \mathbb{N}} \textsc{dtime}\left(n^k\right)}$};
      \node[align=center] at (3,3) {$\alert{\displaystyle \textsc{exptime} = \bigcup_{k\in \mathbb{N}} \textsc{dtime}\left(2^{n^k}\right)}$};
      \node[align=center] at (1,1) {$\alert{\displaystyle \textsc{pspace} = \bigcup_{k\in \mathbb{N}} \textsc{dspace}\left(n^k\right)}$};
      \node[align=center] at (3,1) {$\alert{\displaystyle \textsc{expspace} = \bigcup_{k\in \mathbb{N}} \textsc{dspace}\left(2^{n^k}\right)}$};

      \node[align=center, above] at (1,4) {Complexité \structure{polynomiale}};
      \node[align=center, above] at (3,4) {Complexité \structure{exponentielle}};
      \node[align=right,  left ] at (0,1) {Complexité\\[-2pt]\structure{spatiale}};
      \node[align=right,  left ] at (0,3) {Complexité\\[-2pt]\structure{temporelle}};
    \end{tikzpicture}
    
    $$ \textsc{dtime}\left(\log(n)\right) \subsetneq \alert{\textsc{p}} \subsetneq \textsc{dtime}\left(n^{\log(n)}\right) \subsetneq \alert{\textsc{exptime}} \subsetneq \textsc{dtime}\left(n!\right) $$
  \end{block}
  
\end{frame}

\endgroup
\endinput

% SPDX-License-Identifier: CC-BY-SA-4.0
% Author: Matthieu Perrin
% Part: <Nom de la partie>
% Section: <Nom de la section>
% Sub-section: <Nom de la sous-section>  % (facultatif, laisser vide si non utilisé)
% Frame: <Titre de la slide>

\begingroup

\begin{frame}{Classes de complexité importantes}

  \begin{block}{Classes de complexité \alert{non-déterministe} importantes}

    \begin{tikzpicture}[y=8mm, x=22mm]\footnotesize
      \draw (0,0) -- (4,0);
      \draw (0,2) -- (4,2);
      \draw (0,4) -- (4,4);
      \draw (0, 0) -- (0, 4);
      \draw (2, 0) -- (2, 4);
      \draw (4, 0) -- (4, 4);
      
      \node[align=center] at (1,3) {
        $\begin{array}{@{}r@{~}c@{~}l@{}}
          \alert{\textsc{np}} & \alert{\eqdef} & \displaystyle\alert{\bigcup_{k\in \mathbb{N}} \textsc{ntime}\left(n^k\right)}\\
          \textsc{co-np} & \eqdef & \left\{ \overline{L} \mid L \in \textsc{np} \right\}
        \end{array}$
      };
      \node[align=center] at (3,3) {
        $\begin{array}{@{}r@{~}c@{~}l@{}}
          \alert{\textsc{nexptime}} & \alert{\eqdef} & \displaystyle \alert{\bigcup_{k\in \mathbb{N}} \textsc{ntime}\left(2^{n^k}\right)}\\
          \textsc{co-nexptime} & \eqdef & \left\{ \overline{L} \mid L \in \textsc{nexptime} \right\}
        \end{array}$
      };
      \node[align=center] at (1,1) {
        $\begin{array}{@{}r@{~}c@{~}l@{}}
          \alert{\textsc{npspace}} & \alert{\eqdef} & \displaystyle\alert{\bigcup_{k\in \mathbb{N}} \textsc{nspace}\left(n^k\right)}\\
          \textsc{co-npspace} & \eqdef & \left\{ \overline{L} \mid L \in \textsc{npspace} \right\}
        \end{array}$
      };
      \node[align=center] at (3,1) {
        $\begin{array}{@{}r@{~}c@{~}l@{}}
          \alert{\textsc{nexpspace}} & \alert{\eqdef} & \displaystyle \alert{\bigcup_{k\in \mathbb{N}} \textsc{nspace}\left(2^{n^k}\right)}\\
          \textsc{co-nexpspace} & \eqdef & \left\{ \overline{L} \mid L \in \textsc{nexpspace} \right\}
        \end{array}$
      };
      \node[align=center, above] at (1,4) {Complexité \structure{polynomiale}};
      \node[align=center, above] at (3,4) {Complexité \structure{exponentielle}};
      \node[align=right,  left ] at (0,1) {Complexité\\[-2pt]\structure{spatiale}};
      \node[align=right,  left ] at (0,3) {Complexité\\[-2pt]\structure{temporelle}};
    \end{tikzpicture}
  \end{block}

  \structure{Remarque :} classes stables pour les extensions du modèle de MTND

  \begin{exampleblock}{Exemple -- Le problème \textsc{Subset-Sum}}
    \begin{itemize}
    \item On a une MTND étendue qui reconnaît \textsc{Subset-Sum} en temps linéaire
    \item Donc il y a une MTND stricte qui reconnaît \textsc{Subset-Sum} en temps $\textsc{poly}$
    \item Donc \example{$\textsc{Subset-Sum} \in \textsc{np}$}
    \end{itemize}
  \end{exampleblock}
  
\end{frame}

\endgroup

% SPDX-License-Identifier: CC-BY-SA-4.0
% Author: Matthieu Perrin
% Part: <Nom de la partie>
% Section: <Nom de la section>
% Sub-section: <Nom de la sous-section>  % (facultatif, laisser vide si non utilisé)
% Frame: <Titre de la slide>

\begingroup

\begin{frame}{Inclusion des classes de complexité}

  \on[y=10mm]{
  \begin{tikzpicture}[x=6mm, y=3.5mm]\small
    \node[draw=example,   fill=example!10,   rounded corners, fit={(0,0)(15,15)}] (se) {};
    \node[                fill=white,        rounded corners, fit={(2,2)(14,14)}]      {};
    \node[draw=structure, fill=structure!10, rounded corners, fit={(1,1)(13,13)}] (ne) {};
    \node[draw=black,                        rounded corners, fit={(2,2)(14,14)}] (ce) {};
    \node[draw=alert,     fill=alert!10,     rounded corners, fit={(3,3)(12,12)}] (de) {};
    \node[draw=example,   fill=example!10,   rounded corners, fit={(4,4)(11,11)}] (sp) {};
    \node[                fill=white,        rounded corners, fit={(6,6)(10,10)}]      {};
    \node[draw=structure, fill=structure!10, rounded corners, fit={(5,5)(9,9)}]   (np) {};
    \node[draw=black,                        rounded corners, fit={(6,6)(10,10)}] (cp) {};
    \node[draw=alert,     fill=alert!10,     rounded corners, fit={(7,7)(8,8)}]   (dp) {};

    \node[example,   below] at (se.north)  {$\textsc{expspace} = \textsc{nexpspace} = \textsc{co-nexpspace}$ \footnotesize (Théorème de Savitch)};
    \node[structure, below] at (ne.north)  {$\textsc{nexptime}$};
    \node[black,     below] at (ce.north)  {$\textsc{co-nexptime}$};
    \node[alert,     below] at (de.north)  {$\textsc{exptime}$};
    \node[example,   below] at (sp.north)  {$\textsc{pspace} = \textsc{npspace} = \textsc{co-npspace}$};
    \node[black,     below] at (cp.north)  {$\textsc{co-np}$};
    \node[structure, below] at (np.north)  {$\textsc{np}$};
    \node[alert,          ] at (dp.center) {$\textsc{p}$};
  \end{tikzpicture}
  }
  
  \onBlock[y=-17mm, anchor=north, left=.5\textwidth]{Quelques résultats connus}{
    \begin{itemize}
    \item $\textsc{pspace} = \textsc{npspace} = \textsc{co-npspace}$
    \item $\textsc{p} \subsetneq \textsc{exptime}$
    \end{itemize}
  }

  \onBlock[y=-17mm, anchor=north, right=.5\textwidth]{Quelques résultats inconnus}{
    \begin{itemize}
    \item $\textsc{p} \stackrel{?}{=} \textsc{np} \stackrel{?}{=} \textsc{co-np} \stackrel{?}{=} \textsc{pspace}$
    \item $\textsc{pspace} \stackrel{?}{=} \textsc{exptime}$
    \end{itemize}
  }

  \footnoteref{W. J. Savitch. \textit{Relationships between nondeterministic and deterministic tape complexities.} JCSS (1970)}
  
\end{frame}

\endgroup
\endinput

 
\subsection{Déterminisation de la machine de Turing}
% SPDX-License-Identifier: CC-BY-SA-4.0
% Author: Matthieu Perrin
% Part: <Nom de la partie>
% Section: <Nom de la section>
% Sub-section: <Nom de la sous-section>  % (facultatif, laisser vide si non utilisé)
% Frame: <Titre de la slide>

\begingroup

\begin{frame}{Déterminisation d'une machine de Turing}
  
  \onBlock[top=-3mm]{Théorème -- Équivalence entre MTND et MTD}{
    Soit $\Sigma$ un alphabet, et $L \subseteq \Sigma^\star$ un langage sur $\Sigma$. 
    \begin{itemize}
    \item Si $L$ est reconnaissable, alors $L$ est semi-décidable.
    \item Si $L$ est généré par une MTND, alors $L$ est récursivement énumérable.
    \end{itemize}
  }

  \obBlock<1>[anchor=north, y=12mm]{Conséquence}{
    Les cinq notions suivantes sont équivalentes :
    \begin{itemize}
    \item $L$ est reconnaissable
    \item $L$ est semi-décidable
    \item $L$ est généré par une MTND
    \item $L$ est engendré par une grammaire
    \item $L$ est récursivement énumérable
    \end{itemize}
  }

  \onBlock<2>[anchor=north, y=12mm]{Démonstration}{
    Soit $M$ une MTND reconnaissant $L$.\\
    On construit une MTD $M_D$ reconnaissant $L$.
    \begin{itemize}
    \item On considère l'arbre d'exécution pour un mot $u$
      \begin{itemize}
      \item La racine est la configuration initiale $C_{\mathit{init}}(u)$
      \item Les fils de $c$ sont les $c'$ telles que $c\leadsto_M c'$
        \begin{itemize}
        \item Le nombre de fils de $c$ est borné par $|\rightarrow|$
        \end{itemize}
      \end{itemize}
    \item $M_D$  cherche une configuration acceptante
      \begin{itemize}
      \item Une exploration en profondeur ne fonctionne pas :
        \begin{itemize}
        \item certaines branches peuvent être infinies
        \end{itemize}
      \item $M_D$ exécute une exploration en largeur
        \begin{itemize}
        \item Nécessite une file (FIFO) de configurations
        \end{itemize}
      \end{itemize}
    \end{itemize}
  }

  \on<2>[x=40mm,y=5mm] {
    \begin{tikzpicture}[turingMachine, example, x=15mm]
      \state[accepting]     (2) at (0,0) {$2$}; 
      \state[initial above] (0) at (1,0) {$0$}; 
      \state                (1) at (2,0) {$1$}; 

      \path (0) edge[bend left=5mm] node       {$\smTMtransR{a}{a}$} (1);
      \path (1) edge[bend left=5mm] node       {$\smTMtransL{a}{a}$} (0);
      \path (0) edge                node[swap] {$\smTMtransR{a}{a}$} (2);
    \end{tikzpicture}
  }
  
  \on<2>[x=42mm,y=-23mm] {
    \begin{tikzpicture}[tree, example, x=17mm, y=10mm]\small
      \tree[edges=leadsto, xshift=-.5]{$\langle \langle \varepsilon, aa \rangle, 0 \rangle$}{
        \tree{$\langle \langle a, a \rangle, 1 \rangle$}{
          \tree[xshift=-.5]{$\langle \langle \varepsilon, aa \rangle, 0 \rangle$}{
            \tree{$\langle \langle a, a \rangle, 1 \rangle$}{}           
            \tree{$\langle \langle a, a \rangle, 2 \rangle$}{}            
          }
        }           
        \tree[xshift=-1]{$\langle \langle a, a \rangle, 2 \rangle$}{}
      }
    \end{tikzpicture}
  }

\end{frame}

\endgroup

% SPDX-License-Identifier: CC-BY-SA-4.0
% Author: Matthieu Perrin
% Part: <Nom de la partie>
% Section: <Nom de la section>
% Sub-section: <Nom de la sous-section>  % (facultatif, laisser vide si non utilisé)
% Frame: <Titre de la slide>

\begingroup

\begin{frame}[fragile]{Simulation d'une machine de Turing déterministe}
  \small
  
  \begin{block}{Théorèmes : équivalence entre langage IMP et MTD}
    \begin{itemize}
    \item Toute MTD peut être \structure{simulée} par un algorithme en IMP
      
      \begin{algorithm}[H]
        \SetKwFunction{Forcebrute}{force\_brute}
        \SetKwData{A}{lu}
        \SetKwData{Etat}{etat}
        \SetKwData{Tete}{tete}
        \SetKwData{Ruban}{ruban}
        \SetKwFunction{Ecrire}{ecrit}
        \SetKwFunction{Deplace}{deplace}
        \SetKwFunction{Trans}{transition}
        \Fn{simule($M$ : MTD, $u$ : mot) : booléen}{
          $\Ruban \leftarrow u$;
          $\Tete \leftarrow 1$;
          $\Etat \leftarrow q_0$\;
          \Tantque{$\Etat\neq q_a \land \Etat\neq q_r$}{
            $\A \leftarrow \Ruban[\Tete]$\;
            $\Ruban[\Tete] \leftarrow \Ecrire[\Etat][\A]$\;
            $\Tete \leftarrow \Tete+\Deplace[\Etat][\A]$\;
            $\Etat \leftarrow \Trans[\Etat][\A]$\;
            
            \uSi{$\Tete=0$} {
              $\Ruban \leftarrow \text{``$\blank$''} + \Ruban$; $\Tete \leftarrow 1$
            }
            \SinonSi{$\Tete>|\Ruban|$} {
              $\Ruban \leftarrow \Ruban + \text{``$\blank$''} $
            }
          }
          \Retourner $\Etat=q_a$\;
        }
      \end{algorithm}
    \item<2-> Réciproquement, tout algorithme IMP peut être \structure{compilé} en une MTD.
    \end{itemize}
  \end{block}
  
\end{frame}
\endgroup

% SPDX-License-Identifier: CC-BY-SA-4.0
% Author: Matthieu Perrin
% Part: <Nom de la partie>
% Section: <Nom de la section>
% Sub-section: <Nom de la sous-section>  % (facultatif, laisser vide si non utilisé)
% Frame: <Titre de la slide>

\begingroup

\begin{frame}{Classes de calculabilité}
  \small 
  Soient $\Sigma$ un alphabet, et $L_A$ et $L_B$ deux langages de $\Sigma^\star$. 
  \begin{block}{Propriété : $\leq_M$ est un préordre sur $\mathscr{P}(\Sigma^\star)$}
      \begin{itemize}
      \item $\leq_M$ est réflexive
      \item $\leq_M$ est transitive
      \item $\leq_M$ n'est ni symétrique ni antisymétrique. 
      \end{itemize}
      $L_A$ et $L_B$ sont \structure{équivalents par mappage}, noté \alert{$L_A \equiv_m L_B$}, si $L_A \leq_m L_B$ et $L_B \leq_m L_A$.
  \end{block}
  
  \begin{exampleblock}{Notion de préordre}
    \begin{itemize}
    \item Un \example{préordre} est une relation $\sqsubseteq$ réflexive et transitive. 
      \begin{itemize}
      \item On définit $\equiv$ par $x \equiv y$ si $x \sqsubseteq y$ et $y \sqsubseteq x$
      \begin{itemize}
      \item $\equiv$ est une relation d'équivalence
      \item $\sqsubseteq$ est une relation d'ordre sur les classes d'équivalence de $\equiv$
      \end{itemize}
      \end{itemize}
    \item Par exemple, sur $\{a, b\}^\star$, $u \sqsubseteq v$ si toutes les lettres de $u$ sont dans $v$. 
      \begin{itemize}
      \item $u \equiv v$ si $u$ et $v$ utilisent les mêmes lettres. 
      \item Quatre classes d'équivalence : \example{$\{\varepsilon\}$}, \example{$\mathcal{L}(a^\star)$}, \example{$\mathcal{L}(b^\star)$} et \example{$\mathcal{L}(\Sigma^\star (ab|ba) \Sigma^\star)$}
      \end{itemize}
    \end{itemize}

  \end{exampleblock}
\end{frame}
\endgroup

% SPDX-License-Identifier: CC-BY-SA-4.0
% Author: Matthieu Perrin
% Part: <Nom de la partie>
% Section: <Nom de la section>
% Sub-section: <Nom de la sous-section>  % (facultatif, laisser vide si non utilisé)
% Frame: <Titre de la slide>

\begingroup

\begin{frame}{Complexité de la MT déterminisée}

  \begin{itemize}
  \item Soit $M = \langle \Sigma, \Gamma, \blank, Q, q_0, F, \rightarrow \rangle$ une MTND.
    \begin{itemize}
    \item $M$ reconnaît $\mathcal{L}(M)$ en temps \structure{$NT_M(n)$}
    \item $M$ possède au plus \structure{$|\rightarrow|$}
    \end{itemize}
  \item $M_D$ explore jusqu'à $\mathcal{O}\left(|\rightarrow|^{NT_M(n)}\right)$ n\oe uds
  \item Autres complexités de $M_D$ polynomiales
    \begin{itemize}
    \item Gestion de la file et des configurations
    \end{itemize}
  \item Donc \alert{$T_{M_D}(n) \in \mathcal{O}\left(|\rightarrow|^{NT_M(n)} \times \textsc{poly}\right)$}
  \end{itemize}

  \vspace{2mm}
  En particulier, on a :
  {\Large
    $$\alert{\textsc{p} \subseteq \textsc{np} \subseteq \textsc{exptime}}$$
  }
  
  \vspace{-2mm}
  \begin{description}[$\textsc{np} \subseteq \textsc{exptime}$ : ]
  \item[$\textsc{p} \subseteq \textsc{np}$ : ] Toute MTD est une MTND.
  \item[$\textsc{np} \subseteq \textsc{exptime}$ : ] Si \structure{$NT_M(n) \in \mathcal{O}\left(n^k\right)$}, alors \structure{$T_{M_D}(n) \in \mathcal{O}\left(|\rightarrow|^{n^k} \times \textsc{poly}\right)$}
  \end{description}
  
  \begin{alertblock}{Peut-on faire mieux ?}
    \begin{itemize}
    \item La machine déterminisée n'est peut-être pas optimale
      \begin{itemize}
      \item Pour l'exemple, le même langage peut être décidé en $\mathcal{O}(1)$
      \end{itemize}
    \item Peut-on décider \textsc{Subset-sum} en temps polynomial ? 
    \end{itemize}
    \begin{center}
      \Large
      \alert{A-t-on $\textsc{p} = \textsc{np}$ ?}
    \end{center}
  \end{alertblock}

  \on[x=45mm,y=-30mm] {
    \begin{tikzpicture}[turingMachine, draw=example, text=example, x=14mm]
      \state[initial]   (0) at (0,0) {$0$}; 
      \state[accepting] (2) at (1,0) {$2$}; 
      \path (0) edge node {$\smTMtransR{a}{a}$} (2);
    \end{tikzpicture}
  }

  \on[x=33mm,y=18mm] {
    \begin{tikzpicture}[tree, x=4.5mm, y=15mm, draw=example, text=example]\small
       \tikzset{thebrace/.style={decorate, decoration={brace, amplitude=5pt, raise=3pt, mirror}},}
      \tree[edges=leadsto]{$C_0$}{
        \tree[edges=leadsto]{$C_1$}{
          \tree{$C_4$}{}
          \tree{$C_5$}{}
          \tree{$C_6$}{}
        }
        \tree[edges=leadsto]{$C_2$}{
          \tree{$C_7$}{}
          \tree{$C_8$}{}
          \tree{$C_9$}{}
        }
        \tree[edges=leadsto]{$C_3$}{
          \tree{$C_{10}$}{}
          \tree{$C_{11}$}{}
          \tree{$C_{12}$}{}
        }
      }
      \draw[thebrace] (3.5,-3) -- node[midway,below=7pt]{$|\rightarrow|$} (6.5,-3);
      \draw[<->] (9.7,-3.1) -- (9.7,-.9);
      \node[left] at (9.7,-1.2) {$NT_M(n)$};
      
    \end{tikzpicture}
  }

\end{frame}

\endgroup

% SPDX-License-Identifier: CC-BY-SA-4.0
% Author: Matthieu Perrin
% Part: <Nom de la partie>
% Section: <Nom de la section>
% Sub-section: <Nom de la sous-section>  % (facultatif, laisser vide si non utilisé)
% Frame: <Titre de la slide>

\begingroup

\begin{frame}{Un grand problème ouvert : $P \stackrel{?}{=} NP$}

  \on[width=.7\textwidth, x=9mm, y=25mm]{
    \myquote{William Gasarch}{
      La plupart des théoriciens pensent que $\textsc{p} \neq \textsc{np}$.
    }
  }

  \on[width=.7\textwidth, x=-21mm, y=7mm]{
    \myquote{Stephen Cook}{
      Une réponse positive aurait des conséquences pratiques et philosophiques profondes.
    }
  }

  \on[width=\textwidth, x=-5mm, y=-12mm]{
    \myquote{Scott Aaronson}{
      Si $\textsc{p} = \textsc{np}$, trouver des solutions serait aussi facile que les
      vérifier ; la plupart des systèmes cryptographiques s'effondreraient.
    }
  }

    \onAlertBlock[bottom]{L'un des sept \og Millennium Prize Problems \fg}{
    \begin{itemize}
    \item L'équivalent des problèmes de Hilbert pour le XXI$^{\text{e}}$ siècle
    \item Le \structure{Clay Mathematics Institute} offre \alert{1\,000\,000\$} à quiconque y répond
    \end{itemize}
  }

\end{frame}

\endgroup

 
\section{Caractérisations des langages semi-décidables}

\subsection{Problèmes d'existence d'une solution}
\input{turing_machine/characterizations/solution/certificate}
% SPDX-License-Identifier: CC-BY-SA-4.0
% Author: Matthieu Perrin
% Part: <Nom de la partie>
% Section: <Nom de la section>
% Sub-section: <Nom de la sous-section>  % (facultatif, laisser vide si non utilisé)
% Frame: <Titre de la slide>

\begingroup

\begin{frame}{Exemples de problèmes semi-décidables}

  \vspace{-2mm}
  \begin{block}{Intersection de langages algébriques}

  \vspace{-2mm}
    \Probleme{Intersect-Alg}{
      Deux grammaires algébriques $G$ et $H$
    }{
      Existe-t-il un mot $w \in \mathcal{L}(G) \cap \mathcal{L}(H)$ ?
    }

  \vspace{-2mm}
    \begin{itemize}
    \item \structure{Vérifier} que $w \in \mathcal{L}(G) \cap \mathcal{L}(H)$ est \alert{décidable} par CYK
    \item \textsc{Intersect-Alg} est \alert{semi-décidable} (théorème précédent)
    \item \textsc{Intersect-Alg} est \alert{indécidable} (preuve en TD)
    \end{itemize}
  \end{block}

  \vspace{-2mm}
  \begin{block}{Équations Diophantiennes}

  \vspace{-2mm}
    \Probleme{Diophantine}{
      Un polynôme $P$ à coefficients entiers
    }{
      Existe-t-il un entier $x\in \mathbb{N}, P(x) = 0$ ?
    }

  \vspace{-2mm}
    \begin{itemize}
    \item \structure{Vérifier} que $P(x) = 0$ est \alert{décidable}
    \item \textsc{Diophantine} est \alert{semi-décidable}, mais \alert{indécidable} (admis)
    \end{itemize}
  \end{block}
  
\end{frame}

\endgroup

% SPDX-License-Identifier: CC-BY-SA-4.0
% Author: Matthieu Perrin
% Part: <Nom de la partie>
% Section: <Nom de la section>
% Sub-section: <Nom de la sous-section>  % (facultatif, laisser vide si non utilisé)
% Frame: <Titre de la slide>

\begingroup

\newcommand\PROBLEM{\textsc{problem}}

\begin{frame}{Problème de décision}

  \begin{block}{Définition -- Problème de décision formel}
    Un \structure{problème de décision formel} est un langage \alert{$\PROBLEM{} \in \textsc{lang}$}. 
  \end{block}

  \begin{block}{Définition -- Problème de décision (à domaine) contraint}
    Un \structure{problème de décision contraint} est un couple \alert{$\PROBLEM{} = \langle I, P \rangle$}, où :
    \begin{description}
    \item[$I \in \textsc{lang}$] est le langage des \structure{instances} de \PROBLEM{}
    \item[$P \subseteq I$] est le langage des \structure{instances positives} de \PROBLEM{}
    \end{description}
  \end{block}
  
  \begin{block}{Notations}
    \begin{tikzpicture}[2Darray, x=25mm, y=6mm]
      \arrayColumn[header=\structure{Notation}]{
        \arrayLine{\structure{$\Sigma(\PROBLEM)$}}
        \arrayLine{\structure{$\mathcal{I}(\PROBLEM)$}}
        \arrayLine{\structure{$\textsc{pos}(\PROBLEM)$}}
        \arrayLine{\structure{$\textsc{neg}(\PROBLEM)$}}
      }
      \arrayColumn[header=\structure{$L \in \textsc{lang}$}]{
        \arrayLine{$\Sigma(L)$}
        \arrayLine{$\Sigma(L)^\star$}
        \arrayLine{$L$}
        \arrayLine{$\Sigma(L)^\star \setminus L$}
      }
      \arrayColumn[header=\structure{$\langle I, P\rangle \in \textsc{lang}^2$}]{
        \arrayLine{$\Sigma(I)$}
        \arrayLine{$I$}
        \arrayLine{$P$}
        \arrayLine{$I \setminus P$}
      }
      \arrayColumn[width=30mm, header=\structure{Notion}]{
        \arrayLine{Alphabet}
        \arrayLine{Instances}
        \arrayLine{Instances positives}
        \arrayLine{Instances négatives}
      }
    \end{tikzpicture}
  \end{block}

\end{frame}

\endgroup
\endinput

\input{turing_machine/characterizations/solution/verify_find}
\input{turing_machine/characterizations/solution/find_verify}
% SPDX-License-Identifier: CC-BY-SA-4.0
% Author: Matthieu Perrin
% Part: <Nom de la partie>
% Section: <Nom de la section>
% Sub-section: <Nom de la sous-section>  % (facultatif, laisser vide si non utilisé)
% Frame: <Titre de la slide>

\begingroup

\begin{frame}{Adaptation à la complexité}

  \begin{block}{Caractérisation de \textsc{np} par certificats polynomiaux}
    Un langage $\structure{L \subseteq \Sigma^\star}$ est dans \alert{\textsc{np}}
    si, et seulement s'il existe
    \begin{itemize}
    \item un alphabet $\Gamma$ et un \structure{langage de couples $R \subseteq \Sigma^\star \times \Gamma^\star$} dans \alert{\textsc{p}}, et
    \item un polynôme $P$,  tels que
    \end{itemize}

    \vspace{-2mm}
    $$
    \forall u \in \Sigma^\star,
    \quad
    \alert{u\in L}
    \quad \Leftrightarrow \quad 
    \alert{\exists c} \in \Gamma^\star,
    \left(
    \alert{\langle u,c \rangle \in R}
    \quad \land \quad
    \alert{|c| \le P(|u|)}
    \right)
    $$

    On dit alors que $c$ est un \structure{certificat polynomial} d'appartenance de $u$ à $L$ 
  \end{block}

  \vspace{-2mm}
  \begin{alertblock}{Montrer qu'un problème est dans \textsc{np}}
    On a deux méthodes pour prouver qu'un problème $L$ appartient à \textsc{np} :
    \begin{enumerate}
    \item Preuve d'après la définition :
      \begin{itemize}
      \item Donner une MTND $M$ qui reconnaît $L$
      \item Justifier que la complexité temporelle non-déterministe de $M$ est polynomiale
      \end{itemize}
    \item Preuve d'après la caractérisation :
      \begin{itemize}
      \item Décrire la forme d'un \alert{certificat $c$} qui justifie $u \in L$
      \item Justifier que \alert{la taille} des certificats est \alert{polynomiale} en celle des instances 
      \item Donner un \alert{algorithme} $A(u, c)$ qui décide si $c$ est un certificat pour $u \in L$
      \item Justifier que la \alert{complexité} temporelle déterministe de $A$ est \alert{polynomiale}
      \end{itemize}
    \end{enumerate}
  \end{alertblock}

\end{frame}

\endgroup

% SPDX-License-Identifier: CC-BY-SA-4.0
% Author: Matthieu Perrin
% Part: <Nom de la partie>
% Section: <Nom de la section>
% Sub-section: <Nom de la sous-section>  % (facultatif, laisser vide si non utilisé)
% Frame: <Titre de la slide>

\begingroup

\begin{frame}{Exemple : Remplissage d'une grille de Sudoku}

    \Probleme{Generalized-Sudoku} {
      Une grille $n^2\times n^2$ cases partiellement remplie de nombres entre $1$ et $n^2$.
      \begin{center}
        \begin{tikzpicture}[size=2.5mm]\scriptsize
          \foreach \x in {0,1,...,9}{
            \draw  (\x.5, 0.5) -- (\x.5, 9.5);
            \draw  (0.5, \x.5) -- (9.5, \x.5);
          }
          \foreach \x in {0,3,...,9}{
            \draw[thick] (\x.5, 0.5) -- (\x.5, 9.5);
            \draw[thick] (0.5, \x.5) -- (9.5, \x.5);
          }

          \node at (2,9) {5};
          \node at (3,9) {3};
          \node at (6,9) {8};
          \node at (7,9) {4};
          \node at (8,9) {7};

          \node at (2,8) {4};
          \node at (3,8) {9};
          \node at (5,8) {6};
          \node at (9,8) {3};

          \node at (6,7) {3};
          \node at (8,7) {6};

          \node at (4,6) {2};
          \node at (5,6) {1};
          \node at (9,6) {9};

          \node at (3,5) {8};
          \node at (7,5) {2};

          \node at (1,4) {9};
          \node at (5,4) {7};
          \node at (6,4) {4};

          \node at (2,3) {2};
          \node at (4,3) {4};

          \node at (1,2) {7};
          \node at (5,2) {2};
          \node at (7,2) {8};
          \node at (8,2) {3};

          \node at (2,1) {8};
          \node at (3,1) {5};
          \node at (4,1) {9};
          \node at (7,1) {1};
          \node at (8,1) {2};
          
        \end{tikzpicture}
      \end{center}
    }{
      Existe-t-il une façon de \alert{compléter la grille} telle que chaque nombre apparaisse exactement une fois dans chaque \structure{ligne}, \structure{colonne} et \structure{bloc de $n\times n$ cases} ?
    }

  \vspace{-2mm}
    \begin{itemize}
    \item Un \alert{certificat} est une \structure{grille complète}
    \item Vérifier qu'un certificat est correct est polynomial
    \item Donc {Generalized-Sudoku} est dans \textsc{np}
    \end{itemize}

  \begin{center}
    \alert{
      Y a-t-il des cas où on est \structure{obligé} \\ de \structure{deviner} une case pour pouvoir continuer ?  
    }
  \end{center}

\end{frame}

\endgroup

 
\subsection{Recherche ascendante par force brute}
% SPDX-License-Identifier: CC-BY-SA-4.0
% Author: Matthieu Perrin
% Part: 
% Section: 
% Sub-section: 
% Frame: 

\begingroup

\begin{frame}{Décision de l'appartenance à un langage contextuel}

  \begin{block}{Problème de décision}
    Soit $L$ un langage contextuel
    
    \Probleme{\structure{Decision$_L$}}{
      Un mot \structure{$u \in \Sigma^\star$}
    }{
      Est-ce que \structure{$u \in L$} ?
    }
  \end{block}

  \begin{block}{Algorithme de recherche ascendante par force brute} 

    \begin{description}
    \item [Entrées :]
      \begin{itemize}
      \item Une grammaire $G$, si possible contextuelle
      \item Un mot $u$
      \end{itemize}

    \item [Sortie :] une réponse booléenne sur \structure{$u \in \mathcal{L}(G)$}
    \item [Teminaison :] garantie si $G$ est contextuelle
    \item [Complexité :] exponentielle par rapport à $|u|$
    \end{description}
  \end{block}
  
\end{frame}

\endgroup
\endinput

% SPDX-License-Identifier: CC-BY-SA-4.0
% Author: Matthieu Perrin
% Part: 
% Section: 
% Sub-section: 
% Frame: 

\begingroup

\SetKwFunction{Forcebrute}{force\_brute}
\SetKwData{Old}{old}
\SetKwData{New}{etape}
\SetKwData{Vu}{explore}

\begin{frame}[fragile]{Recherche ascendante par force brute}

  \on[top=-2mm]{
    \begin{algorithm}[H]
      \Fun{$\Forcebrute(G = \langle \Sigma, \Gamma, S, R \rangle \text{: grammaire}, u \in \Sigma^\star) \in \mathbb{B}$}{
        \Alertb<1>{$\New \leftarrow \{u\}$};
        $\Vu \leftarrow \New$;\\
        \While{$\Alertb<5>{\New \neq \emptyset} \land \Alertb<4-5>{S\notin \New}$}{
          \Alertb<2-3>{$\New \leftarrow \{ v\in (\Sigma \cup \Gamma)^\star \,|\, \exists w \in \New, v \vdash w \} \setminus \Vu$};\\
          $\Vu \leftarrow \Vu \cup \New$;\\
        }
        \Return \Alertb<4>{$S \in \New$};\\
      }
    \end{algorithm}
  }

  \onExampleBlock[y=-2mm]{$abc \in L$}{
    $\begin{array}{@{}l@{~~~~}l@{~~}l@{~~}l@{~~}l@{~~}l@{~~}l@{}}
      \alert<1>{\{\alertb<2>{ab}\structureb<2>{c}\}}
      \uncover<2->{&\leftarrow & \{\alertb<2>{aB}\structureb<3>{c}, \alertb<3>{ab}\structureb<2>{C}\}}
      \uncover<3->{&\leftarrow& \{\alertb<3>{aB}\structureb<3>{C}\}} \\
      \uncover<4->{& \leftarrow & \{aBX\} &
        \leftarrow & \{aCX\} &
        \leftarrow & \{aCB\} \\
        & \leftarrow & \{\alert<4>{S}\} \\}
    \end{array}$
  }

  \onExampleBlock<5->[bottom]{$abbc \notin L$}{
    $\begin{array}{@{}l@{~~}c@{~~}l@{~~}c@{~~}l@{}}
      \alert{\{abbc\}} & \leftarrow & \{aBbc, abBc, abbC\} & \leftarrow & \{aBBc, aBbC, abBC\} \\
      & \leftarrow & \{aBBC, abBX\} & \leftarrow & \{aBBX, abCX\} \\
      & \leftarrow & \{aBCX, abCB\} & \leftarrow & \{aBXX, aBCB\} \\
      & \leftarrow & \{aCXX, aBXB\} & \leftarrow & \{aCBX, aCXB\} \\
      & \leftarrow & \{aCCX, aCBB\} & \leftarrow & \{aCCB, aCBB\} \\
      & \leftarrow & \alert{\emptyset} \\
    \end{array}$
  }

  \on[x=35mm]{
    \example{$
      \left\{\begin{array}{rcl}
      S   & \rightarrow & aCB  \\
      C B & \rightarrow & C X  \\
      C X & \rightarrow & B X  \\
      B X & \rightarrow & B C  \\
      \alertb<2,3>{a B} & \alertb<2,3>{\rightarrow} & \alertb<2,3>{a b}  \\
      b B & \rightarrow & b b  \\
      \structureb<2,3>{C}   & \structureb<2,3>{\rightarrow} & \structureb<2,3>{c}
      \end{array}\right.
      $}
  }
  
\end{frame}

\endgroup

\input{turing_machine/characterizations/brute_force/termination}
% SPDX-License-Identifier: CC-BY-SA-4.0
% Author: Matthieu Perrin
% Part: <Nom de la partie>
% Section: <Nom de la section>
% Sub-section: <Nom de la sous-section>  % (facultatif, laisser vide si non utilisé)
% Frame: <Titre de la slide>

\begingroup

\SetKwFunction{Reconnait}{reconnait}

\begin{frame}{Tout langage engendré est reconnaissable}

  \onBlock[top=-5mm]{Théorème -- Engendré $\Rightarrow$ reconnu}{
    Tout langage de type 0 est reconnaissable.
  }

  \onBlock[y=10mm]{Démonstration -- Force brute ascendante non-déterministe}{
    Soit \alert{$G = \langle \Sigma, \Gamma, S, \rightarrow \rangle$} une grammaire non-restreinte. $\mathcal{L}(G)$ est reconnu par :
    
    \begin{algorithm}[H]\small
      \Fun{$\Reconnait_{\mathcal{L}(G)}(u)$}{
        \While{$u\neq S \land \exists \alert{\alpha \rightarrow \beta}, \exists \structure{x}, \structure{y} \in (\Sigma \cup \Gamma)^\star,~ u = \structure{x} \alert{\beta} \structure{y}$}{
          $u \leftarrow \structure{x} \alert{\alpha} \structure{y}$;
          \tcp*[l]{Choix non déterministe de la règle $\alert{\alpha \rightarrow \beta}$ et la position dans $u$}
        }
        \Return $u=S$\;
      }
    \end{algorithm}
  }
  
  \onExampleBlock[y=-10mm]{Exemple -- $\{a^n b^n c^n \mid n>0\}$}{}

  \on[x=12mm, y=-13mm]{\example{
      $\left\{\begin{array}{@{~}r@{~\rightarrow~}l@{~}}
      S  & abc \mid aSBc\\
      cB & Bc  \\
      bB & bb  \\
      \end{array}\right.$
  }}

  \on[bottom=0mm, x=40mm]{
    \begin{tikzpicture}[x=6mm, y=8mm]
      \node (40) at (2,4) {\structureb<10->{\alertb<8-9>{$S$}}};
      \node (30) at (2,3) {\alertb<6-9>{$S$}};     \path[-latex, alert ob=<8-9>] (40) edge (30);
      \node (20) at (0,2) {\alertb<8-9>{$a$}};     \path[-latex, alert ob=<8-9>] (40) edge (20);
      \node (21) at (1,2) {\alertb<6-7>{$a$}};     \path[-latex, alert ob=<6-7>] (30) edge (21);
      \node (23) at (3,2) {\alertb<4-7>{$c$}};     \path[-latex, alert ob=<6-7>] (30) edge (23);
      \node (24) at (4,2) {\alertb<4-5,8-9>{$B$}}; \path[-latex, alert ob=<8-9>] (40) edge (24);
      \node (25) at (5,2) {\alertb<8-9>{$c$}};     \path[-latex, alert ob=<8-9>] (40) edge (25);
      \node (11) at (2,1) {\alertb<2-3,6-7>{$b$}}; \path[-latex, alert ob=<6-7> ] (30) edge (11);
      \node (12) at (3,1) {\alertb<2-5>{$B$}};     \path[-latex, alert ob=<4-5>] (23) edge (12); \path[-latex, alert ob=<4-5>] (24) edge (12);
      \node (13) at (4,1) {\alertb<4-5>{$c$}};     \path[-latex, alert ob=<4-5>] (23) edge (13); \path[-latex, alert ob=<4-5>] (24) edge (13);
      \node (00) at (2,0) {\alertb<2-3>{$b$}};     \path[-latex, alert ob=<2-3>] (11) edge (00); \path[-latex, alert ob=<2-3>] (12) edge (00);
      \node (01) at (3,0) {\alertb<2-3>{$b$}};     \path[-latex, alert ob=<2-3>] (11) edge (01); \path[-latex, alert ob=<2-3>] (12) edge (01);
    \end{tikzpicture}
  }

  \on[bottom=-1mm, x=-6mm]{
    \begin{tikzpicture}[tape, x=6mm, y=6mm]
      \cell[structure ob=<10->]                 {}                                                                          \smheadfromb<10>{1}
      \cell[alert ob=<{8,9}>,structure ob=<10->]{\oneof[$a$]{\on<9->{$S$}}}                                  \smhead<1,9,11>\smheadfromb<8>{1}
      \cell[alert ob=<{6-8}>,structure ob=<10->]{\oneof[$a$]{\on<7->{$S$}\on<9->{}}}                         \smheadb<7,12> \smheadfromb<6>{2}
      \cell[alert ob=<{2-3,6,8}>]               {\oneof[$b$]{\on<7->{$B$}\on<9->{}}}                         \smheadb<3,13> \smheadfromb<2>{-2}
      \cell[alert ob=<{2-5,6,8}>]               {\oneof[$b$]{\on<3->{$B$}\on<5->{$c$}\on<7->{$c$}\on<9->{}}} \smheadb<5>    \smheadfromb<4>{-1}
      \cell[alert ob=<4-5>]                     {\oneof[$c$]{\on<5->{$B$}\on<7->{}}}
      \cell                                     {\oneof[$c$]{\on<7->{}}}
      \cell                                     {}
    \end{tikzpicture}
  }

  \on[bottom=5mm, x=-18mm]{
    \begin{tikzpicture}[turingMachine,x=16mm]
      \state[alert ob=<2-9>, structure ob=<10>, initial above] (0) at (0,0) {\faRandom};
      \state[structure ob=<11>                               ] (1) at (1,0) {\blank};
      \state[structure ob=<12>                               ] (2) at (2,0) {$S$};
      \state[structure ob=<13>, accepting                    ] (3) at (3,0) {\blank};

      \path (0) edge             node {\smTMtransR{\blank}{\blank}} (1);
      \path (1) edge             node {\smTMtransR{S}{S}}           (2);
      \path (2) edge             node {\smTMtransR{\blank}{\blank}} (3);
      \path (0) edge[loop left]  node {
        \smAlign{
          \smTMtransL{\star}{\star}
          \smTMtransR{\star}{\star}
          \smTMtransS{aSBc}{S}
          \smTMtransS{abc}{S}
          \smTMtransS{Bc}{cB}
          \smTMtransS{bb}{bB}
      }} (0);
    \end{tikzpicture}
  }

\end{frame}

\endgroup
\endinput

% SPDX-License-Identifier: CC-BY-SA-4.0
% Author: Matthieu Perrin
% Part: 
% Section: 
% Sub-section: 
% Frame: 

\begingroup

\begin{frame}{Tout langage généré est engendré}

  \onBlock[top=-5mm]{Théorème -- Généré $\Rightarrow$ engendré}{
    Soit $L$ un langage généré par une MTND $M_G$.\\
    Alors il existe une grammaire $G$ qui engendre $L$. 
  }

  \onExampleBlock[y=16mm]{Illustration -- Le langage $\mathcal{L}(ba^\star b)$}{
    On normalise $M_G$ pour qu'elle n'écrive jamais $\blank$ au milieu du ruban.
  }

  \on[bottom, left=40mm]{
    $G = \left\{\begin{array}{rcl}
    \multicolumn{3}{l}{\example{\text{Règles structurelles}}}\\
    \alertb<2>{S}             & \alertb<2>{\rightarrow} & \alertb<2>{\blank 0 \blank}                             \\
    \alertb<3,5,6,7>{\blank}        & \alertb<3,5,6,7>{\rightarrow} & \alertb<3,5,6>{\blank \blank} \mid \alertb<7>{\varepsilon}              \\
    \multicolumn{3}{l}{\example{\text{Transitions } \triangleright}}\\
    \alertb<4>{0 \blank}   & \alertb<4>{\rightarrow} & \alertb<4>{b 1}                               \\
    \alertb<5>{1 \blank}   & \alertb<5>{\rightarrow} & \alertb<5>{a 1}                               \\
    \multicolumn{3}{l}{\example{\text{Transition } \triangleleft}}\\
    \alertb<6>{a 1 \blank} & \alertb<6>{\rightarrow} & \alertb<6>{2 a b}                             \\
    b 1 \blank             & \rightarrow             & 2 b b                                         \\
    \blank 1 \blank            & \rightarrow             & 2 \blank b                                        \\
    \multicolumn{3}{l}{\example{\text{État accepteur}}}\\
    \alertb<7>{2}      & \alertb<7>{\rightarrow} & \alertb<7>{\varepsilon}                       \\
    \end{array}\right.$
  }

  \on[y=0mm, x=20mm]{
    \begin{tikzpicture}[tape, x=7mm, y=7mm]
      \cell{}                
      \cell{\only<4->{$b$}} \smheadb<-3>
      \cell{\only<5->{$a$}} \smhead<4,6>
      \cell{\only<6->{$b$}} \smheadb<5>
      \cell{}                 
    \end{tikzpicture}
  }

  \on[y=-13mm, x=20mm]{
    \begin{tikzpicture}[turingMachine]
      \state [alert ob=<{2-3}>,  initial  ] (0) at (0,0) {$0$}; 
      \state [alert ob=<{4,5}>,          ] (1) at (1,0) {$1$}; 
      \state [alert ob=<{6-}>,  accepting] (2) at (2,0) {$2$}; 

      \path (0) edge             node {$\smTMtransR{\blank}{b}$} (1);
      \path (1) edge[loop below] node {$\smTMtransR{\blank}{a}$} (1);
      \path (1) edge             node {$\smTMtransL{\blank}{b}$} (2);
    \end{tikzpicture}
  }
  
  \on[y=-30mm, x=20mm]{
    $\begin{array}{r@{~~}l@{~~}l@{~~}l@{~~}l} 
      S & \uncover<2->{\vdash\phantom{^\star}} &
      \uncover<2->{\blank \structureb{\alertb<2>{0}} \blank} \\
      &
      \uncover<3->{\vdash^\star} &
      \uncover<3->{\blank 0 \structureb{\alertb<3>{\;\blank}} \blank}\phantom{\blank\blank} &
      \uncover<4->{\vdash} &
      \uncover<4->{\blank \structureb{b \alertb<4>{1}} \blank} \\
      &
      \uncover<5->{\vdash^\star} &
      \uncover<5->{\blank b 1 \structureb{\blank} \blank} &
      \uncover<5->{\vdash} &
      \uncover<5->{\blank b \structureb{\alertb<5>{a 1}} \blank} \\
      &
      \uncover<6->{\vdash^\star} &
      \uncover<6->{\blank b a 1 \structureb{\blank} \blank} &
      \uncover<6->{\vdash} &
      \uncover<6->{\blank b \structureb{\alertb<5>{2 a b}} \blank} \\
      &
      \uncover<7->{\vdash^\star} &
      \uncover<7->{b a b}
    \end{array}$
  }
  
\end{frame}

\endgroup

% SPDX-License-Identifier: CC-BY-SA-4.0
% Author: Matthieu Perrin
% Part: <Nom de la partie>
% Section: <Nom de la section>
% Sub-section: <Nom de la sous-section>  % (facultatif, laisser vide si non utilisé)
% Frame: <Titre de la slide>

\begingroup

\begin{frame}{Classes de calculabilité}
  \small 
  Soient $\Sigma$ un alphabet, et $L_A$ et $L_B$ deux langages de $\Sigma^\star$. 
  \begin{block}{Propriété : $\leq_M$ est un préordre sur $\mathscr{P}(\Sigma^\star)$}
      \begin{itemize}
      \item $\leq_M$ est réflexive
      \item $\leq_M$ est transitive
      \item $\leq_M$ n'est ni symétrique ni antisymétrique. 
      \end{itemize}
      $L_A$ et $L_B$ sont \structure{équivalents par mappage}, noté \alert{$L_A \equiv_m L_B$}, si $L_A \leq_m L_B$ et $L_B \leq_m L_A$.
  \end{block}
  
  \begin{exampleblock}{Notion de préordre}
    \begin{itemize}
    \item Un \example{préordre} est une relation $\sqsubseteq$ réflexive et transitive. 
      \begin{itemize}
      \item On définit $\equiv$ par $x \equiv y$ si $x \sqsubseteq y$ et $y \sqsubseteq x$
      \begin{itemize}
      \item $\equiv$ est une relation d'équivalence
      \item $\sqsubseteq$ est une relation d'ordre sur les classes d'équivalence de $\equiv$
      \end{itemize}
      \end{itemize}
    \item Par exemple, sur $\{a, b\}^\star$, $u \sqsubseteq v$ si toutes les lettres de $u$ sont dans $v$. 
      \begin{itemize}
      \item $u \equiv v$ si $u$ et $v$ utilisent les mêmes lettres. 
      \item Quatre classes d'équivalence : \example{$\{\varepsilon\}$}, \example{$\mathcal{L}(a^\star)$}, \example{$\mathcal{L}(b^\star)$} et \example{$\mathcal{L}(\Sigma^\star (ab|ba) \Sigma^\star)$}
      \end{itemize}
    \end{itemize}

  \end{exampleblock}
\end{frame}
\endgroup


\subsection{Retour sur la hiérarchie de Chomsky}
% SPDX-License-Identifier: CC-BY-SA-4.0
% Author: Matthieu Perrin
% Part: 
% Section: 
% Sub-section: 
% Frame: 

\begingroup

\begin{frame}{Automates à plusieurs piles}

  \on[top, text]{
    \begin{center}
      \structure{Les \alert{automates déterministes à deux piles} \\
        (extension du formalisme des automates à pile)\\
        reconnaissent exactement la classe \textsc{re} des\\
        langages \alert{semi-décidables}}
    \end{center}
  }

  \onExampleBlock[left=.5\textwidth,  y=7mm] {Machine de Turing}{}
  \onExampleBlock[right=.5\textwidth, y=7mm]{Automate à deux piles}{}
  
  \on[x=0mm, y=-3mm]{
    \begin{tikzpicture}[word, size=6mm]
      \cell[open before]{...}
      \cell{}
      \cell{\only<-1>{a}} \smhead<1>  \smheadfromb<4>{2} 
      \cell{\only<-5>{a}} \smheadb<5,8,10> 
      \cell{\only<-7>{b}} \smheadb<7,9> 
      \cell{\only<-3>{b}} \smheadb<3> \smheadfromb<6>{-1} 
      \cell{}             \smheadfromb<2>{-3}
      \cell[open after]{...}

      \cell[open]{}
      \cell[open]{}

      \onlyb<4->{\cell[open]{}}
      \onlyb<8->{\cell[open]{}}
      \onlyb<10->{\cell[open]{}}
      \only<-9>{\cell{\blank}}
      \onlyb<3-4>{\cell{a}}
      \onlyb<3-4,7>{\cell{b}}
      \onlyb<3>{\cell{b}}

      \cell[open after]{...}
      \cell[open]{}
      \cell[open before]{...}
      
      \only<1>{\cell{a}}
      \only<1-2,5>{\cell{a}}
      \only<1-2,5-6>{\cell{b}}
      \only<1-2>{\cell{b}}
      \only<-9>{\cell{\blank}}
      \onlyb<2->{\cell[open]{}}
      \onlyb<6->{\cell[open]{}}
      \onlyb<10->{\cell[open]{}}
    \end{tikzpicture}    
  }

  \on[bottom=3mm, x=-33mm]{
    \begin{tikzpicture}[turingMachine, size=17mm]\footnotesize
      \state[example ob=<{1,5,9}>, initial above] (a) at (1,1) {$a$}; 
      \state[example ob=<{2,6}>                 ] (b) at (1,0) {\faForward}; 
      \state[example ob=<{3,7}>                 ] (c) at (0,0) {$b$}; 
      \state[example ob=<{4,8}>                 ] (d) at (0,1) {\faBackward}; 
      \state[example ob=<10>, accepting         ] (e) at (2,1) {\faCheck}; 
      
      \path                  (a) edge               node         {\smTMtransR{a}{\blank}}      (b);
      \path                  (b) edge[loop right]   node         {\smTMtransR{x}{x}}           (b);
      \path                  (b) edge               node         {\smTMtransL{\blank}{\blank}} (c);
      \path                  (c) edge               node         {\smTMtransR{b}{\blank}}      (d);
      \path                  (d) edge[loop left]    node         {\smTMtransL{x}{x}}           (d);
      \path                  (d) edge               node         {\smTMtransR{\blank}{\blank}} (a);
      \path                  (a) edge               node         {\smTMtransL{\blank}{\blank}} (e);
    \end{tikzpicture}
  }

  \on[bottom=3mm, x=30mm]{
    \begin{tikzpicture}[pushdown, size=17mm]\footnotesize
      \state[example ob=<{1,5,9}>, initial above] (a) at (1,1) {$a$}; 
      \state[example ob=<{2,6}>                 ] (b) at (1,0) {\faForward}; 
      \state[example ob=<{3,7}>                 ] (c) at (0,0) {$b$}; 
      \state[example ob=<{4,8}>                 ] (d) at (0,1) {\faBackward}; 
      \state[example ob=<10>, accepting         ] (e) at (2,1) {\faCheck}; 
      
      \path                  (a) edge               node         {\smPAtrans[g]{\varepsilon}{a}{\varepsilon}}      (b);
      \path                  (b) edge[loop right]   node         {\smGroup{\smPAtrans[g]{\varepsilon}{x}{\varepsilon}\smPAtrans[d]{\varepsilon}{\varepsilon}{x}}}           (b);
      \path                  (b) edge               node         {\smPAtrans[g]{\varepsilon}{\blank}{\blank}} (c);
      \path                  (c) edge               node         {\smPAtrans[d]{\varepsilon}{b}{\varepsilon}}      (d);
      \path                  (d) edge[loop left]    node         {\smGroup{\smPAtrans[g]{\varepsilon}{\varepsilon}{x}\smPAtrans[d]{\varepsilon}{x}{\varepsilon}}}           (d);
      \path                  (d) edge               node         {\smPAtrans[d]{\varepsilon}{\blank}{\blank}} (a);
      \path                  (a) edge               node         {\smGroup{\smPAtrans[g]{\varepsilon}{\blank}{\varepsilon}\smPAtrans[d]{\varepsilon}{\blank}{\varepsilon}}} (e);
    \end{tikzpicture}
  }
  
\end{frame}

\endgroup
\endinput

% SPDX-License-Identifier: CC-BY-SA-4.0
% Author: Matthieu Perrin
% Part: 
% Section: 
% Sub-section: 
% Frame: 

\begingroup

\begin{frame}{Automates linéairement bornés}

  \onBlock[top=-4mm] {Définition -- Automate linéairement borné}{
    Un \structure{automate linéairement borné} est une machine de Turing telle que : 
    \begin{itemize}
    \item La configuration initiale pour $u$ est \alert{$C_\mathit{init}(u) = \langle \vdash, q_0, u \dashv \rangle$}
    \item Les symboles $\vdash$ et $\dashv$ ne sont jamais franchis : $\forall q, q', a, b, d$,
    \end{itemize}
    $$\structure{
      q \xrightarrow{\smTMtrans{\alert{\vdash}}{b}{d}} q'  \Rightarrow (b = \alert{\vdash} \land d = \alert{\triangleright})
      \quad
      \quad
      \quad
      q \xrightarrow{\smTMtrans{\alert{\dashv}}{b}{d}} q'  \Rightarrow (b = \alert{\dashv} \land d = \alert{\triangleleft})
    }$$ 
  }

  \onBlock[left=.5\textwidth, y=6mm, anchor=north] {Théorème -- Kuroda (1964)}{
    \centering
    Les \structure{automates linéairement \\ bornés non-déterministes}\footnote{Problème ouvert : \alert{$\textsc{nspace}(n) \stackrel{?}{=} \textsc{dspace}(n)$}} \\reconnaissent exactement \\la classe \alert{\textsc{cs}} des \\ \structure{langages contextuels}
    $$\alert{\textsc{cs} = \textsc{nspace}(n)}$$
  }
  
  \onExampleBlock[right=.5\textwidth, y=6mm, anchor=north] {Exemple : $\{a^n b^n \mid n\in \mathbb{N}\}$}{}
  
  \on[y=-10mm, x=30mm]{
    \begin{tikzpicture}[word, size=6mm]
      \cell{$\vdash$}
      \cell{\alt<-1>{$a$}{$\vdash$}} \smhead<1>  \smheadfromb<4>{2} 
      \cell{\alt<-5>{$a$}{$\vdash$}} \smheadb<5,8,10> 
      \cell{\alt<-7>{$b$}{$\dashv$}} \smheadb<7,9> 
      \cell{\alt<-3>{$b$}{$\dashv$}} \smheadb<3> \smheadfromb<6>{-1} 
      \cell{$\dashv$}                            \smheadfromb<2>{-3}
    \end{tikzpicture}    
  }

  \on[y=-27mm, x=30mm]{
    \begin{tikzpicture}[turingMachine, x=15mm, y=15mm]\footnotesize
      \state[example ob=<{1,5,9}>, initial above] (a) at (1,1) {$a$}; 
      \state[example ob=<{2,6}>                 ] (b) at (1,0) {\faForward}; 
      \state[example ob=<{3,7}>                 ] (c) at (0,0) {$b$}; 
      \state[example ob=<{4,8}>                 ] (d) at (0,1) {\faBackward}; 
      \state[example ob=<10>, accepting         ] (e) at (2,1) {\faCheck}; 
      
      \path                  (a) edge               node         {\smTMtransR{a}{\vdash}}      (b);
      \path                  (b) edge[loop right]   node         {\smTMtransR{x}{x}}           (b);
      \path                  (b) edge               node         {\smTMtransL{\dashv}{\dashv}} (c);
      \path                  (c) edge               node         {\smTMtransR{b}{\dashv}}      (d);
      \path                  (d) edge[loop left]    node         {\smTMtransL{x}{x}}           (d);
      \path                  (d) edge               node         {\smTMtransR{\vdash}{\vdash}} (a);
      \path                  (a) edge               node         {\smTMtransL{\dashv}{\dashv}} (e);
    \end{tikzpicture}
  }
  
\end{frame}

\endgroup
\endinput

\input{turing_machine/characterizations/chomsky/hierarchy}
% SPDX-License-Identifier: CC-BY-SA-4.0
% Author: Matthieu Perrin
% Part: 
% Section: 
% Sub-section: 
% Frame: 

\begingroup

\begin{frame}{Décidabilité de la reconnaissance}

  \vspace{-2mm}
  \Probleme{\structure{Universal$_{\langle\mathcal{G}, \llbracket \cdot \rrbracket \rangle}$} \normalfont{(où $\mathcal{G}$ est un type de grammaires sur $\Sigma$)}  }{
    \begin{itemize}
    \item Une grammaire \structure{$G \in \mathcal{G}$}
    \item Un mot \structure{$u \in \Sigma^\star$}
    \end{itemize}
  }{
    Est-ce que \structure{$u \in \llbracket G \rrbracket$} ?
  }

  La difficulté du problème dépend du type de $G$ :
 
  \begin{description}
  \item[Type 3 :] Langages rationnels
    \begin{itemize}
    \item Le problème est \structure{décidable}
    \item Complexité \structure{$\mathcal{O}\left(|u|\right)$} avec un AFD
    \end{itemize}
  \item[Type 2 :] Langages algébriques
    \begin{itemize}
    \item Le problème est \structure{décidable}
    \item Complexité \structure{$\mathcal{O}\left(|u|^3\right)$} avec l'algorithme CYK
    \end{itemize}
  \item[Type 1 :] Langages contextuels
    \begin{itemize}
    \item Le problème est \structure{décidable}
    \item Complexité \structure{$\mathcal{O}\left(e^{|u|}\right)$} par force brute
    \end{itemize}
  \item[Type 0 :] Langages récursivement énumérables
    \begin{itemize}
    \item Le problème est \structure{indécidable} mais \structure{semi-décidable}
    \item \alert{Comment le démontrer ?}
    \end{itemize}
  \end{description}
 
\end{frame}

\endgroup



\part{Limites de l'algorithmique}
 
 
\section{Introduction à la Calculabilité}
 
\subsection{Universalité de la Machine de Turing}
% SPDX-License-Identifier: CC-BY-SA-4.0
% Author: Matthieu Perrin
% Part: <Nom de la partie>
% Section: <Nom de la section>
% Sub-section: <Nom de la sous-section>  % (facultatif, laisser vide si non utilisé)
% Frame: <Titre de la slide>

\begingroup

\begin{frame}{Machine de Turing universelle}
  \small
  
  \begin{block}{Remarque}
    \begin{itemize}
    \item Il existe un algorithme qui simule toutes les MTD.
    \item Tout algorithme peut être transformé en une MTD.
    \item Il existe une MTD qui simule toutes les MTD.
    \end{itemize}
  \end{block}

  \pause

  \begin{block}{Théorème -- Machine de Turing universelle}
    Soit $\Sigma$ un alphabet.
    Il existe une MTD $M_{univ}$ qui prend en entrée :
    \begin{itemize}
    \item un encodage $\mathit{encode}(M)$ d'une MTD $M$,
    \item un mot $u \in \Sigma^\star$,
    \end{itemize}
    et telle que :
    \begin{itemize}
    \item \structure{$M_{univ}$ termine sur $\langle \mathit{encode}(M), u \rangle$} si, et seulement si, \structure{$M$ termine sur $u$};
    \item \structure{$M_{univ}$ accepte $\langle \mathit{encode}(M), u \rangle$} si, et seulement si, \structure{$M$ accepte $u$}. 
    \end{itemize}

    \structure{Conséquence : } \alert{$L_{univ} \eqdef \{\langle \mathit{encode}(M), u \rangle | M \text{ décide } u \}$ est semi-décidable}. 
  \end{block}
  
\end{frame}


\endgroup

% SPDX-License-Identifier: CC-BY-SA-4.0
% Author: Matthieu Perrin
% Part: <Nom de la partie>
% Section: <Nom de la section>
% Sub-section: <Nom de la sous-section>  % (facultatif, laisser vide si non utilisé)
% Frame: <Titre de la slide>

\begingroup

\begin{frame}{Encodage des machines de Turing}

  \on[text,y=1.5]{
    Soit $M = \langle \Sigma, \Gamma, \blank, Q, q_0, F, \rightarrow \rangle$ une machine de Turing telle que :
    \begin{itemize}
    \item Encodage des états :
      \begin{itemize}
      \item $\mathit{encode}_Q : Q \rightarrow \{0,1\}^{\left\lceil \log_2(|Q|) \right\rceil} \cup \{F\}$
      \item $\mathit{encode}_Q(q_0) \in 0^\star$
      \item $\mathit{encode}_Q(q_f) = F$
      \end{itemize}
    \item Encodage des transitions :
      \begin{itemize}
      \item $\mathit{encode}_\rightarrow\left(q \xrightarrow{\smTMtrans{a}{b}{d}} q' \right) = a \cdot \mathit{encode}_Q(q) \cdot {\rightarrow} \cdot a' \cdot \mathit{encode}_Q(q')^{\textsc{r}} \cdot d$
      \end{itemize}
    \item Encodage de $M$ : $\displaystyle \mathit{encode}(M) = \prod_{t \in \rightarrow} \mathit{encode}_\rightarrow(t) $
    \end{itemize}
  }

  \onExampleBlock[y=-1]{Exemple :}{}

  \on[y=-2] {
      \begin{tikzpicture}[turingMachine]
        \state (F)  at (0,0) {$F$}; 
        \state (00) at (1,0) {$00$}; 
        \state (01) at (2,0) {$01$}; 
        \state (10) at (3,0) {$10$}; 

        \path (00) edge[bend left] node {\smTMtransR{1}{1}} (10);
        \path (10) edge            node {\smTMtransR{1}{1}} (01);
        \path (01) edge            node {\smTMtransR{1}{1}} (00);
        \path (00) edge            node {\smTMtransL{0}{0}} (F);
    \end{tikzpicture}
  }

  
  \on[y=-3.75] {\scriptsize
    \begin{tikzpicture}[tape, x=4mm, y=4mm]
      \cell{$\blank$}
      \cell{$1$}                    \smsave{T1L}
      \cell{$0$}                    
      \cell{$0$}                    
      \cell{$\rightarrow$}          
      \cell{$1$}                    
      \cell{$0$}                    
      \cell{$1$}                    
      \cell{$\filledtriangleright$} \smsave{T1R}
      \cell{$1$}                    \smsave{T2L}
      \cell{$1$}                    
      \cell{$0$}                    
      \cell{$\rightarrow$}          
      \cell{$1$}                    
      \cell{$1$}                    
      \cell{$0$}                    
      \cell{$\filledtriangleright$} \smsave{T2R}
      \cell{$1$}                    \smsave{T3L}
      \cell{$0$}                    
      \cell{$1$}                    
      \cell{$\rightarrow$}          
      \cell{$1$}                    
      \cell{$0$}                    
      \cell{$0$}                    
      \cell{$\filledtriangleright$} \smsave{T3R}
      \cell{$0$}                    \smsave{T4L}
      \cell{$0$}                    
      \cell{$0$}                    
      \cell{$\rightarrow$}          
      \cell{$F$}                    \smsave{T4R}
      \cell{$\blank$}

      \tikzset{thebrace/.style={decorate, decoration={brace, amplitude=5pt, raise=3pt, mirror}},}
      \draw[thebrace] ([xshift=2pt]T1L.south west) --  ([xshift=-2pt]T1R.south east) node[midway,below=7pt] {$00 \xrightarrow{\smTMtransR{1}{1}} 10$};
      \draw[thebrace] ([xshift=2pt]T2L.south west) --  ([xshift=-2pt]T2R.south east) node[midway,below=7pt] {$10 \xrightarrow{\smTMtransR{0}{0}} 01$};
      \draw[thebrace] ([xshift=2pt]T3L.south west) --  ([xshift=-2pt]T3R.south east) node[midway,below=7pt] {$01 \xrightarrow{\smTMtransR{1}{1}} 00$};
      \draw[thebrace] ([xshift=2pt]T4L.south west) --  ([xshift=-2pt]T4R.south east) node[midway,below=7pt] {$00 \xrightarrow{\smTMtransL{\blank}{\blank}} F$};
    \end{tikzpicture}
  }

\end{frame}

\endgroup

% SPDX-License-Identifier: CC-BY-SA-4.0
% Author: Matthieu Perrin
% Part: <Nom de la partie>
% Section: <Nom de la section>
% Sub-section: <Nom de la sous-section>  % (facultatif, laisser vide si non utilisé)
% Frame: <Titre de la slide>

\begingroup

\begin{frame}{Exemple : $aab \in \mathcal{L}(ab|aab)$ ?}

  \begin{tikzpicture}

    \draw (5,10) node[below]{\begin{minipage}{\textwidth}
        \begin{block}{Machine non-déterministe $M$ :}
          \scalebox{.8}{\begin{tikzpicture}[shorten >=1pt, node distance=1.5cm, on grid, auto]
              \node (nq0)                   {};
              \node (nq1) [above right of=nq0] {};
              \node (nq2) [right of=nq1]    {};
              \node (nq3) [below right of=nq0] {};
              \node (nq4) [right of=nq3]    {};
              \node (nq5) [right of=nq4]    {};

              \node[state, initial, initial text=] (q0) at (nq0) {$q_0$};
              \node[state]                         (q1) at (nq1) {$q_1$};
              \node[state, accepting]              (q2) at (nq2) {$q_2$};
              \node[state]                         (q3) at (nq3) {$q_3$};
              \node[state]                         (q4) at (nq4) {$q_4$};
              \node[state, accepting]              (q5) at (nq5) {$q_5$};

              \node<2-7>[fill=alert!20, state, initial, initial text=] (q0) at (nq0) {$q_0$};
              \node<10>[fill=alert!20, state]                          (q1) at (nq1) {$q_1$};
              \node<8-9>[fill=alert!20, state]                       (q3) at (nq3) {$q_3$};
              \node<11>[fill=alert!20, state]                          (q4) at (nq4) {$q_4$};
              \node<11>[fill=structure!20, state, accepting]                (q5) at (nq5) {$q_5$};


              \path<-3,5-> [->] (q0)       edge[bend left] node {$\smTMtransR{a}{a}$} (q1);
              \path<-4,6-> [->] (q1)       edge[bend left] node {$\smTMtransR{b}{b}$} (q2);
              %              \path<-5,7-> [->] (q2)       edge[bend left] node {$\smTMtransR{a}{a}$} (q1);
              \path<-2,4-> [->] (q0)       edge[bend right] node[swap] {$\smTMtransR{a}{a}$} (q3);
              \path<-5,7-8,10-> [->] (q3) edge[bend left] node {$\smTMtransR{a}{a}$} (q4);
              \path<-6,8-10> [->]       (q4)       edge[bend left] node {$\smTMtransR{b}{b}$} (q5);
              %              \path<-8,10-> [->]      (q5)      edge[bend left] node {$\smTMtransR{a}{a}$} (q3);


              \path<4> [alert, ->] (q0)   edge[bend left] node {$\smTMtransR{a}{a}$} (q1);
              \path<5> [alert, ->] (q1)   edge[bend left] node {$\smTMtransR{b}{b}$} (q2);
              %              \path<6> [alert, ->] (q2)   edge[bend left] node {$\smTMtransR{a}{a}$} (q1);
              \path<3> [alert, ->] (q0)   edge[bend right] node[swap] {$\smTMtransR{a}{a}$} (q3);
              \path<6,9> [alert, ->] (q3)   edge[bend left] node {$\smTMtransR{a}{a}$} (q4);
              \path<7,11> [alert, ->] (q4)   edge[bend left] node {$\smTMtransR{b}{b}$} (q5);
              %              \path<9> [alert, ->] (q5)   edge[bend left] node {$\smTMtransR{a}{a}$} (q3);

          \end{tikzpicture}}
        \end{block}
    \end{minipage}};


    \draw (7.5,10) node[below]{\begin{minipage}{.3\textwidth}\begin{block}{Configuration :}\end{block}\end{minipage}};
    \draw (6,8.5) node[right]{\structure{$f$ : }~
      \only<-2>{$\varepsilon$}%
      \only<3-7>{$\mid a\, q_3\, ab$}%
      \only<4-9>{$\mid a\, q_1\, ab$}%
      \only<9-10>{$\mid aa\, q_4\, b$}%
      \only<11>{$\varepsilon$}%
      %      \only<11>{$\mid aab\, q_5$}%
    };
    \draw (6,8) node[right]{\structure{$c$ : }~
      \only<1>{$aab$}%
      \only<2-7>{$\alert{q_0\, a}ab$}%
      \only<8-9>{$a\, \alert{q_3\, a}b$}%
      \only<10>{$a\, \alert{q_1\, a}b$}%
      \only<11>{$aa\, \alert{q_4\, b}$}%
    };
    \draw<11> (6,7.5) node[right]{\structure{Le mot $aab$ est accepté}};

    \draw (5,4) node{\begin{minipage}{\textwidth}
        \begin{block}{Machine déterminisée $M_D$ :}
          \scalebox{.8}{\begin{tikzpicture}[shorten >=1pt, node distance=3cm, on grid, auto]
              \tikzset{mynode/.style={draw, rounded corners=8, align=center, minimum height=0.6cm, minimum width=1cm}}

              \node (nt0)                       {};
              \node (ninit)   [left of=nt0]     {};
              \node (nt1)     [right of=nt0]    {};
              \node (nt2)     [right of=nt1]    {};
              \node (nt3)     [right of=nt2]    {};
              \node (nreinit) at (0,-1.5)       {};
              \node (nt6)     [right of=nreinit]{};
              \node (nt5)     [right of=nt6]    {};
              \node (nt4)     [right of=nt5]    {};

              \node[fill=example!20, mynode]         (init)   at (ninit)    {\footnotesize init};
              \node[fill=structure!20, mynode]            (t0)     at (nt0)      {\footnotesize $q_0 \xrightarrow{\smTMtransR{a}{a}} q_3$};
              \node[fill=structure!20, mynode]            (t1)     at (nt1)      {\footnotesize $q_0 \xrightarrow{\smTMtransR{a}{a}} q_1$};
              \node[fill=structure!20, mynode, accepting] (t2)     at (nt2)      {\footnotesize $q_1 \xrightarrow{\smTMtransR{b}{b}} q_2$};
              \node[fill=structure!20, mynode]            (t4)     at (nt5)      {\footnotesize $q_3 \xrightarrow{\smTMtransR{a}{a}} q_4$};
              \node[fill=structure!20, mynode, accepting] (t5)     at (nt6)      {\footnotesize $q_4 \xrightarrow{\smTMtransR{b}{b}} q_5$};
              \node[fill=example!20, mynode]         (reinit) at (nreinit)  {\footnotesize reinit};

              \node<2>[fill=alert!20, mynode]         (init)   at (ninit)       {\footnotesize init};
              \node<3>[fill=alert!20, mynode]            (t0)     at (nt0)      {\footnotesize $q_0 \xrightarrow{\smTMtransR{a}{a}} q_3$};
              \node<4>[fill=alert!20, mynode]            (t1)     at (nt1)      {\footnotesize $q_0 \xrightarrow{\smTMtransR{a}{a}} q_1$};
              \node<5>[fill=alert!20, mynode, accepting] (t2)     at (nt2)      {\footnotesize $q_1 \xrightarrow{\smTMtransR{b}{b}} q_2$};
              \node<6,9>[fill=alert!20, mynode]            (t4)     at (nt5)   {\footnotesize $q_3 \xrightarrow{\smTMtransR{a}{a}} q_4$};
              \node<7,11>[fill=alert!20, mynode, accepting] (t5)     at (nt6)      {\footnotesize $q_4 \xrightarrow{\smTMtransR{b}{b}} q_5$};
              \node<8,10>[fill=alert!20, mynode]         (reinit) at (nreinit) {\footnotesize reinit};

              \path [->] (init)   edge node[swap] {\scriptsize$c:\smTMtransR{q}{q}$} (t0);
              \path [->] (t0)     edge node[swap] {\scriptsize$c:\smTMtransR{q}{q}$} (t1);
              \path [->] (t1)     edge node[swap] {\scriptsize$c:\smTMtransR{q}{q}$} (t2);
              \path [->] (t2)     edge node[swap] {\scriptsize$c:\smTMtransR{q}{q}$} (t4);
              \path [->] (t4)     edge node[swap] {\scriptsize$c:\smTMtransR{q}{q}$} (t5);
              \path [->] (t5)     edge node[swap] {\scriptsize$c:\smTMtransR{q}{q}$} (reinit);
              \path [->] (reinit) edge node[swap] {\scriptsize$c:\smTMtransR{q}{q}$} (t0);
          \end{tikzpicture}}
        \end{block}
    \end{minipage}};

  \end{tikzpicture}

\end{frame}

\endgroup

% SPDX-License-Identifier: CC-BY-SA-4.0
% Author: Matthieu Perrin
% Part: <Nom de la partie>
% Section: <Nom de la section>
% Sub-section: <Nom de la sous-section>  % (facultatif, laisser vide si non utilisé)
% Frame: <Titre de la slide>

\begingroup

\begin{frame}{Hypothèses pour la suite}

  \begin{block}{Dans toute la suite, on supposera :}
    \begin{itemize}
    \item L'alphabet considéré sera toujours $\Sigma = \{0,1\}$
      \begin{itemize}
      \item On peut toujours encoder n'importe quel alphabet dans $\{0,1\}^\star$
      \end{itemize}
    \item L'encodage des MT sera un langage sur $\{0,1\}$ : $$\forall M, \mathit{encode}(M) \in \{0,1\}^\star$$ 
    \item La machine universelle $M_{univ}$ :
      \begin{itemize}
      \item sera déterministe
      \item pourra simuler des machines des machines de Turing non-déterministes
      \item utilise (au moins) deux rubans : 
        \begin{itemize}
        \item \textsc{m} initialisé à $\mathit{encode}(M)$ 
        \item \textsc{u} initialisé à $u$ 
        \end{itemize}
      \end{itemize}
    \end{itemize}
  \end{block}

\end{frame}


\endgroup

% SPDX-License-Identifier: CC-BY-SA-4.0
% Author: Matthieu Perrin
% Part: <Nom de la partie>
% Section: <Nom de la section>
% Sub-section: <Nom de la sous-section>  % (facultatif, laisser vide si non utilisé)
% Frame: <Titre de la slide>

\begingroup

\begin{frame}{Turing--complet}

  \begin{block}{Rappel -- Thèse de Church--Turing}
    \begin{center}
      \structure{La définition des \og \alert{fonctions calculables} \fg par des \\
        Machines de Turing déterministes  \\
        caractérise la notion intuitive de \og \alert{procédure effective} \fg.}
    \end{center}
  \end{block}

  \begin{block}{Définition -- Complet au sens de Turing}
    Un système formel est dit \structure{Turing-complet} s'il peut \og simuler \fg toute MTD.
  \end{block}

  \begin{exampleblock}{Exemples}
    \begin{itemize}
    \item Le $\lambda$-calcul est équivalent aux machines de Turing
    \item Les langages de programmation généralistes
    \item Minecraft, le jeu de la vie...
    \end{itemize}
  \end{exampleblock}

  \begin{alertblock}{Contre-exemples}
    \begin{itemize}
    \item Automates finis et à pile, HTML, CSS, SQL non récursif, ...
    \end{itemize}
  \end{alertblock}
  
\end{frame}

\endgroup

 
\section{Limites à la Calculabilité}
 
\subsection{Argument par Dénombrement}
% SPDX-License-Identifier: CC-BY-SA-4.0
% Author: Matthieu Perrin
% Part: <Nom de la partie>
% Section: <Nom de la section>
% Sub-section: <Nom de la sous-section>  % (facultatif, laisser vide si non utilisé)
% Frame: <Titre de la slide>

\begingroup

\begin{frame}{Classes de calculabilité}
  \small 
  Soient $\Sigma$ un alphabet, et $L_A$ et $L_B$ deux langages de $\Sigma^\star$. 
  \begin{block}{Propriété : $\leq_M$ est un préordre sur $\mathscr{P}(\Sigma^\star)$}
      \begin{itemize}
      \item $\leq_M$ est réflexive
      \item $\leq_M$ est transitive
      \item $\leq_M$ n'est ni symétrique ni antisymétrique. 
      \end{itemize}
      $L_A$ et $L_B$ sont \structure{équivalents par mappage}, noté \alert{$L_A \equiv_m L_B$}, si $L_A \leq_m L_B$ et $L_B \leq_m L_A$.
  \end{block}
  
  \begin{exampleblock}{Notion de préordre}
    \begin{itemize}
    \item Un \example{préordre} est une relation $\sqsubseteq$ réflexive et transitive. 
      \begin{itemize}
      \item On définit $\equiv$ par $x \equiv y$ si $x \sqsubseteq y$ et $y \sqsubseteq x$
      \begin{itemize}
      \item $\equiv$ est une relation d'équivalence
      \item $\sqsubseteq$ est une relation d'ordre sur les classes d'équivalence de $\equiv$
      \end{itemize}
      \end{itemize}
    \item Par exemple, sur $\{a, b\}^\star$, $u \sqsubseteq v$ si toutes les lettres de $u$ sont dans $v$. 
      \begin{itemize}
      \item $u \equiv v$ si $u$ et $v$ utilisent les mêmes lettres. 
      \item Quatre classes d'équivalence : \example{$\{\varepsilon\}$}, \example{$\mathcal{L}(a^\star)$}, \example{$\mathcal{L}(b^\star)$} et \example{$\mathcal{L}(\Sigma^\star (ab|ba) \Sigma^\star)$}
      \end{itemize}
    \end{itemize}

  \end{exampleblock}
\end{frame}
\endgroup

% SPDX-License-Identifier: CC-BY-SA-4.0
% Author: Matthieu Perrin
% Part: <Nom de la partie>
% Section: <Nom de la section>
% Sub-section: <Nom de la sous-section>  % (facultatif, laisser vide si non utilisé)
% Frame: <Titre de la slide>

\begingroup


\begin{frame}{Rappel : Equipotence}

  \small
  
  \begin{block}{Définition -- Equipotence}
    Soient $E$ et $F$ deux ensembles.
    \begin{itemize}
    \item On dit que $E$ et $F$ sont \structure{équipotents} s'il existe une \alert{bijection entre $E$ et $F$}. 
    \item De manière équivalente, $E$ et $F$ sont équipotents s'il existe
    \begin{itemize}
    \item une fonction injective de $E$ dans $F$, et 
    \item une fonction injective de $F$ dans $E$
    \end{itemize}
    \item On dit de deux ensembles équipotents qu'ils ont \alert{\og le même cardinal \fg}.
    \item L'équipotence est \structure{réflexive}, \structure{symétrique} et \structure{transitive}.
    \begin{itemize}
    \item L'équipotence définit une \structure{relation d'équivalence} sur $\mathcal{P}(E)$.
    \end{itemize}
    \end{itemize}
  \end{block}

  \begin{exampleblock}{Exemple : $\{a, b, c\}$ et $\{1, 2, 3\}$ sont équipotents}
    \vspace{-3mm}
    $$
    \left\{\begin{array}{ccl}
    \{1, 2, 3\} &\rightarrow& \{a, b, c\}\\
    1 &\mapsto& a\\ 
    2 &\mapsto& b\\ 
    3 &\mapsto& c\\ 
    \end{array}\right.
    $$
  \end{exampleblock}
\end{frame}

\endgroup

% SPDX-License-Identifier: CC-BY-SA-4.0
% Author: Matthieu Perrin
% Part: <Nom de la partie>
% Section: <Nom de la section>
% Sub-section: <Nom de la sous-section>  % (facultatif, laisser vide si non utilisé)
% Frame: <Titre de la slide>

\begingroup

\begin{frame}{Dénombrabilité}
  Soient $E$ un ensemble et $n\in \mathbb{N}$.
  
  \begin{block}{Définition -- Ensemble fini}
    \begin{itemize}
    \item On dit que $E$ est \structure{de cardinal $n$} s'il est équipotent à $\{1, ..., n\}$.
      \begin{itemize}
      \item \example{Exemple : $\{a, b, c\}$ est cardinal $3$}.
      \end{itemize}
    \item On dit que $E$ est \structure{fini} s'il existe $m\in \mathbb{N}$ tel que $E$ est de cardinal $m$.
      \begin{itemize}
      \item \example{Exemple : $\{a, b, c\}$ est cardinal $3$}.
      \end{itemize}
    \end{itemize}
  \end{block}

  \begin{block}{Définition -- Ensemble dénombrable}
    \begin{itemize}
    \item On dit que $E$ est \structure{infini dénombrable} s'il est équipotent à $\mathbb{N}$.
      \begin{itemize}
      \item \example{Exemples : $\mathbb{N}$, $\mathbb{N}\setminus \{0\}$, $\mathbb{Z}$ et $\mathbb{Q}$ sont infinis dénombrables}.
      \end{itemize}
    \item On dit que $E$ est \structure{dénombrable} s'il est fini ou infini dénombrable.
      \begin{itemize}
      \item S'il existe une fonction injective de $E$ dans $\mathbb{N}$, alors $E$ est dénombrable.
      \end{itemize}
    \item On dit que $E$ est \structure{indénombrable} s'il n'est pas dénombrable.
      \begin{itemize}
      \item S'il existe une fonction injective de $\mathbb{R}$ dans $E$, alors $E$ est indénombrable.
      \item \example{Exemples : $\mathbb{R}$, $[0, 1[$ et $\mathbb{C}$ sont indénombrables}.
      \end{itemize}
    \end{itemize}
  \end{block}
\end{frame}

\endgroup

% SPDX-License-Identifier: CC-BY-SA-4.0
% Author: Matthieu Perrin
% Part: <Nom de la partie>
% Section: <Nom de la section>
% Sub-section: <Nom de la sous-section>  % (facultatif, laisser vide si non utilisé)
% Frame: <Titre de la slide>

\begingroup


\begin{frame}{Dénombrabilité des langages}

  Soient $\Sigma$ un alphabet et $f$ une bijection de $\Sigma$ dans $\{1, ..., |\Sigma|\}$.

  \begin{block}{Propriété -- Dénombrabilité de chaque langage}
    Tout langage $L \subseteq \Sigma^\star$ est dénombrable.
  \end{block}

  \vspace{-1mm}
  \begin{block}{Démonstration}
    On interprète les mots de $L$ comme des écritures en base $|\Sigma| + 1$.
    \begin{itemize}
    \item La fonction suivante est injective :
      \vspace{-2mm}
      $$
      g = \left\{\begin{array}{ccc}
      L &\rightarrow& \mathbb{N}\\
      u_{n-1}...u_{0} &\mapsto&  \displaystyle \sum_{i=0}^{n-1} f(u_i) \times (|\Sigma| + 1)^{i}  \\ 
      \end{array}\right.
      $$
    \item Exemples : 
    \begin{itemize}
    \item Si $\Sigma = \{1, ..., 9\}$ et $f : x \mapsto x$, on a $\example{g(12345) = 12345}$
    \item Si $\Sigma = \{a, b\}$, avec $f(a) = 1$ et $f(b) = 2$, on a $\example{g(abab) = 50}$
      \begin{itemize}
      \item car $50 = 2\times 3^3 + 1\times 3^2 + 2\times 3^1 + 1\times 3^0$
      \end{itemize}
    \item Si $|\Sigma| = 1$, on a $\example{g(u) = 2^{|u| -1}}$
    \end{itemize}
    \end{itemize}
  \end{block}
\end{frame}
\endgroup

% SPDX-License-Identifier: CC-BY-SA-4.0
% Author: Matthieu Perrin
% Part: <Nom de la partie>
% Section: <Nom de la section>
% Sub-section: <Nom de la sous-section>  % (facultatif, laisser vide si non utilisé)
% Frame: <Titre de la slide>

\begingroup



\begin{frame}{Non-dénombrabilité de l'ensemble des langages}
  \small
  Soient $\Sigma$ un alphabet et $f$ une bijection de $\Sigma$ dans $\{0, ..., |\Sigma|-1\}$.
  \begin{block}{Non-dénombrabilité de l'ensemble des langages}
    L'ensemble $\mathscr{P}(\Sigma^\star)$ des langages sur $\Sigma$ est indénombrable.
  \end{block}

  \begin{block}{Démonstration}
    On encode les réels entre $0$ et $1$ par les préfixes de leur écriture en base $|\Sigma|$.
    \begin{itemize}
    \item La fonction suivante est injective :
      \vspace{-2mm}
      $$
      h = \left\{\begin{array}{rcl}
      [0; 1[ &\rightarrow& \mathscr{P}(\Sigma^\star)\\
      x &\mapsto& \left\{ u_{n-1}...u_{0} \in \Sigma^\star \,\middle|\, \left\lfloor |\Sigma|^{n} \times x \right\rfloor = \sum_{i=0}^{n-1} f(u_i) \times |\Sigma|^{i}  \right\} \\ 
      \end{array}\right.
      $$
    \item Exemples avec $\Sigma = \{0, ..., 9\}$ et $f : x \mapsto x$. On a :
    \begin{itemize}
    \item \example{$h(\pi-3) = \{\varepsilon, 1, 14, 141, 1415, ...\}$}
    \item \example{$h\left(\frac{1}{3}\right) = \mathcal{L}(3^\star)$}
    \item Si $x$ est rationnel, alors $h(x)$ est rationnel
    \end{itemize}
    \item Remarque : on dit que \structure{$x$ est calculable} si \alert{$h(x)$ est décidable}
    \end{itemize}
  \end{block}
\end{frame}

\endgroup

 
\subsection{Problème de l'Arrêt}
% SPDX-License-Identifier: CC-BY-SA-4.0
% Author: Matthieu Perrin
% Part: <Nom de la partie>
% Section: <Nom de la section>
% Sub-section: <Nom de la sous-section>  % (facultatif, laisser vide si non utilisé)
% Frame: <Titre de la slide>

\begingroup

\SetKwData{X}{x}
\SetKwData{I}{i}
\SetKwFunction{AlgoCompter}{compter}
\SetKwFunction{AlgoBoucler}{boucler}

\begin{frame}{Problème de l'arrêt}

  \onBlock[top=3mm]{Définition -- Problème de l'arrêt}{
    \begin{description}
    \item[Entrées :]
      \begin{itemize}
      \item \alert{$\mathit{algo}$} : un \structure{algorithme avec un paramètre} 
      \item \alert{$x$} : un \structure{argument pour $\mathit{algo}$} 
      \end{itemize}
    \item[Question :] est-ce que \alert{l'exécution de $\mathit{algo}(x)$ se termine} ?
    \end{description}
  }

  \onExampleBlock[bottom=3mm,left=.5\textwidth]{Exemple d'instance positive}{
    \begin{itemize}
    \item $\mathit{algo} = \AlgoCompter$
    \item $x = 5$
    \end{itemize}
    \vspace{2mm}
    \begin{algorithm}[H]
      \Proc{$\AlgoCompter(\X \in \mathbb{N})$}{
        \For{$\I$ \From $1$ \To $\X$}{
          $\Print(\I)$\;
        }
      }
    \end{algorithm}
  }

  \onExampleBlock[bottom=3mm,right=.5\textwidth]{Exemple d'instance négative}{
    \begin{itemize}
    \item $\mathit{algo} = \AlgoBoucler$
    \item $x = 5$
    \end{itemize}
    \vspace{2mm}
    \begin{algorithm}[H]
      \Proc{$\AlgoBoucler(\X \in \mathbb{N})$}{
        \While{\True}{
          $\Print(\X)$\;
        }
      }
    \end{algorithm}
  }
\end{frame}



\begin{frame}{Problème de l'arrêt}
  \small
 
  \begin{block}{Définition -- Le langage $L_{halt}$}
    Étant données une MTD $M$ et une entrée $u$, est-ce que $M$ termine sur $u$ ? 
 
    $$L_{\mathit{halt}} \eqdef \{ \langle \mathit{encode}(M), u \rangle \in (\{0,1\}^\star)^2  \mid M \text{ termine sur } u \}$$ 
    
  \end{block}
 
  \pause 
  \begin{block}{Propriétés}
    \begin{enumerate}
    \item<2-> $L_{\mathit{halt}}$ est \structure{récursivement énumérable}.
    \item<3-> $L_{\mathit{halt}}$ est \structure{indécidable}.
    \item<4-> $\overline{L_{\mathit{halt}}}$ n'est \structure{pas récursivement énumérable}.
    \end{enumerate}
  \end{block}
 
  \begin{block}{Démonstration}
    \begin{enumerate}
    \item<2-> $L_{\mathit{halt}}$ est reconnu par la MTD universelle normalisée pour la reconnaissance.
    \item<3-> Un peu plus difficile...
    \item<4-> Sinon, $L_{\mathit{halt}}$ et $\overline{L_{\mathit{halt}}}$ seraient récursivement énumérable, donc décidables.
    \end{enumerate}
  \end{block}
\end{frame}

\endgroup

% SPDX-License-Identifier: CC-BY-SA-4.0
% Author: Matthieu Perrin
% Part: <Nom de la partie>
% Section: <Nom de la section>
% Sub-section: <Nom de la sous-section>  % (facultatif, laisser vide si non utilisé)
% Frame: <Titre de la slide>

\begingroup

\SetKwData{X}{x}
\SetKwData{I}{i}
\SetKwFunction{searchPerfect}{search\_perfect}
\SetKwFunction{perfect}{perfect}
\SetKwData{S}{somme}

\begin{frame}{Un exemple plus compliqué}
  
  \onExampleBlock[top=-2mm,left=.66\textwidth]{Ces instances sont-elles positives ?}{
    \begin{itemize}
    \item $\mathit{algo} = \searchPerfect$
    \item $x = 2$ puis $x = 3$
    \end{itemize}
    \vspace{1mm}
    \begin{algorithm}[H]
      \Proc{$\searchPerfect(\X \in \mathbb{N})$}{
        \lWhile{$\lnot \perfect(\X)$}{
          $\X \leftarrow \X+2$%
        }
      }
      \Fun{$\perfect(\X \in \mathbb{N}) \in \mathbb{B}$}{
        $\S\leftarrow 0$\;
        \For{$\I$ \From $1$ \To $\X-1$}{
          \If{$\X \!\!\mod \I = 0$}{$\S \leftarrow \S + \I$}
        }
        \Return $\S = \X$\;
      }
    \end{algorithm}
  }
  
  \onBlock<2-|handout>[top=-2mm,right=.33\textwidth]{Nombre parfait}{
    Nombre égal à la somme de ses diviseurs propres.
  }

  \onExampleBlock<2-|handout>[right=.33\textwidth,y=2mm]{Exemples}{
    \begin{description}[45 :]
    \item[$4$ :]  $1 + 2 = 3$
    \item[$6$ :]  $1 + 2 + 3 = 6$
    \item[$45$ :] $1 + 3 + 5 + 9 + 15 = 33$
    \end{description}
  }

  \onAlertBlock<3-|handout>[bottom=-9mm]{Conjecture -- Il n'existe pas de nombre parfait impair}{
    \myquote[alert]{Euler}{
      Il reste la question de savoir si un nombre parfait impair peut exister ; c'est une question assurément très difficile, jamais résolue ni par les Anciens ni par les Modernes.
    }
  }

\end{frame}

\endgroup

% SPDX-License-Identifier: CC-BY-SA-4.0
% Author: Matthieu Perrin
% Part: <Nom de la partie>
% Section: <Nom de la section>
% Sub-section: <Nom de la sous-section>  % (facultatif, laisser vide si non utilisé)
% Frame: <Titre de la slide>

\begingroup

\begin{frame}{L'argument de Strachey}
  \small
  \SetKwFunction{F}{f}
  \SetKwFunction{Termine}{termine}

  \hspace{-3mm}
  \begin{tikzpicture}
    \draw[white] (5,10) -- (5,2.5);

    \draw (5,10) node{\begin{minipage}{\textwidth}
        \begin{itemize}
        \item Supposons (par l'absurde) qu'il existe une fonction informatique \structure{\Termine}
          qui décide si un programme passé en argument termine. 
        \item La fonction \alert{$\F$} ci-dessous termine-t-elle ? 
        \end{itemize}
    \end{minipage}};

    
    \draw (3,8.1) node{\begin{minipage}{.4\textwidth}
        \begin{algorithm}[H]
          \Fun{$\F()$}{
            \uIf{$\Termine(\F)$}{
              \lWhile{\True}{}
            }\lElse{\Return}
          }
        \end{algorithm}
    \end{minipage}};

    
    \draw (8.5,8.8) node{\includegraphics[width=1.8cm]{Stachey}};
    \draw[structure] (8.5,7.3) node{Christopher Strachey};


    \draw<2-> (5,6.3) node{\begin{minipage}{\textwidth}
        \begin{itemize}
        \item Si $\F$ termine, $\Termine(\F)$ retourne $\True$ donc $\F$ ne termine pas.
        \item Si $\F$ ne termine pas, $\Termine(\F)$ retourne $\False$ donc $\F$ termine.
        \item<3-> Absurde, donc \alert{$\Termine$ n'existe pas}. 
        \end{itemize}
    \end{minipage}};

    \draw<4-> (5,4.1) node{\begin{minipage}{\textwidth}
        \begin{block}{Difficulté à la formalisation}
          \vspace{-.5mm}
          Quel est le type de l'argument de $\Termine$ ?
            \begin{itemize}
            \item Si c'est le code de $\F$, est-ce que $\F$ est capable de recopier son code ?
              \begin{itemize}
              \item\vspace{-.5mm} Oui, mais difficile à démontrer : notion de \alert{quine} 
              \end{itemize}
            \item Dans les langages modernes, un pointeur de fonction.
            \end{itemize}
        \end{block}
    \end{minipage}};
  \end{tikzpicture}
\end{frame}

\endgroup

% SPDX-License-Identifier: CC-BY-SA-4.0
% Author: Matthieu Perrin
% Part: <Nom de la partie>
% Section: <Nom de la section>
% Sub-section: <Nom de la sous-section>  % (facultatif, laisser vide si non utilisé)
% Frame: <Titre de la slide>

\begingroup

\SetKwFunction{Diagonale}{diag}
\SetKwFunction{Decide}{decide}

\begin{frame}{L'argument classique de Turing}
  \small
  
  \hspace{-3mm}
  \begin{tikzpicture}
    \draw[white] (5,10) -- (5,2.5);

    \draw (5,10) node{\begin{minipage}{\textwidth}
        \begin{itemize}
        \item Supposons (par l'absurde) qu'il existe une MTD \structure{$\Decide_{\mathit{halt}}$}
          qui décide $L_{\mathit{halt}}$.
          \begin{itemize}
          \item Qui utilise deux rubans \textsc{m} et \textsc{u} pour son entrée $\langle \mathit{encode}(M), u \rangle$.
          \end{itemize}
        \item On considère la MTD qui implémente la fonction $\alert{\Diagonale}$ ci-dessous : 
        \end{itemize}
    \end{minipage}};
    
    \draw (3.5,8) node{\begin{minipage}{.4\textwidth}
        \begin{algorithm}[H]
          \Fun{$\Diagonale(u)$}{
            \uIf{$\Decide_{\mathit{halt}} \langle u, u \rangle$}{
              \lWhile{\True}{}
            }\lElse{\Return \True}
          }
        \end{algorithm}
    \end{minipage}};

    \draw (5,6.5) node{\scalebox{.8}{\begin{tikzpicture}[shorten >=1pt, node distance=2.3cm, on grid, auto]

          \draw[fill=structure!20, rounded corners=8]  (7.5,-.25) rectangle (5,2.25); 
          \draw[structure]  (5,2.25) node[below right] {$\Decide_{\mathit{halt}}$}; 
          \draw[structure]  (4,0) node {\scriptsize $\forall x \neq \blank$}; 

          \node[state, initial, initial text=] (A)  at (0.0,1)    {\faPencil}; 
          \node[state] (B)  at (2.5,1)    {\faBackward}; 
          \node[state, fill=structure!20] (D0) at (5.0,1)    {\faPlay}; 
          \node[state, fill=structure!20] (DO) at (7.5,1.75) {\faCheck}; 
          \node[state, fill=structure!20, accepting] (DN) at (7.5,0.25) {\faTimes}; 

          \path [dashed,->, structure]    (D0) edge[bend left=5mm]  (DO);
          \path [dashed,->, structure]    (D0) edge  (DO);
          \path [dashed,->, structure]    (D0) edge[bend right=5mm] (DO);
          
          \path [dashed,->, structure]    (D0) edge[bend left=5mm]  (DN);
          \path [dashed,->, structure]    (D0) edge  (DN);
          \path [dashed,->, structure]    (D0) edge[bend right=5mm] (DN);

          \path [->] (A) edge node[swap]{\scriptsize$\left\{\begin{array}{c}\textsc{m}:\smTMtransL{\blank}{\blank}\\\textsc{u}:\smTMtransL{\blank}{\blank}\end{array}\right.$} (B);
          \path [->] (B) edge node[swap]{\scriptsize$\left\{\begin{array}{c}\textsc{m}:\smTMtransR{\blank}{\blank}\\\textsc{u}:\smTMtransR{\blank}{\blank}\end{array}\right.$} (D0);

          \path [->] (A) edge[loop below, looseness=5] node{\scriptsize$\left\{\begin{array}{c}\textsc{m}:\smTMtransR{\blank}{x}\\\textsc{u}:\smTMtransR{x}{x}\end{array}\right.$} (A);
          \path [->] (B) edge[loop below, looseness=5] node{\scriptsize$\left\{\begin{array}{c}\textsc{m}:\smTMtransL{x}{x}\\\textsc{u}:\smTMtransL{x}{x}\end{array}\right.$} (B);
          \path [->] (DO) edge[loop right, looseness=5] node{\scriptsize$\begin{array}{c}\smTMtransR{y}{y}\\\forall y\in \Gamma\end{array}$} (DO);

    \end{tikzpicture}}};


    \draw (5,4) node{\begin{minipage}{\textwidth}
        \begin{itemize}
        \item Que retourne \alert{$\Decide_{\mathit{halt}} \langle \Diagonale, \Diagonale \rangle$} ?
          \begin{itemize}
          \item<2-> Si $\Diagonale(\Diagonale)$ termine, alors $\Diagonale(\Diagonale)$ ne termine pas.
          \item<2-> Si $\Diagonale(\Diagonale)$ ne termine pas, alors $\Diagonale(\Diagonale)$ termine.
          \end{itemize}
        \item<3-> Absurde, donc $\Decide_{\mathit{halt}}$ n'existe pas et \alert{$L_{\mathit{halt}}$ est indécidable}. 
        \end{itemize}
    \end{minipage}};

  \end{tikzpicture}

\end{frame}

\endgroup

% SPDX-License-Identifier: CC-BY-SA-4.0
% Author: Matthieu Perrin
% Part: <Nom de la partie>
% Section: <Nom de la section>
% Sub-section: <Nom de la sous-section>  % (facultatif, laisser vide si non utilisé)
% Frame: <Titre de la slide>

\begingroup

\SetKwFunction{AlgoSearchPerfect}{search\_perfect}
\SetKwFunction{AlgoCompter}{compter}
\SetKwFunction{AlgoBoucler}{boucler}

\begin{frame}{Signification de l'indécidabilité}
  \begin{block}{Ce que l'indécidabilité signifie :}
    \begin{itemize}
    \item Aucun \alert{algorithme général} ne décide l'arrêt \alert{pour tous les $\langle M, u \rangle$} 
    \item En pratique, il y a bien des algorithmes difficiles à analyser 
    \end{itemize}
  \end{block}

  \begin{center}
    \begin{tikzpicture}
      \begin{scope}
        \clip (0,0) ellipse (2cm and 1cm);

        \begin{scope}
          \clip (-2,0) rectangle (2,1);
          \fill[structure!20] (-2,0) rectangle (2,1);
          \pgfdeclareradialshading{alertToStructure}{\pgfpoint{0cm}{0cm}}{color(0cm)=(alert!20);color(6mm)=(alert!20);color(9mm)=(structure!20);color(1cm)=(structure!20)}
          \shade[shading=alertToStructure] (2,0) ellipse (2.5cm and .75cm);
        \end{scope}
        \begin{scope}
          \clip (-2,0) rectangle (2,-1);
          \fill[example!20] (-2,0) rectangle (2,-1);
          \pgfdeclareradialshading{alertToExample}{\pgfpoint{0cm}{0cm}}{color(0cm)=(alert!20);color(6mm)=(alert!20);color(9mm)=(example!20);color(1cm)=(example!20)}
          \shade[shading=alertToExample] (2,0) ellipse (2.5cm and .75cm);
        \end{scope}

        \draw[black!40,thick] (0,0) ellipse (2cm and 1cm);
        \draw[black!40] (-2,0) -- (2,0);
        \node[structure] at (0, .6) {\footnotesize\begin{tabular}{c}instances \\positives\end{tabular}};
        \node[example]   at (0,-.6) {\footnotesize\begin{tabular}{c}instances \\négatives\end{tabular}};
        \node[alert]     at (1.15,-.03) {\footnotesize\begin{tabular}{c}instances \\difficiles\end{tabular}};

        \node[alert]     at (1.85,0)  {$\bullet$};
        \node[structure] at (-1.5, .5)  {$\bullet$};
        \node[example]   at (-1.5,-.5) {$\bullet$};
      \end{scope}
      \node[alert, right]     at (2,0)      {\small\AlgoSearchPerfect};
      \node[structure, left]  at (-1.7, .5) {\small\AlgoCompter};
      \node[example, left]    at (-1.7,-.5) {\small\AlgoBoucler};
    \end{tikzpicture}
  \end{center}
  
  \begin{block}{Solutions possibles :}
    \begin{itemize}
    \item Restreindre le problème à une sous-classe.
      \begin{itemize}
      \item Par exemple, boucles \KWStyle{pour} avec borne explicite
      \end{itemize}
    \item Utiliser des approximations et des heuristiques
      \begin{itemize}
      \item Trois réponses : ``oui'', ``non'' et ``je ne sais pas''
      \item Deux réponses : ``oui'', ``attention''
      \end{itemize}
    \end{itemize}
  \end{block}
\end{frame}

\endgroup

% SPDX-License-Identifier: CC-BY-SA-4.0
% Author: Matthieu Perrin
% Part: <Nom de la partie>
% Section: <Nom de la section>
% Sub-section: <Nom de la sous-section>  % (facultatif, laisser vide si non utilisé)
% Frame: <Titre de la slide>

\begingroup

  \SetKwFunction{Decide}{decide}

\begin{frame}{Problème de l'arrêt sur chaîne vide}
  \begin{tikzpicture}
    
    \draw[white] (5,8.5) -- (5,1.5);

    \draw (5,8.9) node{\begin{minipage}{\textwidth}
        \begin{description}
        \item [Le langage :]
          Étant donnée une MTD $M$, est-ce que $M$ termine sur $\varepsilon$ ? 
        \end{description}
        $$\alert{L_{\mathit{empty\_halt}} \eqdef \{ \mathit{encode}(M) \mid M \text{ termine sur } \varepsilon \}}$$ 
        \begin{description}
        \item [Théorème :] \vspace{-4mm}Le langage $L_{\mathit{empty\_halt}}$ est indécidable.
        \end{description}
    \end{minipage}};
    
    \draw<2-> (5,5) node{\begin{minipage}{\textwidth}
        \begin{block}{Démonstration}
          \begin{itemize}
          \item Supposons (par l'absurde) que le langage $L_{\mathit{empty\_halt}}$ est décidable.
            \begin{itemize}
            \item Soit $\alert{\Decide_{\mathit{empty\_halt}}}$ une machine de turing déterministe qui le décide.
            \end{itemize}
          \item<3-> On définit la fonction $\alert{\mathit{initialiseur}}$ 
            \begin{description}
            \item[Entrée :] Un couple $\langle M, u \rangle \in L_{\mathit{univ}}$.
            \item[Sortie :] $\mathit{initialiseur}(M, u)$ écrit $u$ sur son ruban puis exécute $M$
            \item<4>[Remarque :] La fonction \structure{$initialiseur$ est calculable}
            \item<3>[Remarque :] \vspace{1.8cm}La fonction \structure{$initialiseur$ est calculable}
            \end{description}
          \end{itemize}
        \end{block}
    \end{minipage}};

    \draw<3> (5,3.5) node{\scalebox{.75}{\begin{tikzpicture}[shorten >=1pt, node distance=2.3cm, on grid, auto]
          \fill[example!20, rounded corners=10]  (0,-.15) rectangle (10,2.15); 
          \draw (9,1) node[cloud, cloud puffs=11 ,cloud puff arc=120, aspect=2, inner ysep=1em, fill=structure!20]{\structure{$M$}};
          \draw[example]  (0,2.15) node[below right] {$\mathit{initialiseur}(M, u)$}; 
          
          \node[state, fill=example!20, initial, initial text=] (q0)  at (0.0,1)  {\faPencil$_1$}; 
          \node[state, fill=example!20] (q1)   at (2,1)                         {\faPencil$_2$}; 
          \node[state, fill=example!20] (q2)   at (4,1)                         {\faPencil$_n$}; 
          \node[state, fill=example!20] (q3)   at (6,1)                         {\faBackward}; 
          \node[state, fill=structure!20] (M0) at (8,1)        {\faPlay}; 
          \node[state, fill=structure!20, accepting] (Mf) at (10,1)        {\faCheck}; 
          
          \path [->]        (q0) edge node{\scriptsize$\smTMtransL{\blank}{u_{|u|}}$} (q1);
          \path [dashed,->] (q1) edge  (q2);
          \path [->]        (q2) edge node{\scriptsize$\smTMtransL{\blank}{u_{1}}$} (q3);
          \path [->]        (q3) edge node{\scriptsize$\smTMtransR{\blank}{\blank}$} (M0);
          
          \path [dashed,->, structure]    (M0) edge[bend left=5mm]  (Mf);
          \path [dashed,->, structure]    (M0) edge[bend right=5mm] (Mf);
    \end{tikzpicture}}};

    \draw<4> (5,3.15) node{\begin{minipage}{\textwidth}
        \begin{itemize}
        \item La fonction $\alert{\Decide_{\mathit{halt}}}$ décide le langage $L_{halt}$ : 
          \begin{algorithm}[H]
            \Fun{$\Decide_{\mathit{halt}} \langle M, u \rangle$ : booléen}{
              \Return $\Decide_{\mathit{empty\_halt}} (\mathit{initialiseur}(M, u))$\;
            }
          \end{algorithm}
        \item Absurde, car $L_{halt}$ est indécidable. Donc $L_{\mathit{empty\_halt}}$ est indécidable.
        \end{itemize}
    \end{minipage}};

  \end{tikzpicture}
\end{frame}
\endgroup

 
\subsection{Notion de réduction}
% SPDX-License-Identifier: CC-BY-SA-4.0
% Author: Matthieu Perrin
% Part: 
% Section: 
% Sub-section: 
% Frame: 

\begingroup

\begin{frame}{Notion de grammaire}
  
  \begin{block}{Définition -- Grammaire non-restreinte}
    Une \structure{grammaire non-restreinte} est un quadruplet \alert{$\langle \Sigma, \Gamma, S, \rightarrow \rangle$} tel que :
    \begin{description}[xxxx]
    \item[\alert{$\Sigma$}] $\in \Omega_f$ l'alphabet des \structure{terminaux}
    \item[\alert{$\Gamma$}] $\in \Omega_f$ : l'alphabet des \structure{non-terminaux}; tel que $\Sigma \cap \Gamma = \emptyset$
    \item[\alert{$S$}] $\in \Gamma$ : l'\structure{axiome}
    \item[\alert{$\rightarrow$}] $\subseteq ((\Sigma \cup \Gamma)^\star \cdot \Gamma \cdot (\Sigma \cup \Gamma)^\star) \times (\Sigma \cup \Gamma)^\star$ : l'ensemble des \structure{règles de production}
    \end{description}

    \vspace{1mm}
    Une \structure{règle de production} est un couple \alert{$\langle \alpha, \beta \rangle \in \rightarrow$}, noté $\alert{\alpha \rightarrow \beta}$ tel que :
    \begin{description}[xxxx]
    \item[\alert{$\alpha$}] $\in (\Sigma \cup \Gamma)^\star \cdot \Gamma \cdot (\Sigma \cup \Gamma)^\star$ : \structure{membre gauche}
    \item[\alert{$\beta$}] $\in (\Sigma \cup \Gamma)^\star$ : \structure{membre droit}
    \end{description}
  \end{block}

  \begin{block}{Remarques}
    Soient $G = \langle \Sigma, \Gamma, S, \rightarrow \rangle$ une grammaire et $\langle \alpha, \beta \rangle \in \rightarrow$ une règle de production. 
    \begin{itemize}
    \item On n'impose pas que $|\alpha| = 1$ 
    \item On impose que $\alpha$ contienne au moins un non-terminal 
    \item On dit que $G$ est \structure{algébrique} si $\forall \langle \alpha, \beta \rangle \in \rightarrow, |\alpha| = 1$
    \end{itemize}
  \end{block}

\end{frame}

\endgroup
\endinput

% SPDX-License-Identifier: CC-BY-SA-4.0
% Author: Matthieu Perrin
% Part: <Nom de la partie>
% Section: <Nom de la section>
% Sub-section: <Nom de la sous-section>  % (facultatif, laisser vide si non utilisé)
% Frame: <Titre de la slide>

\begingroup

\begin{frame}{Classes de calculabilité}
  \small 
  Soient $\Sigma$ un alphabet, et $L_A$ et $L_B$ deux langages de $\Sigma^\star$. 
  \begin{block}{Propriété : $\leq_M$ est un préordre sur $\mathscr{P}(\Sigma^\star)$}
      \begin{itemize}
      \item $\leq_M$ est réflexive
      \item $\leq_M$ est transitive
      \item $\leq_M$ n'est ni symétrique ni antisymétrique. 
      \end{itemize}
      $L_A$ et $L_B$ sont \structure{équivalents par mappage}, noté \alert{$L_A \equiv_m L_B$}, si $L_A \leq_m L_B$ et $L_B \leq_m L_A$.
  \end{block}
  
  \begin{exampleblock}{Notion de préordre}
    \begin{itemize}
    \item Un \example{préordre} est une relation $\sqsubseteq$ réflexive et transitive. 
      \begin{itemize}
      \item On définit $\equiv$ par $x \equiv y$ si $x \sqsubseteq y$ et $y \sqsubseteq x$
      \begin{itemize}
      \item $\equiv$ est une relation d'équivalence
      \item $\sqsubseteq$ est une relation d'ordre sur les classes d'équivalence de $\equiv$
      \end{itemize}
      \end{itemize}
    \item Par exemple, sur $\{a, b\}^\star$, $u \sqsubseteq v$ si toutes les lettres de $u$ sont dans $v$. 
      \begin{itemize}
      \item $u \equiv v$ si $u$ et $v$ utilisent les mêmes lettres. 
      \item Quatre classes d'équivalence : \example{$\{\varepsilon\}$}, \example{$\mathcal{L}(a^\star)$}, \example{$\mathcal{L}(b^\star)$} et \example{$\mathcal{L}(\Sigma^\star (ab|ba) \Sigma^\star)$}
      \end{itemize}
    \end{itemize}

  \end{exampleblock}
\end{frame}
\endgroup

% SPDX-License-Identifier: CC-BY-SA-4.0
% Author: Matthieu Perrin
% Part: <Nom de la partie>
% Section: <Nom de la section>
% Sub-section: <Nom de la sous-section>  % (facultatif, laisser vide si non utilisé)
% Frame: <Titre de la slide>

\begingroup

\begin{frame}{Exemple de réduction}
  \small
  \begin{block}{Théorème -- Indécidabilité de la décision}
    Décider si une MTD $M_a$ donnée décide $\mathcal{L}(a^\star)$ est indécidable.
  \end{block}
  \begin{block}{Démonstration par réduction au problème de l'arrêt sur $\varepsilon$}
    \begin{itemize}
    \item Soient $L_A = \{ M_\varepsilon \mid M_\varepsilon \text{ termine sur } \varepsilon \}$ et $L_B = \{ M_a \mid M_a \text{ décide } \mathcal{L}(a^\star) \}$.
    \item Pour toute MTD $M_\varepsilon$, on construit $M_a = f(M_\varepsilon)$ comme suit : 

      \begin{tikzpicture}
        \draw (0,0) node {\begin{minipage}{.6\textwidth}
              \SetKwFunction{Decider}{decider}
              \begin{algorithm}[H]
                \Algo{$M_a(u)$}{
                  $M_\varepsilon(\varepsilon)$\;
                  \Return $u \in \mathcal{L}(a^\star)$\;
                }
              \end{algorithm}
        \end{minipage}};
        \draw (3.5,0.1) node {\begin{tikzpicture}[shorten >=1pt,node distance=2cm,on grid,auto]

              \draw (9,1) node[cloud, cloud puffs=11 ,cloud puff arc=120, aspect=2, inner ysep=1em, fill=structure!20]{};
              \draw[structure]  (9,1) node {$M_\varepsilon$ sur $R$}; 
              
              \node [state, fill=structure!20, initial, initial text=] (M0) at (8,1)  {}; 
              \node [state, fill=structure!20]                         (Mf) at (10,1) {}; 
              \node [state, accepting]                                 (E)  at (12,1) {}; 

              \path [->] (Mf)  edge[loop above, looseness=5] node {\scriptsize$A:\smTMtransR{a}{a}$} (Mf);
              \path [->] (Mf) edge node {\scriptsize$A:\smTMtransR{\blank}{\blank}$} (E);

              \path [dashed,->, structure]    (M0) edge[bend left=5mm]  (Mf);
              \path [dashed,->, structure]    (M0) edge[bend right=5mm] (Mf);

              \draw[structure]  (10,0.2) node {$M_a$ sur deux rubans $R$ et $A$}; 
        \end{tikzpicture}};
      \end{tikzpicture}

    \item La réduction $f : M_\varepsilon \mapsto M_a$  est calculable.
    \item $\forall M_\varepsilon, M_\varepsilon \in L_A \Leftrightarrow f(M_\varepsilon) \in L_B$.
    \item Or $L_A$ est indécidable, donc $L_B$ est indécidable.
    \end{itemize}
  \end{block}
\end{frame}
\endgroup

% SPDX-License-Identifier: CC-BY-SA-4.0
% Author: Matthieu Perrin
% Part: <Nom de la partie>
% Section: <Nom de la section>
% Sub-section: <Nom de la sous-section>  % (facultatif, laisser vide si non utilisé)
% Frame: <Titre de la slide>

\begingroup

\begin{frame}{Généralisation : le théorème de Rice}
  ~
  
 \vspace{-9mm}
  \begin{tikzpicture}
  \hspace{-3mm}

    \draw (5,5) node[above]{\begin{minipage}{1.1\textwidth}
        \begin{block}{Théorème -- Théorème de Rice}
          Toute propriété sémantique non triviale \\d'un programme est indécidable.

          \vspace{2mm}
          \begin{description}
            \item[\og Sémantique \fg : ] prédicat sur les exécutions
            \item[\og Non-triviale \fg : ] vraie sur certains programmes, \\\hspace{5mm}fausse sur d'autres
          \end{description}
        \end{block}  
    \end{minipage}};

    \draw (5,5) node[below]{\begin{minipage}{1.1\textwidth}
        \begin{exampleblock}{Exemples de propriétés indécidables}
          \begin{itemize}
          \item Le programme s'arrête sur une certaine entrée
          \item Le programme s'arrête sur toutes ses entrées
          \item Le programme retourne le même résultat qu'un autre programme
          \item Le programme retourne un résultat correct par rapport à sa spécification
          \item Le programme ne déréférence pas le pointeur nul
          \item Le programme ne lève pas d'exception
          \item Le programme est un virus
          \end{itemize}
        \end{exampleblock}  
    \end{minipage}};

%    \draw (8.5,6) node{\includegraphics[width=2.5cm]{Rice}};
%    \draw[structure] (8.5,4.4) node{Gordon Henry Rice};
  \end{tikzpicture}
\end{frame}

\endgroup

% SPDX-License-Identifier: CC-BY-SA-4.0
% Author: Matthieu Perrin
% Part: <Nom de la partie>
% Section: <Nom de la section>
% Sub-section: <Nom de la sous-section>  % (facultatif, laisser vide si non utilisé)
% Frame: <Titre de la slide>

\begingroup

\begin{frame}{Quelques problèmes indécidables}

  \on[y=-5mm]{
    \begin{tikzpicture}[y=11mm]
      \begin{scope}[align=center, font=\small, -latex]
        \node (halt)         at ( 0 00,  0.00) {Arrêt};
        \node (pcpi)         at ( 1.50,  0.00) {\textsc{pcpi}};                                       \path (halt)         edge (pcpi);
        \node (pcp)          at ( 3.00,  0.00) {\textsc{pcp}};                                        \path (pcpi)         edge (pcp);
        \node (ambiguT2)     at ( 5.00,  0.00) {Ambiguité d'une\\ grammaire\\ algébrique};            \path (pcp)          edge (ambiguT2);
        \node (intersectT2)  at ( 5.00, -2.00) {Intersection \\ de deux \\ grammaires \\algébriques}; \path (pcp)          edge (intersectT2);
        \node (haltepsilon)  at (-1.00, -1.00) {Arrêt sur\\mot vide};                                 \path (halt)         edge (haltepsilon);
        \node (hilbert)      at (-4.00, -1.00) {\textsc{Entscheidungsproblem}};                       \path (halt)         edge (hilbert);
        \node (consistency)  at (-4.00, -2.00) {Cohérence d'un \\ système logique};                   \path (hilbert)      edge (consistency);
        \node (matrix)       at (-1.25,  2.00) {Matrices mortelles\\ou identité};                     \path (halt)         edge (matrix);
        \node (pavage)       at (-3.25,  2.00) {Pavage \\ du plan};                                   \path (halt)         edge (pavage);
        \node (polynome)     at (-3.00,  0.25) {Résolution \\d'équations \\ diophantienne};           \path (halt)         edge (polynome);
        \node (decT0)        at ( 2.00, -2.00) {Décision\\ d'un mot par une \\ grammaire générale};   \path (halt)         edge (decT0);
        \node (normalLambda) at ( 2.00,  1.00) {Mise d'un $\lambda$-terme\\ sous forme normale};      \path (halt)         edge (normalLambda);
        \node (terminaison)  at ( 2.00,  2.00) {Terminaison d'un\\système de réécriture};             \path (normalLambda) edge (terminaison);
        \node (confluence)   at ( 5.00,  2.00) {Confluence d'un\\ système de réécriture};             \path (pcp)          edge (confluence);
        \node (rice)         at (-1.00, -2.50) {Théorème \\de Rice};                                  \path (haltepsilon)  edge (rice);
        \node (correct)      at (-1.00, -4.00) {Conformité\\ d'un programme à \\sa spécification};    \path (rice)         edge (correct);
        \node (secure)       at (-4.00, -4.00) {Présence d'une\\ vulnérabilité\\ de sécurité};        \path (rice)         edge (secure);
        \node (univT0)       at ( 2.00, -4.00) {Universalité\\ d'une\\ grammaire\\ générale};         \path (rice)         edge (univT0);
        \node (univT2)       at ( 5.00, -4.00) {Universalité\\ d'une\\ grammaire\\ algébrique};       \path (univT0)       edge (univT2);
      \end{scope}

      \begin{scope}[background, text=structure, inner sep=3mm]
        \clip (-6,-5) rectangle (7,2.5);
        \draw[structure!50,fill=white] (0,0)     rectangle (7,3);
        \draw[structure!50,fill=white] (0,-5)    -- (0,-.5) to[out=90,in=180,distance=5mm] (1,.5) -- (7,.5) -- (7,-5);
        \draw[structure!50,fill=white] (-6,0)    rectangle (0,3);
        \draw[structure!50,fill=white] (-6,-.5)  -- (0,-.5) to[out=90,in=0,distance=10mm] (-3,1) -- (-6,1);
        \draw[structure!50,fill=white] (-6,-2.5) rectangle (0,-.5);
        \draw[structure!50,fill=white] (.5,-5)   -- (.5,0) to[out=90,in=0,distance=3mm] (0,.5) -- (-.5,.5) to[out=180,in=0,distance=15mm] (-3.5,-2.5) -- (-6,-2.5) -- (-6,-5);
        
        \node[left,  align=right] at (7,-1)    {Langages \\formels};
        \node[left,  align=right] at (7,1)     {Systèmes de\\ réécriture};
        \node[right, align=left ] at (-6,2)    {Géométrie};
        \node[right, align=left ] at (-6,0.25) {Algèbre};
        \node[right, align=left ] at (-6,-1.4) {Logique};
        \node[right, align=left ] at (-6,-3)   {Vérification \\de programme};
      \end{scope}
    \end{tikzpicture}
  }

\end{frame}

\endgroup

 
\subsection{Le Problème de Correspondance de Post}
% SPDX-License-Identifier: CC-BY-SA-4.0
% Author: Matthieu Perrin
% Part: <Nom de la partie>
% Section: <Nom de la section>
% Sub-section: <Nom de la sous-section>  % (facultatif, laisser vide si non utilisé)
% Frame: <Titre de la slide>

\begingroup

\begin{frame}{Problème de Correspondance de Post}

  \on[text, top]{
    Soit $\Sigma$ un alphabet contenant au moins deux symboles. 
    \begin{block}{Le problème PCP}
      \begin{description}
      \item[Entrée :] \vspace{-2mm}Un ensemble fini de \structure{sortes de dominos} $D \in \mathscr{P}(\Sigma^+ \times \Sigma^+)$.
        \begin{itemize}
        \item Dominos de la forme \VDomino{$\alpha$}{$\beta$}, avec $\alpha$ et $\beta$ des mots sur $\Sigma$
        \end{itemize}
      \item[Solution :] Un mot non-vide
        $\VDomino{\alpha_1}{\beta_1} \cdots \VDomino{\alpha_k}{\beta_k} \in D^+$ de dominos\footnote[frame]{On peut utiliser chaque domino autant de fois qu'on veut} tel que :
        \begin{itemize}
        \item $\alpha_{1} \cdots \alpha_{k} = \beta_{1} \cdots \beta_{k}$
        \end{itemize}
      \item[Sortie :] \alert{\og oui \fg} s'il existe une solution, \alert{\og non \fg} sinon.
      \end{description}
    \end{block}
  }

  \onImage<1>[bottom=5mm]{%
    width=1.5cm,
    title={Emil Post},
    license={Domaine public (\href{https://commons.wikimedia.org/wiki/File:Emil_Leon_Post.jpg}{Wikimedia})},
    img={Post.jpg}
  }

  \onExampleBlock<2-|handout> [bottom=2mm] {Exemples} {
    \begin{itemize}
    \item $\left\{\VDomino{a}{baa}, 
      \VDomino{ab}{aa}, 
      \VDomino{bba}{bb}\right\}$
      \uncover<3-|handout> {a pour solution
        \VDomino{bba}{bb}
        \VDomino{ab}{aa}
        \VDomino{bba}{bb}
        \VDomino{a}{baa}.}
    \item $\left\{\VDomino{a}{ba}, 
      \VDomino{ab}{b}\right\}$
      \uncover<3-|handout> {n'a pas de solution.}
    \end{itemize}
  }
 
\end{frame}

\endgroup

% SPDX-License-Identifier: CC-BY-SA-4.0
% Author: Matthieu Perrin
% Part: <Nom de la partie>
% Section: <Nom de la section>
% Sub-section: <Nom de la sous-section>  % (facultatif, laisser vide si non utilisé)
% Frame: <Titre de la slide>

\begingroup

\begin{frame}{Problème de Correspondance de Post Initialisé}

  \on[text,top]{
    Soit $\Sigma$ un alphabet contenant au moins deux symboles. 
    
    \begin{block}{Le problème PCPI}
      \begin{description}
      \item[Entrées :]
        \begin{itemize}
        \item Un ensemble fini de \structure{sortes de dominos} $D \in \mathscr{P}(\Sigma^+ \times \Sigma^+)$.
        \item Un \structure{domino initial} $\VDomino{\gamma}{\delta} \in D$
        \end{itemize}
      \item[Solution :] Un mot non-vide
        $\VDomino{\alpha_1}{\beta_1} \cdots \VDomino{\alpha_k}{\beta_k} \in D^+$ de dominos tel que :
        \begin{itemize}
        \item $\alpha_{1} \cdots \alpha_{k} = \beta_{1} \cdots \beta_{k}$ et $\VDomino{\alpha_1}{\beta_1} = \VDomino{\gamma}{\delta}$
        \end{itemize}
      \item[Sortie :] \alert{\og oui \fg} s'il existe une solution, \alert{\og non \fg} sinon.
      \end{description}
    \end{block}
  }
  
  \onExampleBlock<2-|handout> [bottom=2mm] {Exemple} {
    \begin{itemize}
    \item $D=\left\{
      \VDomino{a}{ab}, 
      \VDomino{baba}{a}, 
      \VDomino{bb}{ba}
      \right\}, d_0 = \VDomino{a}{ab}$\\
      \uncover<3-|handout> {a pour solution
        \VDomino{a}{ab}\, 
        \VDomino{bb}{ba}\,
        \VDomino{a}{ab}\,
        \VDomino{a}{ab}\,
        \VDomino{baba}{a}.}
    \end{itemize}
  }

  
\end{frame}
\endgroup

% SPDX-License-Identifier: CC-BY-SA-4.0
% Author: Matthieu Perrin
% Part: <Nom de la partie>
% Section: <Nom de la section>
% Sub-section: <Nom de la sous-section>  % (facultatif, laisser vide si non utilisé)
% Frame: <Titre de la slide>

\begingroup

\begin{frame}{Exercice}

  \vspace{-3mm}
  \begin{block}{Décrire les solutions minimales\footnote{C'est-à-dire non préfixe d'une autre solution} de l'instance du PCPI suivante}
    \begin{itemize}
    \item\vspace{1mm} Ensemble de sortes de dominos :
      {\Large
        $$D = \left\{\begin{array}{c}
        \VDomino[2.00cm]{\langle\phantom{0, abab \rangle \leadsto\langle}}{\langle 0, abab \rangle \leadsto\langle},
        \VDomino[2.00cm]{\rangle \leadsto \langle 0, \rangle}{\phantom{\rangle \leadsto \langle 0, }\rangle}, \vspace{1mm} \\
        \VDomino[0.75cm]{0,a}{1,\phantom{a}},
        \VDomino[0.75cm]{0,b}{2,\phantom{b}},
        \VDomino[0.75cm]{1,b}{0,\phantom{b}},
        \VDomino[0.75cm]{1,b}{2,\phantom{b}},
        \VDomino[0.75cm]{2,a}{0,\phantom{a}},
        \VDomino[0.75cm]{2,a}{1,\phantom{a}}, \vspace{1mm}\\
        \VDomino[0.75cm]{a}{a},
        \VDomino[0.75cm]{b}{b},
        \VDomino[0.75cm]{\rangle\leadsto\langle}{\rangle\leadsto\langle}
        \end{array}\right\}$$
      }
    \item\vspace{3mm} Domino initial :
      {\Large
        $$d_0 = \VDomino[2.00cm]{\langle\phantom{0, abab \rangle \leadsto\langle}}{\langle 0, abab \rangle \leadsto\langle}$$
      }
      \vspace{1mm}
    \end{itemize}
    sur l'alphabet $\Sigma \eqdef \{a,b,0,1,2,\leadsto, \langle, \rangle, \text{`$,$'}\}$.
  \end{block}
  
\end{frame}

\endgroup

% SPDX-License-Identifier: CC-BY-SA-4.0
% Author: Matthieu Perrin
% Part: <Nom de la partie>
% Section: <Nom de la section>
% Sub-section: <Nom de la sous-section>  % (facultatif, laisser vide si non utilisé)
% Frame: <Titre de la slide>

\begingroup

\begin{frame}{Solution de l'exercice}

  \vspace{-2mm}
  \begin{block}{Rappel de l'instance :}
    \vspace{-3mm}
           {\small $$\left\{d_0 =
             \VDomino[1.50cm]{\langle\phantom{0, abab \rangle \leadsto\langle}}{\langle 0, abab \rangle \leadsto\langle},
             \VDomino[1.10cm]{\rangle \leadsto \langle 0, \rangle}{\phantom{\rangle \leadsto \langle 0, }\rangle},
             \VDomino[0.55cm]{0,a}{1,\phantom{a}},
             \VDomino[0.55cm]{0,b}{2,\phantom{b}},
             \VDomino[0.55cm]{1,b}{0,\phantom{b}},
             \VDomino[0.55cm]{1,b}{2,\phantom{b}},
             \VDomino[0.55cm]{2,a}{0,\phantom{a}},
             \VDomino[0.55cm]{2,a}{1,\phantom{a}},
             \VDomino[0.55cm]{a}{a},
             \VDomino[0.55cm]{b}{b},
             \VDomino[0.7cm]{\rangle\leadsto\langle}{\rangle\leadsto\langle}
             \right\}$$}
  \end{block}

  \vspace{-1mm}
  \begin{alertblock}{Oui, il y a deux solutions minimales :} 
    \begin{itemize}
    \item\vspace{-1mm} $\langle 0, abab \rangle \leadsto \langle 1, bab\rangle\leadsto\langle 0, ab\rangle\leadsto\langle 1, b\rangle\leadsto\langle 0, \rangle$
    \end{itemize}
        {\footnotesize
          $$
          \VDomino[1.30cm]{\langle\phantom{0, abab \rangle \leadsto\langle}}{\langle 0, abab \rangle \leadsto\langle}\, 
          \VDomino[0.50cm]{0,a}{1,\phantom{a}}                                                                       \, 
          \VDomino[0.30cm]{b}{b}                                                                                     \, 
          \VDomino[0.30cm]{a}{a}                                                                                     \, 
          \VDomino[0.30cm]{b}{b}                                                                                     \, 
          \VDomino[0.60cm]{\rangle\leadsto\langle}{\rangle\leadsto\langle}                                           \, 
          \VDomino[0.50cm]{1,b}{0,\phantom{b}}                                                                       \, 
          \VDomino[0.30cm]{a}{a}                                                                                     \, 
          \VDomino[0.30cm]{b}{b}                                                                                     \, 
          \VDomino[0.60cm]{\rangle\leadsto\langle}{\rangle\leadsto\langle}                                           \, 
          \VDomino[0.50cm]{0,a}{1,\phantom{a}}                                                                       \, 
          \VDomino[0.30cm]{b}{b}                                                                                     \, 
          \VDomino[0.60cm]{\rangle\leadsto\langle}{\rangle\leadsto\langle}                                           \, 
          \VDomino[0.50cm]{1,b}{0,\phantom{b}}                                                                       \, 
          \VDomino[0.90cm]{\rangle \leadsto \langle 0, \rangle}{\phantom{\rangle \leadsto \langle 0, }\rangle}
          $$
        }
        \begin{itemize}
        \item\vspace{-1mm} $\langle 0, abab \rangle \leadsto \langle 1, bab\rangle\leadsto\langle 2, ab\rangle\leadsto\langle 1, b\rangle\leadsto\langle 0, \rangle$
        \end{itemize}
            {\footnotesize
              $$
              \VDomino[1.30cm]{\langle\phantom{0, abab \rangle \leadsto\langle}}{\langle 0, abab \rangle \leadsto\langle}\, 
              \VDomino[0.50cm]{0,a}{1,\phantom{a}}                                                                       \, 
              \VDomino[0.30cm]{b}{b}                                                                                     \, 
              \VDomino[0.30cm]{a}{a}                                                                                     \, 
              \VDomino[0.30cm]{b}{b}                                                                                     \, 
              \VDomino[0.60cm]{\rangle\leadsto\langle}{\rangle\leadsto\langle}                                           \, 
              \VDomino[0.50cm]{1,b}{2,\phantom{b}}                                                                       \, 
              \VDomino[0.30cm]{a}{a}                                                                                     \, 
              \VDomino[0.30cm]{b}{b}                                                                                     \, 
              \VDomino[0.60cm]{\rangle\leadsto\langle}{\rangle\leadsto\langle}                                           \, 
              \VDomino[0.50cm]{2,a}{1,\phantom{a}}                                                                       \, 
              \VDomino[0.30cm]{b}{b}                                                                                     \, 
              \VDomino[0.60cm]{\rangle\leadsto\langle}{\rangle\leadsto\langle}                                           \, 
              \VDomino[0.50cm]{1,b}{0,\phantom{b}}                                                                       \, 
              \VDomino[0.90cm]{\rangle \leadsto \langle 0, \rangle}{\phantom{\rangle \leadsto \langle 0, }\rangle}
              $$
            }
  \end{alertblock}

  \begin{exampleblock}{Remarque -- $abab$ est-il accepté par l'automate ?} 
    \begin{itemize}
    \item\vspace{-1mm} $\langle 0, abab \rangle \leadsto \langle 1, bab\rangle\leadsto\langle 0, ab\rangle\leadsto\langle 1, b\rangle\leadsto\langle 0, \varepsilon \rangle$
    \item $\langle 0, abab \rangle \leadsto \langle 1, bab\rangle\leadsto\langle 2, ab\rangle\leadsto\langle 1, b\rangle\leadsto\langle 0, \varepsilon \rangle$
    \end{itemize}
  \end{exampleblock}
  
  \on[bottom=-2mm,right=3cm] {\footnotesize
    \begin{tikzpicture}[automaton]
      \state[initial, accepting] (0) at (0.00, 3.00) {0};
      \state[                  ] (1) at (2.00, 3.00) {1};
      \state[                  ] (2) at (1.00, 4.41) {2};
      
      \path (0) edge[bend right] node[swap] {$a$} (1);
      \path (0) edge             node[swap] {$b$} (2);
      \path (1) edge             node[swap] {$b$} (0);
      \path (1) edge[bend right] node[swap] {$b$} (2);
      \path (2) edge[bend right] node[swap] {$a$} (0);
      \path (2) edge             node[swap] {$a$} (1);
    \end{tikzpicture}
  }
  
\end{frame}

\endgroup

% SPDX-License-Identifier: CC-BY-SA-4.0
% Author: Matthieu Perrin
% Part: <Nom de la partie>
% Section: <Nom de la section>
% Sub-section: <Nom de la sous-section>  % (facultatif, laisser vide si non utilisé)
% Frame: <Titre de la slide>

\begingroup

\begin{frame}{Généralisation aux machines de Turing}
  
  \onExampleBlock[left=.5\textwidth,top=-5mm] {Exemple -- $\{a^{n+1} b^n \mid n\in \mathbb{N}\}$}{\footnotesize
    \begin{tikzpicture}[automaton]
      \state[alert ob=<{1,5}>,   initial  ] (0) at (0,1.2) {0}; 
      \state[alert ob=<{2,6,11,12,15,16}> ] (1) at (1.5,1.2) {1}; 
      \state[alert ob=<{3,7}>             ] (2) at (1.5,0.0) {2}; 
      \state[alert ob=<{4,13,14}>         ] (3) at (0,0.0) {3}; 
      \state[alert ob=<8>,       accepting] (4) at (3,0.0) {4}; 
      
      \path                   (0) edge             node       {\smTMtransR{a}{\blank}}                       (1);
      \path[alert ob=<15-16>] (1) edge             node       {\smTMtransL{\blank}{\blank}}                  (2);
      \path                   (2) edge             node       {\smTMtransL{b}{\blank}}                       (3);
      \path                   (3) edge             node       {\smTMtransR{\blank}{\blank}}                  (0);
      \path                   (2) edge             node[swap] {\smTMtransR{\blank}{\blank}}                  (4);
      \path[alert ob=<11-12>] (1) edge[loop right] node       {\smAlign{\smTMtransR{a}{a}\smTMtransR{b}{b}}} (1);
      \path[alert ob=<13-14>] (3) edge[loop left]  node       {\smAlign{\smTMtransL{a}{a}\smTMtransL{b}{b}}} (3);
    \end{tikzpicture}
  }
  
  \on[x=-.25\textwidth,y=-2mm] {\footnotesize
    \begin{tikzpicture}[tape, x=7mm, y=7mm]
      \cell{\oneof[\blank]{}}
      \cell{\oneof[\blank]{\on<13,14>{$a$}}}             
      \cell{\oneof[$a$]   {\on<2->{\blank}\on<9->  {$a$}\on<15->{$a$}}}    \smhead<1>            \smheadfromb<4>{1}
      \cell{\oneof[$a$]   {\on<6->{\blank}\on<9->  {$a$}\on<15->{$b$}}}    \smheadb<{5,7}>       \smheadfromb<13-16>{1}
      \cell{\oneof[$b$]   {\on<4->{\blank}\on<9-14>{$b$}}}                 \smheadb<{3,6,8}>     \smheadfromb<11-12>{-1}
      \cell{\oneof[\blank]{\on<11-14>{$b$}}}                               \smheadfromb<2>{-2}
      \cell{\oneof[\blank]{}}
    \end{tikzpicture}
  }
  
  \onBlock<9->[right=.45\textwidth,top=-5mm] {Dominos nécessaires}{\footnotesize
    $
    \begin{array}{rl}
      D=&
      \uncover<10-|handout>{
        \left\{\VDomino[1.50cm]{\langle\phantom{, 0, aab \rangle \leadsto\langle}}{\langle, 0, aab \rangle \leadsto\langle}\right\} 
        \hspace{3mm}d_0=\VDomino[1.50cm]{\langle\phantom{, 0, aab \rangle \leadsto\langle}}{\langle, 0, aab \rangle \leadsto\langle} \vspace{1mm}\\
      }
      \uncover<10-|handout>{\cup &
        \left\{\VDomino[1.50cm]{\uncover<16-|handout>{w}\,\rangle\leadsto\langle \uncover<16-|handout>{x}, 4, \uncover<16-|handout>{y}\rangle}
                       {\phantom{y\,\rangle\leadsto\langle \blank, 4,\,}\uncover<16-|handout>{z}\,\rangle}
                       \only<16-|handout>{\,\middle|\, w, x, y, z \in \{\blank, \varepsilon\}} \right\} \vspace{.5mm}\\
      }
      \uncover<10-|handout>{\cup &
        \left\{\VDomino[1]{\uncover<16-|handout>{w}\,\rangle\leadsto\langle\,\uncover<16-|handout>{x}}
                       {\uncover<16-|handout>{y}\,\rangle\leadsto\langle\,\uncover<16-|handout>{z}} \only<16-|handout>{\,\middle|\, w, x, y, z \in \{\blank, \varepsilon\}} \right\} \vspace{1mm}\\
      }
      \uncover<12-|handout>{\cup &
        \left\{\VDomino[1]{x}{x} \,\middle|\, x\in \Gamma \right\} \\
      }
      \uncover<12-|handout>{\cup &
        \left\{\VDomino[1]{\phantom{y},q, x}{y ,r,\phantom{x}} \,\middle|\, q\xrightarrow{\smTMtransR{x}{y}} r \in \rightarrow \right\}\\
      }
      \uncover<14-|handout>{\cup &
        \left\{\VDomino[1]{z ,q, \phantom{z}\,x}{\phantom{z},r, z\, y} \,\middle|\, q\xrightarrow{\smTMtransL{x}{y}} r \in \rightarrow \land z\in \Gamma \right\}\\
      }
    \end{array}
    $
  }

  \onBlock<17|handout>[y=-1cm,anchor=north] {Solution au PCPI}{\footnotesize
    $$
    \VDomino[1.50cm]{\langle\phantom{, 0, aab \rangle \leadsto\langle}}{\langle, 0, aab \rangle \leadsto\langle}\, 
    %                                                                                                            
    \VDomino[0.80cm]{\phantom{\blank},0,a}{\blank,1,\phantom{\blank}}                                           \, 
    \VDomino[0.30cm]{a}{a}                                                                                      \, 
    \VDomino[0.30cm]{b}{b}                                                                                      \, 
    \VDomino[0.80cm]{\rangle\leadsto\langle\blank}{\rangle\leadsto\langle\phantom{\blank}}                      \, 
    %                                                                                                            
    \VDomino[0.80cm]{\phantom{a},1,a}{a,1,\phantom{a}}                                                          \, 
    \VDomino[0.30cm]{b}{b}                                                                                      \, 
    \VDomino[0.80cm]{\rangle\leadsto\langle}{\rangle\leadsto\langle}                                            \, 
    %                                                                                                            
    \VDomino[0.30cm]{a}{a}                                                                                      \, 
    \VDomino[0.80cm]{\phantom{b},1,b}{b,1,\phantom{b}}                                                          \, 
    \VDomino[0.80cm]{\phantom{\blank}\rangle\leadsto\langle}{\blank\rangle\leadsto\langle}                      \, 
    %                                                                                                            
    \VDomino[0.30cm]{a}{a}                                                                                      \, 
    \VDomino[0.80cm]{b,1,\phantom{b}\blank}{\phantom{b},2,b\blank}                                              \, 
    \VDomino[0.80cm]{\rangle\leadsto\langle\phantom{\blank}}{\rangle\leadsto\langle\blank}                      \, 
    \VDomino[0.80cm]{a,2,b\phantom{\blank}}{\phantom{a},3,a\blank }                                             \, 
    $$
    $$
    %
    \VDomino[0.80cm]{\blank\rangle\leadsto\langle}{\phantom{\blank}\rangle\leadsto\langle}                      \, 
    %                                                                                                           \, 
    \VDomino[0.80cm]{\blank,3,\phantom{\blank}a}{\phantom{\blank},3,\blank a}                                   \, 
    \VDomino[0.80cm]{\blank\rangle\leadsto\langle}{\phantom{\blank}\rangle\leadsto\langle}                      \, 
    %                                                                                                           \, 
    \VDomino[0.80cm]{\phantom{\blank},3,\blank}{\blank,0,\phantom{\blank}}                                      \, 
    \VDomino[0.30cm]{a}{a}                                                                                      \, 
    \VDomino[0.80cm]{\rangle\leadsto\langle\blank}{\rangle\leadsto\langle\phantom{\blank}}                      \, 
    %                                                                                                           \, 
    \VDomino[0.80cm]{\phantom{\blank},0,a}{\blank,1,\phantom{a}}                                                \, 
    \VDomino[0.80cm]{\phantom{\blank}\rangle\leadsto\langle}{\blank\rangle\leadsto\langle}                      \, 
    %                                                                                                           \, 
    \VDomino[0.80cm]{\blank,1,\phantom{\blank}\blank}{\phantom{\blank},2,\blank\blank}                          \, 
    \VDomino[0.80cm]{\rangle\leadsto\langle}{\rangle\leadsto\langle}                                            \, 
    %                                                                                                           \, 
    \VDomino[0.80cm]{\phantom{\blank},2,\blank}{\blank,4,\phantom{\blank}}                                      \, 
    \VDomino[1.50cm]{\blank\rangle\leadsto\langle \blank,4, \rangle}{\phantom{\blank\rangle\leadsto\langle \blank,4,}\rangle}
    %
    $$
  }

  \onBlock<1-8>[y=-1cm,anchor=north] {Exécution pour le mot $aab$}{
    $$
    \begin{array}{@{}c@{\,}c@{\,}c@{\,}c@{\,}c@{\,}c@{\,}c@{}}
      \uncover<1-|handout> {\langle \varepsilon, 0, aab \rangle}                                              
      &\uncover<2-|handout>{\leadsto}& \uncover<2-|handout>{\langle \varepsilon, 1, ab \rangle}              
      &\uncover<2-|handout>{\leadsto}& \uncover<2-|handout>{\langle a, 1, b \rangle}                         
      &\uncover<2-|handout>{\leadsto}& \uncover<2-|handout>{\langle ab, 1, \varepsilon \rangle}             \\
      &\uncover<3-|handout>{\leadsto}& \uncover<3-|handout>{\langle a, 2, b \rangle}                         
      &\uncover<4-|handout>{\leadsto}& \uncover<4-|handout>{\langle \varepsilon, 3, a \rangle}                                                                    
      &\uncover<4-|handout>{\leadsto}& \uncover<4-|handout>{\langle \varepsilon, 3, \blank a \rangle}       \\ 
      &\uncover<5-|handout>{\leadsto}& \uncover<5-|handout>{\langle \varepsilon, 0, a \rangle}               
      &\uncover<6-|handout>{\leadsto}& \uncover<6-|handout>{\langle \varepsilon, 1, \varepsilon \rangle}     
      &\uncover<7-|handout>{\leadsto}& \uncover<7-|handout>{\langle \varepsilon, 2, \varepsilon \rangle}    \\ 
      &\uncover<8-|handout>{\leadsto}& \uncover<8-|handout>{\langle \varepsilon, 4, \varepsilon \rangle}     
    \end{array}
    $$
  }
  
  \onAlertBlock<9>[y=-1cm,anchor=north] {Objectif}{\centering
    Concevoir une instance du PCPI dont les solutions encodent l'exécution de la machine de Turing sur le mot $aab$.
  }
  
  \onBlock<10>[y=-1cm,anchor=north] {Encoder l'exécution
    $\langle C_0 \rangle \leadsto \langle C_1 \rangle \leadsto \langle C_2 \rangle \leadsto \ldots  \leadsto \langle C_n \rangle \leadsto \langle C_f \rangle$}{
    $$
    \VDomino        [1.00cm]{\langle\phantom{C_0 \rangle \leadsto\langle}}{\langle C_0 \rangle \leadsto\langle}          \, 
    \VDominoContextC[1.00cm]{C_0}{C_1}                                         \, 
    \VDomino        [0.70cm]{\rangle\leadsto\langle}{\rangle\leadsto\langle}   \, 
    \VDominoContextC[1.00cm]{C_1}{C_2}                                         \, 
    \VDomino        [0.70cm]{\rangle\leadsto\langle}{\rangle\leadsto\langle}   \, 
    \VDominoContextC[1.00cm]{C_2\rangle\leadsto\ldots}{\ldots\leadsto\langle C_n}                                   \, 
    \VDomino        [0.70cm]{\rangle\leadsto\langle}{\rangle\leadsto\langle}   \, 
    \VDominoContextC[1.00cm]{C_n}{C_f}                                         \, 
    \VDomino        [1.00cm]{\rangle\leadsto\langle C_f \rangle}{\phantom{\rangle\leadsto\langle C_f}\rangle}      \, 
    $$
  }
  
  \onBlock<11,12>[y=-1cm,anchor=north] {Encoder l'action $\langle a, 1, abb \rangle \leadsto \langle aa, 1, bb \rangle$}{
    $$
    \VDominoContextL[2cm]{\ldots}{\ldots \langle a, 1, abb }       \, 
    \VDomino[0.70cm]{\rangle\leadsto\langle}{\rangle\leadsto\langle} \,
    \uncover<12>{
      \VDomino[0.40cm]{a}{a}                                         \, 
      \VDomino[1.00cm]{\phantom{a},1,a}{a,1,\phantom{a}}             \, 
      \VDomino[0.40cm]{b}{b}                                         \, 
      \VDomino[0.40cm]{b}{b}                                         \,
    }
    \VDomino[0.70cm]{\rangle\leadsto\langle}{\rangle\leadsto\langle} \, 
    \VDominoContextR[2cm]{ aa, 1, bb \rangle \ldots}{\ldots}
    $$
  }
  
  \onBlock<13,14>[y=-1cm,anchor=north] {Encoder l'action $\langle aaa, 3, bb \rangle \leadsto \langle aa, 3, abb \rangle$}{
    $$
    \VDominoContextL[1.70cm]{\ldots}{\ldots \langle aaa, 3, bb }          \, 
    \VDomino        [0.70cm]{\rangle\leadsto\langle}{\rangle\leadsto\langle}   \, 
    \uncover<14>{
      \VDomino      [0.40cm]{a}{a}                                           \, 
      \VDomino      [0.40cm]{a}{a}                                           \, 
      \VDomino      [1.00cm]{a,3,\phantom{a}b}{\phantom{a},3,ab}             \, 
      \VDomino      [0.40cm]{b}{b}                                           \,
    }
    \VDomino        [0.70cm]{\rangle\leadsto\langle}{\rangle\leadsto\langle}   \, 
    \VDominoContextR[1.70cm]{ aa, 3, abb \rangle \ldots}{\ldots}
    $$
  }
  
  \onBlock<15,16>[y=-1cm,anchor=north] {Gérer les bords du ruban : $\langle ab, 1, \varepsilon \rangle \leadsto \langle a, 2, b \rangle$}{
    $$
    \VDominoContextL[1.30cm]{\ldots}{\ldots \langle ab, 1,     }          \, 
    \VDomino        [1.20cm]{\phantom{\blank}\rangle\leadsto\langle\phantom{\blank}}{\uncover<16>{\alert{\blank}}\rangle\leadsto\langle\phantom{\blank}}   \, 
    \VDomino        [0.30cm]{a}{a}                                           \, 
    \VDomino        [1.30cm]{b, 1, \phantom{b}\alert{\blank}}{\phantom{b},2,b\alert{\blank}} \, 
    \VDomino        [1.20cm]{\rangle\leadsto\langle}{\rangle\leadsto\langle}   \,
    \VDominoContextC[1.00cm]{,2,b}{\ldots} \,
    \VDomino        [1.20cm]{\uncover<16>{\alert{\blank}}\rangle\leadsto\langle\phantom{\blank}}{\phantom{\blank}\rangle\leadsto\langle\phantom{\blank}}   \,
    \VDominoContextR[1.30cm]{ \ldots~\ldots}{\ldots}
    $$
  }

\end{frame}

\endgroup

% SPDX-License-Identifier: CC-BY-SA-4.0
% Author: Matthieu Perrin
% Part: <Nom de la partie>
% Section: <Nom de la section>
% Sub-section: <Nom de la sous-section>  % (facultatif, laisser vide si non utilisé)
% Frame: <Titre de la slide>

\begingroup

\begin{frame}{Démonstration de l'indécidabilité de PCPI}

    \vspace{-2mm}
  \begin{block}{Théorème -- indécidabilité de PCPI}
    Le Problème de Correspondance de Post Intitialisé est indécidable
  \end{block}
    \vspace{-1mm}
  \begin{block}{Démonstration -- Réduction du problème de l'arrêt}
    \vspace{1mm}

    Posons $\varphi : \left\{\begin{array}{rcl}
    \textsc{halt} &\rightarrow& \textsc{pcpi} \vspace{.5mm}\\
    \left\langle \langle \Sigma, \Gamma, \#, Q, q_0, F, \rightarrow \rangle\footnote[frame]{Sans perte de généralité, on suppose que la machine efface son ruban avant de s'arrêter}, u\right\rangle  &\mapsto& \left\langle D,
    {\footnotesize\VDomino[1.2cm]{\langle\phantom{, q_0, u \rangle \leadsto\langle}}{\langle, q_0, u \rangle \leadsto\langle}} \right\rangle
    \end{array}
    \right.$, avec :

    \vspace{2mm}
    
    {\small    
      $$\begin{array}{@{}r@{\,}l@{}}
        D=&
        \left\{
        \VDomino[1.5]{\langle\phantom{, q_0, u \rangle \leadsto\langle}}{\langle, q_0, u \rangle \leadsto\langle}
        \right\}
        \cup
        \left\{
        \VDomino[1.5]{w\,\rangle\leadsto\langle x, f, y\rangle}{\phantom{y\,\rangle\leadsto\langle \blank, 4,\,}z\,\rangle},
        \VDomino[1.5]{w\,\rangle\leadsto\langle\,x}{y\,\rangle\leadsto\langle\,z}
        \,\middle|\, w, x, y, z \in \{\blank, \varepsilon\} \land f\in F \right\} \vspace{1mm}\\
        \cup &
        \left\{
        \VDomino[.5]{x}{x}
        \,\middle|\, x\in \Gamma \right\}
        \cup
        \left\{
        \VDomino[1]{\phantom{y},q, x}{y ,r,\phantom{x}}
        \,\middle|\, q\xrightarrow{\scriptsize\smTMtransR{x}{y}} r \in \rightarrow \right\}
        \cup
        \left\{
        \VDomino[1]{z ,q, \phantom{z}\,x}{\phantom{z},r, z\, y}
        \,\middle|\, q\xrightarrow{\scriptsize\smTMtransL{x}{y}} r \in \rightarrow \land z\in \Gamma \right\}\\
      \end{array}$$
    }

    \vspace{2mm}

    \begin{itemize}
    \item $M$ termine sur $u$ si, et seulement si, le PCPI $\left\langle D, d_0 \right\rangle$ a une solution
    \item De plus, $\varphi$ est calculable
    \item Donc $\textsc{halt} \le_m \textsc{pcpi}$
    \item Or \textsc{halt} est indécidable, donc \textsc{pcpi} est indécidable
    \end{itemize}
  \end{block}
  
\end{frame}
  
\endgroup

% SPDX-License-Identifier: CC-BY-SA-4.0
% Author: Matthieu Perrin
% Part: <Nom de la partie>
% Section: <Nom de la section>
% Sub-section: <Nom de la sous-section>  % (facultatif, laisser vide si non utilisé)
% Frame: <Titre de la slide>

\begingroup

\begin{frame}{Démonstration de l'indécidabilité de PCP}
  \newcommand\sep{\bullet}
  \small
    \begin{tikzpicture}

      \draw (8, 3.6) node{\begin{minipage}{\textwidth}
          Réduction du problème de correspondance de Post initialisé.
          \begin{itemize}
          \item Soit $\structure{\left\langle D, \HDomino[1.1]{\gamma_1\cdots \gamma_{|\gamma|}}{\rightarrow_1\cdots \rightarrow_{|\rightarrow|}} \right\rangle}$ une instance du PCPI sur $\Sigma$
          \item<1,7->  $\left\langle D, \HDomino{\gamma}{\rightarrow} \right\rangle$ a une solution ssi l'instance $\alert{D'}$ du PCP sur $\alert{\Sigma'}$ a une solution :

            \uncover<7->{
              $$\begin{array}{rcl}
                \Sigma' &=&\Sigma\cup \{\sep\}
                \vspace{1mm}\\
                D' &=&
                \left\{\VDomino[2]{~\alert{\sep}~\gamma_1\sep ...\sep\gamma_{|\gamma|}~~~~}{\alert{\sep}~~\rightarrow_1\sep ... \sep\rightarrow_{|\rightarrow|}~~\alert{\sep}}\right\}
                \vspace{1mm}\\
                &\cup&
                \left\{ \VDomino[2]{\alert{\sep}~~\alpha_1 \sep ... \sep\alpha_{|\alpha|}~~~~}{~~~~\beta_1\sep ... \sep\beta_{|\beta|}~~\alert{\sep}} \,\middle|\,
                \VDomino[1.1]{\alpha_1 ... \alpha_{|\alpha|}}{\beta_1 ... \beta_{|\beta|}} \in D
                \right\}
                \vspace{1mm}\\
                &\cup&
                \left\{
                \VDomino{\alert{\sep~\sep}}{\phantom{\sep}~\alert{\sep}}
                \right\}
              \end{array}
              $$
            }
          \item<1,8-> \vspace{-1mm}Où la réduction $\left\langle D, \HDomino{\gamma}{\rightarrow} \right\rangle \mapsto D'$  est calculable.
          \item<1,8-> Or le problème de correspondance de Post initialisé est indécidable.
          \item<1,8-> Donc le problème de correspondance de Post est indécidable.
          \end{itemize}
      \end{minipage}};

      \draw<1> (8, 3.6) node{$\begin{array}{ccc}\Sigma' &=& ???\\ D' &=& ???\end{array}$};

      
      \draw<2-6> (8, 3) node{\begin{minipage}{\textwidth}
          \begin{exampleblock}{Exemple : $\color{black}\left\langle D=\left\{
              \HDomino{a}{baa},
              \HDomino{ab}{aa},
              \HDomino{bba}{bb}
              \right\},
              d_0 = \HDomino{bba}{bb}
              \right\rangle$}
            \begin{enumerate}
            \item<3-> Nouveau caractère $\sep$ à gauche pour forcer l'initialisation avec $d_0$
            \item<4-> Intercaller $\sep$ entre toutes les lettres pour permettre le découpage
            \item<5-> Ajouter un domino final pour terminer par $\sep\sep$
            \end{enumerate}

            \uncover<3-> {
            $$D' = \left\{
            \only<-3>{
              \VDomino[1.1]{\alert{\sep} b b a}{\alert{\sep} b b},
              \VDomino[1.1]{\alert{\sep} a}{b a a},
              \VDomino[1.1]{\alert{\sep} a b}{a a},
              \VDomino[1.1]{\alert{\sep} b b a}{b b}
            }
            \only<4>{
              \VDomino[1.1]{\sep b \alert{\sep} b \alert{\sep} a} {\sep b \alert{\sep} b \,\alert{\sep}},
              \VDomino[1.1]{\sep a}                               {b \alert{\sep} a \alert{\sep} a \,\alert{\sep}},
              \VDomino[1.1]{\sep a \alert{\sep} b}                {a \alert{\sep} a \,\alert{\sep}},
              \VDomino[1.1]{\sep b \alert{\sep} b \alert{\sep} a} {b \alert{\sep} b \,\alert{\sep}}
            }
            \only<5->{
              \VDomino[1.1]{\sep b \sep b\sep a}{\sep b\sep b\,\sep},
              \VDomino[1.1]{\sep a}{b\sep a\sep a\,\sep},
              \VDomino[1.1]{\sep a\sep b}{a\sep a\,\sep },
              \VDomino[1.1]{\sep b\sep b\sep a}{b\sep b\,\sep }
            }
            \uncover<5->{, \VDomino{\alert<5>{\sep\sep}}{\alert<5>{\sep}}}
            \right\}$$
            }

            \uncover<6-> {
            \begin{tabular}{ll}
              \example{Solution pour $D'$ :}&  \scalebox{.9}{
                $\VDomino[1.1]{\sep b \sep b\sep a}{\sep b\sep b\sep}\,
                \VDomino[1.1]{\sep a\sep b}{a\sep a\sep}\,
                \VDomino[1.1]{\sep b\sep b\sep a}{b\sep b\sep}\,
                \VDomino[1.1]{\sep a}{b\sep a\sep a\sep}\,
                \VDomino{\sep\sep}{\sep}$
              }
            \vspace{1mm}\\
            \example{Solution pour $D$ :}&
            \scalebox{.9}{
              $\VDomino[1.1]{b b a}{ b b}\,
              \VDomino[1.1]{ab}{a a}\,
              \VDomino[1.1]{bba}{b b}\,
              \VDomino[1.1]{a}{baa}$
            }
            \end{tabular}
            }

          \end{exampleblock}
      \end{minipage}};
    \end{tikzpicture}
\end{frame}

\endgroup

% SPDX-License-Identifier: CC-BY-SA-4.0
% Author: Matthieu Perrin
% Part: <Nom de la partie>
% Section: <Nom de la section>
% Sub-section: <Nom de la sous-section>  % (facultatif, laisser vide si non utilisé)
% Frame: <Titre de la slide>

\begingroup

\begin{frame}{Ambiguïté des grammaires algébriques}
  \begin{block}{Définitions -- Grammaire algébrique ambiguë}
    Soit $G = \langle \Sigma, \Gamma, S, R \rangle$ une grammaire. 
    \begin{itemize}
    \item $G$ est \structure{algébrique} si $R \subseteq \Gamma \times (\Sigma\cup \Gamma)^\star$
      \begin{itemize}
      \item Le membre gauche de toutes ses règles est réduit à un non-terminal
      \end{itemize}
    \item $G$ est \structure{ambiguë} si deux arbres de dérivation génèrent le même mot.
    \end{itemize}
  \end{block}

    \vspace{-2mm}
  \begin{exampleblock}{Exemple : $\left\{A \rightarrow A + A \mid a\right.$}
    Le mot $\example{a+a+a}$ a deux dérivations :

    \hspace\fill
    \scalebox{.8}{\begin{tikzpicture}
      \draw (5,7.1) node{$A$};
      \draw (4,6.4) node{$A$};
      \draw (5,6.4) node{$+$};
      \draw (6,6.4) node{$A$};
      \draw (3,5.7) node{$A$};
      \draw (4,5.7) node{$+$};
      \draw (5,5.7) node{$A$};
      \draw (3,5.0) node{$a$};
      \draw (6,5.7) node{$a$};
      \draw (5,5.0) node{$a$};

      \draw[-latex] (4.8,6.9) -- (4.2,6.6); 
      \draw[-latex] (5.0,6.9) -- (5.0,6.6); 
      \draw[-latex] (5.2,6.9) -- (5.8,6.6);
      
      \draw[-latex] (3.8,6.2) -- (3.2,5.9); 
      \draw[-latex] (4.0,6.2) -- (4.0,5.9); 
      \draw[-latex] (4.2,6.2) -- (4.8,5.9);
      
      \draw[-latex] (3.0,5.5) -- (3.0,5.2); 
      \draw[-latex] (5.0,5.5) -- (5.0,5.2); 
      \draw[-latex] (6.0,6.2) -- (6.0,5.9);
    \end{tikzpicture}}
    \hspace\fill
    \scalebox{.8}{\begin{tikzpicture}
      \draw (5,7.1) node{$A$};
      \draw (4,6.4) node{$A$};
      \draw (5,6.4) node{$+$};
      \draw (6,6.4) node{$A$};
      \draw (5,5.7) node{$A$};
      \draw (6,5.7) node{$+$};
      \draw (7,5.7) node{$A$};

      \draw (4,5.7) node{$a$};
      \draw (5,5.0) node{$a$};
      \draw (7,5.0) node{$a$};

      \draw[-latex] (4.8,6.9) -- (4.2,6.6); 
      \draw[-latex] (5.0,6.9) -- (5.0,6.6); 
      \draw[-latex] (5.2,6.9) -- (5.8,6.6); 
      \draw[-latex] (5.8,6.2) -- (5.2,5.9); 
      \draw[-latex] (6.0,6.2) -- (6.0,5.9); 
      \draw[-latex] (6.2,6.2) -- (6.8,5.9); 
      \draw[-latex] (4.0,6.2) -- (4.0,5.9);
      \draw[-latex] (5.0,5.5) -- (5.0,5.2);
      \draw[-latex] (7.0,5.5) -- (7.0,5.2);
    \end{tikzpicture}}
    \hspace\fill
    \hspace\fill
  \end{exampleblock}

  \vspace{-2mm}
  \begin{block}{Théorème -- Indécidabilité de l'ambiguïté}
    Savoir si une grammaire $G$ est ambiguë est un problème indécidable.
  \end{block}
\end{frame}
\endgroup

% SPDX-License-Identifier: CC-BY-SA-4.0
% Author: Matthieu Perrin
% Part: <Nom de la partie>
% Section: <Nom de la section>
% Sub-section: <Nom de la sous-section>  % (facultatif, laisser vide si non utilisé)
% Frame: <Titre de la slide>

\begingroup

\begin{frame}{Démonstration de l'indécidabilité de l'ambiguïté}
  \small
  \begin{tikzpicture}
    \draw[white] (8, 3.6) -- (8, -1.5);

    \draw (8, 2.7) node{\begin{minipage}{\textwidth}
        Réduction du problème de correspondance de Post.
        \begin{itemize}
        \item Soit $\structure{D = \left\{d_1 = \HDomino{\alpha_1}{\beta_1}, ..., d_n = \HDomino{\alpha_n}{\beta_n} \right\}}$ une instance du PCP sur $\structure{\Sigma}$.
        \item<1,7->  $D$ a une solution ssi la grammaire $\alert{G_D}$ est ambigüe :

          \vspace{5mm}
          \uncover<7->{
            $$\begin{array}{rcl}
              G_D &=& \left\langle \Sigma \cup D, \{S, A, B\}, S, R \right\rangle
              \vspace{1mm}\\
              R &=& \left\{\begin{array}{rclclcl}
              S &\rightarrow& A \mid B\\
              A &\rightarrow& \alpha_1 \, A \, d_1 &\mid &...& \mid& \alpha_n \, A \, d_n \\
              A &\rightarrow& \alpha_1 \, \phantom{A\,} d_1 &\mid& ... &\mid& \alpha_n \,\phantom{A\,} d_n \\
              B &\rightarrow& \beta_1 \, B \, d_1 &\mid& ... &\mid& \beta_n \, B \, d_n \\
              B &\rightarrow& \beta_1 \, \phantom{B\,} d_1 &\mid& ... &\mid& \beta_n \, \phantom{B\,} d_n \\
              \end{array}\right.
            \end{array}
            $$
          }
          \vspace{5mm}

        \item<1,7-> Où la réduction $D \mapsto G_D$  est calculable.
        \item<1,7-> Or le problème de correspondance de Post est indécidable.
        \item<1,7-> Donc le problème de l'ambiguïté est indécidable. 
        \end{itemize}
    \end{minipage}};

    \draw<1> (8, 3) node{$G_D = ???$};

    
    \draw<2-5> (8, 2) node{\begin{minipage}{\textwidth}
        \begin{exampleblock}{Exemple pour $D = {\color{black}\left\{\HDomino[.7]{a}{baa}, \HDomino[.7]{ab}{aa}, \HDomino[.7]{bba}{bb} \right\}}$}

          \begin{enumerate}
          \item $D$ a pour solution $\footnotesize\HDomino{bba}{bb}\, \HDomino{ab}{aa}\, \HDomino{bba}{bb}\, \HDomino{a}{baa}$.
            \begin{itemize}
            \item<3-> On veut générer $bba ab bba a$ de deux manières différentes
              \begin{itemize}
              \item \structure{$S \vdash A\vdash^\star bba \cdot ab \cdot bba \cdot a$} et \example{$S \vdash B\vdash^\star bb \cdot aa \cdot bb \cdot baa$}
              \end{itemize}
            \end{itemize}
          \item<4-> $A$ et $B$ ne doivent pas engendrer $\varepsilon$
          \item<5-> $A$ et $B$ ne doivent pas être ambigües 
          \end{enumerate}

          \vspace{-4mm}
          $$G_D : \left\{\begin{array}{rcrcrcrcr}
          S &\rightarrow& \multicolumn{5}{l}{\uncover<3->{A \mid B}}\\
          \uncover<3->{A} &\uncover<3->{\rightarrow}&
          \uncover<3->{
            \structure{a}   \,A\, \uncover<5->{\footnotesize \HDomino{a}{baa}}  &\mid&
            \structure{ab}  \,A\, \uncover<5->{\footnotesize \HDomino{ab}{aa}}  &\mid&
            \structure{bba} \,A\, \uncover<5->{\footnotesize \HDomino{bba}{bb}} &\uncover<3>{\mid}&
            \uncover<3>{\varepsilon}
          }
          \\
          \uncover<4->{A} &\uncover<4->{\rightarrow}&
          \uncover<4->{
            \structure{a} \phantom{\,A\,} \uncover<5->{\footnotesize \HDomino{a}{baa}} &\mid&
            \structure{ab} \phantom{\,A\,} \uncover<5->{\footnotesize \HDomino{ab}{aa}} &\mid&
            \structure{bba} \phantom{\,A\,} \uncover<5->{\footnotesize \HDomino{bba}{bb}}
          }
          \\
          \uncover<3->{B} &\uncover<3->{\rightarrow}&
          \uncover<3->{
            \example{baa} \,B\, \uncover<5->{\footnotesize \HDomino{a}{baa}} &\mid&
            \example{aa} \,B\, \uncover<5->{\footnotesize \HDomino{ab}{aa}} &\mid&
            \example{bb} \,B\, \uncover<5->{\footnotesize \HDomino{bba}{bb}}&\uncover<3>{\mid}&
            \uncover<3>{\varepsilon}
          }
          \\
          \uncover<4->{B} &\uncover<4->{\rightarrow}&
          \uncover<4->{
            \example{baa} \phantom{\,B\,} \uncover<5->{\footnotesize \HDomino{a}{baa}} &\mid&
            \example{aa} \phantom{\,B\,} \uncover<5->{\footnotesize \HDomino{ab}{aa}} &\mid&
            \example{bb} \phantom{\,B\,} \uncover<5->{\footnotesize \HDomino{bba}{bb}}
          }
          \\
          \end{array}\right.$$
        \end{exampleblock}
    \end{minipage}};


    \draw<6> (8, 2.467) node{\begin{minipage}{\textwidth}
        \begin{exampleblock}{Exemple pour $D = {\color{black}\left\{\HDomino[.7]{a}{baa}, \HDomino[.7]{ab}{aa}, \HDomino[.7]{bba}{bb}\right\}}$}

          \vspace{-4mm}

          $$G_D : \left\{\begin{array}{rcrcrcrcr}
          S &\rightarrow& \multicolumn{5}{l}{\uncover<3->{A \mid B}}\\
          \uncover<3->{A} &\uncover<3->{\rightarrow}&
          \uncover<3->{
            \structure{a} \,A\, \uncover<5->{\footnotesize\HDomino{a}{baa}} &\mid&
            \structure{ab} \,A\, \uncover<5->{\footnotesize\HDomino{ab}{aa}} &\mid&
            \structure{bba} \,A\, \uncover<5->{\footnotesize\HDomino{bba}{bb}} &\uncover<3>{\mid}&
            \uncover<3>{\varepsilon}
          }
          \\
          \uncover<4->{A} &\uncover<4->{\rightarrow}&
          \uncover<4->{
            \structure{a} \phantom{\,A\,} \uncover<5->{\footnotesize\HDomino{a}{baa}} &\mid&
            \structure{ab} \phantom{\,A\,} \uncover<5->{\footnotesize\HDomino{ab}{aa}} &\mid&
            \structure{bba} \phantom{\,A\,} \uncover<5->{\footnotesize\HDomino{bba}{bb}}
          }
          \\
          \uncover<3->{B} &\uncover<3->{\rightarrow}&
          \uncover<3->{
            \example{baa} \,B\, \uncover<5->{\footnotesize\HDomino{a}{baa}} &\mid&
            \example{aa} \,B\, \uncover<5->{\footnotesize\HDomino{ab}{aa}} &\mid&
            \example{bb} \,B\, \uncover<5->{\footnotesize\HDomino{bba}{bb}}&\uncover<3>{\mid}&
            \uncover<3>{\varepsilon}
          }
          \\
          \uncover<4->{B} &\uncover<4->{\rightarrow}&
          \uncover<4->{
            \example{baa} \phantom{\,B\,} \uncover<5->{\footnotesize\HDomino{a}{baa}} &\mid&
            \example{aa} \phantom{\,B\,} \uncover<5->{\footnotesize\HDomino{ab}{aa}} &\mid&
            \example{bb} \phantom{\,B\,} \uncover<5->{\footnotesize\HDomino{bba}{bb}}
          }
          \\
          \end{array}\right.$$

          \begin{itemize}
          \item<6-> $G_D$ a deux dérivations pour : 
            $bbaabbbaa\, \scriptsize
            \HDomino{a}{baa}\, 
            \HDomino{bba}{bb}\,
            \HDomino{ab}{aa}\,
            \HDomino{bba}{bb}
            $.
            \begin{itemize}
            \item $S \vdash A \vdash^\star \structure{bba}\cdot \structure{ab} \cdot \structure{bba} \cdot \structure{a} \cdot \scriptsize
              \HDomino{a}{baa}\cdot
              \HDomino{bba}{bb}\cdot
              \HDomino{ab}{aa}\cdot
              \HDomino{bba}{bb}$
            \item $S \vdash B \vdash^\star \example{bb}\cdot \example{aa} \cdot \example{bb} \cdot \example{baa} \cdot\hspace{-.2mm} \scriptsize
              \HDomino{a}{baa}\cdot
              \HDomino{bba}{bb}\cdot
              \HDomino{ab}{aa}\cdot
              \HDomino{bba}{bb}$
            \end{itemize}
          \end{itemize}
        \end{exampleblock}
    \end{minipage}};
  \end{tikzpicture}
\end{frame}
\endgroup

 
\section{Limites de Complexité}

\end{document}
 
\endinput
